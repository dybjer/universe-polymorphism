%\documentclass[12pt,a4paper]{amsart}
\documentclass[11pt,a4paper]{article}
%\ifx\pdfpageheight\undefined\PassOptionsToPackage{dvips}{graphicx}\else%
%\PassOptionsToPackage{pdftex}{graphicx}
\PassOptionsToPackage{pdftex}{color}
%\fi

%\usepackage{diagrams}

%\usepackage[all]{xy}
\usepackage{url}
\usepackage[utf8]{inputenc}
\usepackage{verbatim}
\usepackage{latexsym}
\usepackage{amssymb,amstext,amsmath,amsthm}
\usepackage{epsf}
\usepackage{epsfig}
% \usepackage{isolatin1}
\usepackage{a4wide}
\usepackage{verbatim}
\usepackage{proof}
\usepackage{latexsym}
%\usepackage{mytheorems}
\newtheorem{theorem}{Theorem}[section]
\newtheorem{corollary}{Corollary}[theorem]
\newtheorem{lemma}{Lemma}[theorem]
\newtheorem{proposition}{Proposition}[theorem]
\theoremstyle{definition}
\newtheorem{definition}[theorem]{Definition}
\newtheorem{remark}{Remark}[theorem]
\newtheorem{TODO}{TODO}[theorem]

\usepackage{float}
\floatstyle{boxed}
\restylefloat{figure}


%%%%%%%%%copied from SymmetryBook by Marc

% hyperref should be the package loaded last
%% \usepackage[backref=page,
%%             colorlinks,
%%             citecolor=linkcolor,
%%             linkcolor=linkcolor,
%%             urlcolor=linkcolor,
%%             unicode,
%%             pdfauthor={CAS},
%%             pdftitle={Symmetry},
%%             pdfsubject={Mathematics},
%%             pdfkeywords={type theory, group theory, univalence axiom}]{hyperref}
% - except for cleveref!
\usepackage[capitalize]{cleveref}
%\usepackage{xifthen}
\usepackage{xcolor}
\definecolor{linkcolor}{rgb}{0,0,0.5}

%%%%%%%%%
\def\oge{\leavevmode\raise
.3ex\hbox{$\scriptscriptstyle\langle\!\langle\,$}}
\def\feg{\leavevmode\raise
.3ex\hbox{$\scriptscriptstyle\,\rangle\!\rangle$}}

%%%%%%%%%
\newcommand\myfrac[2]{
 \begin{array}{c}
 #1 \\
 \hline \hline
 #2
\end{array}}


\newcommand*{\Scale}[2][4]{\scalebox{#1}{$#2$}}%
\newcommand*{\Resize}[2]{\resizebox{#1}{!}{$#2$}}

\newcommand{\II}{\mathbb{I}}
\newcommand{\refl}{\mathsf{refl}}
\newcommand{\mkbox}[1]{\ensuremath{#1}}


\newcommand{\Id}{\mathsf{Id}}
\newcommand{\conv}{=}
%\newcommand{\conv}{\mathsf{conv}}
\newcommand{\lam}[2]{{\langle}#1{\rangle}#2}
\def\NN{\mathsf{N}}
\def\UU{\mathsf{U}}
\def\JJ{\mathsf{J}}
\def\Level{\mathsf{Level}}
%\def\Type{\hbox{\sf Type}}
\def\ZERO{\mathsf{0}}
\def\SUCC{\mathsf{S}}

\newcommand{\type}{\mathsf{type}}
\newcommand{\LAM}{\mathsf{lam}}
\newcommand{\APP}{\mathsf{app}}
\newcommand{\mypi}[3]{\Pi_{#1:#2}#3}
\newcommand{\mylam}[3]{\lambda_{#1:#2}#3}
\newcommand{\mysig}[3]{\Sigma_{#1:#2}#3}
\newcommand{\N}{\mathsf{N}}
\newcommand{\Set}{\mathsf{Set}}
\newcommand{\El}{\mathsf{El}}
%\newcommand{\U}{\mathsf{U}} clashes with def's in new packages
\newcommand{\T}{\mathsf{T}}
\newcommand{\sT}{\mathsf{t}}
\newcommand{\Usuper}{\UU_{\mathrm{super}}}
\newcommand{\Tsuper}{\T_{\mathrm{super}}}
%\newcommand{\conv}{\mathrm{conv}}
\newcommand{\idtoeq}{\mathsf{idtoeq}}
\newcommand{\isEquiv}{\mathsf{isEquiv}}
\newcommand{\ua}{\mathsf{ua}}
\newcommand{\UA}{\mathsf{UA}}
%\newcommand{\Level}{\mathrm{Level}}
\def\Constraint{\mathsf{Constraint}}
\def\Ordo{\mathcal{O}}

\newcommand{\Con}{\mathsf{ Con}}
\newcommand{\Elem}{\mathsf{Elem}}
\newcommand{\Type}{\mathsf{Type}}
\newcommand{\id}{\mathsf{id}}
\newcommand{\pp}{\mathsf{p}}
\newcommand{\qq}{\mathsf{q}}

\def\Ctx{\mathrm{Ctx}}
\def\Ty{\mathrm{Ty}}
\def\Tm{\mathrm{Tm}}

\def\CComega{\mathrm{CC}^\omega}
\setlength{\oddsidemargin}{0in} % so, left margin is 1in
\setlength{\textwidth}{6.27in} % so, right margin is 1in
\setlength{\topmargin}{0in} % so, top margin is 1in
\setlength{\headheight}{0in}
\setlength{\headsep}{0in}
\setlength{\textheight}{9.19in} % so, foot margin is 1.5in
\setlength{\footskip}{.8in}

% Definition of \placetitle
% Want to do an alternative which takes arguments
% for the names, authors etc.

\newcommand{\natrec}{\mathsf{natrec}}
%\rightfooter{}
\newcommand{\set}[1]{\{#1\}}
\newcommand{\sct}[1]{[\![#1]\!]}
%\usepackage{diagrams}
\usepackage{color}
\newcommand\coloremph[2][red]{\textcolor{#1}{\emph{#2}}}
\newcommand\norm[1]{\left\lVert #1 \right\rVert}
\newcommand\greenemph[2][green]{\textcolor{#1}{\emph{#2}}}
\newcommand{\EMP}[1]{\emph{\textcolor{red}{#1}}}




\begin{document}

\title{Dependent Type Theory with a Cumulative Hierarchy of Universes}

\author{}
\date{}
\maketitle

\begin{abstract}
  We first explain what is a model of type theory with a cumulative hieararchy of universes.
  We present three versions of dependent type theory, that all can be used as a presentation of the initial model.
\end{abstract}

\section{Introduction}\label{sec:intros}


\section{Model of type theory}

\subsection{What is a model of type theory?}

\subsection{Definition}

 We present a formal system, which at the same time can be thought of  describing  the
syntax of basic dependent type theory, with \EMP{explicit substitutions} and
a \EMP{name-free} (de Bruijn index) presentation, and defining what is a model
of type theory.

 A model of type theory consists of one set of \EMP{contexts}. If
$\Gamma$ and $\Delta$ are contexts they determine a set $\Delta\rightarrow\Gamma$
of \EMP{substitutions}. If $\Gamma$ is a context, it determines a set
$\Type(\Gamma)$ of \EMP{types} in the context $\Gamma$. Finally, if
$A$ is in $\Type(\Gamma)$ then this determines a set
$\Elem(\Gamma,A)$ of elements of type $A$ in the context $\Gamma$.

 This describes the \EMP{sort} of type theory. We describe now the \EMP{operations}
and the equations they have to satisfy. For any context $\Gamma$
we have an identity substitution $\id:\Gamma\rightarrow\Gamma$.
We also have a composition operator $\sigma\delta:\Theta\rightarrow\Gamma$ if
$\delta:\Theta\rightarrow\Delta$ and $\sigma:\Delta\rightarrow\Gamma$. 
The equations are
$$\sigma~ \id = \id~ \sigma = \sigma~~~~~~~~(\theta\sigma)\delta = \theta(\sigma\delta)$$

We have a terminal context $1$ and for, any context $\Gamma$, a map
$():\Gamma\rightarrow 1$. Furthermore $\sigma = ()$ if
$\sigma:\Gamma\rightarrow 1$.

 If $A$ in $\Type(\Gamma)$ and $\sigma:\Delta\rightarrow\Gamma$
 we should have $A\sigma$ in $\Type(\Delta)$
Furthermore 
$$A~\id = A~~~~~~(A\sigma)\delta = A(\sigma\delta)$$
If $a$ in $\Elem(\Gamma,A)$ and $\sigma:\Delta\rightarrow\Gamma$
we should have $a\sigma$ in $\Elem(\Delta,A\sigma)$.
Furthermore 
$$a~\id = a~~~~~~(a\sigma)\delta = a(\sigma\delta)$$

 We have a \EMP{context extension} operation: if $A$ in $\Type(\Gamma)$ we have
a new context $\Gamma.A$. Furthermore there is a projection
$\pp:\Gamma.A\rightarrow \Gamma$ and a special element
$\qq$ in $\Elem(\Gamma.A,A\pp)$. If $\sigma: \Delta\rightarrow \Gamma$ and
$A$ in $\Type(\Gamma)$ and $a$ in $\Elem(\Delta,A\sigma)$
we have
an extension operation $(\sigma,a):\Delta\rightarrow \Gamma.A$.
We should have 
$$\pp (\sigma,a) = \sigma~~~~~~~~~\qq (\sigma,a) = a~~~~~~~~~
(\sigma,a)\delta = (\sigma\delta,a\delta)~~~~~~~~~~(\pp,\qq) = \id$$

 If $a$ in $\Elem(\Gamma,A)$ we write $[a]= (\id,a):\Gamma\rightarrow \Gamma.A$.
 Thus if $B$ in $\Type(\Gamma.A)$ and $a$ in $\Elem(\Gamma,A)$
 we have $B[a]$ in $\Type(\Gamma)$.
 If furthermore $b$ in $\Elem(\Gamma.A,B)$ we have $b[a]$ in $\Elem(\Gamma,B[a])$.

 If $\sigma:\Delta\rightarrow\Gamma$ and $A$ in $\Type(\Gamma)$
 we define $\sigma^+:\Delta.A\sigma\rightarrow\Gamma.A$ to be
 $(\sigma\pp,\qq)$.

 The extension operation can then be defined as $(\sigma,u) = [u]\sigma^+$.
 Thus instead of the extension operation, we could have chosen the operations
 $[u]$ and $\sigma^+$ as primitive, like in \cite{Ehrhard}. Our argument is independent of this choice
 of primitive operations.

 We suppose furthermore one operation $\Pi~A~B$ such that
$\Pi~A~B$ in $\Type(\Gamma)$ if $A$ in $\Type(\Gamma)$ and $B$ in $\Type(\Gamma.A)$.
We should have $(\Pi~A~B)\sigma = \Pi~(A\sigma)~(B\sigma^+)$.

We have an abstraction operation $\lambda b$ in $\Elem(\Gamma,\Pi~A~B)$
for $b$ in $\Elem(\Gamma.A,B)$
and an application operation $c~a$ in $\Elem(\Gamma,B[a])$
for $c$ in $\Elem(\Gamma,\Pi~A~B)$ and $a$ in $\Elem(\Gamma,A)$.
These operations should satisfy the equations
$$
{(\lambda b)}~{a} = b[a],~~~~~~c = \lambda (c\pp~\qq),~~~~~
(\lambda b)\sigma = \lambda (b\sigma^+),~~~~
({c}~{a})\sigma = {c\sigma}~{(a\sigma)}
$$


\subsection{Cumulative hierarchy of universes}

We have a family of constants $\UU_n$ in $\Type(\Gamma)$ with $\UU_n\sigma = \UU_n$ and
$\T_n(a)$ in $\Type(\Gamma)$ for $a$ in $\Elem(\Gamma,\UU_n)$.

We also have $\Pi^n(a,b)$ in $\Elem(\Gamma,\UU_n)$ if $a$ in $\Elem(\Gamma,\UU_n)$ and
$b$ in $\Elem(\Gamma.\T_n(a),\UU_n)$. We have
$\T_n(\Pi^n(a,b)) = \Pi~\T_n(a)~\T_n(b)$.

We have $\sT_n^m(a)$ in $\Elem(\Gamma,\UU_m)$ if $n\leqslant m$ and $a$ in $\Elem(\Gamma,\UU_n)$
with $\T_m(\sT_n^m(a)) = \T_n(a)$ in $\Type(\Gamma)$
and $\sT_n^m(\Pi^n(a,b)) = \Pi^m(\sT_n^m(a),\sT_n^m(b))$.

Finally we have $\UU^m_n$ in $\Elem(\Gamma,\UU_m)$ with $\T_m(\UU^m_n) = \UU_n$ in $\Type(\Gamma)$ if
$n<m$.


\subsection{Initial/Term model}

We have a syntactic version $T_0$ which presents the initial model, like for equational theory.

The term model presents remarkable properties: e.g. we have that $\Pi~A~B = \Pi~C~D$ implies
$A = B$ in $\Type(\Gamma)$ and $C = D$ in $\Type(\Gamma.A)$. This is not the case for an arbitrary model; for instance
we always have the trivial model where all sorts have exactly one element.


\section{Annotated type theory $T_1$ with Tarski style universes}

We now present the system $T_1$ with implicit substitutions. The connection between $T_0$ and $T_1$ is elementary and $T_1$ can
be used to present the initial model as well as $T_0$.

The syntax for types is
$$
A,B~::=~\Pi_{A}B~|~\UU_n~|~\T_n a
$$
and the syntax for terms, with $v_i$ de Bruijn index
$$
a,b~::=~v_i~|~\APP(A,B,c,a)~|~\LAM(A,B,b)~|~\sT^k_n a~|~\UU^k_n~|~\Pi^n~a~b
$$

\medskip

Renaming and substitutions are now \EMP{defined} operations. They are defined by the syntax
$$
r~::=~\pp~|~r^+~~~~~~~~~\sigma~::=~[a]~|~\sigma^+
$$
and we define first $ar,~Ar$ by induction on $a,A$ and then $a\sigma,~A\sigma$ by induction on $a,A$. The main clauses are
$$
v_i\pp = v_{i+1}~~~~~~v_0r^+ = v_0\sigma^+ = v_0~~~~~~v_0[a] = a~~~~~~~~~~v_{i+1}r^+ = (v_ir)\pp~~~~~~~v_{i+1}\sigma^+ = (v_i\sigma)\pp
$$


\medskip



$$
\frac{\Gamma\vdash A}{\Gamma.A\vdash}~~~~~~\frac{}{()\vdash}~~~~~~~
\frac{\Gamma\vdash}{\Gamma.A\vdash v_0:A\pp}~~~~~~~
\frac{\Gamma\vdash v_i:A}{\Gamma.B\vdash v_{i+1}:A\pp}~~~~~~~
$$
The judgment $\Gamma\vdash A$ expresses that $A$ is a type in context $\Gamma$.
We may write it $A~\type~(\Gamma)$ and may omit the global context $\Gamma$.
$$
\frac{A~\type~~~~~~B~\type~(A)}{\Pi_AB~\type}~~~~~~~~~
\frac{b:B~(A)}{\LAM(A,B,b):\Pi_AB}~~~~~~~~
\frac{c:\Pi_AB~~~~~~a:A}
     {\APP(A,B,c,a):B[a]}
$$

We write $\conv$ for definitional equality (or conversion).
The main conversion rules are\footnote{We omit systematically {\em congruence} rules
  such as the rule that $c~a~\conv~c'~a'$ follows from $c~\conv~c'$ and $a~\conv~a'$.}
$$
\frac{ a:A~~~~~~ A~ \conv~ B}{ a:B}~~~~~~~~~
\frac{ a ~\conv~a':A~~~~~~ A  ~\conv~ B}{ a ~\conv~a':B}
$$
$$
\frac{b:B~(A)~~~~~~~~ a:A}{ \APP(A,B,\LAM(A,B,b),a)  ~\conv~ b[a]:B[a]}
~~~~~~~
\frac{f\pp~v_0 = g\pp~v_0:B~(A)}{ f = g : \Pi_AB}
$$

\section{Rules for an external sequence of universes}

$$
\UU_n~\type~~~~~~
\frac{A:\UU_{n}}{\T_{n}(A)~\type}~~~~~~
{\UU^{k}_l}:\UU_{k}~~~~~~~~~\T_{k}({\UU^{k}_l}) = \UU_{l}
~~~~~~~~~\frac{A:\UU_{n}}
{\sT_{n}^{m}(A):\UU_{m}}
n\leqslant m
$$
We require for all $n,m$
\[
\T_m(\sT_{n}^{m}(A)) = \T_{n}(A) 
\]
We add for all $n,m,p$ with $m\leqslant n\leqslant p$
$$
\sT_{n}^n(a) = a \quad\text{and}\quad \sT_{n}^p\sT_{m}^n = \sT_m^p.
$$
$$
\frac{a:\UU_{n}~~~~~~b:\T_{n}(A)\rightarrow \UU_{n}}
     {\Pi^{n} a b:\UU_{n}}~~~~~~~~~
$$
with conversion rules
$$
\T_{n}~(\Pi^{n} a b) = \Pi_{\T_{n}(a)}{\T_{n}(b)}~~~~~~~
$$
and
$$
\sT_{n}^{m}~(\Pi^{n} a b) = \Pi^{m} \sT_{n}^{m}(a) \sT_{n}^{m}(b)~~~~~~
$$

\section{Rules for a type theory $T_2$ with Russell style universes}

The syntax is now for types and terms
$$
A,B,a,b~::=~v_i~|~c~a~|~\lambda b~|~\Pi_{A}B~|~\UU_n
$$

$$
\frac{A~\type~~~~~~B~\type~(A)}{\Pi_AB~\type}~~~~~~~~~
\frac{b:B~(A)}{\lambda b:\Pi_AB}~~~~~~~~
\frac{c:\Pi_AB~~~~~~a:A}
     {c~a:B[a]}
$$

$$
\UU_n~\type~~~~~~
\frac{A:\UU_{n}}{A~\type}
~~~~~~\frac{A:\UU_{n}}{A:\UU_{n+1}}
~~~~~~~
{\UU_l}:\UU_{l+1}
$$
$$
\frac{A:\UU_{n}~~~~~~B:\UU_n(A)}
     {\Pi_AB:\UU_{n}}$$

\medskip


     Note that the system $T_2$ is \EMP{not} closed under $\eta$-reduction. For instance, in a context
     $x:\UU_1\rightarrow\UU_1$ we have $\lambda_yx~y:\UU_0\rightarrow\UU_2$ but we do \EMP{not} have
     $x:\UU_0\rightarrow\UU_2$.


\medskip


We clearly have a stripping function from $T_1$ to $T_2$. We define $|M|$ where we forget the annotations.
We have $M:A$ in $T_1$ implies $|M|:|A|$ in $T_2$.

\section{Equivalence between $T_1$ and $T_2$}

\begin{lemma}
  If $\Gamma\vdash t:A$ and $\Gamma\vdash u:A$ and $|t| = |u|$ is $\beta$-normal then $\Gamma\vdash t = u:A$.
  If $\Gamma\vdash t:A$ and $\Gamma\vdash u:B$ and $|t| = |u|$ is neutral then
  \begin{itemize}
    \item either $A = B$
      and $\Gamma\vdash t = u:A$
    \item or $A = \UU_k$ and $B = \UU_l$ and $\sT_k^m(t) = \sT_l^m(u):\UU_m$ with $m = max(k,l)$.
  \end{itemize}
\end{lemma}

\begin{theorem}
  If $\Gamma\vdash J$ in the system $T_2$ then there exists a unique, up to conversion, judgement $\Gamma_1\vdash J_1$
  in the system $T_1$ such that $|\Gamma_1| = \Gamma$ and $|J_1| = J$.
\end{theorem}





\bibliographystyle{plain}
\bibliography{refs}

\end{document}
