\documentclass[11pt,a4paper]{article}
%\ifx\pdfpageheight\undefined\PassOptionsToPackage{dvips}{graphicx}\else%
%\PassOptionsToPackage{pdftex}{graphicx}
\PassOptionsToPackage{pdftex}{color}
%\fi

%\usepackage{diagrams}

%\usepackage[all]{xy}
\usepackage{url}
\usepackage[utf8]{inputenc}
\usepackage{verbatim}
\usepackage{latexsym}
\usepackage{amssymb,amstext,amsmath,amsthm}
\usepackage{epsf}
\usepackage{epsfig}
\usepackage{float}
\usepackage{a4wide}
\usepackage{verbatim}
\usepackage{proof}
\usepackage{latexsym}
%\usepackage{mytheorems}
\newtheorem{theorem}{Theorem}[section]
\newtheorem{corollary}{Corollary}[theorem]
\newtheorem{lemma}{Lemma}[theorem]
\newtheorem{proposition}{Proposition}[theorem]
\theoremstyle{definition}
%\newtheorem{definition}[theorem]{Definition}
%\newtheorem{remark}{Remark}[theorem]
\newtheorem{TODO}{TODO}[theorem]
\newtheorem{remark}{Remark}
\newtheorem{definition}{Definition}
\newtheorem{example}{Example}
\newtheorem{conjecture}{Conjecture}

\usepackage{float}
\floatstyle{boxed}
\restylefloat{figure}


%%%%%%%%%copied from SymmetryBook by Marc

% hyperref should be the package loaded last
\usepackage[backref=page,
            colorlinks,
            citecolor=linkcolor,
            linkcolor=linkcolor,
            urlcolor=linkcolor,
            unicode,
            pdfauthor={BCDE},
            pdftitle={Universes},
            pdfsubject={Mathematics},
            pdfkeywords={type theory, universes}]{hyperref}
% - except for cleveref!
\usepackage[capitalize,noabbrev]{cleveref}
%\usepackage{xifthen}
\usepackage{xcolor}
\definecolor{linkcolor}{rgb}{0,0,0.5}

%%%%%%%%%
\def\oge{\leavevmode\raise
.3ex\hbox{$\scriptscriptstyle\langle\!\langle\,$}}
\def\feg{\leavevmode\raise
.3ex\hbox{$\scriptscriptstyle\,\rangle\!\rangle$}}

%%%%%%%%%

\newcommand{\mkbox}[1]{\ensuremath{#1}}
\newcommand{\eraser}[1]{}

\newcommand{\conv}{=}
%\newcommand{\conv}{\mathsf{conv}}

\newcommand{\Id}{\mathsf{Id}}
\newcommand{\Eq}{\mathsf{Eq}}
\newcommand{\id}{\mathsf{id}}
\newcommand{\NN}{\mathsf{N}}
\newcommand{\Nat}{\mathbb{N}}
\newcommand{\UU}{\mathsf{U}}
\newcommand{\JJ}{\mathsf{J}}
\newcommand{\AgdaLevel}{\mathsf{Level}}
\newcommand{\Level}{\mathsf{level}}
\newcommand{\Lev}{{\mathbb{L}}}
%\newcommand{\Type{\hbox{\sf Type}}
\newcommand{\ZERO}{\mathsf{0}}
\newcommand{\SUCC}{\mathsf{S}}
\newcommand{\valid}{\mathsf{valid}}
\newcommand{\level}{\mathsf{level}}
%\newcommand{\type}{\mathsf{type}}
\newcommand{\const}{\mathsf{const}}
\newcommand{\lam}[1]{{\langle}#1{\rangle}}
\newcommand{\mylam}[3]{\lambda_{#1:#2}#3}
\newcommand{\mypi}[3]{\Pi_{#1:#2}#3}
\newcommand{\Upi}[3]{\Pi^{#1}\,#2\,#3}
\newcommand{\mysig}[3]{\Sigma_{#1:#2}#3}
\newcommand{\Usig}[3]{\Sigma^{#1}\,#2\,#3}
\newcommand{\app}[2]{{#1\,#2}} % many applications still hard-coded with ~
\newcommand{\Sapp}[1]{\sapp{\SUCC}{#1}}
\newcommand{\sapp}[2]{{#1(#2)}} % strict app for Id, refl, J, natrec, not S (!)
\newcommand{\Idapp}[3]{\sapp{\Id}{#1,#2,#3}}
\newcommand{\Idnapp}[4]{\sapp{\Id^#4}{#1,#2,#3}}
\newcommand{\NRapp}[4]{\sapp{\RR}{#1,#2,#3,#4}}
\newcommand{\Rfapp}[2]{\sapp{\refl}{#1,#2}}
\newcommand{\Japp}[6]{\sapp{\JJ}{#1,#2,#3,#4,#5,#6}}
\newcommand{\RR}{\mathsf{R}}
%\newcommand{\Set}{\mathsf{Set}}
\newcommand{\Group}{\mathsf{Group}}
%\newcommand{\El}{\mathsf{El}}
%\newcommand{\T}{\mathsf{T}}
%\newcommand{\Usuper}{\UU_{\mathrm{super}}}
%\newcommand{\Tsuper}{\T_{\mathrm{super}}}
%\newcommand{\idtoeq}{\mathsf{idtoeq}}
%\newcommand{\isEquiv}{\mathsf{isEquiv}}
%\newcommand{\Equiv}{\mathsf{Equiv}}
\newcommand{\isContr}{\mathsf{isContr}}
%\newcommand{\ua}{\mathsf{ua}}
%\newcommand{\UA}{\mathsf{UA}}
%\newcommand{\natrec}{\mathsf{natrec}}
%\newcommand{\set}[1]{\{#1\}}
%\newcommand{\sct}[1]{[\![#1]\!]}
%\newcommand{\refl}{\mathsf{refl}}
\newcommand{\ttt}[1]{\text{\tt #1}}

\newcommand{\Constraint}{\mathsf{Constraint}}
\newcommand{\Ordo}{\mathcal{O}}
\newcommand{\AFu}{\mathcal{A}}
\newcommand{\Fu}{\mathit{Fu}}

\newcommand{\Ctx}{\mathrm{Ctx}}
\newcommand{\Ty}{\mathrm{Ty}}
\newcommand{\Tm}{\mathrm{Tm}}
\newcommand{\op}{\mathrm{op}}

\newcommand{\CComega}{\mathrm{CC}^\omega}
\setlength{\oddsidemargin}{0in} % so, left margin is 1in
\setlength{\textwidth}{6.27in} % so, right margin is 1in
\setlength{\topmargin}{0in} % so, top margin is 1in
\setlength{\headheight}{0in}
\setlength{\headsep}{0in}
\setlength{\textheight}{9.19in} % so, foot margin is 1.5in
\setlength{\footskip}{.8in}

% Definition of \placetitle
% Want to do an alternative which takes arguments
% for the names, authors etc.

\newcommand\myfrac[2]{
 \begin{array}{c}
 #1 \\
 \hline \hline
 #2
\end{array}}

\def\levelctx{\mathrm{lctx}}
\def\lhom{\mathrm{lhom}}
\def\psiab{\psi_{\alpha\beta}}

\newcommand*{\Scale}[2][4]{\scalebox{#1}{$#2$}}%
\newcommand*{\Resize}[2]{\resizebox{#1}{!}{$#2$}}

\newcommand{\II}{\mathbb{I}}
\newcommand{\refl}{\mathsf{refl}}
%\newcommand{\mkbox}[1]{\ensuremath{#1}}


%\newcommand{\Id}{\mathsf{Id}}
%\newcommand{\conv}{=}
%\newcommand{\conv}{\mathsf{conv}}
%\newcommand{\lam}[2]{{\langle}#1{\rangle}#2}
\def\NN{\mathsf{N}}
\def\UU{\mathsf{U}}
\def\JJ{\mathsf{J}}
%\def\Type{\hbox{\sf Type}}
\def\ZERO{\mathsf{0}}
\def\SUCC{\mathsf{S}}

\newcommand{\RawCtx}{{\tt Ctx}}
\newcommand{\RawSub}{{\tt Sub}}
\newcommand{\RawTy}{{\tt Ty}}
\newcommand{\RawTm}{{\tt Tm}}
\newcommand{\type}{\mathsf{type}}
\newcommand{\N}{\mathsf{N}}
\newcommand{\Set}{\mathsf{Set}}
\newcommand{\El}{\mathsf{El}}
%\newcommand{\U}{\mathsf{U}} clashes with def's in new packages
\newcommand{\T}{\mathsf{T}}
\newcommand{\Usuper}{\UU_{\mathrm{super}}}
\newcommand{\Tsuper}{\T_{\mathrm{super}}}
%\newcommand{\conv}{\mathrm{conv}}
\newcommand{\idtoeq}{\mathsf{idtoeq}}
\newcommand{\isEquiv}{\mathsf{isEquiv}}
\newcommand{\ua}{\mathsf{ua}}
\newcommand{\UA}{\mathsf{UA}}
\def\Constraint{\mathsf{Constraint}}
\def\Ordo{\mathcal{O}}
\def\Pihat{\Pi}

\def\Ctx{\mathrm{ctx}}
\def\Ty{\mathrm{ty}}
\def\Tm{\mathrm{tm}}
\def\Obj{\mathrm{obj}}
\def\Hom{\mathrm{hom}}
\def\id{\mathrm{id}}
\def\lHom{\mathrm{lhom}}
\def\lctx{\mathrm{lctx}}
\def\lty{\mathrm{level}}
\def\ltm{\mathrm{ltm}}
\def\ltmq{\mathrm{ltmq}}
\def\leq{\mathrm{leq}}
\def\lrefl{\mathrm{lrefl}}
\def\lp{\mathrm{lp}}
\def\lq{\mathrm{lq}}
\def\s{\mathrm{s}}
\def\lid{\mathrm{lid}}
\def\cctx{\mathrm{cctx}}
\def\cty{\mathrm{cty}}
\def\ctm{\mathrm{ctm}}
\def\cid{\mathrm{cid}}
\def\cp{\mathrm{cp}}
\def\cq{\mathrm{cq}}
\def\chom{\mathrm{chom}}

\newcommand{\ctx}{\mathrm{ctx}}
\newcommand{\sub}{\mathrm{sub}}
\newcommand{\ty}{\mathrm{ty}}
\newcommand{\tm}{\mathrm{tm}}
%\newcommand{\hom}{\mathrm{hom}}
\newcommand{\tuple}[1]{\langle #1 \rangle}
\def\CComega{\mathrm{CC}^\omega}
\newcommand{\cext}{.}
\def\p{\mathrm{p}}
\def\q{\mathrm{q}}
\def\app{\mathsf{app}}
\def\U{\mathsf{U}}
\def\T{\mathcal{T}}
\newcommand{\Ta}{\mathrm{T}}
\newcommand{\ta}{\mathrm{t}}

\newcommand{\natrec}{\mathsf{natrec}}
%\rightfooter{}
\newcommand{\set}[1]{\{#1\}}
\newcommand{\sct}[1]{[\![#1]\!]}
\def\R{\mathcal{R}}

\def\L{{\mathcal{L}}}
\def\F{\mathcal{C}}
\def\CwF{\mathrm{CwF}}
\def\SCwF{\mathrm{SCwF}}
\def\C{\mathcal{C}}


\begin{document}
\section{Introduction}
The aim of Voevodsky's {\em initiality conjecture project} \cite{voevodsky:initiality} is to provide a general definition of class of dependent type theories and to develop generic metatheory for theories in this class. For example, one would like a generic construction showing that these theories are initial in corresponding categories of models. Such proofs may seem straightforward superficially, but depend on subtle details in the formulation of grammar and inference rules. Hence, Voevodsky insisted on calling such theorems ``conjectures'' until proven rigorously and ideally implemented in a proof assistant.

An example of such an initiality proof is Brunerie and de Boer's \cite{Brunerie:initiality,deBoer:lic} proof in Agda that a version of Martin-Löf type theory with an external tower of universes is an initial contextual category \cite{cartmell:phd,cartmell:apal} with appropriate extra structure. %Theversion of type theory with de Bruijn variables and implicit substitutions. 

A possible approach to Voevodsky's project is based on Cartmell's notion of a {\em generalized algebraic theory (gat)} \cite{cartmell:phd,cartmell:apal}. The most basic rules of dependent type theory can be captured by the gat of categories with families (cwfs) \cite{dybjer:torino}. Moreover, as shown in our paper {\em On generalized algebraic theories and categories with families} \cite{bezem:hofmann} there is a generic construction of an initial model of an arbitrary gat. 
%Maybe an approach to a general notion of dependent type theory is as an initial model of an extension of the gat of cws?

Furthermore, as shown by Castellan, Clairambault, and Dybjer \cite{castellan:lambek}, by considering simply typed cwfs (scwfs) and unityped cwfs (ucwfs) we can also capture various simply typed and untyped logical systems as gats and thus widening the scope of {\em uniform categorical logic} based on gats and cwfs.

In this article we shall introduce the notion of an indexed cwf, that is, a base category $\C$ and a cwf-valued presheaf
$$
T : \C^\op \to \CwF
$$
We shall show that cwfs indexed by (the base category of) a ucwf $\L$ of universe levels (with suitable extra structure) forms a suitable notion of model of Martin-Löf type theory with explicit universe polymorphism along the lines of our paper {\em Type Theory with Explicit Universe Polymorphism} \cite{BezemCDE22}. The benefit is twofold. First, it is a case study for our approach to the initiality conjecture project. Type theory with explicit universe polymorphism is an example of a dependent type theory, and we show that it can be captured by a gat based on a variation of cwfs. Second, it provides an alternative view of our syntactic type theory with explicit universe polymorphism. A proof that a (modification of) the latter forms an initial model of our gat is however beyond the scope of the paper, and we only outline the relationship. 

Moreover, further variations of indexed cwfs extend the scope of gat and cwf-based uniform categorical. For example, untyped predicate logic can be captured by ucwf-indexed scwfs with extra structure for the logical constants. Moreover, typed predicate logic can be captured by scwf-indexed scwfs, and dependently typed predicate logic by cwf-indexed scwfs, both with suitable extra structure for type formers and logical constants.

(* Something about cwfs as intermediate between mainstream categorical notions and syntactic logical systems, and this holds also applies to variations such as ucwfs, scwfs, and indexed cwfs. The important thing is that these notions can be presented as gats, and hence have generic initial model constructions. Moreover, these gats resemble syntactically presented explicit substitution calculi. *)

\paragraph{Universe polymorphism.} We refer to our previous paper on explicit universe polymorphism \cite{BezemCDE22} for a presentation of the inference rules of Martin-Löf type theory with explicit universe polymorphism. There the reader can also find motivation and examples. Here we only give a brief overview. 

An implicit form of universe polymorphism in dependent type theory was introduced by Huet \cite{Huet87} and is an essential feature of the proof assistant Coq (Rocq) \cite{coq:general}. Alternatively, Agda \cite{agda-wiki} and Lean \cite{moura:lean} employ versions of universe polymorphism, where universe levels are explicitly declared. 

In our paper we followed Courant's approach \cite{Courant02} to explicit universe polymorphism and introduced special universe level {\em judgments}:
$$
l\ \level
\hspace{5em}
l = l'
$$
in addition to the usual judgment forms of type theory. Moreover, all judgments may depend on universe level variables as well as ordinary variables declared in the context. We emphasized that, unlike in Agda, universe levels do {\em not} form a {\em type} in our setting, and instead we added the above judgment forms. 

Furthermore, we presented an extension where equational constraints between universe levels can be declared \cite{BezemCDE22}, building on a proposal by Voevodsky \cite{VV}. This extension can also be described by a gat, but we will postpone this topic to a forthcoming paper.

\paragraph{Plan of the paper.} In Section 2 we present the infinitary gat of cwfs with extra structure for the type formers and an externally indexed tower of universes. We consider both cumulative and non-cumulative universes. In Section 3 we present the {\em finitary} gat of level-indexed type theory with extra structure for the type formers and an internally indexed tower of universes. (* In Section 4 we present the generalized algebraic theory of cwfs with extra structure doubly indexed by levels and level constraints *). In Section 5 we discuss models. In Section 6 we conclude. The appendix contains generalized algebraic theories appearing in previous publications. 

\paragraph{Acknowledgement.} This paper is written in honour of professor Stefano Berardi, the University of Torino, on the occasion of his 60th birthday. Stefano is a valued friend and colleague who spent the autumn of 1994 (check?) in Göteborg with the second and third author. He has made fundamental contributions to type theory and constructivity, in particular to the understanding of the constructive content of classical logic.



Postpone details of gats of cwfs and notational convention until section 2 and 3 (when needed). Something about qiits.

\bibliographystyle{plain}
\bibliography{../refs}
\end{document}
