\documentclass[11pt,a4paper]{article}
%\ifx\pdfpageheight\undefined\PassOptionsToPackage{dvips}{graphicx}\else%
%\PassOptionsToPackage{pdftex}{graphicx}
\PassOptionsToPackage{pdftex}{color}
%\fi

%\usepackage{diagrams}

%\usepackage[all]{xy}
\usepackage{url}
\usepackage[utf8]{inputenc}
\usepackage{verbatim}
\usepackage{latexsym}
\usepackage{amssymb,amstext,amsmath,amsthm}
\usepackage{epsf}
\usepackage{epsfig}
\usepackage{float}
\usepackage{a4wide}
\usepackage{verbatim}
\usepackage{proof}
\usepackage{latexsym}
%\usepackage{mytheorems}
\newtheorem{theorem}{Theorem}[section]
\newtheorem{corollary}{Corollary}[theorem]
\newtheorem{lemma}{Lemma}[theorem]
\newtheorem{proposition}{Proposition}[theorem]
\theoremstyle{definition}
%\newtheorem{definition}[theorem]{Definition}
%\newtheorem{remark}{Remark}[theorem]
\newtheorem{TODO}{TODO}[theorem]
\newtheorem{remark}{Remark}
\newtheorem{definition}{Definition}
\newtheorem{example}{Example}
\newtheorem{conjecture}{Conjecture}

\usepackage{float}
\floatstyle{boxed}
\restylefloat{figure}


%%%%%%%%%copied from SymmetryBook by Marc

% hyperref should be the package loaded last
\usepackage[backref=page,
            colorlinks,
            citecolor=linkcolor,
            linkcolor=linkcolor,
            urlcolor=linkcolor,
            unicode,
            pdfauthor={BCDE},
            pdftitle={Universes},
            pdfsubject={Mathematics},
            pdfkeywords={type theory, universes}]{hyperref}
% - except for cleveref!
\usepackage[capitalize,noabbrev]{cleveref}
%\usepackage{xifthen}
\usepackage{xcolor}
\definecolor{linkcolor}{rgb}{0,0,0.5}

%%%%%%%%%
\def\oge{\leavevmode\raise
.3ex\hbox{$\scriptscriptstyle\langle\!\langle\,$}}
\def\feg{\leavevmode\raise
.3ex\hbox{$\scriptscriptstyle\,\rangle\!\rangle$}}

%%%%%%%%%

\newcommand{\mkbox}[1]{\ensuremath{#1}}
\newcommand{\eraser}[1]{}

\newcommand{\conv}{=}
%\newcommand{\conv}{\mathsf{conv}}

\newcommand{\Id}{\mathsf{Id}}
\newcommand{\Eq}{\mathsf{Eq}}
\newcommand{\id}{\mathsf{id}}
\newcommand{\NN}{\mathsf{N}}
\newcommand{\Nat}{\mathbb{N}}
\newcommand{\UU}{\mathsf{U}}
\newcommand{\JJ}{\mathsf{J}}
\newcommand{\AgdaLevel}{\mathsf{Level}}
\newcommand{\Level}{\mathsf{level}}
\newcommand{\Lev}{{\mathbb{L}}}
%\newcommand{\Type{\hbox{\sf Type}}
\newcommand{\ZERO}{\mathsf{0}}
\newcommand{\SUCC}{\mathsf{S}}
\newcommand{\valid}{\mathsf{valid}}
\newcommand{\level}{\mathsf{level}}
%\newcommand{\type}{\mathsf{type}}
\newcommand{\const}{\mathsf{const}}
\newcommand{\lam}[1]{{\langle}#1{\rangle}}
\newcommand{\mylam}[3]{\lambda_{#1:#2}#3}
\newcommand{\mypi}[3]{\Pi_{#1:#2}#3}
\newcommand{\Upi}[3]{\Pi^{#1}\,#2\,#3}
\newcommand{\mysig}[3]{\Sigma_{#1:#2}#3}
\newcommand{\Usig}[3]{\Sigma^{#1}\,#2\,#3}
\newcommand{\app}[2]{{#1\,#2}} % many applications still hard-coded with ~
\newcommand{\Sapp}[1]{\sapp{\SUCC}{#1}}
\newcommand{\sapp}[2]{{#1(#2)}} % strict app for Id, refl, J, natrec, not S (!)
\newcommand{\Idapp}[3]{\sapp{\Id}{#1,#2,#3}}
\newcommand{\Idnapp}[4]{\sapp{\Id^#4}{#1,#2,#3}}
\newcommand{\NRapp}[4]{\sapp{\RR}{#1,#2,#3,#4}}
\newcommand{\Rfapp}[2]{\sapp{\refl}{#1,#2}}
\newcommand{\Japp}[6]{\sapp{\JJ}{#1,#2,#3,#4,#5,#6}}
\newcommand{\RR}{\mathsf{R}}
%\newcommand{\Set}{\mathsf{Set}}
\newcommand{\Group}{\mathsf{Group}}
%\newcommand{\El}{\mathsf{El}}
%\newcommand{\T}{\mathsf{T}}
%\newcommand{\Usuper}{\UU_{\mathrm{super}}}
%\newcommand{\Tsuper}{\T_{\mathrm{super}}}
%\newcommand{\idtoeq}{\mathsf{idtoeq}}
%\newcommand{\isEquiv}{\mathsf{isEquiv}}
%\newcommand{\Equiv}{\mathsf{Equiv}}
\newcommand{\isContr}{\mathsf{isContr}}
%\newcommand{\ua}{\mathsf{ua}}
%\newcommand{\UA}{\mathsf{UA}}
%\newcommand{\natrec}{\mathsf{natrec}}
%\newcommand{\set}[1]{\{#1\}}
%\newcommand{\sct}[1]{[\![#1]\!]}
%\newcommand{\refl}{\mathsf{refl}}
\newcommand{\ttt}[1]{\text{\tt #1}}

\newcommand{\Constraint}{\mathsf{Constraint}}
\newcommand{\Ordo}{\mathcal{O}}
\newcommand{\AFu}{\mathcal{A}}
\newcommand{\Fu}{\mathit{Fu}}

\newcommand{\Ctx}{\mathrm{Ctx}}
\newcommand{\Ty}{\mathrm{Ty}}
\newcommand{\Tm}{\mathrm{Tm}}
\newcommand{\op}{\mathrm{op}}

\newcommand{\CComega}{\mathrm{CC}^\omega}
\setlength{\oddsidemargin}{0in} % so, left margin is 1in
\setlength{\textwidth}{6.27in} % so, right margin is 1in
\setlength{\topmargin}{0in} % so, top margin is 1in
\setlength{\headheight}{0in}
\setlength{\headsep}{0in}
\setlength{\textheight}{9.19in} % so, foot margin is 1.5in
\setlength{\footskip}{.8in}

% Definition of \placetitle
% Want to do an alternative which takes arguments
% for the names, authors etc.

\newcommand\myfrac[2]{
 \begin{array}{c}
 #1 \\
 \hline \hline
 #2
\end{array}}

\def\levelctx{\mathrm{lctx}}
\def\lhom{\mathrm{lhom}}
\def\psiab{\psi_{\alpha\beta}}

\newcommand*{\Scale}[2][4]{\scalebox{#1}{$#2$}}%
\newcommand*{\Resize}[2]{\resizebox{#1}{!}{$#2$}}

\newcommand{\II}{\mathbb{I}}
\newcommand{\refl}{\mathsf{refl}}
%\newcommand{\mkbox}[1]{\ensuremath{#1}}


%\newcommand{\Id}{\mathsf{Id}}
%\newcommand{\conv}{=}
%\newcommand{\conv}{\mathsf{conv}}
%\newcommand{\lam}[2]{{\langle}#1{\rangle}#2}
\def\NN{\mathsf{N}}
\def\UU{\mathsf{U}}
\def\JJ{\mathsf{J}}
%\def\Type{\hbox{\sf Type}}
\def\ZERO{\mathsf{0}}
\def\SUCC{\mathsf{S}}

\newcommand{\RawCtx}{{\tt Ctx}}
\newcommand{\RawSub}{{\tt Sub}}
\newcommand{\RawTy}{{\tt Ty}}
\newcommand{\RawTm}{{\tt Tm}}
\newcommand{\type}{\mathsf{type}}
\newcommand{\N}{\mathsf{N}}
\newcommand{\Set}{\mathsf{Set}}
\newcommand{\El}{\mathsf{El}}
%\newcommand{\U}{\mathsf{U}} clashes with def's in new packages
\newcommand{\T}{\mathsf{T}}
\newcommand{\Usuper}{\UU_{\mathrm{super}}}
\newcommand{\Tsuper}{\T_{\mathrm{super}}}
%\newcommand{\conv}{\mathrm{conv}}
\newcommand{\idtoeq}{\mathsf{idtoeq}}
\newcommand{\isEquiv}{\mathsf{isEquiv}}
\newcommand{\ua}{\mathsf{ua}}
\newcommand{\UA}{\mathsf{UA}}
\def\Constraint{\mathsf{Constraint}}
\def\Ordo{\mathcal{O}}
\def\Pihat{\Pi}

\def\Ctx{\mathrm{ctx}}
\def\Ty{\mathrm{ty}}
\def\Tm{\mathrm{tm}}
\def\Obj{\mathrm{obj}}
\def\Hom{\mathrm{hom}}
\def\id{\mathrm{id}}
\def\lHom{\mathrm{lhom}}
\def\lctx{\mathrm{lctx}}
\def\lty{\mathrm{level}}
\def\ltm{\mathrm{ltm}}
\def\ltmq{\mathrm{ltmq}}
\def\leq{\mathrm{leq}}
\def\lrefl{\mathrm{lrefl}}
\def\lp{\mathrm{lp}}
\def\lq{\mathrm{lq}}
\def\s{\mathrm{s}}
\def\lid{\mathrm{lid}}
\def\cctx{\mathrm{cctx}}
\def\cty{\mathrm{cty}}
\def\ctm{\mathrm{ctm}}
\def\cid{\mathrm{cid}}
\def\cp{\mathrm{cp}}
\def\cq{\mathrm{cq}}
\def\chom{\mathrm{chom}}

\newcommand{\ctx}{\mathrm{ctx}}
\newcommand{\sub}{\mathrm{sub}}
\newcommand{\ty}{\mathrm{ty}}
\newcommand{\tm}{\mathrm{tm}}
%\newcommand{\hom}{\mathrm{hom}}
\newcommand{\tuple}[1]{\langle #1 \rangle}
\def\CComega{\mathrm{CC}^\omega}
\newcommand{\cext}{.}
\def\p{\mathrm{p}}
\def\q{\mathrm{q}}
\def\app{\mathsf{app}}
\def\U{\mathsf{U}}
\def\T{\mathcal{T}}
\newcommand{\Ta}{\mathrm{T}}
\newcommand{\ta}{\mathrm{t}}

\newcommand{\natrec}{\mathsf{natrec}}
%\rightfooter{}
\newcommand{\set}[1]{\{#1\}}
\newcommand{\sct}[1]{[\![#1]\!]}
\def\R{\mathcal{R}}

\def\L{{\mathcal{L}}}
\def\F{\mathcal{C}}
\def\CwF{\mathrm{CwF}}
\def\SCwF{\mathrm{SCwF}}
\def\C{\mathcal{C}}


\begin{document}
\section{Introduction}
The aim of Voevodsky's {\em initiality conjecture project} \cite{voevodsky:initiality} is to provide a general definition of a notion of a dependent type theory and to develop generic metatheory for these dependent type theories. In particular, one would like to have a generic construction of categorical models and proofs that the corresponding type theories defined in terms of grammars and inference rules form initial models. Such proofs may seem straightforward superficially, but depend on subtle details in the formulation of grammar and inference rules. Hence, Voevodsky insists to call such theorems ``conjectures'' until proven rigourosly and preferably implemented in a proof assistant.

An example of such an initiality proof is de Boer and Brunerie's \cite{Brunerie:initiality,deBoer:lic} proof in Agda that a version of Martin-Löf type theory with an external tower of universes is an initial contextual category \cite{cartmell:phd,cartmell:apal} with appropriate extra structure. %Theversion of type theory with de Bruijn variables and implicit substitutions. 

A possible approach to Voevodsky's project is based on Cartmell's notion of a {\em generalized algebraic theory (gat)} \cite{cartmell:phd,cartmell:apal}. The most basic rules of dependent type theory can be captured by the notion of a category with families \cite{dybjer:torino,hofmann:cambridge} and these can be presented by a certain gat. Moreover, as shown in our paper \cite{bezem:hofmann} there is a generic construction of an initial model of an arbitrary gat. 
%Maybe an approach to a general notion of dependent type theory is as an initial model of an extension of the gat of cws?

Furthermore, as shown by Castellan, Clairambault, and Dybjer \cite{castellan:lambek} by considering simply typed cwfs (scwfs) and unityped cwfs (ucwfs) we can also capture various simply typed and untyped logical systems as gats and thus widening the scope of {\em uniform categorical logic} based on gats and cwfs.

In this note we shall introduce the notion of an indexed cwf, that is, a category $\C$ and a functor 
$$
T : \C^\op \to \CwF
$$
We shall show that cwfs indexed by the base category of a ucwf $\L$ of universe levels (with suitable extra structure) forms a suitable notion of model of Martin-Löf type theory with explicit universe polymorphism along the lines of Bezem et al \cite{BezemCDE22}. The benefit is twofold. First, type theory with explicit universe polymorphism is an example of a dependent type theory, and we show that it is an example of a theory which can be captured by a gat based on variations of cwfs. Second, it provides an alternative view of our syntactic type theory with explicit universe polymorphism. A proof that a (modification of) the latter forms an initial model of our gat is however beyond the scope of the paper, and we only sketch the relationship. 

Moreover, further variations of indexed cwfs extend the scope of gat and cwf-based uniform categorical. For example, untyped predicate logic can be captured by ucwf-indexed scwfs with extra structure for the logical constants. Moreover, typed predicate logic can be captured by scwf-indexed scwfs, and dependentlty typed predicate logic by cwf-indexed scwfs, both with suitable extra structure for type formers and logical constants.




 
\bibliographystyle{plain}
\bibliography{../refs}
\end{document}
