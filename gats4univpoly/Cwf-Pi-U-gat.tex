\section{The gat of cwfs with $\Pi$-types and a single universe closed under them}\label{sec:gatPiU}

\paragraph{The gat of categories.}
\begin{tiny}
Sort symbols:
\begin{eqnarray*}
&\vdash& \Obj\\
\Delta, \Gamma : \Obj &\vdash& \Hom(\Delta,\Gamma)\\
\end{eqnarray*}

Operator symbols:
\begin{eqnarray*}
\Gamma : \Obj &\vdash& \id_\Gamma : \Hom(\Gamma,\Gamma)\\
\Xi,\Delta,\Gamma : \Obj, \gamma : \Hom(\Delta,\Gamma), \delta : \Hom(\Xi,\Delta) &\vdash&
\gamma \circ \delta : \Hom(\Xi,\Gamma)
\end{eqnarray*}

Equations:
\begin{eqnarray*}
\Delta, \Gamma : \Obj, \gamma : \Hom(\Delta,\Gamma) &\vdash& \id_\Gamma \circ \gamma = \gamma : \Hom(\Delta,\Gamma)\\
\Delta, \Gamma : \Obj, \gamma : \Hom(\Delta,\Gamma) &\vdash& \gamma \circ \id_\Delta = \gamma : \Hom(\Delta,\Gamma)\\
\Theta, \Xi,\Delta,\Gamma : \Obj, \gamma : \Hom(\Delta,\Gamma), \delta : \Hom(\Xi,\Delta), \xi : \Hom(\Theta,\Xi) &\vdash&
(\gamma \circ \delta) \circ \xi = \gamma \circ (\delta \circ \xi): \Hom(\Theta,\Gamma)
\end{eqnarray*}
Note: officially $\circ$ has five arguments rather than two.
\end{tiny}

\paragraph{Adding a family valued functor}

\begin{tiny}
Sort symbols ($\ctx = \Obj$):
\begin{eqnarray*}
\Gamma : \ctx &\vdash& \ty(\Gamma)\\
\Gamma : \ctx, A:\ty(\Gamma) &\vdash& \tm(\Gamma,A)
\end{eqnarray*}

Operator symbols:
\begin{eqnarray*}
\Gamma,\Delta : \ctx, A:\ty(\Gamma), \gamma : \Hom(\Delta,\Gamma) &\vdash&
A[\gamma] : \ty(\Delta)\\
\Gamma,\Delta : \ctx, A:\ty(\Gamma), \gamma : \Hom(\Delta,\Gamma), a:\tm(\Gamma,A) &\vdash&  a[\gamma] : \tm(\Delta,A[\gamma])
\end{eqnarray*}

Equations:
\begin{eqnarray*}
\Gamma : \ctx, A:\ty(\Gamma) &\vdash& A[\id_\Gamma] = A : \ty(\Gamma)\\
\Gamma : \ctx, A:\ty(\Gamma), a:\tm(\Gamma,A) &\vdash& a[\id_\Gamma] = a : \tm(\Gamma,A)\\
\Xi,\Delta,\Gamma : \ctx, \delta : \Hom(\Xi,\Delta), \gamma : \Hom(\Delta,\Gamma),
A:\ty(\Gamma) &\vdash& A[\gamma\circ\delta] = A[\gamma][\delta]: \ty(\Xi)\\
\Xi,\Delta,\Gamma : \ctx, \delta : \Hom(\Xi,\Delta), \gamma : \Hom(\Delta,\Gamma),
A:\ty(\Gamma), a:\tm(\Gamma,A) &\vdash&
a[\gamma\circ\delta] = a[\gamma][\delta]: \tm(\Xi,A[\gamma\circ\delta])
\end{eqnarray*}
We have dropped some of the official arguments here too, and will do so in the following as well.
\end{tiny}


\paragraph{Adding a terminal object}

\begin{tiny}
%Sort symbols: none
Operator symbols:
\begin{eqnarray*}
&\vdash& 1 : \ctx\\
\Gamma : \ctx &\vdash& \tuple{}_\Gamma : \Hom(\Gamma,1)
\end{eqnarray*}

Equations:
\begin{eqnarray*}
 &\vdash& \id_1 = \tuple{}_1 : \Hom(1,1)\\
\Gamma,\Delta : \ctx, \gamma : \Hom(\Delta,\Gamma) &\vdash&
\tuple{}_\Gamma\circ\gamma = \tuple{}_\Delta : \Hom(\Delta,1)
\end{eqnarray*}
\end{tiny}


%(The latter two equations are better for term rewriting than the
%obvious single one expressing the uniqueness of $\tuple{}_\Gamma$.)

\paragraph{Adding context comprehension}

%No new sorts are added.
\begin{tiny}
Operator symbols:
\begin{eqnarray*}
\Gamma : \ctx, A:\ty(\Gamma) &\vdash& \Gamma\cext A : \ctx\\
\Gamma,\Delta : \ctx, A:\ty(\Gamma), \gamma : \Hom(\Delta,\Gamma), a:\tm(\Delta,A[\gamma]) &\vdash& \tuple{\gamma,a} : \Hom(\Delta,\Gamma\cext A)\\
\Gamma : \ctx, A:\ty(\Gamma) &\vdash& \p: \Hom(\Gamma\cext A,\Gamma)\\
\Gamma : \ctx, A:\ty(\Gamma) &\vdash& \q: \tm(\Gamma\cext A,A[\p])
\end{eqnarray*}

Equations:
\begin{eqnarray*}
\Gamma,\Delta : \ctx, A:\ty(\Gamma), \gamma : \Hom(\Delta,\Gamma), a:\tm(\Delta,A[\gamma]) &\vdash& \p\circ\tuple{\gamma,a} = \gamma : \Hom(\Delta,\Gamma)\\
\Gamma,\Delta : \ctx, A:\ty(\Gamma), \gamma : \Hom(\Delta,\Gamma), a:\tm(\Delta,A[\gamma]) &\vdash& \q[\tuple{\gamma,a}] = a : \tm(\Delta,A[\gamma]) \\
\Gamma,\Delta,\Xi : \ctx, A:\ty(\Gamma), \gamma : \Hom(\Delta,\Gamma), a:\tm(\Delta,A[\gamma]), \delta : \Hom(\Xi,\Delta) &\vdash&
\tuple{\gamma,a} \circ \delta = \tuple{\gamma\circ\delta,a[\delta]} :
\Hom(\Xi,\Gamma\cext A) \\
\Gamma : \ctx, A:\ty(\Gamma) &\vdash&
\id_{\Gamma\cext A} = \tuple{\p,\q} : \Hom(\Gamma\cext A,\Gamma\cext A)
\end{eqnarray*}
\end{tiny}


\paragraph{Adding $\Pi$-types}
%We add three operator symbols in addition to the operator symbols for cwfs in Section 5.2 and 5.3:
\begin{tiny}
Operator symbols:
\begin{eqnarray*}
\Gamma : \ctx, A : \ty(\Gamma), B : \ty(\Gamma.A)&\vdash& \Pi(A,B) : \ty(\Gamma)\\
\Gamma : \ctx, A : \ty(\Gamma), B : \ty(\Gamma.A), b : \tm(\Gamma.A, B) &\vdash& \lambda(b) : \tm(\Gamma,\Pi(A,B))\\
\Gamma : \ctx, A : \ty(\Gamma), B : \ty(\Gamma.A), c :  \tm(\Gamma,\Pi(A,B)), a : \tm(\Gamma, A) &\vdash& \app(c,a) : \tm(\Gamma, B[\tuple{\id,a}])
\end{eqnarray*}
Equations (omitting the context and type of the equalities):
 \begin{eqnarray*}
 \app(\lambda(b),a) &=& b[\tuple{\id,a}]\\
 \lambda(\app(c[\p],\q)) &=& c
 \end{eqnarray*}
 Equations for commutativity of operator symbols wrt substitution:
 \begin{eqnarray*}
\Pi(A,B)[ \gamma ] &=& \Pi(A [ \gamma ], B[ \gamma^\dagger ])\\
\lambda(b) [ \gamma ] &=& \lambda(b[\gamma^\dagger ])\\
\app(c,a) [ \gamma ] &=& \app(c[ \gamma ], a[ \gamma ] )
\end{eqnarray*}
where $\gamma^\dagger = \tuple{\gamma \circ \p, \q}$.
\end{tiny}


\paragraph{Adding a universe closed under $\Pi$}

\footnote{PD: for comparison only. Should we remove it?}
\begin{tiny}
Operator symbols:
\begin{eqnarray*}
\Gamma : \ctx &\vdash& \U_\Gamma : \ty(\Gamma)\\
\Gamma : \ctx, a : \tm(\Gamma,\U_\Gamma) &\vdash& {\Ta}(a) : \ty(\Gamma)\\
%\Gamma : \ctx &\vdash& (\N^0)_\Gamma : \tm(\Gamma,\U_\Gamma) \\
\Gamma : \ctx,
a : \tm(\Gamma,\U_\Gamma),
b :  \tm(\Gamma \cdot \Ta(a), \U_\Gamma))
&\vdash&
 \Pi^0(a,b) : \tm(\Gamma,\U_\Gamma)
\end{eqnarray*}
%$\U_\Gamma$ is the universe (a type) relative to the context $\Gamma$; $\Ta$ is the decoding operation mapping a term in the universe to the corresponding type; $\N^0$ is the code for $\N$ in the universe, and $\Pi^0$ forms codes for $\Pi$-types in the universe. (Note that we have dropped the context argument of $\Ta$ and $\Pi^0$.)

Equation:
\begin{eqnarray*}
%\Ta(\N^0_\Gamma) &=& \N_\Gamma\\
\Ta(\Pi^0(a,b)) &=& \Pi(\Ta(a),\Ta(b))
\end{eqnarray*}
 Equations for commutativity of operator symbols wrt substitution:
 \begin{eqnarray*}
{\U}_\Gamma [ \gamma ] &=& {\U}_\Delta\\
\Ta(a) [ \gamma ] &=& \Ta(a[ \gamma ] )\\
%\N^0_\Gamma [ \gamma ] &=&\N^0_\Delta\\
\Pi^0(a,b)[ \gamma ] &=& \Pi^0(a [ \gamma ], b[ \gamma^\dagger ])
\end{eqnarray*}
\end{tiny}
%where $\gamma^\dagger = \tuple{\gamma \circ \p, \q}$.