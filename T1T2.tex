%\documentclass[12pt,a4paper]{amsart}
\documentclass[11pt,a4paper]{article}
%\ifx\pdfpageheight\undefined\PassOptionsToPackage{dvips}{graphicx}\else%
%\PassOptionsToPackage{pdftex}{graphicx}
\PassOptionsToPackage{pdftex}{color}
%\fi

%\usepackage{diagrams}

%\usepackage[all]{xy}
\usepackage{url}
\usepackage[utf8]{inputenc}
\usepackage{verbatim}
\usepackage{latexsym}
\usepackage{amssymb,amstext,amsmath,amsthm}
\usepackage{epsf}
\usepackage{epsfig}
% \usepackage{isolatin1}
\usepackage{a4wide}
\usepackage{verbatim}
\usepackage{proof}
\usepackage{latexsym}
%\usepackage{mytheorems}
\newtheorem{theorem}{Theorem}[section]
\newtheorem{corollary}{Corollary}[theorem]
\newtheorem{lemma}{Lemma}[theorem]
\newtheorem{proposition}{Proposition}[theorem]
\theoremstyle{definition}
\newtheorem{definition}[theorem]{Definition}
\newtheorem{remark}{Remark}[theorem]
\newtheorem{TODO}{TODO}[theorem]

\usepackage{float}
\floatstyle{boxed}
\restylefloat{figure}


%%%%%%%%%copied from SymmetryBook by Marc

% hyperref should be the package loaded last
%% \usepackage[backref=page,
%%             colorlinks,
%%             citecolor=linkcolor,
%%             linkcolor=linkcolor,
%%             urlcolor=linkcolor,
%%             unicode,
%%             pdfauthor={CAS},
%%             pdftitle={Symmetry},
%%             pdfsubject={Mathematics},
%%             pdfkeywords={type theory, group theory, univalence axiom}]{hyperref}
% - except for cleveref!
\usepackage[capitalize]{cleveref}
%\usepackage{xifthen}
\usepackage{xcolor}
\definecolor{linkcolor}{rgb}{0,0,0.5}

%%%%%%%%%
\def\oge{\leavevmode\raise
.3ex\hbox{$\scriptscriptstyle\langle\!\langle\,$}}
\def\feg{\leavevmode\raise
.3ex\hbox{$\scriptscriptstyle\,\rangle\!\rangle$}}

%%%%%%%%%
\newcommand\myfrac[2]{
 \begin{array}{c}
 #1 \\
 \hline \hline
 #2
\end{array}}


\newcommand*{\Scale}[2][4]{\scalebox{#1}{$#2$}}%
\newcommand*{\Resize}[2]{\resizebox{#1}{!}{$#2$}}

\newcommand{\II}{\mathbb{I}}
\newcommand{\refl}{\mathsf{refl}}
\newcommand{\mkbox}[1]{\ensuremath{#1}}


\newcommand{\Id}{\mathsf{Id}}
\newcommand{\conv}{=}
%\newcommand{\conv}{\mathsf{conv}}
\newcommand{\lam}[2]{{\langle}#1{\rangle}#2}
\def\NN{\mathsf{N}}
\def\UU{\mathsf{U}}
\def\JJ{\mathsf{J}}
\def\Level{\mathsf{Level}}
\def\List{\mathsf{List}}
\def\Cons{\mathsf{Cons}}
\def\Nil{\mathsf{Nil}}
%\def\Type{\hbox{\sf Type}}
\def\ZERO{\mathsf{0}}
\def\SUCC{\mathsf{S}}

\newcommand{\type}{\mathsf{type}}
\newcommand{\LAM}{\lambda}
\newcommand{\APP}{\mathsf{app}}
\newcommand{\mypi}[3]{\Pi_{#1:#2}#3}
\newcommand{\mylam}[3]{\lambda_{#1:#2}#3}
\newcommand{\mysig}[3]{\Sigma_{#1:#2}#3}
\newcommand{\N}{\mathsf{N}}
\newcommand{\Set}{\mathsf{Set}}
\newcommand{\El}{\mathsf{El}}
%\newcommand{\U}{\mathsf{U}} clashes with def's in new packages
\newcommand{\T}{\mathsf{T}}
\newcommand{\sT}{\mathsf{t}}
\newcommand{\Usuper}{\UU_{\mathrm{super}}}
\newcommand{\Tsuper}{\T_{\mathrm{super}}}
%\newcommand{\conv}{\mathrm{conv}}
\newcommand{\idtoeq}{\mathsf{idtoeq}}
\newcommand{\isEquiv}{\mathsf{isEquiv}}
\newcommand{\ua}{\mathsf{ua}}
\newcommand{\UA}{\mathsf{UA}}
%\newcommand{\Level}{\mathrm{Level}}
\def\Constraint{\mathsf{Constraint}}
\def\Ordo{\mathcal{O}}

\newcommand{\Con}{\mathsf{ Con}}
\newcommand{\Elem}{\mathsf{Elem}}
\newcommand{\Type}{\mathsf{Type}}
\newcommand{\id}{\mathsf{id}}
\newcommand{\pp}{\mathsf{p}}
\newcommand{\qq}{\mathsf{q}}

\def\Ctx{\mathrm{Ctx}}
\def\Ty{\mathrm{Ty}}
\def\Tm{\mathrm{Tm}}

\def\CComega{\mathrm{CC}^\omega}
\setlength{\oddsidemargin}{0in} % so, left margin is 1in
\setlength{\textwidth}{6.27in} % so, right margin is 1in
\setlength{\topmargin}{0in} % so, top margin is 1in
\setlength{\headheight}{0in}
\setlength{\headsep}{0in}
\setlength{\textheight}{9.19in} % so, foot margin is 1.5in
\setlength{\footskip}{.8in}

% Definition of \placetitle
% Want to do an alternative which takes arguments
% for the names, authors etc.

\newcommand{\natrec}{\mathsf{natrec}}
%\rightfooter{}
\newcommand{\set}[1]{\{#1\}}
\newcommand{\sct}[1]{[\![#1]\!]}
%\usepackage{diagrams}
\usepackage{color}
\newcommand\coloremph[2][red]{\textcolor{#1}{\emph{#2}}}
\newcommand\norm[1]{\left\lVert #1 \right\rVert}
\newcommand\greenemph[2][green]{\textcolor{#1}{\emph{#2}}}
\newcommand{\EMP}[1]{\emph{\textcolor{red}{#1}}}




\begin{document}

\title{Type Theory with a Cumulative Hierarchy of Universes}

\author{}
\date{}
\maketitle

%\begin{abstract}
%\end{abstract}

\section{Introduction}\label{sec:intros}

 We prove the equivalence between $T_1$ and $T_2$.

\section{Main Lemma}

We assume some meta properties of $T_1$. We have that $\Pi$ is one-to-one for conversion, i.e.
$\Gamma\vdash \Pi(A,B) = \Pi(C,D)$ implies $\Gamma\vdash A = C$ and
$\Gamma.A\vdash B = D$. We also assume that $\sT^k_n$ is one-to-one for conversion, and
$\UU_n = \UU_m$ implies $n = m$ and $\Pi(A,B)$ cannot be convertible to some $\UU_n$ or
to some type of the form $\T_n(X)$.

We introduce the following notations. We write $\Pi(\Delta,A)$ and $\lambda(\Delta,t)$
for $\Gamma.\Delta\vdash A$ and $\Gamma.\Delta\vdash t:A$
as iterated product and iterated abstraction respectively. If $\Gamma\vdash t : \Pi(\Delta,\UU_n)$, we can write
$t = \lambda(\Delta,u)$ for some $u$, unique up to conversion, such that
$\Gamma.\Delta\vdash u : \UU_n$. We define then
$\sT^k_{n,\Delta}(t) = \lambda(\Delta,\sT^k_n(u))$.

\begin{lemma}
  If $\Gamma\vdash A$ and $\Gamma\vdash B$ and $|A| = |B|$ then $\Gamma\vdash A = B$.

  If $\Gamma\vdash t:A$ and $\Gamma\vdash u:B$ and $|t| = |u|$ then
  \begin{enumerate}
  \item either $\Gamma\vdash A = B$ and $\Gamma\vdash t = u:A$
  \item or there exists $\Delta$ and $n,m$ such that $A = \Pi(\Delta,\UU_n)$ and
    $B = \Pi(\Delta,\UU_m)$ and
    $\Gamma\vdash \sT^k_{n,\Delta}(t) = \sT^k_{m,\Delta}(u):\Pi(\Delta,\UU_k)$ for
    $n\leqslant k,~m\leqslant k$.
\end{enumerate}
\end{lemma}

Note that this implies that if
$\Gamma\vdash t:A$ and $\Gamma\vdash u:B$ and $|t| = |u|$ {\em and} $\Gamma\vdash A = B$ then
we have $\Gamma\vdash t = u : A$. This follows from the fact that $\sT^k_n$ is one-to-one for conversion.

\begin{proof}
  We prove the two statements by mutual induction.

  Let us cover some cases.

  If $|A| = |B| = \UU_n$ then we have $A$ has to be convertible to $\UU_n$ and the same holds for $B$.
  Hence, we have $\Gamma\vdash A = B$.

  If $|A| = |B|$ is a dependent product then $A$ is 
  convertible to some $\Pi(E,F)$ where $E$ and $F$ are uniquely determined. Similarly, 
  $B$ will be convertible to some $\Pi(E_1,F_1)$. We have $|E| = |E_1|$ and $|F| = |F_1|$. By induction, we have
  $\Gamma\vdash E = E_1$ and $\Gamma.E \vdash F = F_1$.

  If $t$ is of the form $\sT^k_n(v)$ then the result is direct by induction: we should have $|v| = |u|$
  and then $A = \UU_k$ and $v:\UU_n$.

  We can thus assume that neither $t$ nor $u$ are of the form $\sT^k_n(v)$.

  If $|t|=|u|$ is an application, we can write $t = \APP(E,F,c,a)$ and $u = \APP(E_1,F_1,c_1,a_1)$.
  We have $|c| = |c_1|$. By induction we have $\Pi(E,F) = \Pi(E_1,F_1)$ or
  $F = \Pi(\Delta,\UU_n)$ and $F_1 = \Pi(\Delta,\UU_m)$ and $E = E_1$ and
  $\sT^k_{n,E.\Delta}(c) = \sT^k_{m,E.\Delta}(c_1)$. We also have then by induction $a = a_1:E$.
  We deduce $\sT^k_{n,\Delta[a]}(t) = \sT^k_{m,\Delta[a]}(u)$.

  If $|t| = |u|$ is an abstraction, the result follows from the fact that $T_2$ has typed abstractions.
  Indeed, we can write $t = \lambda(E,F,t')$ and $u = \lambda(E_1,F_1,u')$ and we have
  $|E| = |E_1|$. By induction $\Gamma\vdash E = E_1$. We also have $|u| = |u'|$. Hence by induction
  we have $F = F_1$ and $\Gamma.E\vdash u = u':F$ or $F = \Pi(\Delta,\UU_n)$ and $F' = \Pi(\Delta,\UU_m)$
  and $\sT^k_{n,\Delta}(u) = \sT^k_{m,\Delta}(u'):\Pi(\Delta,\UU_k)$. We then have
  $\sT^k_{n,E.\Delta}(t) = \sT^k_{m,E.\Delta}(u):\Pi(E.\Delta,\UU_k)$.
\end{proof}



\bibliographystyle{plain}
\bibliography{refs}

\end{document}
