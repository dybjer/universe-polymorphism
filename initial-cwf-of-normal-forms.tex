\documentclass{lmcs}
%\usepackage{etex}
\usepackage[utf8]{inputenc}

\usepackage{color}
\usepackage{hyperref}
\usepackage{float}
\usepackage{amsmath}
\usepackage{amsfonts}
\usepackage{amsthm}
\usepackage{amssymb}
\usepackage{proof}
\usepackage{mathpartir}
\usepackage{mathrsfs}
\usepackage{stmaryrd}
\usepackage{cmll}
\usepackage{enumerate}
\usepackage{graphicx}
\usepackage[all]{xy}
\usepackage{listings}
\usepackage{todonotes}
\usepackage{Guyboxes}
%\DeclareMathOperator{\Ker}{Ker}
%\DeclareMathOperator{\nf}{nf}
%\DeclareMathOperator{\domain}{dom}
%\DeclareMathOperator{\codomain}{cod}
%\DeclareMathOperator{\cod}{cod}
%\DeclareMathOperator{\dom}{dom}
%\DeclareMathOperator{\ctxof}{ctx-of}
%\DeclareMathOperator{\typeof}{type-of}
%\DeclareMathOperator{\fix}{fix}

%\newcommand{\vdashS}{\ \vdash\ }
%\newcommand{\vdashS}{\vdash}
\newcommand {\emptyContext}{1}
\newcommand {\emptyContextI}{\diamond}
\newcommand {\emptyContextS}{\textbf 1}
\newcommand {\contextExtension}[2]{#1 \cdot #2}
\newcommand {\contextExtensionI}[2]{#1 \cdot #2}
\newcommand {\contextExtensionS}[2]{#1 \cdot #2}
\newcommand {\contextExtensionC}[2]{#1 \cdot_\C #2}

\newcommand {\GammaA}{\contextExtension \Gamma A}
\newcommand {\DeltaA}{\contextExtension \Delta A}
\newcommand {\setI}{\text{set}}
\newcommand {\setS}{\textbf{set}}
\newcommand {\depProd}[3]{\Pi(#1, #2, #3)}
\newcommand {\depProdI}[2]{\Pi(#1, #2)}
\newcommand {\depProdS}{\textbf{$\Pi$}}
\newcommand {\el}[2]{{\tt el}(#1, #2)}
\newcommand {\elI}[1]{{\tt el}(#1)}
\newcommand {\elS}{\textbf{el}}
\newcommand {\subType}[4]{{\tt subType}(#3, #4, #1, #2)}
\newcommand {\subTypeI}[2]{\text{subType}(#1, #2)}
\newcommand {\subTypeS}[2]{#1\{#2\}}
\newcommand{\subTypeC}[4]{\mathrm{subType}_\C(#3, #4, #1, #2)}
\newcommand {\q}[2]{{\tt q}_{#1, #2}}
\newcommand {\qI}{{\tt q}}
\newcommand {\qS}{\textbf{q}}
\newcommand{\lambdaAbs}[4]{\lambda(#1, #2, #3, #4)}
\newcommand{\lambdaAbsI}[1]{\lambda(#1)}
\newcommand{\lambdaAbsS}{\textbf{$\lambda$}}
\newcommand{\application}[5]{{\tt app}(#1, #2, #3, #4, #5)}
\newcommand{\applicationI}[2]{\text{app}(#1, #2)}
\newcommand{\applicationS}{\textbf{application}}
\newcommand{\subTerm}[5]{{\tt subTerm}(#4, #5, #1,#2,#3)}
\newcommand{\subTermI}[2]{\text{subTerm}(#1,#2)}
\newcommand{\subTermS}[2]{#1\{#2\}}
\newcommand{\idSub}[1]{{\tt id}(#1)}
\newcommand{\idSubI}{{\tt id}}
\newcommand{\idSubS}{\text{id}}
\newcommand{\proj}[2]{{\tt p}(#1, #2)}
\newcommand{\projI}{{\tt p}}
\newcommand{\projS}{\textbf{p}}
\newcommand{\comp}[5]{{\tt comp}(#1, #2, #3, #4, #5)}
\newcommand{\compI}[2]{{\tt comp}(#1, #2)}
\newcommand{\compS}[2]{#2 \circ #1}
\newcommand{\emptySub}[1]{\emptySubI_{#1}}
\newcommand{\emptySubI}{\langle\rangle}
\newcommand{\emptySubS}{\textbf !}
\newcommand{\extSub}[5]{\text{extension}(#1, #2, #3, #4, #5)}
\newcommand{\extSubI}[2]{\text{extension}(#1, #2)}
\newcommand{\extSubS}[2]{\langle #1, #2\rangle}
\newcommand{\Ctx}{\mathrm{Ctx}}
\newcommand{\Sub}{\mathrm{Sub}}
\newcommand{\Ty}{\mathrm{Ty}}
\newcommand{\Tm}{\mathrm{Tm}}
\newcommand{\C}{{\mathcal C}}
\newcommand{\I}{{\mathcal I}}
\newcommand{\T}{{\mathcal T}}

\newcommand{\Timp}{\T_{\text{imp}}}
\newcommand{\arrow}{{\rightarrow}}
\newcommand{\RawCtx}{{\tt Ctx}}
\newcommand{\RawSub}{{\tt Sub}}
\newcommand{\RawTy}{{\tt Ty}}
\newcommand{\RawTm}{{\tt Tm}}

\newcommand{\scomp}[6]{\mathrm{comp}_#1(#2, #3, #4, #5,#6)}

\newcommand{\inte}[1]{\llbracket #1 \rrbracket}
\newcommand{\intCtx}[1]{\llbracket #1 \rrbracket}
\newcommand{\intSub}[3]{\llbracket #3 \rrbracket_{#1,#2}}
\newcommand{\intTy}[2]{\llbracket #2 \rrbracket_#1}
\newcommand{\intTm}[3]{\llbracket #3 \rrbracket_{#1,#2}}
\newcommand{\ICtx}{{\I_0}}
\newcommand{\ISub}{{\I_1}}
\newcommand{\ITy}{{\I_2}}
\newcommand{\ITm}{{\I_3}}
\newcommand{\iniCtx}[1]{\overline{\llbracket #1 \rrbracket}}
\newcommand{\iniSub}[3]{\overline{\llbracket #3 \rrbracket}_{#1,#2}}
\newcommand{\iniTy}[2]{\overline{\llbracket #2 \rrbracket}_{#1}}
\newcommand{\iniTm}[3]{\overline{\llbracket #3 \rrbracket}_{#1,#2}}

\newcommand{\mejl}[3]{#1$\bigcirc\!\!\!\!\!\alpha\,$#2${}_{\cdot}$#3}

\newcommand{\bbN}[0]{{\mathbb N}}
\newcommand{\bbZ}[0]{{\mathbb Z}}
\newcommand{\bbQ}[0]{{\mathbb Q}}
\newcommand{\bbR}[0]{{\mathbb R}}
\newcommand{\bbB}[0]{{\mathbb B}}
\newcommand{\mU}[0]{{\mathcal U}}
\newcommand{\mT}[0]{{\mathcal T}}
\newcommand{\ve}[0]{{\varepsilon}}
\newcommand{\vf}[0]{{\varphi}}

\newcommand{\wellincluded}[0]{\, \Subset \,}

\newcommand{\memof}[0]{\, \epsilon \,}
\newcommand{\subseteqof}[0]{\, \dot{\subseteq} \,}

\newcommand{\mono}[0]{\to/ >->/}
\newcommand{\pto}[0]{\rightharpoondown}
\newcommand{\wellcov}[0]{{\lll}}
\newcommand{\waybelow}[0]{\ll}
\newcommand{\formint}[0]{\land}
\newcommand{\cov}[0]{{\, \lhd \,}}
\newcommand{\kov}[0]{{\, \lessdot \,}}
\newcommand{\kkov}[0]{{\, <: \,}}
\newcommand{\mutcov}[0]{\sim}
\newcommand{\balcov}[0]{\sqsubseteq}
\newcommand{\bal}[0]{{\sf b}}
\newcommand{\sat}[1]{{\rm Sat}(#1)}
\newcommand{\set}[0]{{\rm Set}}
\newcommand{\Set}[0]{{\bf Set}}
\newcommand{\true}[0]{{\sf T}}
\newcommand{\monus}{\stackrel{{}^{\scriptstyle .}}{\smash{-}}}

\newcommand{\refl}[0]{{\rm ref}}

\newcommand{\inl}[1]{{\sf inl}(#1)}
\newcommand{\inr}[1]{{\sf inr}(#1)}
\newcommand{\nat}[0]{{\mathbb N}}

\newcommand{\nattype}[0]{{\rm N}}
\newcommand{\bool}[0]{{\rm Bool}}
\newcommand{\ext}[1]{\langle #1 \rangle}


\newcommand{\bintree}[0]{{\rm T}_2}

%\newcommand{\sequent}[0]{\vdash}


\renewcommand{\conv}[0]{\approx}
\newcommand{\intimpl}[0]{\supset}

\newcommand{\omitthis}[1]{}

\newcommand{\changenote}[1]{}


 \newcommand{\Id}[0]{{\rm I}}
 

\newcommand{\longtext}[1]{}
\newcommand{\shorttext}[1]{}
\newcommand{\commentaway}[1]{}

\newcommand{\Setoid}[0]{{\bf Setoid}}

\definecolor{Red}{rgb}{1,0,0}
\newcommand{\red}[1]{{\color{Red}#1}}
%\newcommand{\red}[1]{}
\renewcommand{\bar}[1]{\overline{#1}}

%\newdir{pb}{:(1,-1)@^{|-}}
%\def\pb#1{\save[]+<16 pt,0 pt>:a(#1)\ar@{pb{}}[]\restore}

\newcommand{\Fam}{\textbf{Fam}}
\newcommand{\nilc}{1}
\newcommand{\cext}{.}
\newcommand{\indexed}[1]{\boldsymbol{#1}}
\newcommand{\Cat}{\mathrm{Cat}}
\newcommand{\op}{\text{op}}
\newcommand{\iso}{\cong}
\newcommand{\subst}[1]{\langle #1 \rangle}
\newcommand{\applyopen}[2]{\{ #1 \}  #2 }

% added by Marc to get things going. IMPROVE!

\def\N{\mathsf{N}}
\def\U{\mathsf{U}}
\def\F{\mathsf{F}}
\def\app{\mathsf{app}}
\def\Cop{\C^\op}
\def\Cobj{{\mathcal{C}_0}}
\def\p{\mathrm{p}}
\def\q{\mathrm{q}}
\newcommand{\tuple}[1]{\langle #1 \rangle}

\newtheorem{remark}{Remark}
\newtheorem{definition}{Definition}

\def\N{\mathrm{N}}
\def\U{\mathrm{U}}
\def\p{{\tt p}}
\def\ev{{\tt ev}}
\def\q0{{\tt q}}
\def\r{{\tt r}}
\def\arrow{\rightarrow}
\def\Hom{\mathrm{Hom}}
\def\GammaA{\Gamma_{+,\times}}
\def\GammaCL{\Gamma_{\mathrm{CL}}}

\def\Dp{\mathrm{D}_p}
\def\notnotDp{\neg\neg\Dp}
\def\F{\mathcal{F}}
\def\HA{\mathbf{HA}}
\def\PA{\mathbf{PA}}
\def\I{\mathrm{I}}
\def\refl{\mathrm{r}}
\def\id{{\tt id}}
\def\idT{\mathrm{id}_\T}
\def\idC{\mathrm{id}_\C}
\newcommand{\pair}{\mathrm{pair}}
\newcommand{\fst}{\mathrm{fst}}
\newcommand{\interp}[1]{ \overline{\llbracket #1 \rrbracket}}
\newcommand{\Cwf}{\textbf{CwF}}
\newcommand{\Cwfs}{\Cwf_s}
\newcommand{\D}{\mathcal{D}}
\newcommand{\snd}{\mathrm{snd}}
\newcommand{\ap}{\mathrm{app}}
%\newcommand{\app}{\mathrm{app}}
\newcommand{\ini}[1]{\iniCtx{[#1]}}
\DeclareMathOperator{\cod}{cod}
\DeclareMathOperator{\dom}{dom}
\DeclareMathOperator{\ctxof}{ctx-of}
\DeclareMathOperator{\typeof}{type-of}
\newcommand{\vdashS}{\ \vdash\ }
\DeclareMathOperator{\domain}{dom}
\DeclareMathOperator{\codomain}{cod}


\newcommand{\isoCtx}[1]{\stackrel{#1}{\cong}}
\newcommand{\isoTy}[2]{\stackrel{#1}{\cong}_{#2}}
\newcommand{\equSub}[1]{=_{#1}}
\newcommand{\equTm}[2]{=_{#1,#2}}
\newcommand{\TT}{\mathbf{T}}

\def\UU{\mathrm{U}}
\def\Ta{\mathrm{T}}
\def\Pihat{\hat{\Pi}}
\def\UUhat{\hat{\UU}}
\def\Level{\mathrm{Level}}
\def\t{\mathrm{t}}
\def\RawNe{\texttt{Ne}}
\def\var{\mathrm{var}}
\def\Ctx{\mathrm{Ctx}}
\def\Ty{\mathrm{Ty}}
\def\Tm{\mathrm{Tm}}
\def\Ne{\mathrm{Ne}}
\def\Obj{\mathrm{obj}}
\def\Hom{\mathrm{hom}}
\def\id{\mathrm{id}}
\def\lHom{\mathrm{lhom}}
\def\lctx{\mathrm{lctx}}
\def\lty{\mathrm{level}}
\def\ltm{\mathrm{ltm}}
\def\lid{\mathrm{lid}}
\def\lp{\mathrm{lp}}
\def\lq{\mathrm{lq}}
\def\s{\mathrm{s}}

\newcommand{\ctx}{\mathrm{ctx}}
\newcommand{\sub}{\mathrm{sub}}
\newcommand{\ty}{\mathrm{ty}}
\newcommand{\tm}{\mathrm{tm}}
%\newcommand{\hom}{\mathrm{hom}}
%\newcommand{\tuple}[1]{\langle #1 \rangle}
\def\CComega{\mathrm{CC}^\omega}
\def\p{\mathrm{p}}
\def\q{\mathrm{q}}
\def\app{\mathsf{app}}
\def\U{\mathsf{U}}
\def\T{\mathcal{T}}
%\newcommand{\Ta}{\mathrm{T}}
\newcommand{\ta}{\mathrm{t}}

\newcommand{\natrec}{\mathsf{natrec}}
%\rightfooter{}
%\newcommand{\set}[1]{\{#1\}}
\newcommand{\sct}[1]{[\![#1]\!]}
\def\R{\mathcal{R}}

\def\L{{\mathcal{L}}}
\def\F{\mathcal{C}}
\def\CwF{\mathrm{CwF}}





\newcommand{\intCtxU}[1]{\llbracket #1 \rrbracket^\RawCtx}
\newcommand{\intSubU}[1]{\llbracket #1 \rrbracket^\RawSub}
\newcommand{\intTyU}[1]{\llbracket #1 \rrbracket^\RawTy}
\newcommand{\intTmU}[1]{\llbracket #1 \rrbracket^\RawTm}
\newcommand{\intNeU}[1]{\llbracket #1 \rrbracket^\RawNe}

\title{Initial Categories with Families\\with Universe Tower Structures}

\begin{document}

\maketitle

\begin{abstract}
We define the notion of a cumulative $\Pi\UU$-tower-cwf as a category with families with a $\Pi$-structure and a structure for a tower of universes $\UU$ closed under $\Pi$. Then we construct two initial cumulative $\Pi\UU$-tower-cwfs: one formulated a la Tarski and the other formulated a la Russell. It follows that the system a la Russell is isomorphic to the system a la Tarski in the category of cumulative $\Pi\UU$-tower-cwf. (** We should also consider the non-cumulative case **)
 \end{abstract}

%\tableofcontents
\allowdisplaybreaks
\section{Introduction}

We consider type theory with $\Pi$ types and an external countable cumulative tower of universes $\UU_n$. We formulate a generalized algebraic theory and initial models with normal forms a la Tarski and normal forms a la Russell.

\section{The generalized algebraic theory of $\Pi\UU$-tower-cwfs}

\subsection{The generalized algebraic theory of categories}
\hspace{1em}

Sort symbols:
\begin{eqnarray*}
&\vdash& \Obj\\
\Delta, \Gamma : \Obj &\vdash& \Hom(\Delta,\Gamma)\\
\end{eqnarray*}

Operator symbols:
\begin{eqnarray*}
\Gamma : \Obj &\vdash& \id_\Gamma : \Hom(\Gamma,\Gamma)\\
\Xi,\Delta,\Gamma : \Obj, \gamma : \Hom(\Delta,\Gamma), \delta : \Hom(\Xi,\Delta) &\vdash&
\gamma \circ \delta : \Hom(\Xi,\Gamma)
\end{eqnarray*}

Equations:
\begin{eqnarray*}
\Delta, \Gamma : \Obj, \gamma : \Hom(\Delta,\Gamma) &\vdash& \id_\Gamma \circ \gamma = \gamma : \Hom(\Delta,\Gamma)\\
\Delta, \Gamma : \Obj, \gamma : \Hom(\Delta,\Gamma) &\vdash& \gamma \circ \id_\Delta = \gamma : \Hom(\Delta,\Gamma)\\
\Theta, \Xi,\Delta,\Gamma : \Obj, \gamma : \Hom(\Delta,\Gamma), \delta : \Hom(\Xi,\Delta), \xi : \Hom(\Theta,\Xi) &\vdash&
(\gamma \circ \delta) \circ \xi = \gamma \circ (\delta \circ \xi): \Hom(\Theta,\Gamma)
\end{eqnarray*}
Note: officially $\circ$ has five arguments rather than two.

\subsection{Adding a family valued functor}
\hspace{1em}

Sort symbols ($\ctx = \Obj$):
\begin{eqnarray*}
\Gamma : \ctx &\vdash& \ty(\Gamma)\\
\Gamma : \ctx, A:\ty(\Gamma) &\vdash& \tm(\Gamma,A)
\end{eqnarray*}

Operator symbols:
\begin{eqnarray*}
\Gamma,\Delta : \ctx, A:\ty(\Gamma), \gamma : \Hom(\Delta,\Gamma) &\vdash&
A[\gamma] : \ty(\Delta)\\
\Gamma,\Delta : \ctx, A:\ty(\Gamma), \gamma : \Hom(\Delta,\Gamma), a:\tm(\Gamma,A) &\vdash&  a[\gamma] : \tm(\Delta,A[\gamma])
\end{eqnarray*}

Equations:
\begin{eqnarray*}
\Gamma : \ctx, A:\ty(\Gamma) &\vdash& A[\id_\Gamma] = A : \ty(\Gamma)\\
\Gamma : \ctx, A:\ty(\Gamma), a:\tm(\Gamma,A) &\vdash& a[\id_\Gamma] = a : \tm(\Gamma,A)\\
\Xi,\Delta,\Gamma : \ctx, \delta : \Hom(\Xi,\Delta), \gamma : \Hom(\Delta,\Gamma),
A:\ty(\Gamma) &\vdash& A[\gamma\circ\delta] = A[\gamma][\delta]: \ty(\Xi)\\
\Xi,\Delta,\Gamma : \ctx, \delta : \Hom(\Xi,\Delta), \gamma : \Hom(\Delta,\Gamma),
A:\ty(\Gamma), a:\tm(\Gamma,A) &\vdash&
a[\gamma\circ\delta] = a[\gamma][\delta]: \tm(\Xi,A[\gamma\circ\delta])
\end{eqnarray*}
We have dropped some of the official arguments here too, and will do so in the following as well.

\subsection{Adding a terminal object}
\hspace{1em}

%Sort symbols: none
Operator symbols:
\begin{eqnarray*}
&\vdash& 1 : \ctx\\
\Gamma : \ctx &\vdash& \tuple{}_\Gamma : \Hom(\Gamma,1)
\end{eqnarray*}

Equations:
\begin{eqnarray*}
 &\vdash& \id_1 = \tuple{}_1 : \Hom(1,1)\\
\Gamma,\Delta : \ctx, \gamma : \Hom(\Delta,\Gamma) &\vdash&
\tuple{}_\Gamma\circ\gamma = \tuple{}_\Delta : \Hom(\Delta,1)
\end{eqnarray*}

%(The latter two equations are better for term rewriting than the
%obvious single one expressing the uniqueness of $\tuple{}_\Gamma$.)

\subsection{Adding context comprehension}
\hspace{1em}
%No new sorts are added.

Operator symbols:
\begin{eqnarray*}
\Gamma : \ctx, A:\ty(\Gamma) &\vdash& \Gamma\cext A : \ctx\\
\Gamma,\Delta : \ctx, A:\ty(\Gamma), \gamma : \Hom(\Delta,\Gamma), a:\tm(\Delta,A[\gamma]) &\vdash& \tuple{\gamma,a} : \Hom(\Delta,\Gamma\cext A)\\
\Gamma : \ctx, A:\ty(\Gamma) &\vdash& \p: \Hom(\Gamma\cext A,\Gamma)\\
\Gamma : \ctx, A:\ty(\Gamma) &\vdash& \q: \tm(\Gamma\cext A,A[\p])
\end{eqnarray*}

Equations:
\begin{eqnarray*}
\Gamma,\Delta : \ctx, A:\ty(\Gamma), \gamma : \Hom(\Delta,\Gamma), a:\tm(\Delta,A[\gamma]) &\vdash& \p\circ\tuple{\gamma,a} = \gamma : \Hom(\Delta,\Gamma)\\
\Gamma,\Delta : \ctx, A:\ty(\Gamma), \gamma : \Hom(\Delta,\Gamma), a:\tm(\Delta,A[\gamma]) &\vdash& \q[\tuple{\gamma,a}] = a : \tm(\Delta,A[\gamma]) \\
\Gamma,\Delta,\Xi : \ctx, A:\ty(\Gamma), \gamma : \Hom(\Delta,\Gamma), a:\tm(\Delta,A[\gamma]), \delta : \Hom(\Xi,\Delta) &\vdash&
\tuple{\gamma,a} \circ \delta = \tuple{\gamma\circ\delta,a[\delta]} :
\Hom(\Xi,\Gamma\cext A) \\
\Gamma : \ctx, A:\ty(\Gamma) &\vdash&
\id_{\Gamma\cext A} = \tuple{\p,\q} : \Hom(\Gamma\cext A,\Gamma\cext A)
\end{eqnarray*}

\subsection{Adding $\Pi$-types}
\hspace{1em}
%We add three operator symbols in addition to the operator symbols for cwfs in Section 5.2 and 5.3:

Operator symbols:
\begin{eqnarray*}
\Gamma : \ctx, A : \ty(\Gamma), B : \ty(\Gamma.A)&\vdash& \Pi(A,B) : \ty(\Gamma)\\
\Gamma : \ctx, A : \ty(\Gamma), B : \ty(\Gamma.A), b : \tm(\Gamma.A, B) &\vdash& \lambda(b) : \tm(\Gamma,\Pi(A,B))\\
\Gamma : \ctx, A : \ty(\Gamma), B : \ty(\Gamma.A), c :  \tm(\Gamma,\Pi(A,B)), a : \tm(\Gamma, A) &\vdash& \app(c,a) : \tm(\Gamma, B[\tuple{\id,a}])
\end{eqnarray*}

Equations (omitting the context and type of the equalities):
 \begin{eqnarray*}
 \app(\lambda(b),a) &=& b[\tuple{\id,a}]\\
 \lambda(\app(c[\p],\q)) &=& c
 \end{eqnarray*}
 
 Equations for commutativity of operator symbols wrt substitution:
 \begin{eqnarray*}
\Pi(A,B)[ \gamma ] &=& \Pi(A [ \gamma ], B[ \gamma^+ ])\\
\lambda(b) [ \gamma ] &=& \lambda(b[\gamma^+ ])\\
\app(c,a) [ \gamma ] &=& \app(c[ \gamma ], a[ \gamma ] )
\end{eqnarray*}
where $\gamma^+ = \tuple{\gamma \circ \p, \q}$.

\subsection{Adding a universe closed under $\Pi$}
\hspace{1em}

Operator symbols:
\begin{eqnarray*}
\Gamma : \ctx &\vdash& \U_\Gamma : \ty(\Gamma)\\
\Gamma : \ctx, a : \tm(\Gamma,\U_\Gamma) &\vdash& {\Ta}(a) : \ty(\Gamma)\\
%\Gamma : \ctx &\vdash& (\N^0)_\Gamma : \tm(\Gamma,\U_\Gamma) \\
\Gamma : \ctx,
a : \tm(\Gamma,\U_\Gamma),
b :  \tm(\Gamma \cdot \Ta(a), \U_\Gamma))
&\vdash&
 \Pi^0(a,b) : \tm(\Gamma,\U_\Gamma)
\end{eqnarray*}
%$\U_\Gamma$ is the universe (a type) relative to the context $\Gamma$; $\Ta$ is the decoding operation mapping a term in the universe to the corresponding type; $\N^0$ is the code for $\N$ in the universe, and $\Pi^0$ forms codes for $\Pi$-types in the universe. (Note that we have dropped the context argument of $\Ta$ and $\Pi^0$.)

Equation:
\begin{eqnarray*}
%\Ta(\N^0_\Gamma) &=& \N_\Gamma\\
\Ta(\Pi^0(a,b)) &=& \Pi(\Ta(a),\Ta(b))
\end{eqnarray*}

 Equations for commutativity of operator symbols wrt substitution:
 \begin{eqnarray*}
{\U}_\Gamma [ \gamma ] &=& {\U}_\Delta\\
\Ta(a) [ \gamma ] &=& \Ta(a[ \gamma ] )\\
%\N^0_\Gamma [ \gamma ] &=&\N^0_\Delta\\
\Pi^0(a,b)[ \gamma ] &=& \Pi^0(a [ \gamma ], b[ \gamma^+ ])
\end{eqnarray*}

\subsection{Adding an external countable non-cumulative tower of universes closed under $\Pi$}

An infinitary theory! $n, n'$ are external natural numbers. (** Note that this is for the non-cumulative case, the Russell and Tarski constructions are for the cumulative case **)

Operator symbols:
\begin{eqnarray*}
\Gamma : \ctx &\vdash& (\U_n)_\Gamma : \ty(\Gamma)\\
\Gamma : \ctx, a : \tm(\Gamma,(\U_n)_\Gamma) &\vdash& {\Ta_n}(a) : \ty(\Gamma)\\
%\Gamma : \ctx &\vdash& (\N^0)_\Gamma : \tm(\Gamma,(\U_0)_\Gamma) \\
\Gamma : \ctx,
a : \tm(\Gamma,(\U_n)_\Gamma),
b :  \tm(\Gamma \cdot \Ta_n(a), (\U_{n'})_\Gamma))
&\vdash&
 \Pihat^{n,n'}(a,b) : \tm(\Gamma,(\U_{\max(n,n')})_\Gamma)\\
\Gamma : \ctx&\vdash&\UU^n_\Gamma: \tm(\Gamma,(\UU_{n+1})_\Gamma)
\end{eqnarray*}
%$(\U_0)_\Gamma$ is the universe (a type) relative to the context $\Gamma$; $\Ta_0$ is the decoding operation mapping a term in the universe to the corresponding type; $\N^0$ is the code for $\N$ in the universe, and $\Pi^0$ forms codes for $\Pi$-types in the universe. (Note that we have dropped the context argument of $\Ta_0$ and $\Pi^0$.)

Equations:
\begin{eqnarray*}
%\Ta_n(\N^n_\Gamma) &=& \N_\Gamma\\
\Ta_{\max(n,n')}(\Pihat^{n,n'}(a,b)) &=& \Pi(\Ta_n(a),\Ta_{n'}(b))\\
\Ta_{n+1}(\UU^n_\Gamma) &=& (\UU_n)_\Gamma
\end{eqnarray*}

 Equations for commutativity of operator symbols wrt substitution:
 \begin{eqnarray*}
{(\U_n)}_\Gamma [ \gamma ] &=& {(\U_n)}_\Delta\\
\Ta_n(a) [ \gamma ] &=& \Ta_n(a[ \gamma ] )\\
%\N^n_\Gamma [ \gamma ] &=&\N^n_\Delta\\
\Pihat^{n,n'}(a,b)[ \gamma ] &=& \Pihat^{n,n'}(a [ \gamma ], b[ \gamma^+ ])\\
\UU^n_\Gamma [ \gamma ] &=& \UU^n_\Delta
\end{eqnarray*}

\subsection{Adding an external countable cumulative tower of universes closed under $\Pi$}

In the cumulative case $\Pihat^n$ is singly indexed.

Rules for cumulativity
\begin{eqnarray*}
\Gamma : \ctx, a : \Tm(\Gamma,(\UU_{n})_\Gamma &\vdash& \t_n(a) : \Tm(\Gamma,(\UU_{n+1})_\Gamma)
\end{eqnarray*}

\begin{eqnarray*}
\t_n(\Pihat^n(a,b)) &=& \Pihat^{n+1}(\t_n(a),\t_n(b))\\ 
\Ta_{n+1}(\t_n(a)) &=& \Ta_n(a)
\end{eqnarray*}

{\bf Meaning explanations ("standard model").} If we want to justify the rules for cumulativity, $\t_n(a)$ either has to be a canonical form ($\t_n$ is a constructor) or $\t_n$ is defined by induction on $\UU_n$, see Palmgren. In the latter case the rule 
$$\t_n(\Pihat^n(a,b)) = \Pihat^{n+1}(\t_n(a),\t_n(b))$$ is a definitional equality, but in the former case it is not valid, since the head construtors are different. The rule 
$$\Ta_{n+1}(\t_n(ne)) = \Ta_n(ne)$$ on the other hand, is a definitional equality in the former case. It is also satisfied in the latter case, but now only as an extensional equality. They should also be satisfied in the model in {\bf Set} (also constructively).

{\bf Model of normal forms.}
We shall prove that both rules are satisfied in both the Tarski and the Russell model of normal forms. 

We first focus on the case with $\beta$-rules only (and not $\eta$). For the purpose of relating the Tarski and Russell-style models we introduce lifting functions
\begin{eqnarray*}
\t^n_m &=& \t_{n-1} \circ \cdots \circ \t_m : \UU_m \to \UU_n\\
\UUhat^n_m &=& \t^n_{m+1}(\UUhat_m) : \UU_n
\end{eqnarray*}

With $\eta$-expanded normal forms we will need lifting functions such as
$$
i_{k,m,n} : \Pi(A_1,\ldots, \Pi(A_k,\UU_m)\cdots)\to \Pi(A_1,\ldots, \Pi(A_k,\UU_n)\cdots)
$$
defined by $i_{k,m,n}(c)  = \lambda^k(\t^n_m(\ap^k(c,\q)))$. (** roughly **) ).

(** We briefly looked the addition of other type formers such as $\Sigma$, $\N$, etc. We should check this. **)

\section{An initial $\Pi\UU$-tower-cwf of normal forms a la Tarski}
\label{sec:syntax}


\subsection{Raw syntax}\label{sec:grammar}

%This first construction is inspired by the generalised algebraic theory of $\Pi\UU$-tower-cwfs, but here we define it as a usual logical system with a grammar and a collection of inference rules. 
The grammar has five syntactic categories of raw normal expressions: contexts $\RawCtx$,
substitutions $\RawSub$, normal types $\RawTy$, normal terms $\RawTm$ and neutral terms $\RawNe$.
\begin{eqnarray*}
\Gamma \in \RawCtx &::=& 1  \ |\ \Gamma.A\\
\gamma \in \RawSub \ &::=&  \langle\rangle \ |\ \langle \gamma, a \rangle\\
A \in \RawTy &::=& \Pi(A,A)\ |\  \UU_{n} \ |\ \Ta_n(ne) \\ 
a \in \RawTm &::=& ne \ |\  \lambda(a)\ |\ \Pihat^n(a,a)\ |\ \UUhat^n_m \ |\  \t^n_m(ne) \\
ne \in \RawNe &::=& \var(i)  \ |\  \ap(ne,a)
\end{eqnarray*}
Note that $\t^n_m(ne)$ cannot be neutral if we have the equation $\Ta_{n}(\t^n_m(ne)) = \Ta_m(ne)$, because then these are distinct normal forms.

We can define all the $\Pi\UU$-tower-cwf-operations on this raw syntax, including hereditary substitution, using normalization by evaluation. Note that the inference rules refer to these operations.

However, $\eta$-expansion depends on the type and will be considered later.

\subsection{Inference rules}
We have inference rules for the four forms of judgment:
\begin{eqnarray*}
&&\Gamma \vdash\\
&&\Delta \vdash \gamma : \Gamma\\
&&\Gamma \vdash A\\
&&\Gamma \vdash a : A
\end{eqnarray*}
and an auxiliary form of judgment for type inference of neutral terms:
\begin{eqnarray*}
&&\Gamma \vdash ne \Rightarrow A
\end{eqnarray*}

These inference rules (see the boxed displays) are introductory clauses for inductive definitions of predicates on raw experessions. 
We do not need inference rules for equality judgments, since they are syntactic equalities.

\begin{figure}[!htb]
  \centering
  \label{fig:cwf3}
\boxit[Normal forms for pure cwf-structure]{
  \begin{mathpar}
% Contextes
\\
    	\inferrule
		{} 
		{1 \vdash }
	\and
	\inferrule
		{%\Gamma \vdash \\ 
		 \Gamma \vdash A} 
		{\Gamma.A \vdash} 
	\and
    	\inferrule
		{\Gamma \vdashS}
		{\Gamma \vdash \emptySub{} : \emptyContext}
	\and

	\inferrule
		{\Gamma \vdash A \\ 
		 \Delta \vdash \gamma : \Gamma \\ 
		 \Delta \vdash a : A[\gamma]} 
		{\Delta\vdash \langle \gamma,a \rangle : \Gamma.A} 
	\and
	\inferrule{\Gamma \vdash ne \Rightarrow A}
	{\Gamma \vdash ne : A}
	\and
	\inferrule
	{\Gamma \vdash A}
	{\Gamma.A \vdash \var(0) \Rightarrow A[\p]}
	\and
	\inferrule
	{\Gamma \vdash A \and \Gamma \vdash \var(i) : B}
	{\Gamma.A \vdash \var(i+1) \Rightarrow B[\p]}

  \end{mathpar}
}
\end{figure}

\begin{figure}[!htb]
\boxit[$\beta$-normal forms for $\Pi$-types]{
\begin{mathpar}
\\
    \infer {\Gamma \vdash \Pi(A,B)} 
    {\Gamma.A \vdash B}
    \and
    \infer{ \Gamma  \vdash   \lambda (b) :  \Pi (A, B)}
    { \Gamma .A  \vdash  b : B} 
    \and
    \infer{ \Gamma   \vdash  \ap(ne, a) \Rightarrow B[\langle \id, a \rangle] }{  %\Gamma .A  \vdash  B & 
     \Gamma   \vdash  ne \Rightarrow \Pi (A, B) &  \Gamma   \vdash  a : A} 
    
  \end{mathpar}
  }
%  \caption{Inferences rules for $\Pi$-types}
  \label{fig:pi}
\end{figure}
%\footnote{Note that the rules for $\Pi$ need to be completed. 
%$\gamma^+ = \langle \gamma \circ \p_{A[\gamma]}, \qI_{A[\gamma]}\rangle$
%}


\begin{figure}[!htb]
\boxit[The nth universe a la Tarski has a universe structure]{
\begin{mathpar}
\\
\infer{\Gamma \vdash \UU_n}
{\Gamma  \vdash}
\and
\infer{\Gamma \vdash  \Ta_n(ne) }
{\Gamma \vdash ne  \Rightarrow \UU_n}
\and
\infer{\Gamma \vdash \Pihat^n(a,b) : \UU_n}
{\Gamma \vdash a : \UU_n\and\Gamma.\Ta_n(a) \vdash b : \UU_n}
%\and
%\infer{(\UU_n)_\Gamma[\gamma] = (\UU_n)_\Delta}
%{\Delta   \vdash   \gamma  :  \Gamma }
%\and
%   \infer{ \Delta   \vdash   \Pi(a, b) [\gamma] =  \Pi(a [\gamma] , b [\gamma ^+]): (\UU_n)_\Delta}{ \Gamma   \vdash  a : (\UU_n)_\Gamma &  \Gamma .a \vdash  b : (\UU_n)_{\Gamma .a} &  \Delta   \vdash   \gamma  :  \Gamma } \
\end{mathpar}
  }
  \label{fig:uu}
\end{figure}

\begin{figure}[!htb]
\boxit[The $m$-th universe a la Tarski is in the $n$-th ($m < n$)]{
\begin{mathpar}
\\
\infer{\Gamma \vdash \hat{\UU}^n_m : \UU_n}{\Gamma \vdash}
\end{mathpar}
}
\label{fig:uu-code}
\end{figure}
\begin{figure}[!htb]
\boxit[Cumulativity a la Tarski ($m \leq n$)
]{
\begin{mathpar}
\\
\infer{\Gamma \vdash \t^n_m(ne)  :  \UU_{n}}
{\Gamma \vdash ne \Rightarrow \UU_m}
 \end{mathpar}
  }
  \label{fig:uu}
\end{figure}
\footnote{
This old rule with $\t_n(ne) : \UU_{n+1}$ does not suffice, since it cannot be iterated.
}

In this system, not only the neutral terms, but all normal tems $a$ in a well-formed context $\Gamma$ has a unique type. This is computed by a type-inference algorithm $$\typeof : \RawCtx \to \RawTm \to \RawTy$$ such that $\Gamma \vdash$ implies $\typeof(\Gamma,a) = A$ iff $\Gamma \vdash a : A$. 

\subsection{The syntactic $\Pi\UU$-tower-cwf $\T$ of normal forms a la  Tarski}

\begin{eqnarray*}
\Ctx_ \T &=& \{ \Gamma \ |\ \Gamma \vdash \}\\
\Sub_ \T(\Delta,\Gamma) &=&  \{ \gamma \ |\ \Delta \vdash \gamma : \Gamma  \}\\
\Ty_ \T(\Gamma) &=&  \{ A \ |\ \Gamma \vdash A  \}\\
\Tm_ \T(\Gamma,A) &=&  \{ a \ |\ \Gamma \vdash a : A  \}
\end{eqnarray*}
We also define the well-typed neutral terms
\begin{eqnarray*}
\Ne_\T(\Gamma,A) &=&  \{ ne \ |\ \Gamma \vdash ne \Rightarrow A  \} \subseteq \Tm_ \T(\Gamma,A) 
\end{eqnarray*}

We have already defined all the untyped $\Pi\UU$-tower-cwf-combinators using normalization of the raw expressions. In order to prove that they yield a $\Pi\UU$-tower-cwf, we need to prove that they yield well-typed normal forms and that they satisfy all equations of  $\Pi\UU$-tower-cwfs. 

Furthermore, we can prove that $\Gamma \vdash A$ implies $\Gamma \vdash$, $\Gamma \vdash a : A$ implies $\Gamma \vdash A$, $\Delta \vdash \gamma : \Gamma$ implies $\Delta, \Gamma \vdash$, and $\Gamma \vdash ne : A$  implies $\Gamma \vdash A$.

%This algorithm relies on an algorithm $istype(\Gamma,A)$ which checks whether a raw type $A$ is well-typed with respect to the context $\Gamma$, and an algorithm $iscontext(\Gamma)$ which checks whether $\Gamma \vdash$.


\subsection{Initiality of $ \T$}
We have total interpretation functions for each of the syntactic categories:
\begin{eqnarray*}
\intCtxU{-} &:& \Ctx_\T \to \Ctx_\C\\
\intSubU{-}_{\Delta,\Gamma}&:& 
\Sub_\T(\Delta, \Gamma)
\to
\Sub_\C(\intCtxU{\Delta}, \intCtxU{\Gamma})\\
\intTyU{-}_\Gamma&:& \Ty_\T(\Gamma) \to \Ty_\C(\intCtxU{\Gamma})\\
\intTmU{-}_{\Gamma,A} &:& \Tm_\T(\Gamma,A) \to
\Tm_\C (\intCtxU{\Gamma}, \intTyU{A})
\end{eqnarray*}
and for the neutral terms
\begin{eqnarray*}
\intNeU{-}_{\Gamma,A} &:& \Ne_\T(\Gamma,A) \to \Tm_\C (\intCtxU{\Gamma}, \intTyU{A})
\end{eqnarray*}
where the type $A$ is computed by type-inference from $\Gamma$ and the neutral term.

We first define partial interpretation functions by induction on raw expression, and then prove that they are total on the subsets of well-typed expressions. 

We begin with the interpretation of neutral terms:
\begin{eqnarray*}
\intNeU{\var(0)}_{\Gamma.A,A[\p]} &=& (\qI_{\C})_{\intCtxU{\Gamma},\intTyU{A}}\\
\intNeU{\var(n+1)}_{\Gamma.A,B[\p]} &=& \inte{\var(n)}_{\Gamma,B}[\p_\C]_\C\\
\intNeU{\ap(ne,a)}_{\Gamma,B[\langle \id,a \rangle]} &=& \ap_\C(\intNeU{ne}_{\Gamma,\Pi(A,B)},\intTmU{a}_{\Gamma,A})
\end{eqnarray*}
Note that $A$ in the third clause and $B$ in the second and third clause is computed by type-inference.

We then interpret the normal contexts, substitutions, types, and terms:
\begin{eqnarray*}
\intCtxU{1} &=& 1_\C\\
\intCtxU { \Gamma .A} &=& \intCtxU \Gamma ._\C \intCtxU{A}\\
%\end{eqnarray*}
%\inte{\id}_\Gamma & = & {(\idC)}_{\inte{\Gamma}}\\
%\begin{eqnarray*}
\intSubU{ \langle  \rangle }_{\Delta,1} &=& \langle \rangle_\C\\
\intSubU{ \langle  \gamma , a \rangle }_{\Delta,\Gamma.A} &=& \langle \intSubU{\gamma}_{\Delta,\Gamma} , \intSubU{a}_{\Delta,A[\gamma]} \rangle_\C\\
%\end{eqnarray*}
%where in the last clause $\Gamma = 1.A_1. \cdots . A_i . \cdots . A_n$, $A = A_i[\p^{n-i}]$, and $(\pi_i^n)_\C = q[p] \cdots [p]$ (check!). We utilize the type inforence function.
%\begin{eqnarray*}
\intTyU{\Pi(A,B)}_\Gamma&=& \Pi_\C(\intTyU{A}_\Gamma,\intTyU{B}_{\Gamma.A})\\
%\end{eqnarray*}
%\begin{eqnarray*}
\intTmU{\lambda(b)}_{\Gamma,\Pi(A,B)} &=& \lambda_\C(\intTmU{b}_{\Gamma.A, B})\\
\intTmU{ne}_{\Gamma,A} &=& \intNeU{ne}_{\Gamma,A}
\end{eqnarray*}

The clauses for the tower of universes are as follows:
\begin{eqnarray*}
\intTyU{\UU_n}_\Gamma &=& ((\UU_n)_\C)_{\intCtxU{\Gamma}}\\ %\in \Ty_\C(\intCtxU{\Gamma})\\
\intTyU{\Ta_n(ne)}_\Gamma &=& (\Ta_n)_\C(\intNeU{ne}_{\Gamma,\UU_n})\\% \in \Ty_\C(\intCtxU{\Gamma})\\
\intTmU{\Pihat^n(a,b)}_{\Gamma,\UU_n}&=& \Pihat^n_\C(\intTmU{a}_{\Gamma,\UU_n},\intTmU{b}_{\Gamma.\Ta_n(a),\UU_n})\\% \in \Tm_\C(\intCtxU{\Gamma},((\UU_n)_\C)_{\intCtxU{\Gamma}})\\
\intTmU{\hat{\UU}^n_m}_{\Gamma,\UU_{n+1}} &=& ((\UUhat^n_m)_\C)_{\intCtxU{\Gamma}}\\% \in \Tm_\C(\intCtxU{\Gamma},((\UU_{n+1})_\C)_{\intCtxU{\Gamma}})\\
\intTmU{\t^n_m(ne)}_{\Gamma,\UU_{n+1}} &=& 
(\t^n_m)_\C(\intTmU{ne}_{\Gamma,\UU_n})%& \in \Tm_\C(\intCtxU{\Gamma},(\UU_{n+1})_\C)
\end{eqnarray*}
where $\UUhat^n_m$ and $\t^n_m$ are defined in the section on generalized algebraic theories \ref{}.

Note that all clauses say intuitively that the expressions in normal forms are interpreted by their "intended meaning" in an arbitrary $\Pi\UU$-tower cwf $\C$. One can view this as an injection of normal forms into the language of $\Pi\UU$-tower-cwf-combinators followed by the "obvious" initial map into $\C$. In particular the doubly indexed normal form constructs $\UUhat^n_m$ and $\t^n_m$ are interpreted by iterating the singly indexed $\Pi\UU$-tower-cwf-combinators, $\UUhat_m$ and $\t_m$.

%provided
%$\intTmU{a} \in \Tm_\C(\intCtxU{\Gamma},((\UU_n)_\C)_{\intCtxU{\Gamma}})$ and 
%$\intTmU{b} \in \Tm_\C(\intCtxU{\Gamma.a},((\UU_n)_\C)_{\intCtxU{\Gamma.a}})$.
We need to prove that this is a $\Pi\UU$-tower-cwf morphism and that it is the unique one into $\C$. (This is the correctness proof of the normalization algorithm.) The latter part ("completeness") is direct, since there is a unique structure-preserving way to interpret each syntactic construction of well-typed normal forms. For the former part ("soundness") we need in particular to prove that
\begin{eqnarray*}
\intTyU{A[\gamma]} &=& \intTyU{A}[\intSubU{\gamma}]_\C\\
\intTmU{a[\gamma]} &=& \intTmU{a}[\intSubU{\gamma}]_\C
\end{eqnarray*}

(** The idea is to use a standard nbe-proof, like Coquand 2018 or Abel et al 2007, but extend it to explicit substitutions and add an interpretation of substitution in types and terms. The normal form of $a [ \gamma ]$ is $nf (i a [ i \gamma ]')$, where $i$ is the injection of normal forms and $-[-]'$ is the explicit substitution operation. **)

\begin{thm}\label{theom:initial_cwf}
  $\T$ is an initial $\Pi\UU$-tower-cwf.
\end{thm}

{\bf Remark.} What about $\eta$?

\section{An initial $\Pi\UU$-tower-cwf a la  Russell}

\subsection{Raw syntax of normal and neutral terms}

It seems that we need normal terms to construct an initial model a la Russell. (* It is only for neutral terms that we have type inference *) Then we do not need annotations on $\app(A,B,c,a)$. There is a difference between the cumulative and the non-cumulative system. For the non-cumulative system we have uniqueness of types. For the cumulative system we have least types.
\begin{eqnarray*}
\Gamma \in \RawCtx &::=& 1  \ |\ \Gamma.a\\
\gamma \in \RawSub \ &::=& \langle\rangle \ | \langle \gamma, a \rangle\\
a \in \RawTm &::=& ne  \ |\ \lambda(a) \ |\ \Pi(a,a)  \ |\  \UU_n\\
ne \in \RawNe &::=& \var(i)   \ |\ \ap(ne,a) 
\end{eqnarray*}
We can then define operations for all the operations of $\Pi\UU$-tower cwfs in terms of the normalization algorithm. (** This algorithm should be a slight modification of the algorithm for a la Tarski **). In the sequel we will use notations for these. We will also be able to derive typing rules for these operations, e g $\Gamma \vdash \id_\Gamma : \Gamma$ (we need the subscript, because $\id_1 = \langle\rangle, \id_{\Gamma.A} = ?$, etc.)

\subsection{Inference rules}

Can we prove $\Gamma \vdash A$ implies $\Gamma \vdash$, etc? Since the inference rules rely on defined $\Pi\UU$-tower cwfs, these have to be defined first.

\begin{figure}[!htb]
\boxit[The nth universe a la Russell has a universe structure]{
\begin{mathpar}
\\
\infer{\Gamma \vdash \UU_n}
{\Gamma  \vdash}
\and
\infer{\Gamma \vdash  ne }
{\Gamma \vdash ne  \Rightarrow \UU_n}
\and
\infer{\Gamma \vdash \Pi(a,b) : \UU_n}
{\Gamma \vdash a : \UU_n\and\Gamma.a \vdash b : \UU_n}
%\and
%\infer{(\UU_n)_\Gamma[\gamma] = (\UU_n)_\Delta}
%{\Delta   \vdash   \gamma  :  \Gamma }
%\and
%   \infer{ \Delta   \vdash   \Pi(a, b) [\gamma] =  \Pi(a [\gamma] , b [\gamma ^+]): (\UU_n)_\Delta}{ \Gamma   \vdash  a : (\UU_n)_\Gamma &  \Gamma .a \vdash  b : (\UU_n)_{\Gamma .a} &  \Delta   \vdash   \gamma  :  \Gamma } \
\end{mathpar}
  }
  \label{fig:uu}
\end{figure}

\begin{figure}[!htb]
\boxit[The $m$-th universe a la Russell is in the $n$-th ($m < n$)]{
\begin{mathpar}
\\
\infer{\Gamma \vdash \UU_m : \UU_n}{\Gamma \vdash}
\end{mathpar}
}
\label{fig:uu-code}
\end{figure}
\begin{figure}[!htb]
\boxit[Cumulativity a la Russell ($m \leq n$)
]{
\begin{mathpar}
\\
\infer{\Gamma \vdash ne  :  \UU_{n}}
{\Gamma \vdash ne \Rightarrow \UU_m}
 \end{mathpar}
  }
  \label{fig:uu}
\end{figure}
\footnote{
This old rule with $ne : \UU_{n+1}$ does not suffice, since it cannot be iterated.
}
The Russell-rules for universes are in one-to-one correspondence with the Tarski-rules, and are obtained by stripping the Tarski-annotations.

%As for a la Tarski we have type-inference for neutral term $$\typeof : \RawCtx \to \RawNe \to \RawTy$$. 
%\begin{eqnarray*}
%\typeof(\Gamma.A, \var(0)) &=& A[\p]\\
%\typeof(\Gamma.A,\var(i+1)) &=& \typeof(\Gamma.A,\var(i)) [\p]\\
%\typeof(\Gamma,\app(c,a))&=&B[\tuple{ \id,a} ]
%\end{eqnarray*}
%where $\typeof(\Gamma,c) = \Pi(A,B)$ in the last clause. 
%
%However, we do not have uniqueness of types for a la Russell because of the cumulativity rule. This states that if $m \leq n$, and $\Gamma \vdash ne \Rightarrow \UU_m$, then $\Gamma \vdash ne : \UU_n$. Hence in the interpretation function (and translation to a la Tarski) we need to insert $\t^n_m$ in front of such terms.
%
%(** What about the following? **) Note also that without cumulativity there is a one-to-one correspondence between derivations a la Tarski and a la Russell, and type-inference for normal terms is defined by
%\begin{eqnarray*}
%\typeof(\Gamma.A,\lambda(b)) &=& \Pi(A,\typeof(\Gamma,b))\\
%\typeof(\Pi(A,B)) &=& \UU_{\max(m,n)}\\
%\typeof(\Gamma,\UU_n) &=& \UU_{n+1}
%\end{eqnarray*}
%where in the clause for $\Pi$, we have $\typeof(A) = \UU_m, \typeof(B) = \UU_n$.
%
%{\bf Lemma.} (In the presence of $\eta$) We have a type inference function which returns the least type in the following subtype relation. First we have $\typeof(\Gamma,a) =  \UU_n$ 
%iff $\Gamma \vdash a : \UU_m \mathrm{\ for\ all\ }m \geq n$.
%More generally, we have
%$$\typeof(\Gamma,a) =  \Pi(A_1,\ldots, \Pi(A_k,\UU_n)\cdots)$$ 
%iff $$\Gamma \vdash a : \Pi(A_1,\ldots, \Pi(A_k,\UU_m)\cdots)\mathrm{\ for\ all\ }m \geq n.$$
%When we prove the "if" direction, we use the following inference rules: If $k=0$ we insert 
%$\t^{m-n}$ liftings. If $k=1$, we map $c$ to $c\,x$ to $c\,(\t^{m-n}\,x)$ to $\lambda x.c\,(\t^{m-n}\,x)$.
%
%What if $\typeof(\Gamma,a) \neq \Pi(A_1,\ldots, \Pi(A_k,\UU_n)\cdots)$ for all $k, n$? E g $x : \UU_0, y : x \vdash y$? In variable free notation $\UU_0.(\q.1) \vdash \q :\ ?$ Is it the case that $\typeof$ returns the unique type then?

\def\R{\mathcal{R}}

\subsection{The syntactic $\Pi\UU$-tower-cwf $\R$ of normal forms a la Russell}

We define:
\begin{eqnarray*}
\Ctx_\R &=& \{ \Gamma \ |\ \Gamma \vdash \}\\
\Sub_\R(\Delta,\Gamma) &=&  \{ \gamma \ |\ \Delta \vdash \gamma : \Gamma  \}\\
\Ty_\R(\Gamma) &=&  \{ A \ |\ \Gamma \vdash A  \}\\
\Tm_\R(\Gamma,A) &=&  \{ a \ |\ \Gamma \vdash a : A  \}
\end{eqnarray*}
and the auxiliary:
\begin{eqnarray*}
\Ne_\R(\Gamma,A) &=&  \{ ne \ |\ \Gamma \vdash ne \Rightarrow A  \}
\end{eqnarray*}
\subsection{Initiality of $ \R$}

We have total interpretation functions for each of the syntactic categories:
\begin{eqnarray*}
\intCtxU{-} &:& \Ctx_\R \to \Ctx_\C\\
\intSubU{-}_{\Delta,\Gamma}&:& 
\Sub_\R(\Delta, \Gamma)
\to
\Sub_\C(\intCtxU{\Delta}, \intCtxU{\Gamma})\\
\intTyU{-}_\Gamma&:& \Ty_\R(\Gamma) \to \Ty_\C(\intCtxU{\Gamma})\\
\intTmU{-}_{\Gamma,A} &:& \Tm_\R(\Gamma,A) \to
\Tm_\C (\intCtxU{\Gamma}, \intTyU{A})
%\intNeU{-}_{\Gamma,A} &:& \Ne_\R(\Gamma,A) \to
%\Tm_\C (\intCtxU{\Gamma}, \intTyU{A})
\end{eqnarray*}
and for the neutral terms
\begin{eqnarray*}
\intNeU{-}_{\Gamma,A} &:& \Ne_\R(\Gamma,A) \to \Tm_\C (\intCtxU{\Gamma}, \intTyU{A})
\end{eqnarray*}
where the type $A$ is computed by type-inference from $\Gamma$ and the neutral term.

The interpretation clauses for the $\Pi$-cwf-clauses are the same for a la Russell and a la Tarski.

The clauses for the tower of universes a la Russell are as follows. Note that they are the same as for a la Tarski, but with the Tarski-annotations removed.
\begin{eqnarray*}
\intTyU{\UU_n}_\Gamma &=& ((\UU_n)_\C)_{\intCtxU{\Gamma}}\\ %\in \Ty_\C(\intCtxU{\Gamma})\\
\intTyU{ne}_\Gamma &=& (\Ta_n)_\C(\intNeU{ne}_{\Gamma,\UU_n})
\end{eqnarray*}
where $\typeof(\Gamma,ne) = \UU_n$ (that is, $\Gamma \vdash ne \Rightarrow \UU_n$).
\begin{eqnarray*}
% \in \Ty_\C(\intCtxU{\Gamma})\\
\intTmU{\Pi(a,b)}_{\Gamma,\UU_n}&=& \Pihat^n_\C(\intTmU{a}_{\Gamma,\UU_n},\intTmU{b}_{\Gamma.a,\UU_n})\\% \in \Tm_\C(\intCtxU{\Gamma},((\UU_n)_\C)_{\intCtxU{\Gamma}})\\
\intTmU{\UU_m}_{\Gamma,\UU_n} &=& ((\UUhat^n_m)_\C)_{\intCtxU{\Gamma}}
\end{eqnarray*}
where $m < n$.
\begin{eqnarray*}
\intTmU{ne}_{\Gamma,\UU_n} &=& 
(\t^n_m)_\C(\intNeU{ne}_{\Gamma,\UU_m})%& \in \Tm_\C(\intCtxU{\Gamma},(\UU_{n+1})_\C)
\end{eqnarray*}
where $m \leq n$ and $\typeof(\Gamma,ne) = \UU_m$ (that is, $\Gamma \vdash ne \Rightarrow \UU_m$). Recall that $\UUhat^n_m$ and $\t^n_m$ are defined in the section on generalized algebraic theories \ref{}.

Note that all clauses say intuitively that the expressions in normal forms are interpreted by their "intended meaning" in an arbitrary $\Pi\UU$-tower cwf $\C$. One can view this as an injection of normal forms into the language of $\Pi\UU$-tower-cwf-combinators followed by the "obvious" initial map into $\C$. A crucial differences between the Russell-interpretation and the Tarski-interpretation is that in the clauses for neutrals the coercions are implicit and we need to insert the constructs $\UUhat^n_m$ and $\t^n_m$ which are defined by iterating the singly indexed $\Pi\UU$-tower-cwf-combinators, $\UUhat_m$ and $\t_m$.

We need to prove that this is indeed a $\Pi\UU$-tower-cwf morphism and that it is the unique one into $\C$.
\begin{thm}\label{theom:initial_cwf}
  $\R$ is an initial $\Pi\UU$-tower-cwf.
\end{thm}

%{\bf Remark.} What about $\eta$? With $\eta$ we have the conversions Previously $\intTmU{\var(0)}_{\Gamma.A,B}$ was defined provided $B = A[\p]$. Now it can be the case that $B$ is a higher type (in the ordering above). In this case we need to insert lifting $c \mapsto \lambda x.\t^{n-m}(\ap(c,x))$. \begin{eqnarray*}
%\intTmU{\var(0)}_{\Gamma.A,C} &=& i_{k,m,n}((\qI_{\C})_{\intCtxU{\Gamma},\intTyU{A}})
%\end{eqnarray*}
%where $\typeof(\Gamma,\var(0)) = \Pi(A_1,\ldots, \Pi(A_k,\UU_m)\cdots)$ and $C = \Pi(A_1,\ldots, \Pi(A_k,\UU_n)\cdots) = A[\p]$ and $m \geq n$, and $i_{k,m,n} : \Pi(A_1,\ldots, \Pi(A_k,\UU_m)\cdots) \to B = \Pi(A_1,\ldots, \Pi(A_k,\UU_n)\cdots)$.
%

\section{The $\Pi\UU$-cwf (only $\beta$) isomorphism between $\T$ and $\R$}

We can now look at the instances of the interpretation morphisms in both directions.
We only present the clauses for universes.

\subsection{Tarski to Russell.} 
\begin{eqnarray*}
\intTyU{\UU_n}_\Gamma &=& \UU_n \in \Ty_\R(\intCtxU{\Gamma})\\
\intTyU{\Ta_n(ne)}_\Gamma &=& \intNeU{ne}_{\Gamma,\UU_n}\in 
\Ne_\R(\intCtxU{\Gamma},\UU_n) \subseteq \Ty_\R(\intCtxU{\Gamma})\\
\intTmU{\Pihat^n(a,b)}_{\Gamma,\UU_n}
&=& \Pi(\intTmU{a}_{\Gamma,\UU_n},\intTmU{b}_{\Gamma.\Ta_n(a),\UU_n})
\in \Tm_\R(\intCtxU{\Gamma},\UU_n)\\
\intTmU{\UUhat^n_m}_{\Gamma,\UU_n} &=& \UU_m \in \Tm_\R(\intCtxU{\Gamma},\UU_n)\\
\intTmU{\t^n_m(ne)}_{\Gamma,\UU_n} &=& 
\intNeU{ne}_{\Gamma,\UU_{m}} \in \Ne_\R(\intCtxU{\Gamma},\UU_{m}) \subseteq \Tm_\R(\intCtxU{\Gamma},\UU_n)
\end{eqnarray*}
Note that the interpretation morphism is a pure stripping of raw expressions. It is independent of the subscripts.

\subsection{Russell to Tarski} 

We first consider two cases for neutrals:
\begin{eqnarray*}
\intTyU{ne}_\Gamma &=& \Ta_n(\intNeU{ne}_{\Gamma,\UU_n})\\
\intTmU{ne}_{\Gamma,\UU_{n}} &=& 
\t_m^n(\intNeU{ne}_{\Gamma,\UU_m})
\end{eqnarray*}
where $\typeof(\Gamma,ne) = \UU_n$ in the first clause and $\typeof(\Gamma,ne) = \UU_m$ and $m \leq n$ in the second clause. 
\begin{eqnarray*}
\intTyU{\UU_n}_\Gamma &=& \UU_n\\ 
\intTmU{\Pi(a,b)}_{\Gamma,\UU_n}&=& \Pihat^n(\intTmU{a}_{\Gamma,\UU_n},\intTmU{b}_{\Gamma.a,\UU_n})\\
\intTmU{\UU_m}_{\Gamma,\UU_{n}} &=& \UUhat_m^n
\end{eqnarray*}
Note that this interpretation morphism makes use of the subscripts.

{\bf Theorem.} The interpretation morphisms form an isomorphism of $\Pi\UU$-tower-cwfs.

{\bf Proof.} Both $\T$ and $\R$ are initial and hence isomorphic. For a direct proof, we note that it is easy to see that we have an isomorphism, but to show that it preserves $\Pi\UU$-tower-cwf structure, we need to inspect the hereditary substitution (and all other operation) for both a la Russell and a la Tarski.

\bibliographystyle{plain}
\bibliography{refs}
\end{document}
