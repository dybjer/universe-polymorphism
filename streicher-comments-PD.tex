%\documentclass[12pt,a4paper]{amsart}
\documentclass[11pt,a4paper]{article}
%\ifx\pdfpageheight\undefined\PassOptionsToPackage{dvips}{graphicx}\else%
%\PassOptionsToPackage{pdftex}{graphicx}
\PassOptionsToPackage{pdftex}{color}
%\fi

%\usepackage{diagrams}

%\usepackage[all]{xy}
\usepackage{url}
\usepackage[utf8]{inputenc}
\usepackage{verbatim}
\usepackage{latexsym}
\usepackage{amssymb,amstext,amsmath,amsthm}
\usepackage{epsf}
\usepackage{epsfig}
% \usepackage{isolatin1}
\usepackage{a4wide}
\usepackage{verbatim}
\usepackage{proof}
\usepackage{latexsym}
%\usepackage{mytheorems}
\newtheorem{theorem}{Theorem}[section]
\newtheorem{corollary}{Corollary}[theorem]
\newtheorem{lemma}{Lemma}[theorem]
\newtheorem{proposition}{Proposition}[theorem]
\theoremstyle{definition}
\newtheorem{definition}[theorem]{Definition}
\newtheorem{remark}{Remark}[theorem]
\newtheorem{TODO}{TODO}[theorem]

\usepackage{float}
\floatstyle{boxed}
\restylefloat{figure}


%%%%%%%%%copied from SymmetryBook by Marc

% hyperref should be the package loaded last
%% \usepackage[backref=page,
%%             colorlinks,
%%             citecolor=linkcolor,
%%             linkcolor=linkcolor,
%%             urlcolor=linkcolor,
%%             unicode,
%%             pdfauthor={CAS},
%%             pdftitle={Symmetry},
%%             pdfsubject={Mathematics},
%%             pdfkeywords={type theory, group theory, univalence axiom}]{hyperref}
% - except for cleveref!
\usepackage[capitalize]{cleveref}
%\usepackage{xifthen}
\usepackage{xcolor}
\definecolor{linkcolor}{rgb}{0,0,0.5}

%%%%%%%%%
\def\oge{\leavevmode\raise
.3ex\hbox{$\scriptscriptstyle\langle\!\langle\,$}}
\def\feg{\leavevmode\raise
.3ex\hbox{$\scriptscriptstyle\,\rangle\!\rangle$}}

%%%%%%%%%
\newcommand\myfrac[2]{
 \begin{array}{c}
 #1 \\
 \hline \hline
 #2
\end{array}}


\newcommand*{\Scale}[2][4]{\scalebox{#1}{$#2$}}%
\newcommand*{\Resize}[2]{\resizebox{#1}{!}{$#2$}}

\newcommand{\II}{\mathbb{I}}
\newcommand{\refl}{\mathsf{refl}}
\newcommand{\mkbox}[1]{\ensuremath{#1}}


\newcommand{\Id}{\mathsf{Id}}
\newcommand{\conv}{=}
%\newcommand{\conv}{\mathsf{conv}}
\newcommand{\lam}[2]{{\langle}#1{\rangle}#2}
\def\NN{\mathsf{N}}
\def\UU{\mathsf{U}}
\def\JJ{\mathsf{J}}
\def\Level{\mathsf{Level}}
\def\List{\mathsf{List}}
\def\Cons{\mathsf{Cons}}
\def\Nil{\mathsf{Nil}}
%\def\Type{\hbox{\sf Type}}
\def\ZERO{\mathsf{0}}
\def\SUCC{\mathsf{S}}

\newcommand{\type}{\mathsf{type}}
\newcommand{\LAM}{\mathsf{lam}}
\newcommand{\APP}{\mathsf{app}}
\newcommand{\mypi}[3]{\Pi_{#1:#2}#3}
\newcommand{\mylam}[3]{\lambda_{#1:#2}#3}
\newcommand{\mysig}[3]{\Sigma_{#1:#2}#3}
\newcommand{\N}{\mathsf{N}}
\newcommand{\Set}{\mathsf{Set}}
\newcommand{\El}{\mathsf{El}}
%\newcommand{\U}{\mathsf{U}} clashes with def's in new packages
\newcommand{\T}{\mathsf{T}}
\newcommand{\sT}{\mathsf{t}}
\newcommand{\Usuper}{\UU_{\mathrm{super}}}
\newcommand{\Tsuper}{\T_{\mathrm{super}}}
%\newcommand{\conv}{\mathrm{conv}}
\newcommand{\idtoeq}{\mathsf{idtoeq}}
\newcommand{\isEquiv}{\mathsf{isEquiv}}
\newcommand{\ua}{\mathsf{ua}}
\newcommand{\UA}{\mathsf{UA}}
%\newcommand{\Level}{\mathrm{Level}}
\def\Constraint{\mathsf{Constraint}}
\def\Ordo{\mathcal{O}}

\newcommand{\Con}{\mathsf{ Con}}
\newcommand{\Elem}{\mathsf{Elem}}
\newcommand{\Type}{\mathsf{Type}}
\newcommand{\id}{\mathsf{id}}
\newcommand{\pp}{\mathsf{p}}
\newcommand{\qq}{\mathsf{q}}

\def\Ctx{\mathrm{Ctx}}
\def\Ty{\mathrm{Ty}}
\def\Tm{\mathrm{Tm}}

\def\CComega{\mathrm{CC}^\omega}
\setlength{\oddsidemargin}{0in} % so, left margin is 1in
\setlength{\textwidth}{6.27in} % so, right margin is 1in
\setlength{\topmargin}{0in} % so, top margin is 1in
\setlength{\headheight}{0in}
\setlength{\headsep}{0in}
\setlength{\textheight}{9.19in} % so, foot margin is 1.5in
\setlength{\footskip}{.8in}

% Definition of \placetitle
% Want to do an alternative which takes arguments
% for the names, authors etc.

\newcommand{\natrec}{\mathsf{natrec}}
%\rightfooter{}
\newcommand{\set}[1]{\{#1\}}
\newcommand{\sct}[1]{[\![#1]\!]}
%\usepackage{diagrams}
\usepackage{color}
\newcommand\coloremph[2][red]{\textcolor{#1}{\emph{#2}}}
\newcommand\norm[1]{\left\lVert #1 \right\rVert}
\newcommand\greenemph[2][green]{\textcolor{#1}{\emph{#2}}}
\newcommand{\EMP}[1]{\emph{\textcolor{red}{#1}}}




\begin{document}

\title{Dependent Type Theory with a Cumulative Hierarchy of Universes}

\author{}
\date{}
\maketitle

\begin{abstract}
\end{abstract}

\section{Introduction}\label{sec:intros}

  We present two type systems with a cumulative hiearchy of universes, one annotated Tarski style version and
  one Russell style version. There is a natural forgetful map from the Tarski version to the Russell version. Modulo some (known)
  meta-properties of both system, and using the technique presented in \cite{Streicher:semtt},
  we show that any Russell judgement can be lifted in a unique way to a Tarski judgement.

  We think our statements are ready for formalisation, since we try to be precise about the treatment of variables and substitions.


\section{Annotated type theory $T_1$ with Tarski style universes}

We now present the system $T_1$ with Tarski style universes. 

The syntax for types is
$$
A,B~::=~\Pi~{A}~B~|~\UU_n~|~\T_n a
$$
and the syntax for terms, with $v_i$ de Bruijn index
$$
a,b~::=~v_i~|~\APP(A,B,c,a)~|~\LAM(A,B,b)~|~\sT^k_n a~|~\UU^k_n~|~\Pi^n~a~b
$$

\medskip

Renaming and substitutions are now \EMP{defined} operations. 
$$
r~::=~\pp~|~r^+~~~~~~~~~\sigma~::=~[a]~|~\sigma^+~~~~~~~~~~\alpha~::=~r~|~\sigma
$$
and we define first the renaming operations $ar,~Ar$ by induction on $a,A$ and then the substitution operations
$a\sigma,~A\sigma$ by induction on $a,A$. The actions on variables are
$$
v_i\pp = v_{i+1}~~~~~~v_0[a] = a~~~~~~~~~~v_0\alpha^+ = v_0~~~~~~v_{i+1}\alpha^+ = (v_i\alpha)\pp
$$
We can then define $A\alpha$ (resp. $a\alpha$) by induction on $A$ (resp. $a$).

For instance $(\Pi~A~B)\alpha = \Pi~(A\alpha)~(B\alpha^+)$
and $\APP(A,B,c,a)\alpha = \APP(A\alpha,B\alpha^+,c\alpha,a\alpha)$.

\begin{lemma}
  We have $(B[a])\alpha = B\alpha^+[a\alpha]$ and $(b[a])\alpha = b\alpha^+[a\alpha]$.
\end{lemma}

\medskip
\footnote{Explain context?}
The type system describes how to derive judgements of the form $\Gamma\vdash J$ where $J$ is of the form
$A~\type$ or $a:A$ or $A=B$ or $a=b : A$.

\medskip

$$
\frac{\Gamma\vdash A}{\Gamma.A\vdash}~~~~~~\frac{}{()\vdash}~~~~~~~
\frac{\Gamma\vdash A}{\Gamma.A\vdash v_0:A\pp}~~~~~~~
\frac{\Gamma\vdash v_i:A}{\Gamma.B\vdash v_{i+1}:A\pp}~~~~~~~
$$
The judgment $\Gamma\vdash A$ expresses that $A$ is a type in context $\Gamma$.
We may write it $A~\type~(\Gamma)$ and may omit the global context $\Gamma$.
$$
\frac{A~\type~~~~~~B~\type~(A)}{\Pi~A~B~\type}~~~~~~~~~
\frac{b:B~(A)}{\LAM(A,B,b):\Pi~A~B}~~~~~~~~
\frac{c:\Pi~A~B~~~~~~a:A}
     {\APP(A,B,c,a):B[a]}
$$

We write $\conv$ for definitional equality (or conversion).
The main conversion rules are\footnote{We omit systematically {\em congruence} rules
  such as the rule that $c~a~\conv~c'~a'$ follows from $c~\conv~c'$ and $a~\conv~a'$.}
  \footnote{Also symmetry and transitivity of equality.}
$$
\frac{ a:A~~~~~~ A~ \conv~ B}{ a:B}~~~~~~~~~
\frac{ a ~\conv~a':A~~~~~~ A  ~\conv~ B}{ a ~\conv~a':B}
$$
$$
\frac{b:B~(A)~~~~~~~~ a:A}{ \APP(A,B,\LAM(A,B,b),a)  ~\conv~ b[a]:B[a]}
~~~~~~~
\frac{f\pp~v_0 = g\pp~v_0:B~(A)}{ f = g : \Pi~A~B}$$

\footnote{$\T_{n}A$ instead of $\T_{n}(A)$ etc. It's a mixture of applicative notation $fa$ and mathematical notation $f(a)$.}
$$
\UU_n~\type~~~~~~
\frac{A:\UU_{n}}{\T_{n}(A)~\type}~~~~~~
\frac{}{\UU^{k}_l:\UU_{k}}k<l~~~~~~~~~\frac{}{\T_{k}({\UU^{k}_l}) = \UU_{l}}k<l
~~~~~~~~~\frac{A:\UU_{n}}
{\sT_{n}^{m}(A):\UU_{m}}
n\leqslant m
$$
We require for $n\leqslant m$
\[
\T_m(\sT_{n}^{m}(A)) = \T_{n}(A) 
\]
We add for $m\leqslant n\leqslant p$
$$
\sT_{n}^n(a) = a \quad\text{and}\quad \sT_{n}^p\sT_{m}^n = \sT_m^p.
$$
$$
\frac{a:\UU_{n}~~~~~~b:\UU_n~(\T_{n}(a))}
     {\Pi^{n} a b:\UU_{n}}~~~~~~~~~
$$
with conversion rules
$$
\T_{n}~(\Pi^{n} a b) = \Pi~{\T_{n}(a)}~{\T_{n}(b)}~~~~~~~
$$
and
$$
\sT_{n}^{m}~(\Pi^{n} a b) = \Pi^{m} \sT_{n}^{m}(a) \sT_{n}^{m}(b)~~~~~~
$$

 We define $\alpha:\Delta\rightarrow\Gamma$ by induction on $\alpha$
 \footnote{inductively?}.
We have $\pp:\Gamma.A\rightarrow\Gamma$ and $[a]:\Gamma\rightarrow \Gamma.A$\footnote{if $\Gamma \vdash a : A$} and
$\alpha^+:\Delta.A\alpha\rightarrow\Gamma.A$ if $\alpha:\Delta\rightarrow\Gamma$.

\begin{lemma}
  The following rule is admissible: if $\Gamma\vdash J$ and $\alpha:\Delta\rightarrow\Gamma$
  then $\Delta\vdash J\alpha$.
\end{lemma}

Assuming $\Pi$ one-to-one, we can then show closure under $\beta$-reduction.

\begin{lemma}
  If $\APP(A,B,\LAM(A',B',b),a):T$ then $b[a]:T$.
\end{lemma}

\begin{proof}
  Using $\Pi$ one-to-one, we get $A=A'$ and $B=B'~(A)$\footnote{I don't understand this argument. Where is the $\Pi$? Are we assuming $b : B' (A')$ so $\LAM(A',B',b) : \Pi A' B' = \Pi A B$?}. We then have $b:B~(A)$ and $T = B[a]$.
  We deduce $b[a]:B[a]$ by substitution and $b[a]:T$ by the type equality rule.
\end{proof}

\section{Rules for a type theory $T_2$ with Russell style universes}

The syntax is now for types and terms
$$
A,B,a,b~::=~v_i~|~c~a~|~\lambda b~|~\Pi~{A}~B~|~\UU_n
$$

$$
\frac{A~\type~~~~~~B~\type~(A)}{\Pi~A~B~\type}~~~~~~~~~
\frac{b:B~(A)}{\lambda b:\Pi~A~B}~~~~~~~~
\frac{c:\Pi~A~B~~~~~~a:A}
     {c~a:B[a]}
$$

$$
\UU_n~\type~~~~~~
\frac{A:\UU_{n}}{A~\type}
~~~~~~\frac{A:\UU_{n}}{A:\UU_{n+1}}
~~~~~~~
{\UU_l}:\UU_{l+1}
$$
$$
\frac{A:\UU_{n}~~~~~~B:\UU_n(A)}
     {\Pi~A~B:\UU_{n}}$$

The main conversion rules are
$$
\frac{ a:A~~~~~~ A~ \conv~ B}{ a:B}~~~~~~~~~
\frac{ a ~\conv~a':A~~~~~~ A  ~\conv~ B}{ a ~\conv~a':B}
$$
$$
\frac{b:B~(A)~~~~~~~~ a:A}{ (\lambda b)~a  ~\conv~ b[a]:B[a]}
~~~~~~~
\frac{f\pp~v_0 ~\conv~ g\pp~v_0:B~(A)}{ f ~\conv~ g : \Pi~A~B}
$$



\medskip

     Note that the system $T_2$ is \EMP{not} closed under $\eta$-reduction. For instance, in a context
     $$x:\UU_1\rightarrow\UU_1$$
     we have $\lambda_yx~y:\UU_0\rightarrow\UU_2$ but we do \EMP{not} have
     $x:\UU_0\rightarrow\UU_2$.


\medskip

We have an operation which removes the annotation from $T_1$ to $T_2$.
$$
|\UU_n| = \UU_n~~~~~~~~|\Pi~A~B| = \Pi~|A|~|B|~~~~~~~~~|\T_n(a)| = |a|
$$
$$
|\APP(A,B,c,a)| = |c|~|a|~~~~~~~~~|v_i| = v_i~~~~~~~~~|\LAM(A,B,b)| = |b|~~~~~~~~|\sT_k^n(a)| = |a|
$$
The following remark has a direct proof.

\begin{lemma}
  If $\Gamma\vdash J$ in $T_1$ then $|\Gamma|\vdash |J|$ in $T_2$.
\end{lemma}


\section{Equivalence between $T_1$ and $T_2$}

We can describe neutral and $\beta$-normal forms for the non annotated terms.
$$
k~::=~v_i~|~k~n~~~~~~~~~~~n~::=~\UU_k~|~\Pi~n~n~|~\lambda n~|~k
$$


\begin{lemma}
  If $\Gamma\vdash t:A$ and $\Gamma\vdash u:A$ and $|t| = |u|$ is $\beta$-normal then $\Gamma\vdash t = u:A$.
  If $\Gamma\vdash t:A$ and $\Gamma\vdash u:B$ and $|t| = |u|$ is neutral then
  \begin{itemize}
    \item either $A = B$
      and $\Gamma\vdash t = u:A$
    \item or $A = \UU_k$ and $B = \UU_l$ and $\sT_k^m(t) = \sT_l^m(u):\UU_m$ with $m = max(k,l)$.
  \end{itemize}
\end{lemma}

\begin{theorem}\label{equivalence-tarski-russell}
  If $\Gamma\vdash J$ in the system $T_2$ then there exists a unique, up to conversion, judgement $\Gamma_1\vdash J_1$
  in the system $T_1$ such that $|\Gamma_1| = \Gamma$ and $|J_1| = J$\footnote{syntactic equalities?}.
\end{theorem}

\section{Addition of data types}

The proof extends to the corresponding systems with data types.\footnote{We add only the rules of list-formation, list-introduction, list-elimination, list-equality, and closure of all universes under lists, e g $A : \UU_l$ implies $\List~A : \UU_l$, just like any other type-former.}

For the case of the list types $\List~A$ with constructors $\Nil$ and $\Cons$, we can define in the Russell system
$$
f~:~\List~\UU_0~\rightarrow~\List~\UU_1~~~~~~~~~~~
f~\Nil = \Nil~~~~~~f~(\Cons~x~xs) = \Cons~x~(f~xs)
$$
which can be lifted to in the Tarski version
$$
f~:~\List~\UU_0~\rightarrow~\List~\UU_1~~~~~~~~~~~
f~\Nil = \Nil~~~~~~f~(\Cons~x~xs) = \Cons~(\sT_0^1 x)~(f~xs)
$$
One should not expect however to have $xs:\List~\UU_1~(xs:\List~\UU_0)$\footnote{for either system}.

\section{Initial models}

It is possible to present type theory with a cumulative hierarchy of universes as a generalised algebraic theory.
The system $T_1$ can be used to present the initial/term model of this theory, following the method presented in \cite{Streicher:semtt}
(which was formalised in the work \cite{brunerie:initiality}).
Our result shows that Russell's system $T_2$ can be used as well to present this initial/term model.
\footnote{Let me try to spell out some of the details here. The operations of the Russell-term model need to be defined. For example, application $\APP(\Gamma,A,B,c,a)$ where $\Gamma,A,B,c,a$ are Russell-terms is defined as $c~a$. One then checks that the equations of the gat are satisfied. To provide the initial map to the Tarski model, we use Theorem \ref{equivalence-tarski-russell} which e g maps a Russell judgment $\Gamma \vdash a : A$ to a Tarski judgment $\Gamma_1 \vdash a_1 : A_1$ such that 
$| \Gamma_1 | = \Gamma, | a_1 | = a, | A_1 | = A$. 
We then need to check that all operations of the generalised algebraic theory are preserved, e g
that $\Gamma_1 \vdash (A~\sigma)_1 = A_1~\sigma_1$.
}



\bibliographystyle{plain}
\bibliography{refs}

\end{document}
