\documentclass{lmcs}
%\usepackage{etex}
\usepackage[utf8]{inputenc}

\usepackage{color}
%\newcommand{\FYI}[1]{{\color{red}#1}}
\newcommand{\FYI}[1]{{#1}}
\usepackage{hyperref}
\usepackage{float}
\usepackage{amsmath}
\usepackage{amsfonts}
\usepackage{amsthm}
\usepackage{amssymb}
\usepackage{proof}
\usepackage{mathpartir}
\usepackage{mathrsfs}
\usepackage{stmaryrd}
\usepackage{cmll}
\usepackage{enumerate}
\usepackage{url}

%usepackage{graphicx}
%\usepackage[all]{xy}
\usepackage{listings}
%\usepackage{todonotes}
%\DeclareMathOperator{\Ker}{Ker}
%\DeclareMathOperator{\nf}{nf}
%\DeclareMathOperator{\domain}{dom}
%\DeclareMathOperator{\codomain}{cod}
%\DeclareMathOperator{\cod}{cod}
%\DeclareMathOperator{\dom}{dom}
%\DeclareMathOperator{\ctxof}{ctx-of}
%\DeclareMathOperator{\typeof}{type-of}
%\DeclareMathOperator{\fix}{fix}

%\newcommand{\vdashS}{\ \vdash\ }
%\newcommand{\vdashS}{\vdash}
\newcommand {\emptyContext}{1}
\newcommand {\emptyContextI}{\diamond}
\newcommand {\emptyContextS}{\textbf 1}
\newcommand {\contextExtension}[2]{#1 \cdot #2}
\newcommand {\contextExtensionI}[2]{#1 \cdot #2}
\newcommand {\contextExtensionS}[2]{#1 \cdot #2}
\newcommand {\contextExtensionC}[2]{#1 \cdot_\C #2}

\newcommand {\GammaA}{\contextExtension \Gamma A}
\newcommand {\DeltaA}{\contextExtension \Delta A}
\newcommand {\setI}{\text{set}}
\newcommand {\setS}{\textbf{set}}
\newcommand {\depProd}[3]{\Pi(#1, #2, #3)}
\newcommand {\depProdI}[2]{\Pi(#1, #2)}
\newcommand {\depProdS}{\textbf{$\Pi$}}
\newcommand {\el}[2]{{\tt el}(#1, #2)}
\newcommand {\elI}[1]{{\tt el}(#1)}
\newcommand {\elS}{\textbf{el}}
\newcommand {\subType}[4]{{\tt subType}(#3, #4, #1, #2)}
\newcommand {\subTypeI}[2]{\text{subType}(#1, #2)}
\newcommand {\subTypeS}[2]{#1\{#2\}}
\newcommand{\subTypeC}[4]{\mathrm{subType}_\C(#3, #4, #1, #2)}
\newcommand {\q}[2]{{\tt q}_{#1, #2}}
\newcommand {\qI}{{\tt q}}
\newcommand {\qS}{\textbf{q}}
\newcommand{\lambdaAbs}[4]{\lambda(#1, #2, #3, #4)}
\newcommand{\lambdaAbsI}[1]{\lambda(#1)}
\newcommand{\lambdaAbsS}{\textbf{$\lambda$}}
\newcommand{\application}[5]{{\tt app}(#1, #2, #3, #4, #5)}
\newcommand{\applicationI}[2]{\text{app}(#1, #2)}
\newcommand{\applicationS}{\textbf{application}}
\newcommand{\subTerm}[5]{{\tt subTerm}(#4, #5, #1,#2,#3)}
\newcommand{\subTermI}[2]{\text{subTerm}(#1,#2)}
\newcommand{\subTermS}[2]{#1\{#2\}}
\newcommand{\idSub}[1]{{\tt id}(#1)}
\newcommand{\idSubI}{{\tt id}}
\newcommand{\idSubS}{\text{id}}
\newcommand{\proj}[2]{{\tt p}(#1, #2)}
\newcommand{\projI}{{\tt p}}
\newcommand{\projS}{\textbf{p}}
\newcommand{\comp}[5]{{\tt comp}(#1, #2, #3, #4, #5)}
\newcommand{\compI}[2]{{\tt comp}(#1, #2)}
\newcommand{\compS}[2]{#2 \circ #1}
\newcommand{\emptySub}[1]{\emptySubI_{#1}}
\newcommand{\emptySubI}{\langle\rangle}
\newcommand{\emptySubS}{\textbf !}
\newcommand{\extSub}[5]{\text{extension}(#1, #2, #3, #4, #5)}
\newcommand{\extSubI}[2]{\text{extension}(#1, #2)}
\newcommand{\extSubS}[2]{\langle #1, #2\rangle}
\newcommand{\Ctx}{\mathrm{Ctx}}
\newcommand{\Sub}{\mathrm{Sub}}
\newcommand{\Ty}{\mathrm{Ty}}
\newcommand{\Tm}{\mathrm{Tm}}
\newcommand{\C}{{\mathcal C}}
\newcommand{\I}{{\mathcal I}}
\newcommand{\T}{{\mathcal T}}

\newcommand{\Timp}{\T_{\text{imp}}}
\newcommand{\arrow}{{\rightarrow}}
\newcommand{\RawCtx}{{\tt Ctx}}
\newcommand{\RawSub}{{\tt Sub}}
\newcommand{\RawTy}{{\tt Ty}}
\newcommand{\RawTm}{{\tt Tm}}

\newcommand{\scomp}[6]{\mathrm{comp}_#1(#2, #3, #4, #5,#6)}

\newcommand{\inte}[1]{\llbracket #1 \rrbracket}
\newcommand{\intCtx}[1]{\llbracket #1 \rrbracket}
\newcommand{\intSub}[3]{\llbracket #3 \rrbracket_{#1,#2}}
\newcommand{\intTy}[2]{\llbracket #2 \rrbracket_#1}
\newcommand{\intTm}[3]{\llbracket #3 \rrbracket_{#1,#2}}
\newcommand{\ICtx}{{\I_0}}
\newcommand{\ISub}{{\I_1}}
\newcommand{\ITy}{{\I_2}}
\newcommand{\ITm}{{\I_3}}
\newcommand{\iniCtx}[1]{\overline{\llbracket #1 \rrbracket}}
\newcommand{\iniSub}[3]{\overline{\llbracket #3 \rrbracket}_{#1,#2}}
\newcommand{\iniTy}[2]{\overline{\llbracket #2 \rrbracket}_{#1}}
\newcommand{\iniTm}[3]{\overline{\llbracket #3 \rrbracket}_{#1,#2}}

\newcommand{\mejl}[3]{#1$\bigcirc\!\!\!\!\!\alpha\,$#2${}_{\cdot}$#3}

\newcommand{\bbN}[0]{{\mathbb N}}
\newcommand{\bbZ}[0]{{\mathbb Z}}
\newcommand{\bbQ}[0]{{\mathbb Q}}
\newcommand{\bbR}[0]{{\mathbb R}}
\newcommand{\bbB}[0]{{\mathbb B}}
\newcommand{\mU}[0]{{\mathcal U}}
\newcommand{\mT}[0]{{\mathcal T}}
\newcommand{\ve}[0]{{\varepsilon}}
\newcommand{\vf}[0]{{\varphi}}

\newcommand{\wellincluded}[0]{\, \Subset \,}

\newcommand{\memof}[0]{\, \epsilon \,}
\newcommand{\subseteqof}[0]{\, \dot{\subseteq} \,}

\newcommand{\mono}[0]{\to/ >->/}
\newcommand{\pto}[0]{\rightharpoondown}
\newcommand{\wellcov}[0]{{\lll}}
\newcommand{\waybelow}[0]{\ll}
\newcommand{\formint}[0]{\land}
\newcommand{\cov}[0]{{\, \lhd \,}}
\newcommand{\kov}[0]{{\, \lessdot \,}}
\newcommand{\kkov}[0]{{\, <: \,}}
\newcommand{\mutcov}[0]{\sim}
\newcommand{\balcov}[0]{\sqsubseteq}
\newcommand{\bal}[0]{{\sf b}}
\newcommand{\sat}[1]{{\rm Sat}(#1)}
\newcommand{\set}[0]{{\rm Set}}
\newcommand{\Set}[0]{{\bf Set}}
\newcommand{\true}[0]{{\sf T}}
\newcommand{\monus}{\stackrel{{}^{\scriptstyle .}}{\smash{-}}}

\newcommand{\refl}[0]{{\rm ref}}

\newcommand{\inl}[1]{{\sf inl}(#1)}
\newcommand{\inr}[1]{{\sf inr}(#1)}
\newcommand{\nat}[0]{{\mathbb N}}

\newcommand{\nattype}[0]{{\rm N}}
\newcommand{\bool}[0]{{\rm Bool}}
\newcommand{\ext}[1]{\langle #1 \rangle}


\newcommand{\bintree}[0]{{\rm T}_2}

%\newcommand{\sequent}[0]{\vdash}


\renewcommand{\conv}[0]{\approx}
\newcommand{\intimpl}[0]{\supset}

\newcommand{\omitthis}[1]{}

\newcommand{\changenote}[1]{}


 \newcommand{\Id}[0]{{\rm I}}
 

\newcommand{\longtext}[1]{}
\newcommand{\shorttext}[1]{}
\newcommand{\commentaway}[1]{}

\newcommand{\Setoid}[0]{{\bf Setoid}}

\definecolor{Red}{rgb}{1,0,0}
\newcommand{\red}[1]{{\color{Red}#1}}
%\newcommand{\red}[1]{}
\renewcommand{\bar}[1]{\overline{#1}}

%\newdir{pb}{:(1,-1)@^{|-}}
%\def\pb#1{\save[]+<16 pt,0 pt>:a(#1)\ar@{pb{}}[]\restore}

\newcommand{\Fam}{\textbf{Fam}}
\newcommand{\nilc}{1}
\newcommand{\cext}{.}
\newcommand{\indexed}[1]{\boldsymbol{#1}}
\newcommand{\Cat}{\mathrm{Cat}}
\newcommand{\op}{\text{op}}
\newcommand{\iso}{\cong}
\newcommand{\subst}[1]{\langle #1 \rangle}
\newcommand{\applyopen}[2]{\{ #1 \}  #2 }

% added by Marc to get things going. IMPROVE!

\def\N{\mathsf{N}}
\def\U{\mathsf{U}}
\def\F{\mathsf{F}}
\def\app{\mathsf{app}}
\def\Cop{\C^\op}
\def\Cobj{{\mathcal{C}_0}}
\def\p{\mathrm{p}}
\def\q{\mathrm{q}}
\newcommand{\tuple}[1]{\langle #1 \rangle}

\newtheorem{remark}{Remark}
\newtheorem{definition}{Definition}

%\def\N{\mathrm{N}}
\def\U{\mathrm{U}}
\def\p{{\tt p}}
\def\ev{{\tt ev}}
\def\q0{{\tt q}}
\def\r{{\tt r}}
\def\arrow{\rightarrow}
\def\Hom{\mathrm{Hom}}
\def\GammaA{\Gamma_{+,\times}}
\def\GammaCL{\Gamma_{\mathrm{CL}}}

\def\Dp{\mathrm{D}_p}
\def\notnotDp{\neg\neg\Dp}
\def\F{\mathcal{F}}
\def\HA{\mathbf{HA}}
\def\PA{\mathbf{PA}}
\def\I{\mathrm{I}}
\def\refl{\mathrm{r}}
\def\id{{\tt id}}
\def\idT{\mathrm{id}_\T}
\def\idC{\mathrm{id}_\C}
\newcommand{\pair}{\mathrm{pair}}
\newcommand{\fst}{\mathrm{fst}}
\newcommand{\interp}[1]{ \overline{\llbracket #1 \rrbracket}}
\newcommand{\Cwf}{\textbf{CwF}}
\newcommand{\Cwfs}{\Cwf_s}
\newcommand{\D}{\mathcal{D}}
\newcommand{\snd}{\mathrm{snd}}
\newcommand{\ap}{\mathrm{app}}
%\newcommand{\app}{\mathrm{app}}
\newcommand{\ini}[1]{\iniCtx{[#1]}}
\DeclareMathOperator{\cod}{cod}
\DeclareMathOperator{\dom}{dom}
\DeclareMathOperator{\ctxof}{ctx-of}
\DeclareMathOperator{\typeof}{type-of}
\newcommand{\vdashS}{\ \vdash\ }
\DeclareMathOperator{\domain}{dom}
\DeclareMathOperator{\codomain}{cod}


\newcommand{\isoCtx}[1]{\stackrel{#1}{\cong}}
\newcommand{\isoTy}[2]{\stackrel{#1}{\cong}_{#2}}
\newcommand{\equSub}[1]{=_{#1}}
\newcommand{\equTm}[2]{=_{#1,#2}}
\newcommand{\TT}{\mathbf{T}}

\newtheorem{theorem}{Theorem}
\newcommand{\s}{\mathrm{s}}
\newcommand{\Rec}{\mathrm{R}}
\newcommand{\Ta}{\mathrm{T}}
\newcommand{\ta}{\mathrm{t}}
\newcommand{\Ru}{\mathcal{R}}
\newcommand{\Nhat}{\hat{\N}}
\newcommand{\Pihat}{\hat{\Pi}}
\newcommand{\Tan}{\Ta_n}
\newcommand{\Un}{\U_n}
\newcommand{\Nhatn}{\N^n}
\newcommand{\Pihatn}{\Pi^n}
\newcommand{\Nn}{\Nhatn}
\newcommand{\Pin}{\Pihatn}
\newcommand{\TRu}{\Ta_\Ru}
\newcommand{\URu}{\U_\Ru}
\newcommand{\NRu}{\N_\Ru}
\newcommand{\PiRu}{\Pi_\Ru}
\newcommand{\TRun}{{(\Ta_\Ru)}_n}
\newcommand{\URun}{{(\U_\Ru)}_n}
\newcommand{\NRun}{{(\N_\Ru)}^n}
\newcommand{\PiRun}{{(\Pi_\Ru)}^n}
\newcommand{\TRum}{{(\Ta_\Ru)}_m}
\newcommand{\URum}{{(\\U_\Ru)}_m}
\newcommand{\TC}{\Ta_\C}
\newcommand{\UC}{\U_\C}
\newcommand{\NC}{\N_\C}
\newcommand{\PiC}{\Pi_\C}
\newcommand{\Level}{\mathrm{Level}}
\def\Sort{\mathcal{S}}
\def\Op{\mathcal{O}}
\def\Eq{\mathcal{E}}
\def\D{\mathcal{D}}
\def\V{\mathrm{V}}
\def\Cwf{\mathbf{CwF}}
\def\Obj{\mathrm{obj}}
\def\Ctx{\mathrm{Ctx}}
\def\Hom{\mathrm{hom}}
\def\id{\mathrm{id}}
\def\Mon{\mathrm{M}}
\def\idmon{\mathrm{e}}
\def\comp{\mathrm{*}}
\newcommand{\ctx}{\mathrm{ctx}}
\newcommand{\sub}{\mathrm{sub}}
\newcommand{\ty}{\mathrm{ty}}
\newcommand{\tm}{\mathrm{tm}}
%\newcommand{\hom}{\mathrm{hom}}
\def\nt{\mathrm{nat}}
\def\fun{\mathrm{fun}}


\title[Generalized Algebraic Theories and Categories with Families]{On Generalized Algebraic Theories\\and Categories with Families}\author{Marc Bezem, Thierry Coquand, Peter Dybjer, Mart\'in Escard\'o}

\begin{document}

\maketitle

\begin{abstract}
We give a syntax independent formulation of finitely presented generalized algebraic theories as initial objects in categories of categories with families (cwfs) with extra structure.
%Such a generalized algebraic theory is presented by a \FYI{presentation} consisting of a finite list of sort symbols, operator symbols, and equations.
To this end we simultaneously define the notion of a \FYI{\em{presentation}} $\Sigma$ of a generalized algebraic theory and the associated category $\Cwf_\Sigma$ of small cwfs with a $\Sigma$-structure and cwf-morphisms that preserve $\Sigma$-structure on the nose.  Our definition refers to the purely semantic notion of {\em uniform family} of contexts, types, and terms in $\Cwf_\Sigma$. Furthermore, we show how to syntactically construct an initial cwf with a $\Sigma$-structure. This result can be viewed as a generalization of Birkhoff's completeness theorem for equational logic. It is obtained by extending Castellan, Clairambault, and Dybjer's construction of an initial cwf. We provide examples of generalized algebraic theories for monoids, categories, categories with families, and categories with families with extra structure for some type formers of Martin-Löf type theory. The models of these are internal monoids, internal categories, and internal categories with families (with extra structure) in a small category with families. Finally, we show how to extend our definition to some generalized algebraic theories that are not finitely presented, such as the theory of contextual cwfs.
\end{abstract}

\section{Introduction}

Martin-Löf type theory can be characterized in a syntax independent way as the initial category with families (cwf)  with extra structure for the type formers \cite{castellan:tlca2015,castellan:lmcs}. The main contribution of this note is a similar syntax independent characterization of finitely presented generalized algebraic theories as initial cwfs with extra structure.

Generalized algebraic theories were introduced by Cartmell in his PhD thesis \cite{cartmell:phd} as a dependently typed generalization of many sorted algebraic theories. Each generalized algebraic theory is presented by (possibly infinite) sets of sort symbols, operator symbols, and equations. Cartmell's definition of generalized algebraic theories \cite{cartmell:phd,cartmell:apal} is based on a notion of {\em derived rule} expressed in terms of a traditional syntactic system for dependent type theory. He also defines a notion of model whereby sort symbols are interpreted as families of sets.

Categories with families (cwfs) \cite{dybjer:torino} were introduced as a new notion of model of dependent type theory. Cwfs arise by reformulating the notion of category with attributes in Martin Hofmann's sense \cite{hofmann:csl}. The key point is that cwfs arise as models of a certain generalized algebraic theory closely related to Martin-Löf's substitution calculus \cite{martinlof:gbg92}. As such the notion of cwf becomes a useful intermediary between traditional syntactic systems for dependent type theory and a variety of categorical notions of model.

The generalized algebraic theory of cwfs can be seen as a kind of idealized formal system of dependent type theory. In contrast to Martin-Löf's substitution calculus and other syntactic systems for dependent type theory, it is {\em not} formulated in terms of grammars and inference rules for the forms of judgment of type theory. Instead it is formulated in terms of the sort symbols (corresponding to the judgment forms), operator symbols (corresponding to the formation, introduction, and elimination rules), and equations (corresponding to the equality rules for the type formers) of the generalized algebraic theory. Some of the general reasoning (about equality, substitution, and assumptions) is taken care of by the underlying infrastructure of dependent types. This makes it possible to abstract away from details in the formulation of grammars and inference rules.
%In contrast to the various syntactic systems, the generalized algebraic theory of cwfs has a canonical flavour.
%We may define dependent type theory in a syntax independent way as the initial object in a category of cwfs with extra structure for interpreting the type formers.
%However, the reader may now object that this looks like a circular definition.
%We learn what dependent type theory is if we already know what dependent type theory is!

In this note we explore the interdependence between generalized algebraic theories and cwfs. We already explained that cwfs can be defined as models of a generalized algebraic theory.
%(This generalized algebraic theory is presented in section \ref{gat-cwf}.)
In the other direction, the notion of generalized algebraic theory relies on the notion of cwf, in the sense that the latter models the underlying infrastructure of dependent types.

%We shall here define a new finitely presented notion of generalized algebraic theory and simultaneously a general categorical notion of model. To this end we define what it means to be a \FYI{presentation} $\Sigma$ of a generalized algebraic theory and the associated category $\Cwf_\Sigma$ with extra structure for $\Sigma$. Our definition refers to {\em uniform families} of contexts, types, and terms in $\Cwf_\Sigma$, a purely semantic notion. Afterwards, we construct initial objects $\T_\Sigma \in \Cwf_\Sigma$ by extending Castellan, Clairambault, and Dybjer's  construction of an initial object in the category $\Cwf$ of cwfs \cite{castellan:tlca2015,castellan:lmcs}.

%Once one has the appropriate definition of generalized algebraic theories in terms of cwfs, the details become natural and there are no surprises. The definition becomes canonical once we accept the abstract definition of dependent type theories as initial cwfs with extra structure.

\subsection*{Plan of the paper}

In Section 2 we recall the definition of the category $\Cwf$ of small cwfs and morphisms preserving cwf-structure on the nose. Section 3 contains our main definition of a syntax independent notion of \FYI{presentation} $\Sigma$ of a generalized algebraic theory and the category $\Cwf_\Sigma$ of small cwfs with a $\Sigma$-structure. In Section 4 we construct an initial object $\T_\Sigma$ in $\Cwf_\Sigma$. In Section 5 we show several examples of generalized algebraic theories: for monoids, categories, cwfs, and cwfs with extra structure for $\Pi$-types, a type of natural numbers, and a universe. We point out that small cwfs with extra structure for generalized algebraic theories of monoids, categories, and cwfs have an internal monoid, internal category, and internal cwf, respectively. We also sketch how to extend our approach to some countably presented generalized algebraic theories, and show the example of contextual cwfs, a variant of Cartmell's contextual categories \cite{cartmell:phd,cartmell:apal}. Finally, in Section 6 we discuss related work with connections to Voevodsky's initiality conjecture \cite{voevodsky:initiality} and to Altenkirch and Kaposi's quotient inductive-inductive types \cite{altenkirch:qiits}.

Our development can be formulated in a constructive set theory,
as described for instance by Aczel \cite{MR519801,aczel:relate}, although the set theory
we use for formulating the notion of cwf with a $\Sigma$-structure is probably
much weaker. In the general theory we need to distinguish between small and large sets, and hence we assume that our set theory comes with a Grothendieck universe $\V$ of {\em small sets}. Furthermore, in Section \ref{sec:examples} we provide an example of a small cwf with an internal cwf. For this purpose we need two more Grothendieck universes $\V'$ and $\V''$, where $\V''  \in \V' \in \V$.
%provide the example of the small cwf $\Set'$ of tiny sets with an internal cwf of very tiny sets. This relies on our set theory having two more Grothendieck universes: $\V'$ and $\V''$, where $\V''  \in \V' \in \V$.

As emphasized by Voevodsky~\cite{voevodsky:initiality}, we study structures invariant
under {\em isomorphisms} and not under {\em equivalences}, and it is actually misleading
to call them ``category'' (and this is why Voevodsky used the term ``$C$-system''
for what Cartmell called ``contextual category'').
As he also noticed, this
important distinction between categories and notions invariant under isomorphisms becomes
precise in the setting of univalent foundations where not all collections of objects
are constructed from sets.
%% We remark that we work in set-theoretic metalanguage throughout the paper. Everything we do is constructive and should be possible to formalize in Aczel's CZF \cite{MR519801}. We also remark that it would be interesting to work in type-theoretic metalanguage, for example, in homotopy type theory and Voevodsky's univalent foundations. As such it could contribute to the recent work by several researchers inspired by Voevodsky's initiality conjecture.

%\subsection*{To the memory of Martin Hofmann}
%The content of this paper  is closely related to several of Martin's
%important contributions to the semantics of dependent type theory.
%Martin was an extremely gifted and generous person,
%many researchers have benefited from his collaboration.
%He is truly missed.

\subsection*{Remarks on terminology and notation}
Like Cartmell, we have chosen to use the term {\em sort symbol} from many-sorted universal algebra. However, in our semantic notion of \FYI{presentation} sort symbols are interpreted by {\em type families} in a cwf. A cwf consists of a base category where the objects of the base category are (semantic) contexts and the morphisms are (semantic) substitutions. Moreover, we have a family-valued presheaf mapping contexts to families of (semantic) terms indexed by (semantic) types. Thus the reader should be aware of the mismatch between the word {\em sort} from universal algebra and the word {\em type} in the cwf semantics.

Another possible source of confusion is that cwfs appear on two different levels. In Section \ref{sec:def_cwf} we recall the definition of cwf in set-theoretic metalanguage, where we use $\Ty$ to denote the family of types indexed by contexts and $\Tm$ to denotpapere family of terms indexed by contexts and types. This notion of cwf is then used to define the semantic notions of \FYI{presentation} and category of models of a generalized algebraic theory. Then in Section \ref{gat-cwf} we define the generalized algebraic theory of {\em internal cwfs}. This generalized algebraic theory has sort symbols $\ty$ for {\em internal types} and $\tm$ for {\em internal terms} using lower case to highlight the difference from $\Ty$ and $\Tm$ in the model cwf.

Furthermore, we often use the same notation both on the semantic and the syntactic level. 
For example, in Section~\ref{gat-sig-mod}, where we are syntax
independent, we use the same letter $S$ as in
Section~\ref{initial-gat-section}, where we syntactically construct the initial model.

\subsection*{To the memory of Martin Hofmann}

We have written this paper to honour the memory of Martin Hofmann. The topic is categorical models of dependent type theory, an area that Martin made seminal contributions to. In particular, he did much to clarify the relationship between intensional and extensional type theory.
His thesis was the first investigation of the setoid model \cite{hofmann:phd}. His and Streicher's groupoid model \cite{hofmann:groupoid} refutes uniqueness of identity proofs and identity reflection, the two rules that separate extensional from intensional type theory. The groupoid model also validates the principle of universe extensionality, a special case of Voevodsky's univalence axiom. As a consequence this work is a forerunner to Voevodsky's univalent foundations.

Other notable contributions to dependent type theory include the interpretation of extensional type theory in locally cartesian closed categories \cite{hofmann:csl,curien-garner-hofmann}, the use of a presheaf model to prove that the Logical Framework version of Martin-Löf type theory is a conservative extension of the original version \cite{hofmann:cambridge}, and a method for eliminating extensional identity types \cite{hofmann:conservativity}. Martin also wrote a widely read introduction to the syntax and semantics of dependent types \cite{hofmann:cambridge}.

Martin was an extremely gifted and generous person,
many researchers have benefited from his collaboration.
He is truly missed.

\section{Categories with families}\label{sec:def_cwf}

\subsection{The category of cwfs and strict cwf-morphisms}

%In this section (whole paper?), the meta-language is set-theoretic.
%Much of the vocabulary is category-theoretic, but we freely use
%equality of objects in a categorical context.
%We first define the category $\Fam$, and then the category $\Cwf$.

\begin{definition}\label{def:catFam}
$\Fam$ is a category whose objects are
set-indexed families of sets $(U_x)_{x\in X}$.
A morphism of $\Fam$ with source $(U_x)_{x\in X}$ and target $(V_y)_{y\in Y}$
consists of a re-indexing function $f: X\to Y$ together with a family
$(g_x)_{x\in X}$ of functions $g_x : U_x \to V_{f(x)}$. %, for all $x\in X$.
\end{definition}

The next step is to define the category $\Cwf$.
We split this definition in two: first the objects,
which are called \emph{categories with families}, in Definition~\ref{def:Cwfobj},
and then the morphisms in Definition~\ref{def:Cwfmor}.
Since $\Cwf$ has been developed as a categorical framework for the semantics of
type theory, much of the terminology (contexts, substitutions,
types, terms) refers to the syntax of type theory,
suggesting the intended interpretation of this syntax in the
so-called $\Cwf$-semantics.

The main novelty of this paper is to use $\Cwf$ as a framework
for defining a notion of finitely presented generalized algebraic theory.
Contexts, substitutions, types, and terms also make
sense in relation to generalized algebraic theories.
% Note however that we use the term {\em ``sort symbol"} from universal algebra for what might have been called ``type symbol".

\begin{definition}\label{def:Cwfobj}
A category with families (cwf) consists of the following data:

\begin{itemize}
\item A category $\C$;

\item A $\Fam$-valued presheaf on $\C$, that is, a functor
$T : \Cop \to \Fam$;

\item A terminal object $1\in \C$, and unique maps
$\tuple{}_\Gamma \in \C(\Gamma, 1)$ for all objects $\Gamma$ of $\C$;

\item Operations ${\cext\,},~\tuple{\_,\_},~\p$ and $\q$
explained in the following paragraphs.
These four operations and their associated equations
are referred to as \emph{context comprehension}.
\end{itemize}

We let $\Gamma, \Delta,\ldots$ range over objects of $\C$,
and refer to them as \emph{contexts}.
We let $\delta, \gamma,\ldots$ range over morphisms,
and refer to them as \emph{substitutions}.
We refer to $1$ as the \emph{empty} context; the terminal maps
$\tuple{}_\Gamma$ represent the \emph{empty} substitutions.

If $T(\Gamma) = (U_x)_{x\in X}$, we write $\Ty(\Gamma)$ for the set $X$.
We call the elements of $\Ty(\Gamma)$ \emph{types in context $\Gamma$},
and let $A, B, C$ range over such types.
Furthermore, for $A \in \Ty(\Gamma)$, we write $\Tm(\Gamma, A)$ for the set $U_A$
and call the elements of $\Tm(\Gamma, A)$
\emph{terms of type $A$ in context $\Gamma$}.

For $\gamma : \Delta \to \Gamma$,
the functorial action of $T$ yields a morphism
\[
T(\gamma) \in  \Fam\left((\Tm(\Gamma, A))_{A\in \Ty(\Gamma)}, % \to
                (\Tm(\Delta, B))_{B\in \Ty(\Delta)}\right)
\]
consisting of a reindexing function $\_\,[\gamma] : \Ty(\Gamma) \to
\Ty(\Delta)$ referred to as \emph{substitution in types}, and for each $A\in
\Ty(\Gamma)$ a function $\_\,[\gamma] : \Tm(\Gamma, A) \to \Tm(\Delta,
A[\gamma])$, referred to as \emph{substitution in terms}.

Now we turn to the explanation of the operations
${\cext\,},~\tuple{\_,\_},~\p,~\q$.
Given $\Gamma \in \C$, $A \in \Ty(\Gamma)$, $\gamma : \Delta \to \Gamma$,
and $a\in \Tm(\Delta, A[\gamma])$, we have
\[
\Gamma \cext A \in \C
\quad\qquad
\p_{\Gamma, A} : \Gamma \cext A \to \Gamma
\quad\qquad
\q_{\Gamma, A} \in \Tm(\Gamma\cext A, A[\p_{\Gamma,A}])
\quad\qquad
\tuple{\gamma, a}_A : \Delta \to \Gamma \cext A.
\]
We call $\Gamma \cext A$ the \emph{extended} context
and $\tuple{\gamma, a}_A$ the \emph{extended} substitution.

The operations  ${\cext\,},~\tuple{\_,\_},~\p,~\q$
satisfy the following universal property:
$\tuple{\gamma, a}_A$ is the unique substitution satisfying
\[
\p_{\Gamma, A} \circ \tuple{\gamma, a}_A = \gamma
\qquad \text{and}\qquad
\q_{\Gamma, A} [\tuple{\gamma, a}_A] = a\,.
\]
We refer (colloquially) to $\p$ as the \emph{first projection},
and to $\q$ as the \emph{second projection}. %\footnote%
{Note that the first equation implies that
$\Tm(\Delta,A[\p_{\Gamma,A}][\tuple{\gamma, a}]) = \Tm(\Delta,A[\gamma])$
so that $\q_{\Gamma, A} [\tuple{\gamma, a}]$ and $a$ are elements of the same set.}
Here and below, subscripts are omitted from ${\cext\,},~\tuple{\_,\_},~\p,~\q$
when they can be reconstructed from the context (no pun intended).
(End Definition~\ref{def:Cwfobj}.)
\end{definition}

A cwf is thus a structure $(\C,1,\tuple{},T,\cext\, , \tuple{\_,\_},\p, \q)$,
subject to equations, for the category and the presheaf, and universal
properties, formulated purely equationally, for the terminal object and for context comprehension.
The morphisms to be defined next preserve this structure,
even in a strict way, `on the nose'.
We often shorten the notation of a cwf to $(\C,T)$, or even just $\C$,
leaving the remaining structure implicit.

\begin{definition}\label{def:Cwfmor}
A \emph{(strict) cwf-morphism $F$ between cwfs $(\C,T_\C)$ and $(\D,T_\D)$}
consists of

\begin{itemize}

\item A functor $F_\fun : \C \to \D$;
\item A natural transformation $F_\nt : T_\C \Rightarrow (T_\D \circ F_\fun^\op)$;
\item The terminal object is preserved on the nose: $F_\fun(1_{\C}) = 1_{\D}$;
\item Context comprehension is preserved on the nose, see below.
\end{itemize}

Since $F_\nt$ is a natural transformation between $\Fam$-valued presheaves,
$F_\nt$ has a component for any object $\Gamma$ of $\C$, and
these components are morphisms in $\Fam(T_C(\Gamma),T_\D(F_\fun(\Gamma)))$.
Recall that morphisms in $\Fam$ consist of a reindexing function
and a family of functions. It is convenient to denote $F_\fun$,
all reindexing functions, as well as all members of the families of functions,
simply by $F$. Thus we have $F(A) \in \Ty_\D(F(\Gamma))$
and $F(a) \in \Tm_\D(F(\Gamma), F(A))$, for all $\Gamma$
and $A\in\Ty_\C(\Gamma)$ and $a\in \Tm_\C(\Gamma, A)$.

Naturality of $F_\nt$
amounts to preservation of substitution, {i.e.}, for all
$\gamma : \Delta \to \Gamma$ in $\C$, we have
\[
F(A[\gamma]) = F(A)[F(\gamma)] \qquad \qquad
F(a[\gamma]) = F(a)[F(\gamma)]\,.
\]

Last but not least, we turn to the preservation of context comprehension
on the nose, and require
\[
F(\Gamma\cext A) = F(\Gamma)\cext F(A) \qquad
%F(\tuple{\gamma,a}) = \tuple{F(\gamma),F(a)} \qquad
F(\p_{\Gamma, A}) = \p_{F(\Gamma), F(A)} \qquad
F(\q_{\Gamma, A}) = \q_{F(\Gamma), F(A)}\,.
\]

Note that the universal property implies that
$F(\tuple{\gamma,a}) = \tuple{F(\gamma),F(a)}$.
The same is true for the terminal maps:
$F(\tuple{}_\Gamma) = \tuple{}_{F(\Gamma)}$.
(End Definition~\ref{def:Cwfmor}.)
\end{definition}

Small cwfs with strict cwfs-morphisms form a category, written $\Cwf$.
%\footnote{PD: we should be precise here. "Small cwfs are those with small sets of objects, morphisms, types, and terms, that is, these sets are members of $\V_1$. The category $\Cwf$ is large, that is, it has large sets of objects and morphisms, that is, these sets are members of $\V_2$.}

\section{\FYI{presentations} and models of generalized algebraic theories}\label{gat-sig-mod}

In universal algebra one has the notion of a \emph{signature},
which consists of a set of sort symbols and a set of typed operator symbols.
Using the vocabulary of the signature one then specifies the set
of equational axioms. 

For generalized algebraic theories the situation is more complicated.
First, sorts may depend on other sorts and even on operators,
so that sorts and operators cannot be presented as two separated sets.
Even more so, sorts and operators may depend on equations to
be well-typed, so that separation of these three syntactic
categories is not possible. We give an example of the interdependency 
in the next paragraph, more examples can be found in Section~\ref{sec:examples}.

For readability we give this example in the language of type theory.
Consider a sort $X$ with two operators $x_0 :X$ and $x_1 : X$
and an axiom $x_0 = x_1$, all in the empty context. Consider now a sort
$Y(x)$ in the context $x:X$, and a sort $Z(y,y')$ in the context
$y:Y(x_0), y':Y(x_1)$. Finally, consider a sort $W(y,z)$ in the
context $y:Y(x_0), z:Z(y,y)$. The sort $W(y,z)$ is only well-typed
since $x_0 = x_1$. The same is true for an operator $w(y,z): W(y,z)$
in the same context as $W(y,z)$. The equation $x_0 = x_1$ could come
after the sorts $Y(x)$ and $Z(y,y')$, but has to come before $W(y,z)$.

%We now come to the main point of this note.
For reasons mentioned above we need a more general notion
than that of signature in universal algebra, encompassing not only the types of the sort and operator symbols, but also the equations.  
We call this notion the \emph{presentation} of a generalized algebraic theory,
or presentation for short. It is non-trivial to define what a presentation
is and how it presents a generalized algebraic theory.
This requires several steps.

We first define the notion of a \FYI{presentation} $\Sigma$ and the associated category $\Cwf_{\Sigma}$ of cwfs with a $\Sigma$-structure.
Each object of $\Cwf_{\Sigma}$ is a cwf with extra structure and
each morphism is a cwf-morphism preserving $\Sigma$-structure.
%\footnote{PD: Add "We shall show that $\Cwf_\Sigma$ will be a large (but not very large) category for all \FYI{presentation}s $\Sigma$.}
For this definition, we will need the following auxiliary notions.

A {\em uniform family of contexts} is a family $\Gamma = (\Gamma_{\C})$ with $\Gamma_\C$ a context in
$\C$ for each $\C \in \Cwf_{\Sigma}$, such that
$F(\Gamma_\C) = \Gamma_\D$ for all morphisms $F \in \Cwf_{\Sigma}(\C,\D)$.
If $\Gamma$ is such a family, a {\em uniform family of types} over $\Gamma$ is a
family of types $A = (A_{\C})$ with $A_{\C}$ a type over $\Gamma_{\C}$ and
$F(A_{\C}) = A_{\D}$ for all morphisms $F \in \Cwf_{\Sigma}(\C,\D)$.
Finally, given $\Gamma$ and $A$, a {\em uniform family of terms} is a family
of terms $a = (a_{\C})$ with $a_\C \in \Tm_{\C}(\Gamma_{\C},A_{\C})$ such that
$F(a_{\C}) = a_{\D}$ for all morphisms $F \in \Cwf_{\Sigma}(\C,\D)$.

\begin{remark}
Uniform families appear in Freyd's proof of the adjoint functor theorem \cite{freyd:abelian}, in Reynolds' \cite{reynolds:impredicative} and Reynolds and Plotkin's construction \cite{plotkin-reynolds} of an initial algebra for an endofunctor from an impredicative encoding of an inductive type, and in Awodey, Frey, and Speight's  \cite{awodey:impredicative} construction of an impredicative encoding of a higher inductive type. The common idea in these works is to first construct a weakly initial object and then the initial object is obtained by taking uniform families.
\end{remark}

\begin{definition}\label{def-sig-mod}
We define the notion of a \FYI{presentation} $\Sigma$ and the category $\Cwf_\Sigma$ of cwfs with a $\Sigma$-structure and cwf-morphisms that preserve $\Sigma$-structure. The definition is by induction on the length of $\Sigma$. We have the following base case:
\begin{description}
\item[The empty \FYI{presentation}] The only \FYI{presentation} of length zero is the empty one $\emptyset$. We let $\Cwf_\emptyset = \Cwf$.
%\footnote{PD: Add "Note that a uniform family of contexts $(\Gamma_{\C})$ indexed by $\C \in \Cwf$ is a large set. Thus a \FYI{presentation} will also be a large set, since it is essentially a sequence of uniform families. As a consequence the set of all \FYI{presentations} is a very large set."}
\end{description}
Assume now that we have defined a \FYI{presentation} $\Sigma$ of length $n$ and the associated
category $\Cwf_{\Sigma}$.
%\footnote{PD: Add "Moreover, assume that we have shown that $\Cwf_{\Sigma}$ is a large but not very large category".}
Then we define $\Sigma'$ of length $n+1$ and the associated category $\Cwf_{\Sigma'}$, where $\Sigma'$ is obtained from $\Sigma$ by adding a new sort symbol, or a new operator symbol, or a new equation, as follows.
\begin{description}
\item[Adding a sort symbol]
  Let $\Gamma = (\Gamma_\C)$ be a uniform family of contexts indexed by $\C \in \Cwf_{\Sigma}$.
  Then we can extend $\Sigma$ with a new sort symbol $S$ relative to $\Gamma$, to obtain
  the generalized algebraic theory $\Sigma' = (\Sigma,(\Gamma,S))$.
  The objects of $\Cwf_{\Sigma'}$ are pairs $(\C,S_{\C})$, where $\C$ is an object of $\Cwf_{\Sigma}$
  and $S_{\C} \in \Ty_\C(\Gamma_{\C})$.
  A morphism in $\Cwf_{\Sigma'}((\C,S_{\C}), (\D,S_{\D}))$
  is a morphism $F \in \Cwf_{\Sigma}(\C,\D)$ such that $F(S_\C) = S_\D$.
\item[Adding an operator symbol]
  If $\Gamma$ is a uniform family of contexts and $A$ a uniform family of
  types over $\Gamma$,
  %both indexed by $\C \in \Cwf_{\Sigma}$,
  then we can extend $\Sigma$ with a new operator
  symbol $f$ relative to $\Gamma$ and $A$, to obtain
  the generalized algebraic theory $\Sigma' = (\Sigma,(\Gamma,A,f))$.
  An object of $\Cwf_{\Sigma'}$
  is a pair $(\C,f_{\C})$ where $\C$ is an object in $\Cwf_{\Sigma}$ and $f_{\C} \in \Tm_\C(\Gamma_\C,A_\C)$.
  A morphism in $\Cwf_{\Sigma'}((\C,f_{\C}),(\D,f_{\D}))$ is a morphism $F \in \Cwf_{\Sigma}(\C,\D)$ such that $F(f_{\C}) = f_{\D}$.
\item[Adding an equation]
  If $\Gamma$ is a uniform family of contexts,
  $A$ is a uniform family of types over $\Gamma$
  and $a,a'$ are uniform families of terms in $A$,
  %all indexed by $\C \in \Cwf_{\Sigma}$,
  then we can extend $\Sigma$ with a new equation $a = a'$ relative to $\Gamma$ and $A$, to obtain
  the generalized algebraic theory $\Sigma' = (\Sigma,(\Gamma,A,a,a'))$.
  In this case,
  $\Cwf_{\Sigma'}$ is a full subcategory of $\Cwf_{\Sigma}$. An object $\C$ in
  $\Cwf_{\Sigma'}$ is an object $\C$ of $\Cwf_\Sigma$ such that $a_{\C} = a'_{\C}$.
\end{description}

\end{definition}

This definition is {\em syntax independent}. In the next section we show how to {\em syntactically} construct an {\em initial object} $\T_{\Sigma}$ in $\Cwf_{\Sigma}$ (for an arbitrary \FYI{presentation} $\Sigma$) in terms of grammars and inference rules. A context in $\T_{\Sigma}$ will be an equivalence class $[ \Gamma ]$ of raw contexts, and similarly for substitutions, types, and terms.

We refer to Section \ref{monoids} where we show a simple instance of this definition: a \FYI{presentation} $\Sigma$ of internal monoids and its associated category of models $\Cwf_\Sigma$ of cwfs with an internal monoid. We also show how to construct the initial cwf with an internal monoid $\T_\Sigma$.

\begin{remark}
There is a bijective correspondence between contexts in $\T_\Sigma$ and uniform families of contexts
indexed by $\C \in \Cwf_\Sigma$. To each context $[ \Gamma ]$ in $\T_\Sigma$ we associate the uniform family $(\inte{ [ \Gamma ] }_\C)$ where $\inte{-}_\C$ is the unique morphism from $\T_\Sigma$ to $\C$. To each family $(\Gamma_\C)$ indexed by $\C \in \Cwf_\Sigma$ we associate the context $\Gamma_{\T_\Sigma}$ in $\T_\Sigma$. Moreover, $\inte{ [ \Gamma ] }_{\T_\Sigma} = [ \Gamma ]$ since $\T_\Sigma$ is initial, and  $\inte{ {\Gamma_{\T_\Sigma}}}_\C  = \Gamma_\C$ because of uniformity and since $\inte{-}_\C$ is a morphism in $\Cwf_{\Sigma}$. For similar reasons there are bijective correspondences between types and terms in $\T_\Sigma$ and uniform families of types and terms.
\end{remark}

\begin{remark}
Note that in general a \FYI{presentation} $\Sigma$ is a large set, since uniform families over $\Cwf_\Sigma$ are. However, because of the bijective correspondence these uniform families can be replaced by contexts, types, and terms in $\T_\Sigma$. This replacement will turn the semantic $\Sigma$ into a small syntactic version.
\end{remark}

\begin{remark}
Cartmell's notion of generalized algebraic theory \cite{cartmell:phd,cartmell:apal} also makes it possible to stipulate equations between type expressions. However, none of our examples makes use of this extra generality. In particular,  in Section \ref{sec:examples} we present the generalized algebraic theory of cwfs with extra structure for $\N, \Pi$ and a first universe $\U_0$ without needing type equations. The reason is that
equations between types become equations between terms in our rendering
of dependent type theory as a generalized algebraic theory. See Remark \ref{remark:typeequations} in Section \ref{sec:u-example} for more explanation.

Like Cartmell, we could consider generalized algebraic theories with type equations, but we prefer not to make such equations part of our notion.


%If we wish to extend our notion of \FYI{presentation} with this new possibility we could add the following clause:
%\begin{description}
%\item[Adding a type equation]
%If $\Gamma$ is a uniform family of contexts, and
%  $A, A'$ are uniform families of types over $\Gamma$
%  then we can extend $\Sigma$ with a new equation $A = A'$, to obtain
%  the generalized algebraic theory $\Sigma' = (\Sigma,(\Gamma,A,A'))$.
%  In this case,
%  $\Cwf_{\Sigma'}$ is a full subcategory of $\Cwf_{\Sigma}$. An object $\C$ in
%  $\Cwf_{\Sigma'}$ is an object $\C$ of $\Cwf_\Sigma$ such that $A_{\C} = A'_{\C}$.
%\end{description}
%The construction of initial objects $\T_\Sigma$ in the next section can easily be extended to this more general notion of \FYI{presentation} $\Sigma$.
\end{remark}


%The construction of $\T_\Sigma$ is the topic of the next section.

\section{The construction of an initial object in $\Cwf_\Sigma$}\label{initial-gat-section}

In Section 3 we gave a {\em syntax independent specification} of a generalized algebraic theory as the initial object of the category $\Cwf_\Sigma$ of models of a (semantic) \FYI{presentation} $\Sigma$. Now we show our main theorem: the {\em syntactic construction} of such an initial object $\T_\Sigma$. This construction is done in several steps. We first define the ``raw" syntactic expressions. Then we define four families of partial equivalence relations (pers) over those raw expressions, corresponding to the four equality judgments. The term model $\T_\Sigma$ is obtained by quotienting with these pers.

The following theorem can be viewed as a generalization of Birkhoff's completeness theorem for equational logic \cite{birkhoff}.
\begin{theorem}\label{initial-gat}
The category $\Cwf_\Sigma$ has an initial object $\T_\Sigma$,
for every \FYI{presentation} $\Sigma$ of a generalized algebraic theory.
\end{theorem}

The construction of $\T_\Sigma$ will be by induction on the construction of $\Sigma$. It is based on construction of initial cwfs in \cite{castellan:tlca2015,castellan:lmcs} and we refer the reader to those papers for more details. Here we only provide a sketch and focus on how to extend the construction to $\T_\Sigma$.

For each $\Sigma$ we will define the following.
\begin{enumerate}
\item
A grammar for the {\em raw syntax}, that is, raw contexts in $\RawCtx_\Sigma$, raw substitutions in $\RawSub_\Sigma$, raw types in $\RawTy_\Sigma$, and raw terms in $\RawTm_\Sigma$.
\item
A system of inference rules that generate four families of partial equivalence relations (pers) by a mutual inductive definition:
$$
\Gamma = \Gamma' \vdash_\Sigma
\qquad
\Gamma \vdash_\Sigma A = A'
\qquad
\Delta \vdash_\Sigma \gamma = \gamma' : \Gamma
\qquad
\Gamma \vdash_\Sigma a = a' : A
$$
where $\Gamma, \Gamma' \in \RawCtx_\Sigma, \gamma, \gamma' \in \RawSub_\Sigma, A, A' \in \RawTy_\Sigma,$ and $a,a' \in \RawTm_\Sigma$. These pers define the valid equality judgments of a variable-free version of dependent type theory with explicit substitutions based on the cwf-combinators. The ordinary judgments will be defined as the reflexive instances of these equality judgments. For example $\Gamma \vdash_\Sigma$, meaning that $\Gamma$ is a valid context, is defined as the reflexive instance $\Gamma = \Gamma \vdash_\Sigma$.
\item
A cwf $\T_\Sigma$ is then constructed from the equivalence classes of derivable judgments. For example, the contexts in $\T_\Sigma$ are equivalence classes $[\Gamma]$, such that $\Gamma \vdash_\Sigma$. We will show that $\T_\Sigma$ is a cwf with a $\Sigma$-structure, that is, an object of $\Cwf_\Sigma$.
\item
A $\Cwf_\Sigma$-morphism $\inte{-} : \T_\Sigma \to \C$ for every $\C \in \Cwf_\Sigma$. This is the {\em interpretation morphism}. This morphism is a partial function defined by induction on the raw syntax, that (whenever it is defined) maps raw contexts to contexts in $\C$, raw substitutions to substitutions in $\C$, raw types to types in $\C$, and raw terms to terms in~$\C$. We show that these partial functions preserve the partial equivalence relations so that we can define the interpretation morphism on the equivalence classes. Finally we show that it indeed is a $\Cwf_\Sigma$-morphism and the unique such into $\C$.
\end{enumerate}

We begin with the construction for the base case: the {\bf empty \FYI{presentation}} $\emptyset$.
\begin{enumerate}
\item
We start with the {\em raw syntax} for the initial pure cwf $\T_\emptyset$ . It is specified by the following grammar for raw contexts, raw substitutions, raw types, and raw terms.
\begin{eqnarray*}
\Gamma \in \RawCtx_\emptyset &::=& 1  \ |\ \Gamma\cext A\\
\gamma \in \RawSub_\emptyset  \ &::=& \gamma \circ \gamma \ |\ \id_\Gamma \ |\ \langle\rangle_\Gamma \ |\ \p_{A} \ |\ \langle \gamma, a \rangle_A\\
A \in \RawTy_\emptyset  &::=& A[\gamma]\\
a \in \RawTm_\emptyset  &::=& a[\gamma] \ |\ \qI_A
\end{eqnarray*}
This grammar generates a language of {\em cwf-combinators}.
\item
The system of inference rules for $\T_\emptyset$ is displayed in \cite{castellan:tlca2015,castellan:lmcs}. It is a system of {\em general rules}, rules for dependent type theory which come before we introduce any sort symbols and operator symbols and equations (or any rules for the type formers of intuitionistic type theory). We do not have room here to display them, but note that they can be divided into four groups:
\begin{itemize}
\item the per rules, amounting to symmetry and transitivity for the four forms of equality judgments;
\item preservation rules for judgments, amounting to substitution of equals for equals (an example of such a rule is the {\em type equality rule});
\item congruence rules for operators expressing that the cwf-combinators preserve equality;
\item conversion rules for the cwf-combinators.
\end{itemize}
\item
Note that the initial cwf $\T_\emptyset$ is trivial: its category of contexts contains only a terminal object (the empty context), and there are no types and terms. Nevertheless, the grammar and inference rules used in its definition form a starting point. The grammar for raw types and raw terms will be extended each time we add a new sort symbol or operator symbol, respectively. For each such new symbol and each new equation we will add a new inference rule. As a consequence we will generate non-trivial $\T_\Sigma$.
\item
The definition of the interpretation morphism  $\inte{-} : \T_\emptyset \to \C$ and its proof of uniqueness are routine and can be found in \cite{castellan:tlca2015,castellan:lmcs}.
\end{enumerate}

Assume now for the induction step that we have defined the grammar, the inference rules, $\T_\Sigma$ and the interpretation morphism $\inte{-} : \T_{\Sigma} \to \C$ in $\Cwf_\Sigma$.
Let $\Sigma'$ be $\Sigma$ extended by a new sort symbol, a new operator symbol, or a new equation. We shall now explain how to define $\T_{\Sigma'}$.
\begin{description}
\item[Adding a sort symbol] If $\Gamma \vdash_\Sigma$, then we can introduce a new sort symbol $S$ in the context $\Gamma$ representing the sequence of types of the arguments of $S$.
\begin{enumerate}
\item
We add a new production for raw types
$$
A ::= S
$$
to the productions for $\T_\Sigma$.
\item
We add  the inference rule
\begin{mathpar}
    \inferrule
    {}
    {\Gamma \vdash_{\Sigma'} S}
  \end{mathpar}
to the inference rules for $\T_\Sigma$.
\item
We define $S_{\T_{\Sigma'}} = [S]$ and $\Sigma' = (\T_\Sigma,[S])$.
%, so that $ \T_{\Sigma'}$ has a $\Sigma'$-structure $(\T_\Sigma,[S])$.
\item
We extend the definition of the interpretation morphism $\inte{-}$  to an interpretation morphism $\inte{-}' : \T_{\Sigma'} \to \C$ by
$$
\inte{[S]}' = S_\C
$$
It follows  that this is a morphism in $\Cwf_{\Sigma'}$ and that it is unique.
\end{enumerate}

\item[Adding an operator symbol] If $\Gamma \vdash_\Sigma A$, then we can introduce a new operator symbol $f$, where the context $\Gamma$ represents the sequence of types of the arguments and $A$ is the type of the result.
\begin{enumerate}
\item
We add a new production for raw terms
$$
a ::= f
$$
to the productions for $\T_\Sigma$.
\item
We add  the inference rule
\begin{mathpar}
    \inferrule
    {}
    {\Gamma \vdash_{\Sigma'} f : A}
  \end{mathpar}
to the inference rules for $\T_\Sigma$.
\item
We define $f_{\T_{\Sigma'}} = [f]$ and $ \T_{\Sigma'} = (\T_\Sigma,[f])$.\item
We extend the definition of the interpretation morphism $\inte{-}$  to an interpretation morphism $\inte{-}' : \T_{\Sigma'} \to \C$ by
$$
\inte{[f]}' = f_\C
$$
It follows  that this is a morphism in $\Cwf_{\Sigma'}$ and that it is unique.
\end{enumerate}

\item[Adding an equation] If $\Gamma \vdash_\Sigma a : A$ and $\Gamma \vdash_\Sigma a' : A$ we can introduce a new equation $a = a'$.
\begin{enumerate}
\item
$\T_\Sigma'$ has the same productions as $\T_\Sigma$.
\item
We add  the inference rule
\begin{mathpar}
    \inferrule
    {}
    {\Gamma \vdash_{\Sigma'} a = a' : A}
  \end{mathpar}
to the inference rules for $\T_\Sigma$.
\item
$\T_{\Sigma'}$ is based on the same raw syntax as $\T_\Sigma$ but the equivalence relation has changed. To show that $\T_{\Sigma'} \in \Cwf_{\Sigma'}$ we just need to show that $[ a ] = [ a' ]$ but this follows from the inference rule $\Gamma \vdash_{\Sigma'} a = a' : A$.
\item
In order to define $\inte{-}'$ we first define the partial function on the raw syntax to be identical to the partial function on the raw syntax for $\inte{-}$. We then prove that this partial function preserves the extended partial equivalence relation and define $\inte{-}'$ on the new equivalence classes. It follows  that $\inte{-}'$ is unique.
\end{enumerate}
\end{description}
This concludes the proof of the theorem. \qed



\section{Examples of generalized algebraic theories}\label{sec:examples}

We will now display the sort symbols, operator symbols, and equations for the generalized algebraic of monoids, categories, and cwfs. We will then show how to add operator symbols and equations when extending cwfs with $\Pi$-types, natural numbers $\N$, and a universe closed under $\Pi$ and $\N$. The models of these theories are internal monoids, internal categories, and internal cwfs (possibly with extra structure for internal $\Pi, \N$, and universes) in $\Cwf_\Sigma$. Moreover, internal monoids, categories, and cwfs in the cwf~$\Set$ are small monoids, categories, and cwfs, respectively. Note that the ``external" cwfs defined in Section 2 need not be small, and hence not internal cwfs in the cwf~$\Set$.

We begin by using the recipe in Definition \ref{def-sig-mod} to construct the \FYI{presentation of} monoids and its associated category of models, that is, of cwfs with an internal monoid. We then follow the recipe in Theorem \ref{initial-gat} and construct the initial cwf with an internal monoid.

For ease of readability, we will only present the sort symbols, operator symbols, and equations in the remaining examples by using an informal notation with named variables, rather than the formal notations using cwf-combinators employed in Definitions \ref{def:Cwfmor} and \ref{initial-gat}.

Our final example is the generalized algebraic theory of contextual cwfs, a variant of Cartmell's contextual categories. The contexts in such contextual cwfs come with a length $n$. We sketch how this can be axiomatized as a generalized algebraic theory with countably many sort symbols $\ctx_n, \sub_n, \ty_n, \tm_n$ for an external natural number $n$ (and similarly for the operator symbols and equations). We also indicate how our framework can be extended to cover such generalized algebraic theories.

\subsection{Internal monoids}\label{monoids}
 The one-sorted algebraic theory of monoids has two operator symbols,
$\idmon$ for identity and $\comp$ for composition, and associativity and identity laws as equations.
As any other one-sorted algebraic theory, the theory of monoids yields a
generalized algebraic theory. In ordinary notation with variables it might be rendered as follows, where $\Mon$ is the only sort:
\begin{eqnarray*}
&\vdash& \Mon\\
&\vdash& e : \Mon\\
x, y : \Mon &\vdash& \comp(x,y) : \Mon\\
y : \Mon &\vdash& \comp(\idmon,y) = y : \Mon\\
x : \Mon &\vdash& \comp(x,\idmon) = x : \Mon\\
x, y, z : \Mon &\vdash& \comp(\comp(x,y),z) = \comp(x,\comp(y,z)) : \Mon
\end{eqnarray*}

We now show how the corresponding official (in the sense of Definition \ref{def-sig-mod}) \FYI{presentation of} monoids $\Sigma$ and its associated category of models $\Cwf_\Sigma$ are constructed step-wise.

As always, we begin with the empty \FYI{presentation} $\emptyset$ and its category of models $\Cwf_\emptyset = \Cwf$.
\begin{description}
\item[Adding the sort symbol $\Mon$] Each cwf $\C$ has a chosen empty context (terminal object) $1_\C$. Since cwf-morphisms preserve empty contexts on the nose, $1 = (1_\C)$ is a uniform family of contexts in $\Cwf_\emptyset$. Hence we can introduce a new sort symbol $\Mon$ in the empty context. The resulting \FYI{presentation} is
$$
\Sigma_1 = (\emptyset, (1,\Mon))
$$
The objects of $\Cwf_{\Sigma_1}$  are pairs $(\C,\Mon_\C)$, where $\C$ is a cwf and $\Mon_\C \in \Ty_\C(1_\C)$.
\item[Adding the operator symbol for the identity]
%Each $(\C,\Mon_\C) \in \Cwf_{\Sigma_1} $ has an empty context $1_\C$ and a type $\Mon_\C \in \Ty_\C(1_\C)$.
Since, morphisms in $\Cwf_{\Sigma_1}$ preserve both empty contexts $1_\C$ and types $\Mon_\C$ on the nose, we have a uniform family of contexts $1 = (1_\C)$ and a uniform family of types $\Mon = (\Mon_\C)$ in $\Cwf_{\Sigma_1}$. Hence we can introduce a new operator symbol $e$ (the identity of the monoid).  The resulting \FYI{presentation} is
$$
\Sigma_2 = (\Sigma_1, (1,\Mon,e))
$$
The objects of $\Cwf_{\Sigma_2}$  are triples $(\C,\Mon_\C,e_\C)$, where $\C$ is a cwf, $\Mon_\C \in \Ty_\C(1_\C)$ and $e_\C \in \Tm_\C(1_\C,\Mon_\C)$.
\item[Adding the operator symbol for composition]
Again using that cwf-morphisms preserve all cwf-structure and $\Mon_\C$, we deduce that we have a uniform family of contexts $1.\Mon.\Mon[\p]$ and a uniform family of types $\Mon[\p][\p]$ in $\Cwf_{\Sigma_2}$. Thus we can add a binary operator symbol $\comp$. The resulting \FYI{presentation} is
$$
\Sigma_3 = (\Sigma_2, (1.\Mon.\Mon[\p],\Mon[\p][\p],\comp))
$$
The objects of $\Cwf_{\Sigma_3}$  are quadruples $(\C,\Mon_\C,e_\C,*_\C)$, where $\C$ is a cwf, $\Mon_\C \in \Ty_\C(1_\C)$, $e_\C \in \Tm_\C(1_\C,\Mon_\C)$, and $*_\C \in \Tm_\C((1.\Mon.\Mon[\p])_\C,(\Mon[\p][\p])_\C)$.
\item[Adding the left identity law]
Furthermore, we extend the \FYI{presentation} with the equations stating that $\idmon$ is a left identity as follows:
$$
\Sigma_4 = (\Sigma_3, (1.\Mon, \Mon[\p], \comp[\tuple{\tuple{\tuple{},\idmon[\tuple{}]},\q}], \q))
$$
The uniform family of context $1.\Mon$ expresses that the equation has one variable of type $\Mon$, the uniform family of types $\Mon[\p]$ expresses that the two sides of the equation have type $\Mon$, and the uniform families of terms $\comp[\tuple{\tuple{\tuple{},\idmon[\tuple{}]},\q}]$ and $\q$ express the two sides of the equation.
$\Cwf_{\Sigma_4}$ is the full subcategory of $\Cwf_{\Sigma_3}$ with objects $\C$
such that $(\comp[\tuple{\tuple{\tuple{},\idmon[\tuple{}]},\q}])_\C = \q_\C$.
\item[Adding the right identity and the associativity laws]
Finally we add the right identity equation and the associativity equation to get the \FYI{presentations} $\Sigma_5$ and $\Sigma_6$. We omit the details.
\end{description}
We define the \FYI{presentation of} monoids to be $\Sigma = \Sigma_6$. The category $\Cwf_\Sigma$ is the category of cwfs with an {\em internal monoid}. This is a cwf-version of the notion of internal monoid which can be defined in any category with finite products. Ordinary (small) monoids come out as internal monoids in $\Set$, the cwf of small sets.

%If we interpret the intial gat in a cwf C with families, then

We also sketch the construction of the initial object $\T_\Sigma$ of $\Cwf_\Sigma$ following the recipe for introducing sort symbols, operators symbols, and equations in Section \ref{initial-gat}. (We omit the index $\Sigma$ in $\vdash_\Sigma$.)
%Note that these inference rules follow the principles of introducing sort symbols, operator symbols, and equations we displayed in Section 4 when building the initial cwf $\T_\Sigma$ for appropriate $\Sigma$.
\begin{description}
\item[Adding the sort symbol $\Mon$]
First, we have $1 \vdash$ for the empty \FYI{presentation}, so we can
add a production for the sort symbol $\Mon$ and the inference rule:
\begin{eqnarray*}
1 &\vdash& \Mon
\end{eqnarray*}
For later use we infer $1.\Mon \vdash$ and, using $\p:1.\Mon \to 1$, $1.\Mon\vdash\Mon[\p]$,
so $1.\Mon.\Mon[\p] \vdash$.
\item[Adding the operator symbol for identity]
We then add a production for the operator symbol $\idmon$ and the inference rule:
\begin{eqnarray*}
1 &\vdash& \idmon : \Mon
\end{eqnarray*}
Again for later use we infer $1.\Mon\vdash \idmon[\p] :\Mon[\p]$.
Note that here $\p = \tuple{}$, the empty substitution $1.\Mon \to 1$,
since there is only one substitution $1.\Mon \to 1$.
\item[Adding the operator symbol for composition]
We then add a production for the binary operator symbol $\comp$.
Using another $\p: 1.\Mon.\Mon[\p] \to 1.\Mon$ (note the different type),
we can derive $1.\Mon.\Mon[\p] \vdash \Mon[\p][\p]$, so we can add the inference rule
\begin{eqnarray*}
1.\Mon.\Mon[\p] &\vdash& \comp : \Mon[\p][\p]
\end{eqnarray*}
Note that we project $\Mon$ on the right twice, reflecting that $\comp$ is binary.
\item[Adding the left identity law]
We can derive $1.\Mon \vdash \q : \Mon[\p]$. With some effort,
using previous inferences, we can  derive
$1.\Mon \vdash \comp[\tuple{\tuple{\tuple{},\idmon[\tuple{}]},\q}] : \Mon[\p]$.
Hence we can add the inference rule for the equation ($\idmon$ is a left identity):
\begin{eqnarray*}
1.\Mon &\vdash& \comp[\tuple{\tuple{\tuple{},\idmon[\tuple{}]},\q}] = \q : \Mon[\p]
\end{eqnarray*}
\item[Adding the right identity and the associativity laws]
We omit the details.
\end{description}
The resulting initial object $\T_\Sigma = \T_{\Sigma_6}$ is generated by a system of grammar and inference rules for dependent type theory with an internal monoid. In this theory we can prove statements such as
$$
\Gamma \vdash a : \Mon
$$
stating that $a$ is a well-formed monoid expression in the context $\Gamma$ and
$$
\Gamma \vdash a = a': \Mon
$$
stating that $a = a'$ is a derivable equation between monoid expressions in the context $\Gamma$. Note that both contexts and monoid expressions use cwf-combinators and are variable-free.

%This example illustrates that generalized algebraic theories indeed generalise ordinary algebraic theories.
Of course, using dependent type theory for reasoning about monoid expressions is overkill; monoids form a single-sorted algebraic theory in the usual sense.
The remaining examples will use dependent types in an essential way. However, for reasons of readability we will from now on only use ordinary notation with named variables. Hopefully, it is clear from the above how to formally construct the corresponding official \FYI{presentations}, categories of models, and initial models using cwf-combinators. For example, these constructions for the theory of internal categories are similar to the constructions for the theory of internal monoids.

% In the remaining examples, we shall mainly use ordinary notation with variables.
% % We have already said that.

\subsection{Internal categories} The generalized algebraic theory of categories was one of Cartmell's motivating examples. It has the following sort symbols, operator symbols, and equations. Again, note that the models are cwfs with an internal category. To emphasize the difference between the internal notions of category and cwf and the external notions (introduced in Section 2), our notation for sort symbols in the generalized algebraic theory of internal cwfs use lower case letters ($\Obj, \Hom, \ty, \tm$). This is in contrast to the upper case letters for the external versions ($\Ty, \Tm$). We will however overload notation for operator symbols, and for example use $\circ$ both for the cwf-combinator and for the operator symbol in the generalized algebraic theory of internal categories.

Sort symbols:
\begin{eqnarray*}
&\vdash& \Obj\\
\Delta, \Gamma : \Obj &\vdash& \Hom(\Delta,\Gamma)\\
\end{eqnarray*}

Operator symbols:
\begin{eqnarray*}
\Gamma : \Obj &\vdash& \id_\Gamma : \Hom(\Gamma,\Gamma)\\
\Xi,\Delta,\Gamma : \Obj, \gamma : \Hom(\Delta,\Gamma), \delta : \Hom(\Xi,\Delta) &\vdash&
\gamma \circ \delta : \Hom(\Xi,\Gamma)
\end{eqnarray*}

Equations:
\begin{eqnarray*}
\Delta, \Gamma : \Obj, \gamma : \Hom(\Delta,\Gamma) &\vdash& \id_\Gamma \circ \gamma = \gamma : \Hom(\Delta,\Gamma)\\
\Delta, \Gamma : \Obj, \gamma : \Hom(\Delta,\Gamma) &\vdash& \gamma \circ \id_\Delta = \gamma : \Hom(\Delta,\Gamma)\\
\Theta, \Xi,\Delta,\Gamma : \Obj, \gamma : \Hom(\Delta,\Gamma), \delta : \Hom(\Xi,\Delta), \xi : \Hom(\Theta,\Xi) &\vdash&
(\gamma \circ \delta) \circ \xi = \gamma \circ (\delta \circ \xi): \Hom(\Theta,\Gamma)
\end{eqnarray*}
%In order to build $\Hom(\Delta,\Gamma)$ for two concrete contexts $\vdash \Delta : \Ctx$ and $\vdash \Gamma : \Ctx$ we need to apply $\Hom$ to the context morphism consisting of $\Delta$ and $\Gamma$ to yield $\Hom(\Delta,\Gamma)$. (There is more to say ...)
Note that composition is officially an operator symbol with five arguments. In the official notation we should write $\gamma \circ_{\Xi,\Delta,\Gamma} \delta$, but we suppress the context arguments $\Xi,\Delta,\Gamma$. We will do so for some other operations too.

The rendering of the generalized algebraic theory of categories in cwf-combinator language and the proof that it indeed yields a \FYI{presentation} are similar to what they were for the generalized algebraic theory of monoids. The inference rules for the two sort symbols in cwf-combinator language are
\begin{eqnarray*}
1 &\vdash& \Obj\\
1.\Obj.\Obj[\p] &\vdash& \Hom
\end{eqnarray*}
and the operator symbols for identity
\begin{eqnarray*}
1.\Obj &\vdash& \idmon : \Hom[\tuple{\id_{1.\Obj},\q_{1,\Obj}}]
\end{eqnarray*}
We omit the verbose cwf-renderings of the operator symbol for composition and the equations.

A cwf with extra structure for the generalized algebraic theory of categories is a cwf with an {\em internal category}. This is a cwf-based analogue of the usual notion of internal category in a category with finite limits. As shown by Martin Hofmann \cite{hofmann:csl,hofmann:cambridge}, every category with finite limits yields a category with attributes, and hence a cwf. However, not every cwf has finite limits. To achieve this we need more structure. As shown by Clairambault and Dybjer \cite{ClairambaultD11,ClairambaultD14} the 2-category of categories with finite limits is biequivalent to the 2-category of democratic cwfs that support $\Sigma$-types and extensional identity types.

An internal category in the cwf $\Set$ of small sets is a small category.

%\subsubsection{How to generate the \FYI{presentation} using the combinator language}
%The \FYI{presentation of} the generalized algebraic theory of categories has sort symbols $\Obj$
%and $\Hom$, operator symbols $\id$ for identity and $\circ$ for composition, and associativity and identity laws as equations. We shall here sketch how to generate the \FYI{presentation of} categories with reference to definition \ref{} and using cwf-combinators.
%\begin{itemize}
%\item The initial cwf $\T_\emptyset$ (as constructed above) has only one object (context) [1], one equivalence class of morphisms $[\id_1]$ (with several representatives: $\tuple{}_1, \id_1 \circ \id_1$, etc), and no types and terms. Hence we can add a new constant sort $\Obj$ of (internal) objects with context $1$ to the \FYI{presentation}.
%\item $\T_{([\Obj],[],[])}$ (as constructed above) contains the context $[(1.\Obj).\Obj[\p_{1,\Obj}]]$ (corresponding to the context $x : \Obj, y : \Obj$ in usual notation). Hence we can introduce a new sort $\Hom$ with this context. This sort represents the family $\Hom(x,y)$ of (internal) morphisms.
%\item $\T_{([\Obj, \Hom],[],[])}$ contains the context $[1.\Obj]$ (corresponding to the context $x : \Obj$) and the type $[\Hom[\tuple{\id_{1.\Obj},\q_{1,\Obj}}]]$ (corresponding to the type $\Hom(x,x)$). Hence we can introduce an operator symbol $\id$ with this context and type.
%\item In a similar way we can add the operator symbol for composition and the equations, but we omit the details.
%\end{itemize}

\subsection{Internal cwfs}\label{gat-cwf}

The generalized algebraic theory of cwfs is obtained by extending the generalized algebraic theory of categories with new sort symbols, operator symbols, and equations for a family valued functor, a terminal object, and context comprehension. We here rename the sort $\Obj$ of objects of the category of contexts to $\ctx$.

\subsubsection{The extension with a family valued functor}
\mbox{ }

Sort symbols:
\begin{eqnarray*}
\Gamma : \ctx &\vdash& \ty(\Gamma)\\
\Gamma : \ctx, A:\ty(\Gamma) &\vdash& \tm(\Gamma,A)
\end{eqnarray*}

Operator symbols:
\begin{eqnarray*}
\Gamma,\Delta : \ctx, A:\ty(\Gamma), \gamma : \Hom(\Delta,\Gamma) &\vdash&
A[\gamma] : \ty(\Delta)\\
\Gamma,\Delta : \ctx, A:\ty(\Gamma), \gamma : \Hom(\Delta,\Gamma), a:\tm(\Gamma,A) &\vdash&  a[\gamma] : \tm(\Delta,A[\gamma])
\end{eqnarray*}

Equations:
\begin{eqnarray*}
\Gamma : \ctx, A:\ty(\Gamma) &\vdash& A[\id_\Gamma] = A : \ty(\Gamma)\\
\Gamma : \ctx, A:\ty(\Gamma), a:\tm(\Gamma,A) &\vdash& a[\id_\Gamma] = a : \tm(\Gamma,A)\\
\Xi,\Delta,\Gamma : \ctx, \delta : \Hom(\Xi,\Delta), \gamma : \Hom(\Delta,\Gamma),
A:\ty(\Gamma) &\vdash& A[\gamma\circ\delta] = A[\gamma][\delta]: \ty(\Xi)\\
\Xi,\Delta,\Gamma : \ctx, \delta : \Hom(\Xi,\Delta), \gamma : \Hom(\Delta,\Gamma),
A:\ty(\Gamma), a:\tm(\Gamma,A) &\vdash&
a[\gamma\circ\delta] = a[\gamma][\delta]: \tm(\Xi,A[\gamma\circ\delta])
\end{eqnarray*}

\subsubsection{The extension with a terminal object}
No new sorts are required.

Operator symbols:
\begin{eqnarray*}
&\vdash& 1 : \ctx\\
\Gamma : \ctx &\vdash& \tuple{}_\Gamma : \Hom(\Gamma,1)
\end{eqnarray*}

Equations:
\begin{eqnarray*}
 &\vdash& \id_1 = \tuple{}_1 : \Hom(1,1)\\
\Gamma,\Delta : \ctx, \gamma : \Hom(\Delta,\Gamma) &\vdash&
\tuple{}_\Gamma\circ\gamma = \tuple{}_\Delta : \Hom(\Delta,1)
\end{eqnarray*}
(The latter two equations are better for term rewriting than the
obvious single one expressing the uniqueness of $\tuple{}_\Gamma$.)

\subsubsection{The extension with context comprehension}

No new sorts are required.

Operator symbols:
\begin{eqnarray*}
\Gamma : \ctx, A:\ty(\Gamma) &\vdash& \Gamma\cext A : \ctx\\
\Gamma,\Delta : \ctx, A:\ty(\Gamma), \gamma : \Hom(\Delta,\Gamma), a:\tm(\Delta,A[\gamma]) &\vdash& \tuple{\gamma,a} : \Hom(\Delta,\Gamma\cext A)\\
\Gamma : \ctx, A:\ty(\Gamma) &\vdash& \p: \Hom(\Gamma\cext A,\Gamma)\\
\Gamma : \ctx, A:\ty(\Gamma) &\vdash& \q: \tm(\Gamma\cext A,A[\p])
\end{eqnarray*}

Equations:
\begin{eqnarray*}
\Gamma,\Delta : \ctx, A:\ty(\Gamma), \gamma : \Hom(\Delta,\Gamma), a:\tm(\Delta,A[\gamma]) &\vdash& \p\circ\tuple{\gamma,a} = \gamma : \Hom(\Delta,\Gamma)\\
\Gamma,\Delta : \ctx, A:\ty(\Gamma), \gamma : \Hom(\Delta,\Gamma), a:\tm(\Delta,A[\gamma]) &\vdash& \q[\tuple{\gamma,a}] = a : \tm(\Delta,A[\gamma]) \\
\Gamma,\Delta,\Xi : \ctx, A:\ty(\Gamma), \gamma : \Hom(\Delta,\Gamma), a:\tm(\Delta,A[\gamma]), \delta : \Hom(\Xi,\Delta) &\vdash&
\tuple{\gamma,a} \circ \delta = \tuple{\gamma\circ\delta,a[\delta]} :
\Hom(\Xi,\Gamma\cext A) \\
\Gamma : \ctx, A:\ty(\Gamma) &\vdash&
\id_{\Gamma\cext A} = \tuple{\p,\q} : \Hom(\Gamma\cext A,\Gamma\cext A)
\end{eqnarray*}
(If $\p\circ\delta = \gamma$ and $\q[\delta]=a$, we get
$\tuple{\gamma,a}=\tuple{\p\circ\delta, \q[\delta]} = \tuple{\p,\q}\circ\delta =
\delta$, the uniqueness requirement of the universal property.
However, the equation for surjective pairing is not left-linear and with
a variable on one side, which is not good for rewriting.)

If $\Sigma$ is the presentation of the generalized algebraic theory of cwfs, then $\Cwf_\Sigma$ is the category of small cwfs with an internal cwf. As mentioned in the introduction, to give an example of such a cwf, we assume that our set theory comes with two more Grothendieck universes $\V'$ and $\V''$, where $\V''  \in \V' \in \V$, and $\V$ is the set of small sets used in the Definition~\ref{def:Cwfobj} of external cwfs. We now consider the $\V$-small cwf $\Set'$ of $\V'$-small sets. This has an internal cwf of $\V''$-small sets obtained by interpreting the sort of objects $\ctx$ as the $\V'$-small set $\V''$, and the sorts of types $\ty(\Gamma)$ also as $\V''$.

%Further examples of internal cwfs inside $\Set$ are provided by cwfs supporting \FYI{presentations} of generalized algebraic theories, as constructed in the previous section.

%\begin{remark}
%Note the difference between the internal notion of cwf defined here and the external notion of cwf defined in Section \ref{sec:def_cwf}. 
%\end{remark}

\subsection{Internal cwfs with $\Pi$-types}
We add three operator symbols in addition to the operator symbols for cwfs in Section 5.2 and 5.3:
\begin{eqnarray*}
\Gamma : \ctx, A : \ty(\Gamma), B : \ty(\Gamma.A)&\vdash& \Pi(A,B) : \ty(\Gamma)\\
\Gamma : \ctx, A : \ty(\Gamma), B : \ty(\Gamma.A), b : \tm(\Gamma.A, B) &\vdash& \lambda(b) : \tm(\Gamma,\Pi(A,B))\\
\Gamma : \ctx, A : \ty(\Gamma), B : \ty(\Gamma.A), c :  \tm(\Gamma,\Pi(A,B)), a : \tm(\Gamma, A) &\vdash& \app(c,a) : \tm(\Gamma, B[\tuple{\id,a}])
\end{eqnarray*}
(again omitting some of the official arguments)
and equations for $\beta, \eta$ (also omitting the context and type of the equality judgment)
 \begin{eqnarray*}
 \app(\lambda(b),a) &=& b[\tuple{\id,a}]\\
 \lambda(\app(c[\p],\q)) &=& c
\end{eqnarray*}
and commutation with respect to substitution:
\begin{eqnarray*}
\Pi(A,B)[ \gamma ] &=& \Pi(A [ \gamma ], B[ \gamma^+ ])\\
\lambda(b) [ \gamma ] &=& \lambda(b[\gamma^+ ])\\
\app(c,a) [ \gamma ] &=& \app(c[ \gamma ], a[ \gamma ] )
\end{eqnarray*}
where $\gamma^+ = \tuple{\gamma \circ \p, \q}$.

\begin{remark}
Cartmell \cite{cartmell:apal} defines a generalized algebraic theory for $\Sigma$-types as follows. If we start with any generalized algebraic theory with a sort symbol $A$ in the empty context and  a sort symbol $B$ in context $x:A$, then we can extend it with a new sort symbol $\Sigma B$ in the empty context and operator symbols and equations for the two projections and the pairing. In a similar way we could extend any such generalized algebraic theory with a new sort symbol $\Pi B$ in the empty context and operator symbols for $\lambda$ and $\app$ and the equations for $\beta$ and $\eta$. The reader should be aware of the difference between the resulting generalized algebraic theories and our generalized algebraic theory for cwfs with $\Pi$-types.
\end{remark}

\begin{remark}
Furthermore, just as we can extend the internal notion of cwf with $\Pi$-types we can extend the external notion of cwf defined in Section \ref{sec:def_cwf} with a structure for $\Pi$-types \cite{castellan:tlca2015,castellan:lmcs}.
\end{remark}

\subsection{Internal cwfs with $\Pi$ and $\N$}
Furthermore, we add the operator symbol
\begin{eqnarray*}
\Gamma : \ctx &\vdash& \N_\Gamma : \ty(\Gamma)
\end{eqnarray*}
We also add operator symbols for $0, \s, \Rec$ and the equations for $\Rec$ and for commutativity with substitution, but omit the details. Note that the type of the primitive recursion operator $\Rec$ relies on the \FYI{presentation of} $\Pi$-types.

\subsection{Internal cwfs with $\U_0$ closed under $\Pi$ and $\N$}\label{sec:u-example}
We add four more operator symbols
\begin{eqnarray*}
\Gamma : \ctx &\vdash& (\U_0)_\Gamma : \ty(\Gamma)\\
\Gamma : \ctx, a : \tm(\Gamma,(\U_0)_\Gamma) &\vdash& {\Ta_0}(a) : \ty(\Gamma)\\
\Gamma : \ctx &\vdash& \N^0_\Gamma : \tm(\Gamma,(\U_0)_\Gamma) \\
\Gamma : \ctx,
a : \tm(\Gamma,(\U_0)_\Gamma),
b :  \tm(\Gamma  .  \Ta_0(a), (\U_0)_\Gamma))
&\vdash&
 \Pi^0(a,b) : \tm(\Gamma,(\U_0)_\Gamma)
\end{eqnarray*}
$(\U_0)_\Gamma$ is the universe (a type) relative to the context $\Gamma$; $\Ta_0$ is the decoding operation mapping a term in the universe to the corresponding type; $\N^0$ is the code for $\N$ in the universe, and $\Pi^0$ forms codes for $\Pi$-types in the
 universe. (Note that we have dropped the context argument of $\Ta_0$ and $\Pi^0$.)

We add the decoding equations:
\begin{eqnarray*}
\Ta_0(\N^0_\Gamma) &=& \N_\Gamma\\
\Ta_0(\Pi^0(a,b)) &=& \Pi(\Ta_0(a),\Ta_0(b))
\end{eqnarray*}
and the equations for preservation of substitution:
\begin{eqnarray*}
{(\U_0)}_\Gamma [ \gamma ] &=& {(\U_0)}_\Delta\\
\Ta_0(a) [ \gamma ] &=& \Ta_0(a[ \gamma ] )\\
\N^0_\Gamma [ \gamma ] &=&\N^0_\Delta\\
\Pi^0(a,b)[ \gamma ] &=& \Pi^0(a [ \gamma ], b[ \gamma^+ ])
\end{eqnarray*}
%where $\gamma^+ = \tuple{\gamma \circ \p, \q}$.

\begin{remark}\label{remark:typeequations}
Note that all equations are between {\em terms} in the generalized algebraic theory of cwfs with extra structure for $\N, \Pi,$ and $\U_0$; we do not need the extra generality of stipulating type equations as discussed in the introduction. For example, $\Ta_0(\N^0_\Gamma) = \N_\Gamma$ is an equation between {\em internal} types, that is, terms of type $\ty(\Gamma)$.
\end{remark}

\begin{remark}
Also note that the generalized algebraic theory for the universe is inevitably {\em \`a la Tarski} in the sense that we distinguish between types and terms in a cwf and we must have an operation decoding a term into a type. However, Martin-Löf's distinction between {\em \`a la Russell} and {\em \`a la Tarski} \cite{martinlof:padova} is a distinction between two different formulations of the raw syntax and inference rules of type theory.
\end{remark}

\subsection{A possible refinement to internal contextual cwfs}

Our treatment can be adapted to some non finitely presented generalized algebraic theories.
If we have an increasing sequence of \FYI{presentations} $\Sigma_n$ we can consider their
union.
%of the theory $T_{\Sigma_n}$.
For instance, we can describe a generalized algebraic theory of contextual cwfs \cite{castellan:lambek} (similar to Cartmell's contextual categories and Voevodsky's $C$-systems) by
the following stratification of the theory of cwfs. We replace the sort $\ctx$
by a sequence of sorts $\ctx_0,\,\ctx_1,\,\dots ,$ where $\ctx_n$ represents the sort
of contexts of length $n$ and a corresponding sequence of sorts
$\ty_n(\Gamma)$ for $\Gamma$ in $\ctx_n$
and $\tm_n(\Gamma,A)$ for $A$ in $\ty_n(\Gamma)$. Context extension $\Gamma.A$ is now in $\ctx_{n+1}$
if $A$ is in $\ty_n(\Gamma)$ and so on.
We also add {\em destructors}: we have
$\mathrm{ft}(\Gamma)$ in $\ctx_n$
and $\mathrm{st}(\Gamma)$ in $\ty_n(\mathrm{ft}(\Gamma))$
with $\Gamma = \mathrm{ft}(\Gamma).\mathrm{st}(\Gamma)$.
Similarly we have a stratification of the sort of substitutions
$\hom_{n,m}(\Delta,\Gamma)$ for $\Delta$ in $\ctx_n$ and $\Gamma$ in $\ctx_m$.
The resulting models are {\em internal contextual cwfs} in a cwf.

\begin{remark}
Generalized algebraic presentations of contextual categories (C-systems) have been suggested by Voevodsky \cite{voevodsky:c-systems} and Cartmell \cite{cartmell:gat-contextual}.
\end{remark}

%\subsection{Cwfs with universe tower structures}

%\subsection{Cwfs with universe tower structures}

%
%The first formulation of intuitionistic type theory with an infinite sequence of universes is due to Martin-Löf
%\cite{martinlof:predicative}. Rules for cumulativity (or lifting) were added in Martin-Löf \cite{martinlof:hannover}. Both formulations have an infinite sequence of universes indexed by external natural numbers, and as a consequence the theories have infinitely many rules.
%
%We shall now formalize a notion of finitary generalized algebraic theory closely related to Martin-Löf's cumulative version. The external natural number indices will be represented by internal level indices in the generalized algebraic theory. To this end we introduce a new sort symbol
%$$
%\vdash \Level
%$$
%in addition to the previous four sort symbols. (However, we do not have a {\em type} of levels.) An element $n : \Level$ represent an external natural number. (Note that the new sort $\Level$ corresponds to adding a new form of judgment $\vdash n\  \Level$ to the formal system.)
%
%We add operator symbols for levels
%\begin{eqnarray*}
%&\vdash& 0 : \Level\\
%n : \Level &\vdash& \s(n) : \Level
%\end{eqnarray*}
%and operator symbols for types and terms:
%\begin{eqnarray*}
%n : \Level, \Gamma : \ctx &\vdash& (\U_n)_\Gamma : \Ty(\Gamma)\\
%n : \Level, \Gamma : \ctx, a : \Tm(\Gamma,(\U_n)_\Gamma) &\vdash& {\Ta_n}(a) : \Ty(\Gamma)\\
%n : \Level, \Gamma : \Ctx &\vdash& (\N^n)_\Gamma : \Tm(\Gamma,(\U_n)_\Gamma) \\
%n : \Level, \Gamma : \Ctx,
%a : \Tm(\Gamma,(\U_n)_\Gamma),
%b :  \Tm(\Gamma  .  \Ta_n(a), (\U_n)_\Gamma))
%&\vdash&
% \Pi^n(a,b) : \Tm(\Gamma,(\U_n)_\Gamma)\\
%n : \Level, \Gamma : \Ctx &\vdash& (\U^n)_\Gamma \in \Tm(\Gamma,(\U_{\s(n)})_\Gamma)\\
%n : \Level, \Gamma : \Ctx, a : \Tm(\Gamma,(\U_n)_\Gamma) &\vdash& \Ta_n^{n+1}(a)\footnote{200804: should we change notation $\Ta_n^{n+1}$ to $\Ta^n$?} : \Tm(\Gamma,(\U_{\s(n)})_\Gamma)
%\end{eqnarray*}
%The last operator symbol is the lifting (or cumulativity) operator. We have the following equations
%\begin{eqnarray*}
%\Tan((\N^n)_\Gamma) &=& \N_\Gamma\\
%\Ta_n(\Pi^{n}(a,b)) &=& \Pi(\Ta_n(a),\Tan(b))\\
%\Ta_{\s(n)}((\U^n)_\Gamma ) &=& (\U_n)_\Gamma\\
%%&\Ta_{n+1}(\Ta_n^{n+1}(a)) &=& \Ta_n(a)\\
%\Ta^{n+1}_n((\N^n)_\Gamma) &=& (\N^{\s(n)})_\Gamma\\
%\Ta^{n+1}_n(\Pi^{n}(a,b)) &=& \Pi^{\s(n)}(\Ta^{n+1}_n(a),\Ta^{n+1}_n(b))
%\end{eqnarray*}
%Finally, all operator symbols commute with substitution:
%\begin{eqnarray*}
%{(\Un)}_\Gamma [ \gamma ] &=& {(\Un)}_\Delta\\
%\Tan(a) [ \gamma ] &=& \Tan(a[ \gamma ] )\\
%\N^n_\Gamma [ \gamma ] &=&\N^n_\Delta\\
%\Pi^{n}(a,b)[ \gamma ] &=& \Pi^{n}(a [ \gamma ], b[ \gamma^+ ])\\
%\U^n[\gamma] &=& \U^n\\
%\Ta_n^{n+1}(a)[\gamma]  &=& \Ta_n^{n+1}(a[\gamma])
%\end{eqnarray*}
%Mention a la Russell initiality?
%
%\subsection{Removing cumulativity} Agda, cumulativity up to equivalence.
%
%\subsection{Cwfs with universe polymorphic tower structures} Here we need to extend the cwf-framework further to take into account contexts with level variables, etc.
%\newpage
\section{Related work}

Streicher \cite{streicher:semtt} defined {\em doctrines of constructions} (contextual categories with suitable extra structure) as a notion of model of the Calculus of Constructions. He also constructed a term model and remarked that it is an initial object in a category of doctrines of constructions.
%Streicher \cite{streicher:semtt} constructs a term model of the Calculus of Constructions and remarks that it is an initial object in a category of {\em doctrines of constructions} (contextual categories with suitable extra structure).
Recently, Brunerie et al \cite{brunerie:initiality} presented a formalized proof in the Agda system that a formal system for Martin-Löf type theory forms an initial object in a category of contextual categories with extra structure for
the type formers.

More generally, Voevodsky \cite{voevodsky:initiality} outlined a new vision of the theory of syntax and semantics of dependent type theories. In this vision formal systems for dependent type theory are proved to be initial in suitable categories of models ({\em the initiality conjecture}). The above-mentioned contributions by Streicher and Brunerie et al are two examples of such characterizations. However, Voevodsky's aim was to go further and characterize a whole class of type theories and prove a general initiality result for this class with the aim to form the basis for a general metatheory of dependent type theory. Our work can be viewed as a contribution to Voevodsky's programme, since we prove an initiality theorem for the whole class of finitely presented generalized algebraic theories. Another characterization of a general class of dependent type theories and their initial models has been proposed by Uemura \cite{uemura:general-framework}. Another related contribution is Palmgren and Vickers' \cite{palmgrenvickers} construction of initial models of essentially algebraic theories.

Altenkirch and Kaposi \cite{altenkirch:qiits} gave several examples of {\em quotient inductive-inductive types (qiits)}. Their main example is a definition of dependent type theory with $\Pi$-types and a universe, as a simultaneous definition in the Agda system \cite{agda-wiki} of the data types $\Ctx$ of contexts, $\Sub(\Delta,\Gamma)$ of substitutions, $\Ty(\Gamma)$, and $\Tm(\Gamma,A)$ of terms. Their definition is {\em inductive-inductive} \cite{nordvallforsberg:iids}, since the index sets of $\Sub, \Ty,$ and $\Tm$ are generated simultaneously, and as a consequence are not indexed inductive definitions in the usual sense where the index sets are fixed in advance. Furthermore, it is a quotient inductive-inductive type since they also have constructors for identity types, as in a {\em higher inductive type}.
There is a close relationship between this qiit and our initial internal cwf with $\Pi$-types and a universe. Like our definition, their qiit-definition uses cwf-combinators. Moreover, our sort symbols correspond to their formation rules (data type constructors), our operator symbols correspond to their introduction rules (constructors), and our equations correspond to their propositional identities. However, a differerence is that our equations are judgmental equalities while theirs are propositional. As a consequence they use transport maps when moving between identical types.

%\footnote{Insert: Furthermore, Altenkirch et al \cite{altenkirch:thessaloniki} introduce a general schema for qiits and Kaposi, Kov{\'{a}}cs, and Altenkirch \cite{kaposi:qiits} construct initial algebras for them. For these constructions they work in cwfs with $\Pi$-types, intensional identity types, and a universe, while we work in plain cwfs. Although generalized algebraic theories and qiits are related notions, neither is a generalization of the other. Gat is a basic dependently typed notion and independent of Martin-Löf type theory, whereas qiit is the latest in the series of more and more general notions of inductive type (inductive family, inductive-recursive type and family, inductive-inductive type, higher inductive type) extending intensional Martin-Löf type theory.
%
%Kaposi, Kov{\'{a}}cs, and Altenkirch \cite{kaposi:qiits} develop a general theory of qiits. This includes a notion of signature for a qiit, the notion of an algebra of such a signature, and a construction of initial algebras. There are obvious parallels with our work, but there are also fundamental diffe\FYI{presentation} $\Sigma$rences:
%}

%Kaposi, Kov{\'{a}}cs, and Altenkirch \cite{kaposi:qiits} developed a general theory of qiits. This includes a notion of signature for a qiit, the notion of an algebra of such a signature, and a construction of initial algebras. For these constructions they work in cwfs with $\Pi$-types, identity types, and a universe. This is in contrast to our work which is based on plain cwfs without extra structure for type formers. Although generalized algebraic theories and qiits are related notions, neither is a generalization of the other. Gat is a basic notion independent of Martin-Löf type theory, whereas qiit is the latest in a series of generalizations of inductive type (inductive family, inductive-recursive type and family, inductive-inductive type, higher inductive type) extending intensional Martin-Löf type theory.

The notion of qiit is the latest in a series of generalizations of inductive type (inductive family, inductive-recursive type and family, inductive-inductive type, higher inductive type) extending Martin-Löf type theory. Kaposi, Kov{\'{a}}cs, and Altenkirch \cite{kaposi:qiits} developed a general theory of qiits. This includes a notion of signature for a qiit, the notion of an algebra of such a signature, and a construction of initial algebras. In particular, they introduce a domain-specific type theory of signatures (and implement it in Agda), and define a signature for a qiit to be a context in this theory. It would be interesting to try to relate such signatures for qiits to the \FYI{presentations} of generalized algebraic theories in our paper, but this is beyond the scope of the present paper.


%Using the terminology of qiits, we here provide a definition of a class of valid qiits (in our modified sense) together with a class of models for each of them.


%Given that we have definition of a \FYI{presentation} $\Sigma$ and a semantics for $\T_\Sigma$ we could ask whether we could add the construction of these to type theory. This would make most sense in extensional type theory, since we can then move seamlessly between propositional and judgmental equalities. It is however, not clear how to provide canonical form semantics in the sense of Martin-Löf's meaning explanations for this notion. However, although our development takes place in set theory, everything we do is constructive and could be formalized in CZF presumably.

\subsection*{Acknowledgements}
We are grateful to the anonymous referees for constructive criticism and pointers to related work. We would also like to thank Andrej Bauer, John Cartmell, and Christian Sattler for useful comments.

\bibliographystyle{plain}
\bibliography{refs}
%\bibliography{}
%\documentclass{lmcs}
%\usepackage{etex}
\usepackage[utf8]{inputenc}

\usepackage{color}
\usepackage{hyperref}
\usepackage{float}
\usepackage{amsmath}
\usepackage{amsfonts}
\usepackage{amsthm}
\usepackage{amssymb}
\usepackage{proof}
\usepackage{mathpartir}
\usepackage{mathrsfs}
\usepackage{stmaryrd}
\usepackage{cmll}
\usepackage{enumerate}
\usepackage{url}

%usepackage{graphicx}
%\usepackage[all]{xy}
\usepackage{listings}
%\usepackage{todonotes}
%\DeclareMathOperator{\Ker}{Ker}
%\DeclareMathOperator{\nf}{nf}
%\DeclareMathOperator{\domain}{dom}
%\DeclareMathOperator{\codomain}{cod}
%\DeclareMathOperator{\cod}{cod}
%\DeclareMathOperator{\dom}{dom}
%\DeclareMathOperator{\ctxof}{ctx-of}
%\DeclareMathOperator{\typeof}{type-of}
%\DeclareMathOperator{\fix}{fix}

%\newcommand{\vdashS}{\ \vdash\ }
%\newcommand{\vdashS}{\vdash}
\newcommand {\emptyContext}{1}
\newcommand {\emptyContextI}{\diamond}
\newcommand {\emptyContextS}{\textbf 1}
\newcommand {\contextExtension}[2]{#1 \cdot #2}
\newcommand {\contextExtensionI}[2]{#1 \cdot #2}
\newcommand {\contextExtensionS}[2]{#1 \cdot #2}
\newcommand {\contextExtensionC}[2]{#1 \cdot_\C #2}

\newcommand {\GammaA}{\contextExtension \Gamma A}
\newcommand {\DeltaA}{\contextExtension \Delta A}
\newcommand {\setI}{\text{set}}
\newcommand {\setS}{\textbf{set}}
\newcommand {\depProd}[3]{\Pi(#1, #2, #3)}
\newcommand {\depProdI}[2]{\Pi(#1, #2)}
\newcommand {\depProdS}{\textbf{$\Pi$}}
\newcommand {\el}[2]{{\tt el}(#1, #2)}
\newcommand {\elI}[1]{{\tt el}(#1)}
\newcommand {\elS}{\textbf{el}}
\newcommand {\subType}[4]{{\tt subType}(#3, #4, #1, #2)}
\newcommand {\subTypeI}[2]{\text{subType}(#1, #2)}
\newcommand {\subTypeS}[2]{#1\{#2\}}
\newcommand{\subTypeC}[4]{\mathrm{subType}_\C(#3, #4, #1, #2)}
\newcommand {\q}[2]{{\tt q}_{#1, #2}}
\newcommand {\qI}{{\tt q}}
\newcommand {\qS}{\textbf{q}}
\newcommand{\lambdaAbs}[4]{\lambda(#1, #2, #3, #4)}
\newcommand{\lambdaAbsI}[1]{\lambda(#1)}
\newcommand{\lambdaAbsS}{\textbf{$\lambda$}}
\newcommand{\application}[5]{{\tt app}(#1, #2, #3, #4, #5)}
\newcommand{\applicationI}[2]{\text{app}(#1, #2)}
\newcommand{\applicationS}{\textbf{application}}
\newcommand{\subTerm}[5]{{\tt subTerm}(#4, #5, #1,#2,#3)}
\newcommand{\subTermI}[2]{\text{subTerm}(#1,#2)}
\newcommand{\subTermS}[2]{#1\{#2\}}
\newcommand{\idSub}[1]{{\tt id}(#1)}
\newcommand{\idSubI}{{\tt id}}
\newcommand{\idSubS}{\text{id}}
\newcommand{\proj}[2]{{\tt p}(#1, #2)}
\newcommand{\projI}{{\tt p}}
\newcommand{\projS}{\textbf{p}}
\newcommand{\comp}[5]{{\tt comp}(#1, #2, #3, #4, #5)}
\newcommand{\compI}[2]{{\tt comp}(#1, #2)}
\newcommand{\compS}[2]{#2 \circ #1}
\newcommand{\emptySub}[1]{\emptySubI_{#1}}
\newcommand{\emptySubI}{\langle\rangle}
\newcommand{\emptySubS}{\textbf !}
\newcommand{\extSub}[5]{\text{extension}(#1, #2, #3, #4, #5)}
\newcommand{\extSubI}[2]{\text{extension}(#1, #2)}
\newcommand{\extSubS}[2]{\langle #1, #2\rangle}
\newcommand{\Ctx}{\mathrm{Ctx}}
\newcommand{\Sub}{\mathrm{Sub}}
\newcommand{\Ty}{\mathrm{Ty}}
\newcommand{\Tm}{\mathrm{Tm}}
\newcommand{\C}{{\mathcal C}}
\newcommand{\I}{{\mathcal I}}
\newcommand{\T}{{\mathcal T}}

\newcommand{\Timp}{\T_{\text{imp}}}
\newcommand{\arrow}{{\rightarrow}}
\newcommand{\RawCtx}{{\tt Ctx}}
\newcommand{\RawSub}{{\tt Sub}}
\newcommand{\RawTy}{{\tt Ty}}
\newcommand{\RawTm}{{\tt Tm}}

\newcommand{\scomp}[6]{\mathrm{comp}_#1(#2, #3, #4, #5,#6)}

\newcommand{\inte}[1]{\llbracket #1 \rrbracket}
\newcommand{\intCtx}[1]{\llbracket #1 \rrbracket}
\newcommand{\intSub}[3]{\llbracket #3 \rrbracket_{#1,#2}}
\newcommand{\intTy}[2]{\llbracket #2 \rrbracket_#1}
\newcommand{\intTm}[3]{\llbracket #3 \rrbracket_{#1,#2}}
\newcommand{\ICtx}{{\I_0}}
\newcommand{\ISub}{{\I_1}}
\newcommand{\ITy}{{\I_2}}
\newcommand{\ITm}{{\I_3}}
\newcommand{\iniCtx}[1]{\overline{\llbracket #1 \rrbracket}}
\newcommand{\iniSub}[3]{\overline{\llbracket #3 \rrbracket}_{#1,#2}}
\newcommand{\iniTy}[2]{\overline{\llbracket #2 \rrbracket}_{#1}}
\newcommand{\iniTm}[3]{\overline{\llbracket #3 \rrbracket}_{#1,#2}}

\newcommand{\mejl}[3]{#1$\bigcirc\!\!\!\!\!\alpha\,$#2${}_{\cdot}$#3}

\newcommand{\bbN}[0]{{\mathbb N}}
\newcommand{\bbZ}[0]{{\mathbb Z}}
\newcommand{\bbQ}[0]{{\mathbb Q}}
\newcommand{\bbR}[0]{{\mathbb R}}
\newcommand{\bbB}[0]{{\mathbb B}}
\newcommand{\mU}[0]{{\mathcal U}}
\newcommand{\mT}[0]{{\mathcal T}}
\newcommand{\ve}[0]{{\varepsilon}}
\newcommand{\vf}[0]{{\varphi}}

\newcommand{\wellincluded}[0]{\, \Subset \,}

\newcommand{\memof}[0]{\, \epsilon \,}
\newcommand{\subseteqof}[0]{\, \dot{\subseteq} \,}

\newcommand{\mono}[0]{\to/ >->/}
\newcommand{\pto}[0]{\rightharpoondown}
\newcommand{\wellcov}[0]{{\lll}}
\newcommand{\waybelow}[0]{\ll}
\newcommand{\formint}[0]{\land}
\newcommand{\cov}[0]{{\, \lhd \,}}
\newcommand{\kov}[0]{{\, \lessdot \,}}
\newcommand{\kkov}[0]{{\, <: \,}}
\newcommand{\mutcov}[0]{\sim}
\newcommand{\balcov}[0]{\sqsubseteq}
\newcommand{\bal}[0]{{\sf b}}
\newcommand{\sat}[1]{{\rm Sat}(#1)}
\newcommand{\set}[0]{{\rm Set}}
\newcommand{\Set}[0]{{\bf Set}}
\newcommand{\true}[0]{{\sf T}}
\newcommand{\monus}{\stackrel{{}^{\scriptstyle .}}{\smash{-}}}

\newcommand{\refl}[0]{{\rm ref}}

\newcommand{\inl}[1]{{\sf inl}(#1)}
\newcommand{\inr}[1]{{\sf inr}(#1)}
\newcommand{\nat}[0]{{\mathbb N}}

\newcommand{\nattype}[0]{{\rm N}}
\newcommand{\bool}[0]{{\rm Bool}}
\newcommand{\ext}[1]{\langle #1 \rangle}


\newcommand{\bintree}[0]{{\rm T}_2}

%\newcommand{\sequent}[0]{\vdash}


\renewcommand{\conv}[0]{\approx}
\newcommand{\intimpl}[0]{\supset}

\newcommand{\omitthis}[1]{}

\newcommand{\changenote}[1]{}


 \newcommand{\Id}[0]{{\rm I}}
 

\newcommand{\longtext}[1]{}
\newcommand{\shorttext}[1]{}
\newcommand{\commentaway}[1]{}

\newcommand{\Setoid}[0]{{\bf Setoid}}

\definecolor{Red}{rgb}{1,0,0}
\newcommand{\red}[1]{{\color{Red}#1}}
%\newcommand{\red}[1]{}
\renewcommand{\bar}[1]{\overline{#1}}

%\newdir{pb}{:(1,-1)@^{|-}}
%\def\pb#1{\save[]+<16 pt,0 pt>:a(#1)\ar@{pb{}}[]\restore}

\newcommand{\Fam}{\textbf{Fam}}
\newcommand{\nilc}{1}
\newcommand{\cext}{.}
\newcommand{\indexed}[1]{\boldsymbol{#1}}
\newcommand{\Cat}{\mathrm{Cat}}
\newcommand{\op}{\text{op}}
\newcommand{\iso}{\cong}
\newcommand{\subst}[1]{\langle #1 \rangle}
\newcommand{\applyopen}[2]{\{ #1 \}  #2 }

% added by Marc to get things going. IMPROVE!

\def\N{\mathsf{N}}
\def\U{\mathsf{U}}
\def\F{\mathsf{F}}
\def\app{\mathsf{app}}
\def\Cop{\C^\op}
\def\Cobj{{\mathcal{C}_0}}
\def\p{\mathrm{p}}
\def\q{\mathrm{q}}
\newcommand{\tuple}[1]{\langle #1 \rangle}

\newtheorem{remark}{Remark}
\newtheorem{definition}{Definition}

%\def\N{\mathrm{N}}
\def\U{\mathrm{U}}
\def\p{{\tt p}}
\def\ev{{\tt ev}}
\def\q0{{\tt q}}
\def\r{{\tt r}}
\def\arrow{\rightarrow}
\def\Hom{\mathrm{Hom}}
\def\GammaA{\Gamma_{+,\times}}
\def\GammaCL{\Gamma_{\mathrm{CL}}}

\def\Dp{\mathrm{D}_p}
\def\notnotDp{\neg\neg\Dp}
\def\F{\mathcal{F}}
\def\HA{\mathbf{HA}}
\def\PA{\mathbf{PA}}
\def\I{\mathrm{I}}
\def\refl{\mathrm{r}}
\def\id{{\tt id}}
\def\idT{\mathrm{id}_\T}
\def\idC{\mathrm{id}_\C}
\newcommand{\pair}{\mathrm{pair}}
\newcommand{\fst}{\mathrm{fst}}
\newcommand{\interp}[1]{ \overline{\llbracket #1 \rrbracket}}
\newcommand{\Cwf}{\textbf{CwF}}
\newcommand{\Cwfs}{\Cwf_s}
\newcommand{\D}{\mathcal{D}}
\newcommand{\snd}{\mathrm{snd}}
\newcommand{\ap}{\mathrm{app}}
%\newcommand{\app}{\mathrm{app}}
\newcommand{\ini}[1]{\iniCtx{[#1]}}
\DeclareMathOperator{\cod}{cod}
\DeclareMathOperator{\dom}{dom}
\DeclareMathOperator{\ctxof}{ctx-of}
\DeclareMathOperator{\typeof}{type-of}
\newcommand{\vdashS}{\ \vdash\ }
\DeclareMathOperator{\domain}{dom}
\DeclareMathOperator{\codomain}{cod}


\newcommand{\isoCtx}[1]{\stackrel{#1}{\cong}}
\newcommand{\isoTy}[2]{\stackrel{#1}{\cong}_{#2}}
\newcommand{\equSub}[1]{=_{#1}}
\newcommand{\equTm}[2]{=_{#1,#2}}
\newcommand{\TT}{\mathbf{T}}

\newtheorem{theorem}{Theorem}
\newcommand{\s}{\mathrm{s}}
\newcommand{\Rec}{\mathrm{R}}
\newcommand{\Ta}{\mathrm{T}}
\newcommand{\ta}{\mathrm{t}}
\newcommand{\Ru}{\mathcal{R}}
\newcommand{\Nhat}{\hat{\N}}
\newcommand{\Pihat}{\hat{\Pi}}
\newcommand{\Tan}{\Ta_n}
\newcommand{\Un}{\U_n}
\newcommand{\Nhatn}{\N^n}
\newcommand{\Pihatn}{\Pi^n}
\newcommand{\Nn}{\Nhatn}
\newcommand{\Pin}{\Pihatn}
\newcommand{\TRu}{\Ta_\Ru}
\newcommand{\URu}{\U_\Ru}
\newcommand{\NRu}{\N_\Ru}
\newcommand{\PiRu}{\Pi_\Ru}
\newcommand{\TRun}{{(\Ta_\Ru)}_n}
\newcommand{\URun}{{(\U_\Ru)}_n}
\newcommand{\NRun}{{(\N_\Ru)}^n}
\newcommand{\PiRun}{{(\Pi_\Ru)}^n}
\newcommand{\TRum}{{(\Ta_\Ru)}_m}
\newcommand{\URum}{{(\\U_\Ru)}_m}
\newcommand{\TC}{\Ta_\C}
\newcommand{\UC}{\U_\C}
\newcommand{\NC}{\N_\C}
\newcommand{\PiC}{\Pi_\C}
\newcommand{\Level}{\mathrm{Level}}
\def\Sort{\mathcal{S}}
\def\Op{\mathcal{O}}
\def\Eq{\mathcal{E}}
\def\D{\mathcal{D}}
\def\V{\mathrm{V}}
\def\Cwf{\mathbf{CwF}}
\def\Obj{\mathrm{obj}}
\def\Ctx{\mathrm{Ctx}}
\def\Hom{\mathrm{hom}}
\def\id{\mathrm{id}}
\def\Mon{\mathrm{M}}
\def\idmon{\mathrm{e}}
\def\comp{\mathrm{*}}
\newcommand{\ctx}{\mathrm{ctx}}
\newcommand{\sub}{\mathrm{sub}}
\newcommand{\ty}{\mathrm{ty}}
\newcommand{\tm}{\mathrm{tm}}
%\newcommand{\hom}{\mathrm{hom}}
\def\nt{\mathrm{nat}}
\def\fun{\mathrm{fun}}


\title[Generalized Algebraic Theories and Categories with Families]{A Note on Generalized Algebraic Theories\\and Categories with Families}\author{Marc Bezem, Thierry Coquand, Peter Dybjer, Mart\'in Escard\'o}

\begin{document}

\maketitle

\begin{abstract}
We give a new syntax independent definition of the notion of a finitely presented generalized algebraic theory as an initial object in a category of categories with families (cwfs) with extra structure. To this end we define inductively how to build a valid signature $\Sigma$ for a generalized algebraic theory and the associated category $\Cwf_\Sigma$ of cwfs with a $\Sigma$-structure and cwf-morphisms that preserve $\Sigma$-structure on the nose.  Our definition refers to the purely semantic notions of {\em uniform family} of contexts, types, and terms. Furthermore, we show how to syntactically construct initial cwfs with $\Sigma$-structures. This result can be viewed as a generalization of Birkhoff's completeness theorem for equational logic. It is obtained by extending Castellan, Clairambault, and Dybjer's construction of an initial cwf. We provide examples of generalized algebraic theories for monoids, categories, categories with families, and categories with families with extra structure for some type formers of dependent type theory. The models of these are internal monoids, internal categories, and internal categories with families (with extra structure) in a category with families. Finally, we show how to extend our definition to some generalized algebraic theories that are not finitely presented, such as the theory of contextual categories with families.
\end{abstract}

\section{Introduction}

Martin-Löf type theory can be characterized in a syntax independent way as the initial category with families (cwf)  with extra structure for the type formers \cite{castellan:tlca2015,castellan:lmcs}. The main contribution of this note is a similar syntax independent characterization of the notion of finitely presented generalized algebraic theory as the initial cwf with extra structure.

Generalized algebraic theories (gats) were introduced by Cartmell in his PhD thesis \cite{cartmell:phd} as a dependently typed generalization of many sorted algebraic theories. Each gat is specified by a signature with (possibly infinite) sets of sort symbols, operator symbols, and equations. Cartmell's definition of gats \cite{cartmell:phd,cartmell:apal} is based on a notion of {\em derived rule} expressed in terms of a traditional syntactic system for dependent type theory. He also defines a notion of model whereby sort symbols are interpreted as families of sets.

Categories with families (cwfs) \cite{dybjer:torino} were introduced as a new notion of model of dependent type theory. Cwfs arise by reformulating the notion of category with attributes in Martin Hofmann's sense \cite{hofmann:csl}. The key point is that cwfs arise as models of a certain generalized algebraic theory closely related to Martin-Löf's substitution calculus \cite{martinlof:gbg92}. As such the notion of cwf becomes a useful intermediary between traditional syntactic systems for dependent type theory and a variety of categorical notions of model.

The gat of cwfs is thus a kind of idealized formal system of dependent type theory. In contrast to Martin-Löf's substitution calculus, and other syntactic systems for dependent type theory, it is {\em not} formulated in terms of grammars and inference rules for the forms of judgment of type theory. Instead it is formulated in terms of the sort symbols (corresponding to the judgment forms), operator symbols (corresponding to the formation, introduction, and elimination rules), and equations (corresponding to the equality rules for the type formers) of the gat. Some of the general reasoning (about equality, substitution, and assumptions) is taken care of by the underlying infrastructure of dependent types. This makes it possible to abstract away from details in the formulation of grammars and inference rules. In contrast to the various syntactic systems, the gat of cwfs has a canonical flavour. 
%We may define dependent type theory in a syntax independent way as the initial object in a category of cwfs with extra structure for interpreting the type formers.
%However, the reader may now object that this looks like a circular definition.
%We learn what dependent type theory is if we already know what dependent type theory is!

In this note we explore the interdependence between gats and cwfs. We already explained that cwfs can be defined as models of a gat. 
%(This gat is presented in section \ref{gat-cwf}.) 
In the other direction, the notion of gat relies on the notion of cwf, in the sense that the latter models the underlying infrastructure of dependent types. 

%We shall here define a new finitely presented notion of gat and simultaneously a general categorical notion of model. To this end we define what it means to be a valid signature $\Sigma$ for a gat and the associated category $\Cwf_\Sigma$ with extra structure for $\Sigma$. Our definition refers to {\em uniform families} of contexts, types, and terms in $\Cwf_\Sigma$, a purely semantic notion. Afterwards, we construct initial objects $\T_\Sigma \in \Cwf_\Sigma$ by extending Castellan, Clairambault, and Dybjer's  construction of an initial object in the category $\Cwf$ of cwfs \cite{castellan:tlca2015,castellan:lmcs}.

%Once one has the appropriate definition of gats in terms of cwfs, the details become natural and there are no surprises. The definition becomes canonical once we accept the abstract definition of dependent type theories as initial cwfs with extra structure.

\subsection*{Plan of the paper}

In Section 2 we recall the definition of the category $\Cwf$ of categories with families and morphisms preserving cwf-structure on the nose. Section 3 contains our main definition of a syntax independent notion of valid signature $\Sigma$ for a gat and the category $\Cwf_\Sigma$ of cwfs with a $\Sigma$-structure. In Section 4 we construct an initial object $\T_\Sigma$ in $\Cwf_\Sigma$. In Section 5 we show several examples of gats: for monoids, categories, cwfs, and cwfs with extra structure for one universe. We point out that cwfs with extra structure for gats of monoids, categories, cwfs are cwfs with an internal monoid, category, and cwf, respectively. We also sketch how to extend our approach to some countably presented gats, and show the example of contextual cwfs, a variant of Cartmell's contextual categories \cite{cartmell:phd,cartmell:apal}. Finally, in Section 6 we discuss related work, for example relating to Voevodsky's initiality conjecture \cite{voevodsky:initiality} and Altenkirch and Kaposi's quotient inductive-inductive types \cite{altenkirch:qiits}.

Our development can be formulated in a constructive set theory,
as described for instance by Aczel \cite{MR519801}, although the set theory
we use for formulating the notion of cwf with a $\Sigma$-structure is probably
much weaker. As emphasized by Voevodsky~\cite{voevodsky:initiality}, we study structures invariant
under {\em isomorphisms} and not under {\em equivalences}, and it is actually misleading
to call them ``category'' (and this is why Voevodsky used the term ``$C$-system''
for what Cartmell called ``contextual category'').
As he also noticed, this
important distinction between categories and notions invariant under isomorphisms becomes
precise in the setting of univalent foundations where not all collections of objects
are constructed from sets.
%% We remark that we work in set-theoretic metalanguage throughout the paper. Everything we do is constructive and should be possible to formalize in Aczel's CZF \cite{MR519801}. We also remark that it would be interesting to work in type-theoretic metalanguage, for example, in homotopy type theory and Voevodsky's univalent foundations. As such it could contribute to the recent work by several researchers inspired by Voevodsky's initiality conjecture.

%\subsection*{To the memory of Martin Hofmann}
%The content of this paper  is closely related to several of Martin's
%important contributions to the semantics of dependent type theory.
%Martin was an extremely gifted and generous person,
%many researchers have benefited from his collaboration.
%He is truly missed.

\subsection*{Remarks on terminology and notation}
Like Cartmell, we have chosen to use the term {\em sort symbol} from many-sorted universal algebra. However, in our semantic notion of signature sort symbols are interpreted as {\em type families} in a cwf. A cwf consists of a base category where the objects of the base category are (semantic) contexts and the morphisms are (semantic) substitutions. Moreover, we have a family-valued presheaf mapping contexts to families of (semantic) terms indexed by (semantic) types. Thus the reader should be aware of the mismatch between the word {\em sort} from universal algebra and the word {\em type} in the cwf semantics.

Another possible source of confusion is that cwfs appear on two different levels. In Section \ref{sec:def_cwf} we recall the definition of cwf in set-theoretic metalanguage, where we use $\Ty$ to denote the family of types indexed by contexts and $\Tm$ to denote the family of terms indexed by contexts and types. This notion of cwf is then used to define the semantic notions of signature and category of models of a gat. Then in Section \ref{gat-cwf} we define the gat of {\em internal cwfs}. This gat has sort symbols $\ty$ for {\em internal types} and $\tm$ for {\em internal terms} using lower case to highlight the difference from $\Ty$ and $\Tm$ in the model cwf. 

Furthermore, we often use the same notation both on the semantic and the syntactic level. For example, in Section \ref{gat-sig-mod},  where we are syntax independent, the letter $S$ denotes a semantic sort symbol, whereas in Section \ref{initial-gat}, where we construct the initial model, it denotes a syntactic sort symbol.

\subsection*{To the memory of Martin Hofmann}

We have written this paper to honour the memory of Martin Hofmann. The topic is categorical models of dependent type theory, an area that Martin made seminal contributions to. In particular, he did much to clarify the relationship between intensional and extensional type theory.
His thesis was the first investigation of the setoid model \cite{hofmann:phd}. His and Streicher's groupoid model \cite{hofmann:groupoid} refutes uniqueness of identity proofs and identity reflection, the two rules that separate extensional from intensional type theory. The groupoid model also validates the principle of universe extensionality, a special case of Voevodsky's univalence axiom. As a consequence this work is a forerunner to Voevodsky's univalent foundations. 

Other notable contributions to dependent type theory include the interpretation of extensional type theory in locally cartesian closed categories \cite{hofmann:csl,curien-garner-hofmann}, the use of a presheaf model to prove that the Logical Framework version of Martin-Löf type theory is a conservative extension of the original version \cite{hofmann:cambridge}, and a method for eliminating extensional identity types \cite{hofmann:conservativity}. Martin also wrote a widely read introduction to the syntax and semantics of dependent types \cite{hofmann:cambridge}. 

Martin was an extremely gifted and generous person,
many researchers have benefited from his collaboration.
He is truly missed.

\section{Categories with families}\label{sec:def_cwf}

\subsection{The category of cwfs and strict cwf-morphisms}

%In this section (whole paper?), the meta-language is set-theoretic.
%Much of the vocabulary is category-theoretic, but we freely use
%equality of objects in a categorical context.
%We first define the category $\Fam$, and then the category $\Cwf$.

\begin{definition}\label{def:catFam}
$\Fam$ is a category whose objects are
set-indexed families of sets, denoted as $(U_x)_{x\in X}$.
A morphism of $\Fam$ with source $(U_x)_{x\in X}$ and target $(V_y)_{y\in Y}$
consists of a re-indexing function $f: X\to Y$ together with a family
$(g_x)_{x\in X}$ of functions $g_x : U_x \to V_{f(x)}$. %, for all $x\in X$.
\end{definition}

The next step is to define the category $\Cwf$.
We split this definition in two: first the objects,
which are called \emph{categories with families}, in Definition~\ref{def:Cwfobj},
and then the morphisms in Definition~\ref{def:Cwfmor}.
Since $\Cwf$ has been developed as a categorical framework for the semantics of
type theory, much of the terminology (contexts, substitutions,
types, terms) refers to the syntax of type theory,
suggesting the intended interpretation of this syntax in the
so-called $\Cwf$-semantics.

The main novelty of this note is to use $\Cwf$ as a framework
to define a new notion of a generalized algebraic theory.
Contexts, substitutions, types, and terms also make
sense in relation to gats.
% Note however that we use the term {\em ``sort symbol"} from universal algebra for what might have been called ``type symbol".

\begin{definition}\label{def:Cwfobj}
A category with families (cwf) consists of the following data:

\begin{itemize}
\item A category $\C$;

\item A $\Fam$-valued presheaf on $\C$, that is, a functor
$T : \Cop \to \Fam$;

\item A terminal object $1\in \C$, and unique maps
$\tuple{}_\Gamma \in \C(\Gamma, 1)$ for all objects $\Gamma$ of $\C$;

\item Operations ${\cext\,},~\tuple{\_,\_},~\p$ and $\q$
explained in the following paragraphs.
These four operations and their associated equations
are referred to as \emph{context comprehension}.
\end{itemize}

We let $\Gamma, \Delta,\ldots$ range over objects of $\C$,
and refer to them as \emph{contexts}.
We let $\delta, \gamma,\ldots$ range over morphisms,
and refer to them as \emph{substitutions}.
We refer to $1$ as the \emph{empty} context; the terminal maps
$\tuple{}_\Gamma$ represent the \emph{empty} substitutions.

If $T(\Gamma) = (U_x)_{x\in X}$, we write $\Ty(\Gamma)$ for the set $X$.
We call the elements of $\Ty(\Gamma)$ \emph{types in context $\Gamma$},
and let $A, B, C$ range over such types.
Furthermore, for $A \in \Ty(\Gamma)$, we write $\Tm(\Gamma, A)$ for the set $U_A$
and call the elements of $\Tm(\Gamma, A)$
\emph{terms of type $A$ in context $\Gamma$}.

For $\gamma : \Delta \to \Gamma$,
the functorial action of $T$ yields a morphism
\[
T(\gamma) \in  \Fam\left((\Tm(\Gamma, A))_{A\in \Ty(\Gamma)}, % \to
                (\Tm(\Delta, B))_{B\in \Ty(\Delta)}\right)
\]
consisting of a reindexing function $\_\,[\gamma] : \Ty(\Gamma) \to
\Ty(\Delta)$ referred to as \emph{substitution in types}, and for each $A\in
\Ty(\Gamma)$ a function $\_\,[\gamma] : \Tm(\Gamma, A) \to \Tm(\Delta,
A[\gamma])$, referred to as \emph{substitution in terms}.

Now we turn to the explanation of the operations
${\cext\,},~\tuple{\_,\_},~\p,~\q$.
Given $\Gamma \in \C$, $A \in \Ty(\Gamma)$, $\gamma : \Delta \to \Gamma$,
and $a\in \Tm(\Delta, A[\gamma])$, we have
\[
\Gamma \cext A \in \C
\quad\qquad
\p_{\Gamma, A} : \Gamma \cext A \to \Gamma
\quad\qquad
\q_{\Gamma, A} \in \Tm(\Gamma\cext A, A[\p_{\Gamma,A}])
\quad\qquad
\tuple{\gamma, a}_A : \Delta \to \Gamma \cext A.
\]
We call $\Gamma \cext A$ the \emph{extended} context
and $\tuple{\gamma, a}_A$ the \emph{extended} substitution.

The operations  ${\cext\,},~\tuple{\_,\_},~\p,~\q$
satisfy the following universal property:
$\tuple{\gamma, a}_A$ is the unique substitution satisfying
\[
\p_{\Gamma, A} \circ \tuple{\gamma, a}_A = \gamma
\qquad \text{and}\qquad
\q_{\Gamma, A} [\tuple{\gamma, a}_A] = a\,.
\]
We refer (colloquially) to $\p$ as the \emph{first projection},
and to $\q$ as the \emph{second projection}. %\footnote%
{Note that the first equation implies that
$\Tm(\Delta,A[\p_{\Gamma,A}][\tuple{\gamma, a}]) = \Tm(\Delta,A[\gamma])$
so that $\q_{\Gamma, A} [\tuple{\gamma, a}]$ and $a$ are elements of the same set.}
Here and below, subscripts are omitted from ${\cext\,},~\tuple{\_,\_},~\p,~\q$
when they can be reconstructed from the context (no pun intended).
(End Definition~\ref{def:Cwfobj}.)
\end{definition}

A cwf is thus a structure $(\C,1,\tuple{},T,\cext\, , \tuple{\_,\_},\p, \q)$,
subject to equations, for the category and the presheaf, and universal
properties, formulated purely equationally, for the terminal object and for context comprehension.
The morphisms to be defined next preserve this structure,
even in a strict way, `on the nose'.
We often shorten the notation of a cwf to $(\C,T)$, or even just $\C$,
leaving the remaining structure implicit.

\begin{definition}\label{def:Cwfmor}
A \emph{(strict) cwf-morphism $F$ between cwfs $(\C,T_\C)$ and $(\D,T_\D)$}
consists of

\begin{itemize}

\item A functor $F_\fun : \C \to \D$;
\item A natural transformation $F_\nt : T_\C \Rightarrow (T_\D \circ F_\fun^\op)$;
\item The terminal object is preserved on the nose: $F_\fun(1_{\C}) = 1_{\D}$;
\item Context comprehension is preserved on the nose, see below.
\end{itemize}

Since $F_\nt$ is a natural transformation between $\Fam$-valued presheaves,
$F_\nt$ has a component for any object $\Gamma$ of $\C$, and
these components are morphisms in $\Fam(T_C(\Gamma),T_\D(F_\fun(\Gamma)))$.
Recall that morphisms in $\Fam$ consist of a reindexing function
and a family of functions. It is convenient to denote $F_\fun$,
all reindexing functions, as well as all members of the families of functions,
simply by $F$. Thus we have $F(A) \in \Ty_\D(F(\Gamma))$
and $F(a) \in \Tm_\D(F(\Gamma), F(A))$, for all $\Gamma$
and $A\in\Ty_\C(\Gamma)$ and $a\in \Tm_\C(\Gamma, A)$.

Naturality of $F_\nt$
amounts to preservation of substitution, {i.e.}, for all
$\gamma : \Delta \to \Gamma$ in $\C$, we have
\[
F(A[\gamma]) = F(A)[F(\gamma)] \qquad \qquad
F(a[\gamma]) = F(a)[F(\gamma)]\,.
\]

Last but not least, we turn to the preservation of context comprehension
on the nose, and require
\[
F(\Gamma\cext A) = F(\Gamma)\cext F(A) \qquad
%F(\tuple{\gamma,a}) = \tuple{F(\gamma),F(a)} \qquad
F(\p_{\Gamma, A}) = \p_{F(\Gamma), F(A)} \qquad
F(\q_{\Gamma, A}) = \q_{F(\Gamma), F(A)}\,.
\]

Note that the universal property implies that
$F(\tuple{\gamma,a}) = \tuple{F(\gamma),F(a)}$.
The same is true for the terminal maps:
$F(\tuple{}_\Gamma) = \tuple{}_{F(\Gamma)}$.
(End Definition~\ref{def:Cwfmor}.)
\end{definition}

Small cwfs with strict cwfs-morphisms form a category, written $\Cwf$.

\section{Signatures and models of generalized algebraic theories}\label{gat-sig-mod}

We now come to the main point of this note.
We define how to build a valid gat signature $\Sigma$ and the associated
category $\Cwf_{\Sigma}$ of cwfs with a $\Sigma$-structure.
Each object of $\Cwf_{\Sigma}$ is a cwf with extra structure and
each morphism is a cwf-morphism preserving $\Sigma$-structure.
For this definition, we will need the following auxiliary notions.

A {\em uniform family of contexts} is a family $\Gamma = (\Gamma_{\C})$ with $\Gamma_\C$ a context in 
$\C$ for each $\C \in \Cwf_{\Sigma}$, such that
$F(\Gamma_\C) = \Gamma_\D$ for all morphisms $F \in \Cwf_{\Sigma}(\C,\D)$.
If $\Gamma$ is such a family, a {\em uniform family of types} over $\Gamma$ is a
family of types $A = (A_{\C})$ with $A_{\C}$ a type over $\Gamma_{\C}$ and
$F(A_{\C}) = A_{\D}$ for all morphisms $F \in \Cwf_{\Sigma}(\C,\D)$.
Finally, given $\Gamma$ and $A$, a {\em uniform family of terms} is a family
of terms $a = (a_{\C})$ with $a_\C \in \Tm_{\C}(\Gamma_{\C},A_{\C})$ such that
$F(a_{\C}) = a_{\D}$ for all morphisms $F \in \Cwf_{\Sigma}(\C,\D)$. 

\begin{remark}
Uniform families appear in Freyd's proof of the adjoint functor theorem \cite{freyd:abelian}, in Reynolds' \cite{reynolds:impredicative} and Reynolds and Plotkin's construction \cite{plotkin-reynolds} of an initial algebra for an endofunctor from an impredicative encoding of an inductive type, and in Awodey, Frey, and Speight's  \cite{awodey:impredicative} construction of an impredicative encoding of a higher inductive type. The common idea in these works is to first construct a weakly initial object and then the initial object is obtained by taking uniform families.
\end{remark}

\begin{definition}\label{def-sig-mod}
We define inductively (actually inductive-recursively) how to build a valid signature $\Sigma$ and the category $\Cwf_\Sigma$ of cwfs with a $\Sigma$-structure and cwf-morphisms that preserve $\Sigma$-structure. First, the base case:
\begin{description}
\item[The empty signature] The empty signature $\emptyset$ is valid and $\Cwf_\emptyset = \Cwf$.
\end{description}
Assume now that we have defined $\Sigma$ as a valid signature and the associated
category $\Cwf_{\Sigma}$.
Then we can add a new sort symbol, or a new operator symbol, or a new equation, to get a new valid signature,
as follows:
\begin{description}
\item[Adding a sort symbol]
  Let $\Gamma = (\Gamma_\C)$ be a uniform family of contexts indexed by $\C \in \Cwf_{\Sigma}$.
  Then we can extend $\Sigma$ with a new sort symbol $S$ relative to $\Gamma$, to obtain
  the gat $\Sigma' = (\Sigma,(\Gamma,S))$.
  The objects of $\Cwf_{\Sigma'}$ are pairs $(\C,S_{\C})$, where $\C$ is an object of $\Cwf_{\Sigma}$
  and $S_{\C} \in \Ty_\C(\Gamma_{\C})$.
  A morphism in $\Cwf_{\Sigma'}((\C,S_{\C}), (\D,S_{\D}))$
  is a morphism $F \in \Cwf_{\Sigma}(\C,\D)$ such that $F(S_\C) = S_\D$.
\item[Adding an operator symbol]
  If $\Gamma$ is a uniform family of contexts and $A$ a uniform family of
  types over $\Gamma$, 
  %both indexed by $\C \in \Cwf_{\Sigma}$, 
  then we can extend $\Sigma$ with a new operator
  symbol $f$ relative to $\Gamma$ and $A$, to obtain
  the gat $\Sigma' = (\Sigma,(\Gamma,A,f))$.
  An object of $\Cwf_{\Sigma'}$
  is a pair $(\C,f_{\C})$ where $\C$ is an object in $\Cwf_{\Sigma}$ and $f_{\C} \in \Tm_\C(\Gamma_\C,A_\C)$.
  A morphism in $\Cwf_{\Sigma'}((\C,f_{\C}),(\D,f_{\D}))$ is a morphism $F \in \Cwf_{\Sigma}(\C,\D)$ such that $F(f_{\C}) = f_{\D}$
\item[Adding an equation]
  If $\Gamma$ is a uniform family of contexts,
  $A$ is a uniform family of types over $\Gamma$
  and $a,a'$ are uniform families of terms in $A$, 
  %all indexed by $\C \in \Cwf_{\Sigma}$,
  then we can extend $\Sigma$ with a new equation $a = a'$ relative to $\Gamma$ and $A$, to obtain
  the gat $\Sigma' = (\Sigma,(\Gamma,A,a,a'))$.
  In this case,
  $\Cwf_{\Sigma'}$ is a full subcategory of $\Cwf_{\Sigma}$. An object $\C$ in
  $\Cwf_{\Sigma'}$ is an object $\C$ of $\Cwf_\Sigma$ such that $a_{\C} = a'_{\C}$.
\end{description}

\end{definition}

This definition is {\em syntax independent}. In the next subsection we then show the purely {\em syntactic} construction of an {\em initial object} $\T_{\Sigma}$ in $\Cwf_{\Sigma}$ (for an arbitrary valid signature $\Sigma$) in terms of grammars and inference rules. A context in $\T_{\Sigma}$ will be an equivalence class $[ \Gamma ]$ of raw contexts, and similarly for substitutions, types, and terms. To give a uniform
  family of contexts $\Gamma_\C$ is then equivalent to giving a context $[ \Gamma ] \in \T_{\Sigma}$, since $\Gamma_\C = \inte{ [ \Gamma ] }_\C$ where $\inte{-}_\C$ is the interpretation morphism from $\T_\Sigma$ to $\C$.
Uniform families of types and terms arise from types and terms in $\T_\Sigma$ in a similar way.

We refer to Section \ref{monoids} where we show a simple example: the construction of a signature $\Sigma$ for internal monoids and its associated category of models $\Cwf_\Sigma$ of cwfs with an internal monoid. We also show how to construct the initial cwf with an internal monoid $\T_\Sigma$. 

\begin{remark}
Cartmell's notion of gat \cite{cartmell:phd,cartmell:apal} also makes it possible to stipulate equations between type expressions. However, neither of our examples makes use of this extra generality. In particular,  in Section \ref{sec:examples} we present the gat of cwfs with extra structure for $\N, \Pi$ and a first universe $\U_0$ without needing type equations. The reason is that
equations between types become equations between terms in our rendering
of dependent type theory as a gat. See Remark \ref{remark:typeequations} in Section \ref{sec:u-example} for more explanation.

Like Cartmell, we could consider gats with type equations, but we prefer not to make such equations part of our notion.


%If we wish to extend our notion of valid signature with this new possibility we could add the following clause:
%\begin{description}
%\item[Adding a type equation]
%If $\Gamma$ is a uniform family of contexts, and
%  $A, A'$ are uniform families of types over $\Gamma$
%  then we can extend $\Sigma$ with a new equation $A = A'$, to obtain
%  the gat $\Sigma' = (\Sigma,(\Gamma,A,A'))$.
%  In this case,
%  $\Cwf_{\Sigma'}$ is a full subcategory of $\Cwf_{\Sigma}$. An object $\C$ in
%  $\Cwf_{\Sigma'}$ is an object $\C$ of $\Cwf_\Sigma$ such that $A_{\C} = A'_{\C}$.
%\end{description}
%The construction of initial objects $\T_\Sigma$ in the next section can easily be extended to this more general notion of signature $\Sigma$.
\end{remark}


%The construction of $\T_\Sigma$ is the topic of the next section.

%\section{Signatures and models of generalized algebraic theories}
%
%We now come to the main point of this note. We define how to build a valid gat signature $\Sigma$ and what it means for a cwf to support it. This definition relies on the construction of initial cwfs $\T_\Sigma$ supporting $\Sigma$, but we will postpone the construction of these until the next section. In this way we separate the abstract definition from the concrete syntactic details employed for building $\T_\Sigma$.
%\footnote{Say something about the fact that we do not assume anything about the relationship between $\T_\Sigma$ and $\T_{\Sigma'}$ because all initial objects are isomorphic, and hence we can move between them.}
%\begin{definition}
%We define inductively how to build a valid signature $\Sigma$ and the category $\Cwf_\Sigma$ of cwfs with support for $\Sigma$ and cwf-morphisms that preserve it. First, the base case:
%\begin{description}
%\item[The empty signature] The empty signature $\emptyset$ is valid and $\Cwf_\emptyset = \Cwf$.
%\end{description}
%Assume now that $\Sigma$ is a valid signature and that the category $\Cwf_\Sigma$ has an initial object $\T_\Sigma$ with $\inte{-} : \T_\Sigma \to \C$ as the unique morphism into an object $\C$. Then we can add a new sort symbol, or a new operator symbol, or a new equation, to get a new valid signature, as follows:
%\begin{description}
%\item[Adding a new sort symbol (Thierry)]
%Let $\Gamma_\C \in \C$ for $\C \in \Cwf_{\Sigma}$ be a uniform family of contexts, that is, if $f(\Gamma_\C) = \Gamma_\D$ for all morphisms $f \in \Cwf_{\Sigma}(\C,\D)$. Then we can extend $\Sigma$ with a new sort symbol to obtain an extended gat $\Sigma' = (\Sigma,F)$, where $F_\C \in \Ty_\C(\Gamma_\C)$ is a uniform family of types, that is $f(F_\C) = F_\D$. The objects of $\Cwf_{\Sigma'}$ are pairs $(\C,F)$, where $\C$ is an object of $\Cwf_{\Sigma}$. A morphism in $\Cwf_{\Sigma'}((\C,F), (\D,G))$ is a morphism $f \in \Cwf_{\Sigma}(\C,\D)$ such that $f(F_\C) = G_\D$.
%\item[Adding a new sort symbol]
%If $\Gamma$ is a context in $\T_\Sigma$, then we can extend $\Sigma$ with a new sort symbol $F$ (with context $\Gamma$) to obtain an extended gat $\Sigma'$. The objects of $\Cwf_{\Sigma'}$ are pairs $(\C,F_\C)$, where $\C$ is an object of $\Cwf_{\Sigma}$ and $F_\C \in \Ty_\C(\inte{\Gamma})$. A morphism in $\Cwf_{\Sigma'}((\C,F_\C), (\D,F_\D))$ is a morphism in $\Cwf_{\Sigma}(\C,\D)$ that maps $F_\C$ to $F_\D$.
%\item[Adding a new operator symbol]
%If $\Gamma$ is a context in $\T_\Sigma$ and $A \in \Ty_{\T_\Sigma}(\Gamma)$, then we can extend $\Sigma$ with a new operator symbol $f$ (with context $\Gamma$ and type $A$). A cwf $\C$ supports $\Sigma'$ if it supports $\Sigma$ and there is $f_\C \in\Tm_\C(\inte{\Gamma},\inte{A}_{\Gamma})$.
%A cwf-morphism in $\C \to \D$ preserves $\Sigma'$ if it preserves $\Sigma$ and maps $f_\C$ to $f_\D$.
%\item[Adding a new equation]
%If $\Gamma$ is a context in $\T_\Sigma$, $A \in \Ty_{\T_\Sigma}(\Gamma)$, and $a, a' \in \Tm_{\T_\Sigma}(\Gamma,A)$, then we can extend $\Sigma$ with a new equation $a = a'$ (with context $\Gamma$ and type $A$). A cwf $\C$ supports $\Sigma'$ if it supports $\Sigma$ and $\inte{a}_{\Gamma,A}= \inte{a'}_{\Gamma,A} \in \Tm_\C(\inte{\Gamma},\inte{A}_\Gamma)$. A cwf-morphism in $\C \to \D$, where $\C$ and $\D$ support $\Sigma'$, preserves $\Sigma'$ iff it preserves $\Sigma$.
%\end{description}
%\end{definition}
%
%Note that the empty signature $\emptyset$ is the only valid signature that can be directly constructed from the definition. In order to form other signatures we need to construct initial cwfs supporting already constructed signatures. This is the topic of the next section.

\section{The construction of an initial object in $\Cwf_\Sigma$}\label{initial-gat}

We shall now show our main theorem. It can be viewed as a generalization of Birkhoff's completeness theorem for equational logic \cite{birkhoff}:
\begin{theorem}
The category $\Cwf_\Sigma$ has an initial object $\T_\Sigma$,
for every valid signature $\Sigma$.
\end{theorem}

The construction of $\T_\Sigma$ will be by induction on the construction of $\Sigma$. It is based on construction of initial cwfs in \cite{castellan:tlca2015,castellan:lmcs} and we refer the reader to those papers for more details. Here we only provide a sketch and focus on how to extend the construction to $\T_\Sigma$.

For each $\Sigma$ we will define the following.
\begin{itemize}
\item
A grammar for raw contexts in $\RawCtx_\Sigma$, raw substitutions in $\RawSub_\Sigma$, raw types in $\RawTy_\Sigma$, and raw terms in $\RawTm_\Sigma$.
\item
A system of inference rules that generate four families of partial equivalence relations (pers) by a mutual inductive definition:
$$
\Gamma = \Gamma' \vdash_\Sigma
\qquad
\Gamma \vdash_\Sigma A = A'
\qquad
\Delta \vdash_\Sigma \gamma = \gamma' : \Gamma
\qquad
\Gamma \vdash_\Sigma a = a' : A
$$
where $\Gamma, \Gamma' \in \RawCtx_\Sigma, \gamma, \gamma' \in \RawSub_\Sigma, A, A' \in \RawTy_\Sigma,$ and $a,a' \in \RawTm_\Sigma$. Instances of these pers correspond to valid equality judgments of a variable-free version of dependent type theory with explicit substitutions based on the cwf-combinators. The ordinary judgments will be defined as the reflexive instances of these equality judgments. For example $\Gamma \vdash_\Sigma$ (meaning ``$\Gamma$ is a valid context") is defined as the reflexive instance $\Gamma = \Gamma \vdash_\Sigma$.
\item
A cwf $\T_\Sigma$ is then constructed from the equivalence classes of derivable judgments. For example, the contexts in $\T_\Sigma$ are equivalence classes $[\Gamma]$, such that $\Gamma \vdash_\Sigma$. We will show that $\T_\Sigma$ is a cwf with a $\Sigma$-structure, that is, an object of $\Cwf_\Sigma$.
\item
A $\Cwf_\Sigma$-morphism $\inte{-} : \T_\Sigma \to \C$ for every $\C \in \Cwf_\Sigma$. This is the {\em interpretation morphism}. This morphism is a partial function defined by induction on the raw syntax, that (whenever it is defined) maps raw contexts to contexts in $\C$, raw substitutions to substitutions in $\C$, raw types to types in $\C$, and raw terms to terms in~$\C$. We show that these partial functions preserve the partial equivalence relations so that we can define the interpretation morphism on the equivalence classes. Finally we show that it indeed is a $\Cwf_\Sigma$-morphism and the unique such into $\C$.
\end{itemize}

We begin with the construction for the base case: the {\bf empty signature} $\emptyset$.
\begin{itemize}
\item
We start with the following grammar for raw contexts, raw substitutions, raw types, and raw terms.
\begin{eqnarray*}
\Gamma \in \RawCtx_\emptyset &::=& 1  \ |\ \Gamma\cext A\\
\gamma \in \RawSub_\emptyset  \ &::=& \gamma \circ \gamma \ |\ \id_\Gamma \ |\ \langle\rangle_\Gamma \ |\ \p_{A} \ |\ \langle \gamma, a \rangle_A\\
A \in \RawTy_\emptyset  &::=& A[\gamma]\\
a \in \RawTm_\emptyset  &::=& a[\gamma] \ |\ \qI_A
\end{eqnarray*}
These grammars generate a language of {\em cwf-combinators}.
\item
The system of inference rules is displayed in \cite{castellan:tlca2015,castellan:lmcs}. It is a system of {\em general rules}, rules for dependent type theory which come before we introduce any sort symbols and operator symbols and equations (or any rules for the type formers of intuitionistic type theory). We do not have room here to display them, but note that they can be divided into four groups:
\begin{itemize}
\item the per rules, amounting to symmetry and transitivity for the four forms of equality judgments;
\item preservation rules for judgments, amounting to substitution of equals for equals (an example of such a rule is the {\em type equality rule});
\item congruence rules for operators expressing that the cwf-combinators preserve equality;
\item conversion rules for the cwf-combinators.
\end{itemize}
\item
Note that the initial cwf $\T_\emptyset$ is rather uninteresting: its category of contexts contains only a terminal object (the empty context), and there are no types and terms. Nevertheless, the grammar and inference rules used in its definition form a starting point. The grammar for raw types and raw terms will be extended each time we add a new sort symbol or operator symbol, respectively. For each such new symbol and each new equation we will add a new inference rule. As a consequence we will generate a non-trivial $\T_\Sigma$.
\end{itemize}

Assume now for the induction step that we have defined the grammar, the inference rules, $\T_\Sigma$ and the interpretation morphism $\inte{-} : \T_{\Sigma} \to \C$ in $\Cwf_\Sigma$.
Let $\Sigma'$ be $\Sigma$ extended by a new sort symbol, a new operator symbol, or a new equation. We shall now explain how to define $\T_{\Sigma'}$.
\begin{description}
\item[Adding a sort symbol] If $\Gamma \vdash_\Sigma$, then we can introduce a new sort symbol $S$ in the context $\Gamma$ representing the sequence of types of the arguments of $S$.
\begin{itemize}
\item
We add a new production for raw types
$$
A ::= S
$$
to the productions for $\T_\Sigma$.
\item
We add  the inference rule
\begin{mathpar}
    \inferrule
    {}
    {\Gamma \vdash_{\Sigma'} S}
  \end{mathpar}
to the inference rules for $\T_\Sigma$.
\item
We define $S_{\T_{\Sigma'}} = [S]$, so that $ \T_{\Sigma'}$ has a $\Sigma'$-structure $(\T_\Sigma,[S])$.
\item
We extend the definition of the interpretation morphism $\inte{-}$  to an interpretation morphism $\inte{-}' : \T_{\Sigma'} \to \C$ by
$$
\inte{[S]}' = S_\C
$$
It follows  that this is a morphism in $\Cwf_{\Sigma'}$ and that it is unique.
\end{itemize}

\item[Adding an operator symbol] If $\Gamma \vdash_\Sigma A$, then we can introduce a new operator symbol $f$, where the context $\Gamma$ represents the sequence of types of the arguments and $A$ is the type of the result.
\begin{itemize}
\item
We add a new production for raw terms
$$
a ::= f
$$
to the productions for $\T_\Sigma$.
\item
We add  the inference rule
\begin{mathpar}
    \inferrule
    {}
    {\Gamma \vdash_{\Sigma'} f : A}
  \end{mathpar}
to the inference rules for $\T_\Sigma$.
\item
We define $f_{\T_{\Sigma'}} = [f]$, so that $ \T_{\Sigma'}$ has a $\Sigma'$-structure $(\T_\Sigma,[f])$.\item
We extend the definition of the interpretation morphism $\inte{-}$  to an interpretation morphism $\inte{-}' : \T_{\Sigma'} \to \C$ by
$$
\inte{[f]}' = f_\C
$$
It follows  that this is a morphism in $\Cwf_{\Sigma'}$ and that it is unique.
\end{itemize}

\item[Adding an equation] If $\Gamma \vdash_\Sigma a : A$ and $\Gamma \vdash_\Sigma a' : A$ we can introduce a new equation $a = a'$.
\begin{itemize}
\item
$\T_\Sigma'$ has the same productions as $\T_\Sigma$.
\item
We add  the inference rule
\begin{mathpar}
    \inferrule
    {}
    {\Gamma \vdash_{\Sigma'} a = a' : A}
  \end{mathpar}
to the inference rules for $\T_\Sigma$.
\item
$\T_{\Sigma'}$ is based on the same raw syntax as $\T_\Sigma$ but the equivalence relation has changed. To show that $\T_{\Sigma'} \in \Cwf_{\Sigma'}$ we just need to show that $[ a ] = [ a' ]$ but this follows from the inference rule $\Gamma \vdash_{\Sigma'} a = a' : A$.
\item
In order to define $\inte{-}'$ we first define the partial function on the raw syntax to be identical to the partial function on the raw syntax for $\inte{-}$. We then prove that this partial function preserves the extended partial equivalence relation and define $\inte{-}'$ on the new equivalence classes. It follows  that $\inte{-}'$ is unique.
\end{itemize}
\end{description}
This concludes the proof of the theorem. \qed



\section{Examples of generalized algebraic theories}\label{sec:examples}

We will now display the sort symbols, operator symbols, and equations for the generalized algebraic of internal monoids, internal categories and of internal cwfs. We will then show how to add operator symbols and equations when adding internal notions of $\Pi, \N$ and a universe closed under $\Pi$ and $\N$ to the gat of internal cwfs. The reason for prefacing these notions by the word ``internal" is that the models of the theories are internal monoids, categories, and cwfs in $\Cwf_\Sigma$ for the respective signatures for these theories. Moreover, internal monoids, categories, and cwfs in the cwf~$\Set$ are small monoids, categories, and cwfs, respectively. Note that the cwfs defined in Section 2 need not be small, and hence not internal cwfs in the cwf~$\Set$.


We begin by using the recipe in Definition \ref{def:Cwfmor} to construct the semantic signature for internal monoids and its associated category of models, that is, of cwfs with an internal monoid. We then follow the recipe in Definition \ref{initial-gat} and construct the initial cwf with an internal monoid. 

For ease of readability, we will only present the sort symbols, operator symbols, and equations in the remaining examples by using an informal notation with named variables, rather than the formal notations using cwf-combinators employed in Definitions \ref{def:Cwfmor} and \ref{initial-gat}.

Our final example is the gat of internal contextual cwfs, a variant of Cartmell's contextual categories. The contexts in such contextual cwfs come with a length $n$. We sketch how this can be axiomatized as a gat with countably many sort symbols $\ctx_n, \sub_n, \ty_n, \tm_n$ for an external natural number $n$ (and similarly for the operator symbols and equations). We also indicate how our framework can be extended to cover such gats.

\subsection{Internal monoids}\label{monoids}
 The one-sorted algebraic theory of monoids has two operator symbols,
$\idmon$ for identity and $\comp$ for composition, and associativity and identity laws as equations.
As any other one-sorted algebraic theory, the theory of monoids yields a
generalized algebraic theory. In ordinary notation with variables it might be rendered as follows, where $\Mon$ is the only sort:
\begin{eqnarray*}
&\vdash& \Mon\\
&\vdash& e : \Mon\\
x, y : \Mon &\vdash& \comp(x,y) : \Mon\\
y : \Mon &\vdash& \comp(\idmon,y) = y : \Mon\\
x : \Mon &\vdash& \comp(x,\idmon) = x : \Mon\\
x, y, z : \Mon &\vdash& \comp(\comp(x,y),z) = \comp(x,\comp(y,z)) : \Mon
\end{eqnarray*}

We now show how the corresponding semantic signature for monoids $\Sigma$ and its associated category of models $\Cwf_\Sigma$ are constructed step-wise in the sense of Definition \ref{def-sig-mod}.

As always, we begin with the empty signature $\emptyset$ and its category of models $\Cwf_\emptyset = \Cwf$.
\begin{description}
\item[Adding the sort symbol $\Mon$] Each cwf $\C$ has a chosen empty context (terminal object) $1_\C$. Since cwf-morphisms preserve empty contexts on the nose, $1 = (1_\C)$ is a uniform family of contexts in $\Cwf_\emptyset$. Hence we can introduce a new constant sort symbol $\Mon$ in the empty context. The resulting signature  is
$$
\Sigma_1 = (\emptyset, (1,\Mon))
$$
The objects of $\Cwf_{\Sigma_1}$  are pairs $(\C,\Mon_\C)$, where $\C$ is a cwf and $\Mon_\C \in \Ty_\C(1_\C)$.
\item[Adding the operator symbol for the identity]
%Each $(\C,\Mon_\C) \in \Cwf_{\Sigma_1} $ has an empty context $1_\C$ and a type $\Mon_\C \in \Ty_\C(1_\C)$. 
Since, morphisms in $\Cwf_{\Sigma_1}$ preserve both empty contexts $1_\C$ and types $\Mon_\C$ on the nose, we have a uniform family of contexts $1 = (1_\C)$ and a uniform family of types $\Mon = (\Mon_\C)$ in $\Cwf_{\Sigma_1}$. Hence we can introduce a new constant operator symbol $e$ (the identity of the monoid).  The resulting signature  is
$$
\Sigma_2 = (\Sigma_1, (1,\Mon,e))
$$
The objects of $\Cwf_{\Sigma_2}$  are triples $(\C,\Mon_\C,e_\C)$, where $\C$ is a cwf, $\Mon_\C \in \Ty_\C(1_\C)$ and $e_\C \in \Tm_\C(1_\C,\Mon_\C)$.
\item[Adding the operator symbol for composition]
Again using that cwf-morphisms preserve all cwf-structure and $\Mon_\C$, we deduce that we have a uniform family of contexts $1.\Mon.\Mon[\p]$ and a uniform family of types $\Mon[\p][\p]$ in $\Cwf_{\Sigma_2}$. Thus we can add a binary operator symbol $\comp$. The resulting signature is
$$
\Sigma_3 = (\Sigma_2, (1.\Mon.\Mon[\p],\Mon[\p][\p],\comp))
$$
The objects of $\Cwf_{\Sigma_3}$  are quadruples $(\C,\Mon_\C,e_\C,*_\C)$, where $\C$ is a cwf, $\Mon_\C \in \Ty_\C(1_\C)$, $e_\C \in \Tm_\C(1_\C,\Mon_\C)$, and $*_\C \in \Tm_\C((1.\Mon.\Mon[\p])_\C,(\Mon[\p][\p])_\C)$.
\item[Adding the left identity law]
Furthermore, we extend the signature with the equations stating that $\idmon$ is a left identity as follows:
$$
\Sigma_4 = (\Sigma_3, (1.\Mon, \Mon[\p], \comp[\tuple{\tuple{\tuple{},\idmon[\tuple{}]},\q}], \q))
$$
The uniform family of context $1.\Mon$ expresses that the equation has one variable of type $\Mon$, the uniform family of types $\Mon[\p]$ expresses that the two sides of the equation have type $\Mon$, and the uniform families of terms $\comp[\tuple{\tuple{\tuple{},\idmon[\tuple{}]},\q}]$ and $\q$ express the two sides of the equation.
$\Cwf_{\Sigma_4}$ is the full subcategory of $\Cwf_{\Sigma_3}$ with objects $\C$
such that $(\comp[\tuple{\tuple{\tuple{},\idmon[\tuple{}]},\q}])_\C = \q_\C$.
\item[Adding the right identity and the associativity laws]
Finally we add the right identity equation and the associativity equation to get the signatures $\Sigma_5$ and $\Sigma_6$. We omit the details. 
\end{description}
We call the resulting syntax independent signature for internal monoids $\Sigma = \Sigma_6$. The category $\Cwf_\Sigma$ is the category of cwfs with an {\em internal monoid}. This is a cwf-version of the notion of internal monoid which can be defined in any category with finite products. Ordinary (small) monoids come out as internal monoids in $\Set$, the cwf of small sets.

We also sketch the construction of the initial object $\T_\Sigma$ of $\Cwf_\Sigma$ following the recipe for introducing sort symbols, operators symbols, and equations in Section \ref{initial-gat}. (We omit the index $\Sigma$ in $\vdash_\Sigma$.)
%Note that these inference rules follow the principles of introducing sort symbols, operator symbols, and equations we displayed in Section 4 when building the initial cwf $\T_\Sigma$ for appropriate $\Sigma$.
\begin{description}
\item[Adding the sort symbol $\Mon$]
First, we have $1 \vdash$ for the empty signature, so we can
add a production for the constant sort symbol $\Mon$ and the inference rule:
\begin{eqnarray*}
1 &\vdash& \Mon
\end{eqnarray*}
For later use we infer $1.\Mon \vdash$ and, using $\p:1.\Mon \to 1$, $1.\Mon\vdash\Mon[\p]$,
so $1.\Mon.\Mon[\p] \vdash$.
\item[Adding the operator symbol for identity]
We then add a production for the constant operator symbol $\idmon$ and the inference rule:
\begin{eqnarray*}
1 &\vdash& \idmon : \Mon
\end{eqnarray*}
Again for later use we infer $1.\Mon\vdash \idmon[\p] :\Mon[\p]$.
Note that here $\p = \tuple{}$, the empty substitution $1.\Mon \to 1$,
since there is only one substitution $1.\Mon \to 1$.
\item[Adding the operator symbol for composition]
We then add a production for the binary operator symbol $\comp$.
Using another $\p: 1.\Mon.\Mon[\p] \to 1.\Mon$ (note the different type),
we can derive $1.\Mon.\Mon[\p] \vdash \Mon[\p][\p]$, so we can add the inference rule
\begin{eqnarray*}
1.\Mon.\Mon[\p] &\vdash& \comp : \Mon[\p][\p]
\end{eqnarray*}
Note that we project $\Mon$ on the right twice, reflecting that $\comp$ is binary.
\item[Adding the left identity law]
We can derive $1.\Mon \vdash \q : \Mon[\p]$. With some effort,
using previous inferences, we can  derive
$1.\Mon \vdash \comp[\tuple{\tuple{\tuple{},\idmon[\tuple{}]},\q}] : \Mon[\p]$.
Hence we can add the inference rule for the equation ($\idmon$ is a left identity):
\begin{eqnarray*}
1.\Mon &\vdash& \comp[\tuple{\tuple{\tuple{},\idmon[\tuple{}]},\q}] = \q : \Mon[\p]
\end{eqnarray*}
\item[Adding the right identity and the associativity laws]
We omit the details.
\end{description}
The resulting initial object $\T_\Sigma = \T_{\Sigma_6}$ is a system of grammar and inference rules for dependent type theory with an internal monoid. In this theory we can prove statements such as 
$$
\Gamma \vdash a : \Mon
$$
stating that $a$ is a well-formed monoid expression in the context $\Gamma$ and
$$
\Gamma \vdash a = a': \Mon
$$
stating that $a = a'$ is a derivable equation between monoid expressions in the context $\Gamma$. Note that both contexts and monoid expressions use cwf-combinators and are variable-free. Of course, using dependent type theory for reasoning about monoid expressions is overkill; this could be done using only simple types.

The remaining examples will use dependent types in an essential way. However, for reasons of readability we will from now on only use ordinary notation with named variables. Hopefully, it is clear from the above how to formally construct the corresponding semantic signatures, categories of models, and initial models using cwf-combinators. For example, these constructions for the theory of internal categories are similar to the constructions for the theory of internal monoids.

% In the remaining examples, we shall mainly use ordinary notation with variables.
% % We have already said that.

\subsection{Internal categories} The gat of categories was one of Cartmell's motivating examples. It has the following sort symbols, operator symbols, and equations. Again, note that in our case the models are internal categories in a cwf. To emphasize the difference between the internal notions of category and cwf and the external notions (introduced in Section 2), our notation for sort symbols in the gat of internal cwfs use lower case letters ($\Obj, \Hom, \ty, \tm$). This is in contrast to the upper case letters for the external versions ($\Ty, \Tm$). We will however overload notation for operator symbols, and for example use $\circ$ both for the cwf-combinator and for the operator symbol in the gat of internal categories.

Sort symbols:
\begin{eqnarray*}
&\vdash& \Obj\\
\Delta, \Gamma : \Obj &\vdash& \Hom(\Delta,\Gamma)\\
\end{eqnarray*}

Operator symbols:
\begin{eqnarray*}
\Gamma : \Obj &\vdash& \id_\Gamma : \Hom(\Gamma,\Gamma)\\
\Xi,\Delta,\Gamma : \Obj, \gamma : \Hom(\Delta,\Gamma), \delta : \Hom(\Xi,\Delta) &\vdash&
\gamma \circ \delta : \Hom(\Xi,\Gamma)
\end{eqnarray*}

Equations:
\begin{eqnarray*}
\Delta, \Gamma : \Obj, \gamma : \Hom(\Delta,\Gamma) &\vdash& \id_\Gamma \circ \gamma = \gamma : \Hom(\Delta,\Gamma)\\
\Delta, \Gamma : \Obj, \gamma : \Hom(\Delta,\Gamma) &\vdash& \gamma \circ \id_\Delta = \gamma : \Hom(\Delta,\Gamma)\\
\Theta, \Xi,\Delta,\Gamma : \Obj, \gamma : \Hom(\Delta,\Gamma), \delta : \Hom(\Xi,\Delta), \xi : \Hom(\Theta,\Xi) &\vdash&
(\gamma \circ \delta) \circ \xi = \gamma \circ (\delta \circ \xi): \Hom(\Theta,\Gamma)
\end{eqnarray*}
%In order to build $\Hom(\Delta,\Gamma)$ for two concrete contexts $\vdash \Delta : \Ctx$ and $\vdash \Gamma : \Ctx$ we need to apply $\Hom$ to the context morphism consisting of $\Delta$ and $\Gamma$ to yield $\Hom(\Delta,\Gamma)$. (There is more to say ...)
Note that composition is officially an operator symbol with five arguments. In the official notation we should write $\gamma \circ_{\Xi,\Delta,\Gamma} \delta$, but we suppress the context arguments $\Xi,\Delta,\Gamma$. We will do so for some other operations too.

The rendering of the gat of categories in cwf-combinator language and the proof that it indeed yields a valid signature are similar to what they were for the gat of monoids. The inference rules for the two sort symbols in cwf-combinator language are
\begin{eqnarray*}
1 &\vdash& \Obj\\
1.\Obj.\Obj[\p] &\vdash& \Hom
\end{eqnarray*}
and the operator symbols for identity
\begin{eqnarray*}
1.\Obj &\vdash& \idmon : \Hom[\tuple{\id_{1.\Obj},\q_{1,\Obj}}]
\end{eqnarray*}
We omit the verbose cwf-renderings of the operator symbol for composition and the equations.

A cwf with extra structure for the generalized algebraic theory of categories is a cwf with an {\em internal category}. This is a cwf-based analogue of the usual notion of internal category in a category with finite limits. As shown by Martin Hofmann \cite{hofmann:csl,hofmann:cambridge}, every category with finite limits yields a category with attributes, and hence a cwf. However, not every cwf has finite limits. To achieve this we need more structure. As shown by Clairambault and Dybjer \cite{ClairambaultD11,ClairambaultD14} the 2-category of categories with finite limits is biequivalent to the 2-category of democratic cwfs that support $\Sigma$-types and extensional identity types.

An internal category in the cwf $\Set$ of small sets is a small category.

%\subsubsection{How to generate the valid signature using the combinator language}
%The signature for the generalized algebraic theory of categories has sort symbols $\Obj$
%and $\Hom$, operator symbols $\id$ for identity and $\circ$ for composition, and associativity and identity laws as equations. We shall here sketch how to generate the valid signature for categories with reference to definition \ref{} and using cwf-combinators.
%\begin{itemize}
%\item The initial cwf $\T_\emptyset$ (as constructed above) has only one object (context) [1], one equivalence class of morphisms $[\id_1]$ (with several representatives: $\tuple{}_1, \id_1 \circ \id_1$, etc), and no types and terms. Hence we can add a new constant sort $\Obj$ of (internal) objects with context $1$ to the signature.
%\item $\T_{([\Obj],[],[])}$ (as constructed above) contains the context $[(1.\Obj).\Obj[\p_{1,\Obj}]]$ (corresponding to the context $x : \Obj, y : \Obj$ in usual notation). Hence we can introduce a new sort $\Hom$ with this context. This sort represents the family $\Hom(x,y)$ of (internal) morphisms.
%\item $\T_{([\Obj, \Hom],[],[])}$ contains the context $[1.\Obj]$ (corresponding to the context $x : \Obj$) and the type $[\Hom[\tuple{\id_{1.\Obj},\q_{1,\Obj}}]]$ (corresponding to the type $\Hom(x,x)$). Hence we can introduce an operator symbol $\id$ with this context and type.
%\item In a similar way we can add the operator symbol for composition and the equations, but we omit the details.
%\end{itemize}

\subsection{Internal cwfs}\label{gat-cwf}

The gat of internal cwfs is obtained by extending the gat of internal categories with new sort symbols, operator symbols, and equations for a family valued functor, and then new operator symbols and equations for a terminal object, and context comprehension. We here rename the sort $\Obj$ of objects of the category of contexts to $\ctx$.

\subsubsection{The extension with a family valued functor}
\mbox{ }

Sort symbols:
\begin{eqnarray*}
\Gamma : \ctx &\vdash& \ty(\Gamma)\\
\Gamma : \ctx, A:\ty(\Gamma) &\vdash& \tm(\Gamma,A)
\end{eqnarray*}

Operator symbols:
\begin{eqnarray*}
\Gamma,\Delta : \ctx, A:\ty(\Gamma), \gamma : \Hom(\Delta,\Gamma) &\vdash&
A[\gamma] : \ty(\Delta)\\
\Gamma,\Delta : \ctx, A:\ty(\Gamma), \gamma : \Hom(\Delta,\Gamma), a:\tm(\Gamma,A) &\vdash&  a[\gamma] : \tm(\Delta,A[\gamma])
\end{eqnarray*}

Equations:
\begin{eqnarray*}
\Gamma : \ctx, A:\ty(\Gamma) &\vdash& A[\id_\Gamma] = A : \ty(\Gamma)\\
\Gamma : \ctx, A:\ty(\Gamma), a:\tm(\Gamma,A) &\vdash& a[\id_\Gamma] = a : \tm(\Gamma,A)\\
\Xi,\Delta,\Gamma : \ctx, \delta : \Hom(\Xi,\Delta), \gamma : \Hom(\Delta,\Gamma),
A:\ty(\Gamma) &\vdash& A[\gamma\circ\delta] = A[\gamma][\delta]: \ty(\Xi)\\
\Xi,\Delta,\Gamma : \ctx, \delta : \Hom(\Xi,\Delta), \gamma : \Hom(\Delta,\Gamma),
A:\ty(\Gamma), a:\tm(\Gamma,A) &\vdash&
a[\gamma\circ\delta] = a[\gamma][\delta]: \tm(\Xi,A[\gamma\circ\delta])
\end{eqnarray*}

\subsubsection{The extension with a terminal object}
No new sorts are required.

Operator symbols:
\begin{eqnarray*}
&\vdash& 1 : \ctx\\
\Gamma : \ctx &\vdash& \tuple{}_\Gamma : \Hom(\Gamma,1)
\end{eqnarray*}

Equations:
\begin{eqnarray*}
 &\vdash& \id_1 = \tuple{}_1 : \Hom(1,1)\\
\Gamma,\Delta : \ctx, \gamma : \Hom(\Delta,\Gamma) &\vdash&
\tuple{}_\Gamma\circ\gamma = \tuple{}_\Delta : \Hom(\Delta,1)
\end{eqnarray*}
(The latter two equations are better for term rewriting than the
obvious single one expressing the uniqueness of $\tuple{}_\Gamma$.)

\subsubsection{The extension with context comprehension}

No new sorts are required.

Operator symbols:
\begin{eqnarray*}
\Gamma : \ctx, A:\ty(\Gamma) &\vdash& \Gamma\cext A : \ctx\\
\Gamma,\Delta : \ctx, A:\ty(\Gamma), \gamma : \Hom(\Delta,\Gamma), a:\tm(\Delta,A[\gamma]) &\vdash& \tuple{\gamma,a} : \Hom(\Delta,\Gamma\cext A)\\
\Gamma : \ctx, A:\ty(\Gamma) &\vdash& \p: \Hom(\Gamma\cext A,\Gamma)\\
\Gamma : \ctx, A:\ty(\Gamma) &\vdash& \q: \tm(\Gamma\cext A,A[\p])
\end{eqnarray*}

Equations:
\begin{eqnarray*}
\Gamma,\Delta : \ctx, A:\ty(\Gamma), \gamma : \Hom(\Delta,\Gamma), a:\tm(\Delta,A[\gamma]) &\vdash& \p\circ\tuple{\gamma,a} = \gamma : \Hom(\Delta,\Gamma)\\
\Gamma,\Delta : \ctx, A:\ty(\Gamma), \gamma : \Hom(\Delta,\Gamma), a:\tm(\Delta,A[\gamma]) &\vdash& \q[\tuple{\gamma,a}] = a : \tm(\Delta,A[\gamma]) \\
\Gamma,\Delta,\Xi : \ctx, A:\ty(\Gamma), \gamma : \Hom(\Delta,\Gamma), a:\tm(\Delta,A[\gamma]), \delta : \Hom(\Xi,\Delta) &\vdash&
\tuple{\gamma,a} \circ \delta = \tuple{\gamma\circ\delta,a[\delta]} :
\Hom(\Xi,\Gamma\cext A) \\
\Gamma : \ctx, A:\ty(\Gamma) &\vdash&
\id_{\Gamma\cext A} = \tuple{\p,\q} : \Hom(\Gamma\cext A,\Gamma\cext A)
\end{eqnarray*}
(If $\p\circ\delta = \gamma$ and $\q[\delta]=a$, we get
$\tuple{\gamma,a}=\tuple{\p\circ\delta, \q[\delta]} = \tuple{\p,\q}\circ\delta =
\delta$, the uniqueness requirement of the universal property.
However, the equation for surjective pairing is not left-linear and with
a variable on one side, which is not good for rewriting.)

A cwf with extra structure supporting the generalized algebraic theory of cwfs is a cwf with an
\emph{internal cwf}. An internal cwf in the cwf $\Set$ of small sets is a {\em small cwf},
that is, it is a cwf in the ordinary sense (see Definition~\ref{def:Cwfobj})
except that it has small sets of objects, morphisms, types, and terms.

An example of a cwf with an internal cwf is provided by the cwf $\Set$ of small sets with an internal category of very small sets. We can make this precise if we work in set theory with two Grothendieck universes $\V_0 \in \V_1$. We call the members of $\V_1$ ``small sets" and the members of $\V_0$ ``very small sets". The category of contexts of the cwf $\Set$ is the usual category of small sets, by which we here mean that its objects are in $\V_1$. Moreover, the types are also the small sets in $\V_1$. To get an internal cwf, we interpret its sort of objects $\ctx$ as the small set $\V_0$ of very small sets, and the sorts of types $\ty(\Gamma)$ also as $\V_0$.

%Further examples of internal cwfs inside $\Set$ are provided by cwfs supporting signatures of generalized algebraic theories, as constructed in the previous section.

\subsection{Internal cwfs with $\Pi$-types}
We add three operator symbols in addition to the operator symbols for cwfs in Section 5.2 and 5.3:
\begin{eqnarray*}
\Gamma : \ctx, A : \ty(\Gamma), B : \ty(\Gamma.A)&\vdash& \Pi(A,B) : \ty(\Gamma)\\
\Gamma : \ctx, A : \ty(\Gamma), B : \ty(\Gamma.A), b : \tm(\Gamma.A, B) &\vdash& \lambda(b) : \tm(\Gamma,\Pi(A,B))\\
\Gamma : \ctx, A : \ty(\Gamma), B : \ty(\Gamma.A), c :  \tm(\Gamma,\Pi(A,B)), a : \tm(\Gamma, A) &\vdash& \app(c,a) : \tm(\Gamma, B[\tuple{\id,a}])
\end{eqnarray*}
(again omitting some of the official arguments)
and equations for $\beta, \eta$ (also omitting the context and type of the equality judgment)
 \begin{eqnarray*}
 \app(\lambda(b),a) &=& b[\tuple{\id,a}]\\
 \lambda(\app(c[\p],\q)) &=& c
\end{eqnarray*}
and commutation with respect to substitution.
\begin{eqnarray*}
\Pi(A,B)[ \gamma ] &=& \Pi(A [ \gamma ], B[ \gamma^+ ])\\
\lambda(b) [ \gamma ] &=& \lambda(b[\gamma^+ ])\\
\app(c,a) [ \gamma ] &=& \app(c[ \gamma ], a[ \gamma ] )
\end{eqnarray*}
where $\gamma^+ = \tuple{\gamma \circ \p, \q}$.

\subsection{Internal cwfs with $\Pi$ and $\N$}
Furthermore, we add the operator symbol
\begin{eqnarray*}
\Gamma : \ctx &\vdash& \N_\Gamma : \ty(\Gamma) 
\end{eqnarray*}
We also add operator symbols for $0, \s, \Rec$ and the equations for $\Rec$ and for commutativity with substitution, but omit the details. Note that the type of the primitive recursion operator $\Rec$ relies on the signature for $\Pi$-types.

\subsection{Internal cwfs with $\U_0$ closed under $\Pi$ and $\N$}\label{sec:u-example}
We add four more operator symbols
\begin{eqnarray*}
\Gamma : \ctx &\vdash& (\U_0)_\Gamma : \ty(\Gamma)\\
\Gamma : \ctx, a : \tm(\Gamma,(\U_0)_\Gamma) &\vdash& {\Ta_0}(a) : \ty(\Gamma)\\
\Gamma : \ctx &\vdash& (\N^0)_\Gamma : \tm(\Gamma,(\U_0)_\Gamma) \\
\Gamma : \ctx,
a : \tm(\Gamma,(\U_0)_\Gamma),
b :  \tm(\Gamma \cdot \Ta_0(a), (\U_0)_\Gamma))
&\vdash&
 \Pi^0(a,b) : \tm(\Gamma,(\U_0)_\Gamma)
\end{eqnarray*}
$(\U_0)_\Gamma$ is the universe (a type) relative to the context $\Gamma$; $\Ta_0$ is the decoding operation mapping a term in the universe to the corresponding type; $\N^0$ is the code for $\N$ in the universe, and $\Pi^0$ forms codes for $\Pi$-types in the
 universe. (Note that we have dropped the context argument of $\Ta_0$ and $\Pi^0$.)

We add the decoding equations:
\begin{eqnarray*}
\Ta_0(\N^0_\Gamma) &=& \N_\Gamma\\
\Ta_0(\Pi^0(a,b)) &=& \Pi(\Ta_0(a),\Ta_0(b))
\end{eqnarray*}
and the equations for preservation of substitution:
\begin{eqnarray*}
{(\U_0)}_\Gamma [ \gamma ] &=& {(\U_0)}_\Delta\\
\Ta_0(a) [ \gamma ] &=& \Ta_0(a[ \gamma ] )\\
\N^0_\Gamma [ \gamma ] &=&\N^0_\Delta\\
\Pi^0(a,b)[ \gamma ] &=& \Pi^0(a [ \gamma ], b[ \gamma^+ ])
\end{eqnarray*}
%where $\gamma^+ = \tuple{\gamma \circ \p, \q}$.

\begin{remark}\label{remark:typeequations}
Note that all equations are between {\em terms} in the gat of cwfs with extra structure for $\N, \Pi,$ and $\U_0$; we do not need the extra generality of stipulating type equations as discussed in the introduction. For example, the equation $\Ta_0(\N^0_\Gamma) = \N_\Gamma$ is not an equation between {\em types}, but between {\em terms} of type $\ty(\Gamma)$.
\end{remark}

\begin{remark}
Also note that the gat for the universe is inevitably {\em \`a la Tarski} in the sense that we distinguish between types and terms in a cwf and we must have an operation decoding a term into a type. However, Martin-Löf's distinction between {\em \`a la Russell} and {\em \`a la Tarski} \cite{martinlof:padova} is a distinction between two different formulation of the raw syntax and inference rules of type theory.
\end{remark}

\subsection{A possible refinement to internal contextual cwfs}

Our treatment can be adapted to some non finitely presented gats.
If we have an increasing sequence of signatures $\Sigma_n$ we can consider their
union.
%of the theory $T_{\Sigma_n}$.
For instance, we can describe a gat of contextual cwfs \cite{castellan:lambek} (similar to Cartmell's contextual categories and Voevodsky's $C$-systems) by
the following stratification of the theory of cwfs. We replace the sort $\ctx$
by a sequence of sorts $\ctx_0,\,\ctx_1,\,\dots ,$ where $\ctx_n$ represents the sort
of contexts of length $n$ and a corresponding sequence of sorts
$\ty_n(\Gamma)$ for $\Gamma$ in $\ctx_n$
and $\tm_n(\Gamma,A)$ for $A$ in $\ty_n(\Gamma)$. Context extension $\Gamma.A$ is now in $\ctx_{n+1}$
if $A$ is in $\ty_n(\Gamma)$ and so on.
We also add {\em destructors}: we have
$\mathrm{ft}(\Gamma)$ in $\ctx_n$
and $\mathrm{st}(\Gamma)$ in $\ty_n(\mathrm{ft}(\Gamma))$
with $\Gamma = \mathrm{ft}(\Gamma).\mathrm{st}(\Gamma)$.
Similarly we have a stratification of the sort of substitutions
$\hom_{n,m}(\Delta,\Gamma)$ for $\Delta$ in $\ctx_n$ and $\Gamma$ in $\ctx_m$.
The resulting models are {\em internal contextual cwfs} in a cwf.

\begin{remark}
Generalized algebraic presentations of contextual categories (C-systems) have been given by Voevodsky \cite{voevodsky:c-systems} and Cartmell \cite{cartmell:gat-contextual}.
\end{remark}

%\subsection{Cwfs with universe tower structures} 

%\subsection{Cwfs with universe tower structures}

%
%The first formulation of intuitionistic type theory with an infinite sequence of universes is due to Martin-Löf
%\cite{martinlof:predicative}. Rules for cumulativity (or lifting) were added in Martin-Löf \cite{martinlof:hannover}. Both formulations have an infinite sequence of universes indexed by external natural numbers, and as a consequence the theories have infinitely many rules.
%
%We shall now formalize a notion of finitary gat closely related to Martin-Löf's cumulative version. The external natural number indices will be represented by internal level indices in the gat. To this end we introduce a new sort symbol
%$$
%\vdash \Level
%$$
%in addition to the previous four sort symbols. (However, we do not have a {\em type} of levels.) An element $n : \Level$ represent an external natural number. (Note that the new sort $\Level$ corresponds to adding a new form of judgment $\vdash n\  \Level$ to the formal system.)
%
%We add operator symbols for levels
%\begin{eqnarray*}
%&\vdash& 0 : \Level\\
%n : \Level &\vdash& \s(n) : \Level
%\end{eqnarray*}
%and operator symbols for types and terms:
%\begin{eqnarray*}
%n : \Level, \Gamma : \ctx &\vdash& (\U_n)_\Gamma : \Ty(\Gamma)\\
%n : \Level, \Gamma : \ctx, a : \Tm(\Gamma,(\U_n)_\Gamma) &\vdash& {\Ta_n}(a) : \Ty(\Gamma)\\
%n : \Level, \Gamma : \Ctx &\vdash& (\N^n)_\Gamma : \Tm(\Gamma,(\U_n)_\Gamma) \\
%n : \Level, \Gamma : \Ctx,
%a : \Tm(\Gamma,(\U_n)_\Gamma),
%b :  \Tm(\Gamma \cdot \Ta_n(a), (\U_n)_\Gamma))
%&\vdash&
% \Pi^n(a,b) : \Tm(\Gamma,(\U_n)_\Gamma)\\
%n : \Level, \Gamma : \Ctx &\vdash& (\U^n)_\Gamma \in \Tm(\Gamma,(\U_{\s(n)})_\Gamma)\\
%n : \Level, \Gamma : \Ctx, a : \Tm(\Gamma,(\U_n)_\Gamma) &\vdash& \Ta_n^{n+1}(a)\footnote{200804: should we change notation $\Ta_n^{n+1}$ to $\Ta^n$?} : \Tm(\Gamma,(\U_{\s(n)})_\Gamma)
%\end{eqnarray*}
%The last operator symbol is the lifting (or cumulativity) operator. We have the following equations
%\begin{eqnarray*}
%\Tan((\N^n)_\Gamma) &=& \N_\Gamma\\
%\Ta_n(\Pi^{n}(a,b)) &=& \Pi(\Ta_n(a),\Tan(b))\\
%\Ta_{\s(n)}((\U^n)_\Gamma ) &=& (\U_n)_\Gamma\\
%%&\Ta_{n+1}(\Ta_n^{n+1}(a)) &=& \Ta_n(a)\\
%\Ta^{n+1}_n((\N^n)_\Gamma) &=& (\N^{\s(n)})_\Gamma\\
%\Ta^{n+1}_n(\Pi^{n}(a,b)) &=& \Pi^{\s(n)}(\Ta^{n+1}_n(a),\Ta^{n+1}_n(b))
%\end{eqnarray*}
%Finally, all operator symbols commute with substitution:
%\begin{eqnarray*}
%{(\Un)}_\Gamma [ \gamma ] &=& {(\Un)}_\Delta\\
%\Tan(a) [ \gamma ] &=& \Tan(a[ \gamma ] )\\
%\N^n_\Gamma [ \gamma ] &=&\N^n_\Delta\\
%\Pi^{n}(a,b)[ \gamma ] &=& \Pi^{n}(a [ \gamma ], b[ \gamma^+ ])\\
%\U^n[\gamma] &=& \U^n\\
%\Ta_n^{n+1}(a)[\gamma]  &=& \Ta_n^{n+1}(a[\gamma])
%\end{eqnarray*}
%Mention a la Russell initiality?
%
%\subsection{Removing cumulativity} Agda, cumulativity up to equivalence.
%
%\subsection{Cwfs with universe polymorphic tower structures} Here we need to extend the cwf-framework further to take into account contexts with level variables, etc.
%\newpage
\section{Related work}

The first proof of initiality of a formal system with dependent types is due to Streicher \cite{streicher:semtt}. He proved that the formal system for the Calculus of Construction forms an initial object in a category of contextual categories with suitable extra structure. Recently, Brunerie et al \cite{brunerie:initiality} presented a formalized proof in the Agda system that a formal system for Martin-Löf type theory forms an initial object in a category of contextual categories with extra structure for
the type formers.

More generally, Voevodsky \cite{voevodsky:initiality} has outlined a new vision of the theory of syntax and semantics of dependent type theories. In this vision formal systems for dependent type theory are proved to be initial in suitable categories of models ({\em the initiality conjecture}). The above-mentioned contributions by Streicher and Brunerie et al are two examples of such characterizations. However, Voevodsky's aim was to go further and characterize a whole class of type theories and prove a general initiality result for them with the aim to form the basis for a general metatheory of dependent type theory. Our work can be viewed as a contribution to Voevodsky's programme, since we prove an initiality theorem for the whole class of finitely presented generalized algebraic theories. A characterization of a more general class of dependent type theories and their initial models has been proposed by Uemura \cite{uemura:general-framework}. Another related contribution is Palmgren and Vickers' \cite{palmgrenvickers} construction of initial models of essentially algebraic theories.

Altenkirch and Kaposi \cite{altenkirch:qiits} give several examples of {\em quotient inductive-inductive types (qiits)}. Their main example is a definition of dependent type theory with $\Pi$-types and a universe, as a simultaneous definition in the Agda system \cite{agda-wiki} of the data types $\Ctx$ of contexts, $\Sub(\Delta,\Gamma)$ of substitutions, $\Ty(\Gamma)$, and $\Tm(\Gamma,A)$ of terms. Their definition is {\em inductive-inductive} \cite{nordvallforsberg:iids}, since the index sets of $\Sub, \Ty,$ and $\Tm$ are generated simultaneously, and as a consequence are not indexed inductive definitions in the usual sense where the index sets are fixed in advance. Furthermore, it is a quotient inductive-inductive type since they also have constructors for identity types, as in a {\em higher inductive type}.
There is a close relationship between this qiit and our initial internal cwf with $\Pi$-types and a universe. Like our definition, their qiit-definition uses cwf-combinators. Moreover, our sort symbols correspond to their formation rules (data type constructors), our operator symbols correspond to their introduction rules (constructors), and our equations correspond to their propositional identities. However, the fact that our equations are judgmental equalities while theirs are propositional identities is an important difference. As a consequence they need to use transport maps when moving between identical types. 

%\footnote{Insert: Furthermore, Altenkirch et al \cite{altenkirch:thessaloniki} introduce a general schema for qiits and Kaposi, Kov{\'{a}}cs, and Altenkirch \cite{kaposi:qiits} construct initial algebras for them. For these constructions they work in cwfs with $\Pi$-types, intensional identity types, and a universe, while we work in plain cwfs. Although gats and qiits are related notions, neither is a generalization of the other. Gat is a basic dependently typed notion and independent of Martin-Löf type theory, whereas qiit is the latest in the series of more and more general notions of inductive type (inductive family, inductive-recursive type and family, inductive-inductive type, higher inductive type) extending intensional Martin-Löf type theory.
%
%Kaposi, Kov{\'{a}}cs, and Altenkirch \cite{kaposi:qiits} develop a general theory of qiits. This includes a notion of signature for a qiit, the notion of an algebra of such a signature, and a construction of initial algebras. There are obvious parallels with our work, but there are also fundamental differences:
%}

Kaposi, Kov{\'{a}}cs, and Altenkirch \cite{kaposi:qiits} develop a general theory of qiits. This includes a notion of signature for a qiit, the notion of an algebra of such a signature, and a construction of initial algebras. For these constructions they work in cwfs with $\Pi$-types, identity types, and a universe. This is in contrast to our work which is based on plain cwfs without extra structure for type formers. Although gats and qiits are related notions, neither is a generalization of the other. Gat is a basic notion independent of Martin-Löf type theory, whereas qiit is the latest in a series of generalizations of inductive type (inductive family, inductive-recursive type and family, inductive-inductive type, higher inductive type) extending intensional Martin-Löf type theory.

%Using the terminology of qiits, we here provide a definition of a class of valid qiits (in our modified sense) together with a class of models for each of them.


%Given that we have definition of a valid signature $\Sigma$ and a semantics for $\T_\Sigma$ we could ask whether we could add the construction of these to type theory. This would make most sense in extensional type theory, since we can then move seamlessly between propositional and judgmental equalities. It is however, not clear how to provide canonical form semantics in the sense of Martin-Löf's meaning explanations for this notion. However, although our development takes place in set theory, everything we do is constructive and could be formalized in CZF presumably.

\bibliographystyle{plain}
\bibliography{refs}
%\bibliography{}
%\input{referenc}
\end{document}

\end{document}
