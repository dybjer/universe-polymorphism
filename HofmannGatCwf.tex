\documentclass{lmcs}
%\usepackage{etex}
\usepackage[utf8]{inputenc}

\usepackage{color}
\usepackage{hyperref}
\usepackage{float}
\usepackage{amsmath}
\usepackage{amsfonts}
\usepackage{amsthm}
\usepackage{amssymb}
\usepackage{proof}
\usepackage{mathpartir}
\usepackage{mathrsfs}
\usepackage{stmaryrd}
\usepackage{cmll}
\usepackage{enumerate}
\usepackage{graphicx}
\usepackage[all]{xy}
\usepackage{listings}
\usepackage{todonotes}
\usepackage{Guyboxes}
%\DeclareMathOperator{\Ker}{Ker}
%\DeclareMathOperator{\nf}{nf}
%\DeclareMathOperator{\domain}{dom}
%\DeclareMathOperator{\codomain}{cod}
%\DeclareMathOperator{\cod}{cod}
%\DeclareMathOperator{\dom}{dom}
%\DeclareMathOperator{\ctxof}{ctx-of}
%\DeclareMathOperator{\typeof}{type-of}
%\DeclareMathOperator{\fix}{fix}

%\newcommand{\vdashS}{\ \vdash\ }
%\newcommand{\vdashS}{\vdash}
\newcommand {\emptyContext}{1}
\newcommand {\emptyContextI}{\diamond}
\newcommand {\emptyContextS}{\textbf 1}
\newcommand {\contextExtension}[2]{#1 \cdot #2}
\newcommand {\contextExtensionI}[2]{#1 \cdot #2}
\newcommand {\contextExtensionS}[2]{#1 \cdot #2}
\newcommand {\contextExtensionC}[2]{#1 \cdot_\C #2}

\newcommand {\GammaA}{\contextExtension \Gamma A}
\newcommand {\DeltaA}{\contextExtension \Delta A}
\newcommand {\setI}{\text{set}}
\newcommand {\setS}{\textbf{set}}
\newcommand {\depProd}[3]{\Pi(#1, #2, #3)}
\newcommand {\depProdI}[2]{\Pi(#1, #2)}
\newcommand {\depProdS}{\textbf{$\Pi$}}
\newcommand {\el}[2]{{\tt el}(#1, #2)}
\newcommand {\elI}[1]{{\tt el}(#1)}
\newcommand {\elS}{\textbf{el}}
\newcommand {\subType}[4]{{\tt subType}(#3, #4, #1, #2)}
\newcommand {\subTypeI}[2]{\text{subType}(#1, #2)}
\newcommand {\subTypeS}[2]{#1\{#2\}}
\newcommand{\subTypeC}[4]{\mathrm{subType}_\C(#3, #4, #1, #2)}
\newcommand {\q}[2]{{\tt q}_{#1, #2}}
\newcommand {\qI}{{\tt q}}
\newcommand {\qS}{\textbf{q}}
\newcommand{\lambdaAbs}[4]{\lambda(#1, #2, #3, #4)}
\newcommand{\lambdaAbsI}[1]{\lambda(#1)}
\newcommand{\lambdaAbsS}{\textbf{$\lambda$}}
\newcommand{\application}[5]{{\tt app}(#1, #2, #3, #4, #5)}
\newcommand{\applicationI}[2]{\text{app}(#1, #2)}
\newcommand{\applicationS}{\textbf{application}}
\newcommand{\subTerm}[5]{{\tt subTerm}(#4, #5, #1,#2,#3)}
\newcommand{\subTermI}[2]{\text{subTerm}(#1,#2)}
\newcommand{\subTermS}[2]{#1\{#2\}}
\newcommand{\idSub}[1]{{\tt id}(#1)}
\newcommand{\idSubI}{{\tt id}}
\newcommand{\idSubS}{\text{id}}
\newcommand{\proj}[2]{{\tt p}(#1, #2)}
\newcommand{\projI}{{\tt p}}
\newcommand{\projS}{\textbf{p}}
\newcommand{\comp}[5]{{\tt comp}(#1, #2, #3, #4, #5)}
\newcommand{\compI}[2]{{\tt comp}(#1, #2)}
\newcommand{\compS}[2]{#2 \circ #1}
\newcommand{\emptySub}[1]{\emptySubI_{#1}}
\newcommand{\emptySubI}{\langle\rangle}
\newcommand{\emptySubS}{\textbf !}
\newcommand{\extSub}[5]{\text{extension}(#1, #2, #3, #4, #5)}
\newcommand{\extSubI}[2]{\text{extension}(#1, #2)}
\newcommand{\extSubS}[2]{\langle #1, #2\rangle}
\newcommand{\Ctx}{\mathrm{Ctx}}
\newcommand{\Sub}{\mathrm{Sub}}
\newcommand{\Ty}{\mathrm{Ty}}
\newcommand{\Tm}{\mathrm{Tm}}
\newcommand{\C}{{\mathcal C}}
\newcommand{\I}{{\mathcal I}}
\newcommand{\T}{{\mathcal T}}

\newcommand{\Timp}{\T_{\text{imp}}}
\newcommand{\arrow}{{\rightarrow}}
\newcommand{\RawCtx}{{\tt Ctx}}
\newcommand{\RawSub}{{\tt Sub}}
\newcommand{\RawTy}{{\tt Ty}}
\newcommand{\RawTm}{{\tt Tm}}

\newcommand{\scomp}[6]{\mathrm{comp}_#1(#2, #3, #4, #5,#6)}

\newcommand{\inte}[1]{\llbracket #1 \rrbracket}
\newcommand{\intCtx}[1]{\llbracket #1 \rrbracket}
\newcommand{\intSub}[3]{\llbracket #3 \rrbracket_{#1,#2}}
\newcommand{\intTy}[2]{\llbracket #2 \rrbracket_#1}
\newcommand{\intTm}[3]{\llbracket #3 \rrbracket_{#1,#2}}
\newcommand{\ICtx}{{\I_0}}
\newcommand{\ISub}{{\I_1}}
\newcommand{\ITy}{{\I_2}}
\newcommand{\ITm}{{\I_3}}
\newcommand{\iniCtx}[1]{\overline{\llbracket #1 \rrbracket}}
\newcommand{\iniSub}[3]{\overline{\llbracket #3 \rrbracket}_{#1,#2}}
\newcommand{\iniTy}[2]{\overline{\llbracket #2 \rrbracket}_{#1}}
\newcommand{\iniTm}[3]{\overline{\llbracket #3 \rrbracket}_{#1,#2}}

\newcommand{\mejl}[3]{#1$\bigcirc\!\!\!\!\!\alpha\,$#2${}_{\cdot}$#3}

\newcommand{\bbN}[0]{{\mathbb N}}
\newcommand{\bbZ}[0]{{\mathbb Z}}
\newcommand{\bbQ}[0]{{\mathbb Q}}
\newcommand{\bbR}[0]{{\mathbb R}}
\newcommand{\bbB}[0]{{\mathbb B}}
\newcommand{\mU}[0]{{\mathcal U}}
\newcommand{\mT}[0]{{\mathcal T}}
\newcommand{\ve}[0]{{\varepsilon}}
\newcommand{\vf}[0]{{\varphi}}

\newcommand{\wellincluded}[0]{\, \Subset \,}

\newcommand{\memof}[0]{\, \epsilon \,}
\newcommand{\subseteqof}[0]{\, \dot{\subseteq} \,}

\newcommand{\mono}[0]{\to/ >->/}
\newcommand{\pto}[0]{\rightharpoondown}
\newcommand{\wellcov}[0]{{\lll}}
\newcommand{\waybelow}[0]{\ll}
\newcommand{\formint}[0]{\land}
\newcommand{\cov}[0]{{\, \lhd \,}}
\newcommand{\kov}[0]{{\, \lessdot \,}}
\newcommand{\kkov}[0]{{\, <: \,}}
\newcommand{\mutcov}[0]{\sim}
\newcommand{\balcov}[0]{\sqsubseteq}
\newcommand{\bal}[0]{{\sf b}}
\newcommand{\sat}[1]{{\rm Sat}(#1)}
\newcommand{\set}[0]{{\rm Set}}
\newcommand{\Set}[0]{{\bf Set}}
\newcommand{\true}[0]{{\sf T}}
\newcommand{\monus}{\stackrel{{}^{\scriptstyle .}}{\smash{-}}}

\newcommand{\refl}[0]{{\rm ref}}

\newcommand{\inl}[1]{{\sf inl}(#1)}
\newcommand{\inr}[1]{{\sf inr}(#1)}
\newcommand{\nat}[0]{{\mathbb N}}

\newcommand{\nattype}[0]{{\rm N}}
\newcommand{\bool}[0]{{\rm Bool}}
\newcommand{\ext}[1]{\langle #1 \rangle}


\newcommand{\bintree}[0]{{\rm T}_2}

%\newcommand{\sequent}[0]{\vdash}


\renewcommand{\conv}[0]{\approx}
\newcommand{\intimpl}[0]{\supset}

\newcommand{\omitthis}[1]{}

\newcommand{\changenote}[1]{}


 \newcommand{\Id}[0]{{\rm I}}
 

\newcommand{\longtext}[1]{}
\newcommand{\shorttext}[1]{}
\newcommand{\commentaway}[1]{}

\newcommand{\Setoid}[0]{{\bf Setoid}}

\definecolor{Red}{rgb}{1,0,0}
\newcommand{\red}[1]{{\color{Red}#1}}
%\newcommand{\red}[1]{}
\renewcommand{\bar}[1]{\overline{#1}}

%\newdir{pb}{:(1,-1)@^{|-}}
%\def\pb#1{\save[]+<16 pt,0 pt>:a(#1)\ar@{pb{}}[]\restore}

\newcommand{\Fam}{\textbf{Fam}}
\newcommand{\nilc}{1}
\newcommand{\cext}{.}
\newcommand{\indexed}[1]{\boldsymbol{#1}}
\newcommand{\Cat}{\mathrm{Cat}}
\newcommand{\op}{\text{op}}
\newcommand{\iso}{\cong}
\newcommand{\subst}[1]{\langle #1 \rangle}
\newcommand{\applyopen}[2]{\{ #1 \}  #2 }

% added by Marc to get things going. IMPROVE!

\def\N{\mathsf{N}}
\def\U{\mathsf{U}}
\def\F{\mathsf{F}}
\def\app{\mathsf{app}}
\def\Cop{\C^\op}
\def\Cobj{{\mathcal{C}_0}}
\def\p{\mathrm{p}}
\def\q{\mathrm{q}}
\newcommand{\tuple}[1]{\langle #1 \rangle}

\newtheorem{remark}{Remark}
\newtheorem{definition}{Definition}

%\def\N{\mathrm{N}}
\def\U{\mathrm{U}}
\def\p{{\tt p}}
\def\ev{{\tt ev}}
\def\q0{{\tt q}}
\def\r{{\tt r}}
\def\arrow{\rightarrow}
\def\Hom{\mathrm{Hom}}
\def\GammaA{\Gamma_{+,\times}}
\def\GammaCL{\Gamma_{\mathrm{CL}}}

\def\Dp{\mathrm{D}_p}
\def\notnotDp{\neg\neg\Dp}
\def\F{\mathcal{F}}
\def\HA{\mathbf{HA}}
\def\PA{\mathbf{PA}}
\def\I{\mathrm{I}}
\def\refl{\mathrm{r}}
\def\id{{\tt id}}
\def\idT{\mathrm{id}_\T}
\def\idC{\mathrm{id}_\C}
\newcommand{\pair}{\mathrm{pair}}
\newcommand{\fst}{\mathrm{fst}}
\newcommand{\interp}[1]{ \overline{\llbracket #1 \rrbracket}}
\newcommand{\Cwf}{\textbf{CwF}}
\newcommand{\Cwfs}{\Cwf_s}
\newcommand{\D}{\mathcal{D}}
\newcommand{\snd}{\mathrm{snd}}
\newcommand{\ap}{\mathrm{app}}
%\newcommand{\app}{\mathrm{app}}
\newcommand{\ini}[1]{\iniCtx{[#1]}}
\DeclareMathOperator{\cod}{cod}
\DeclareMathOperator{\dom}{dom}
\DeclareMathOperator{\ctxof}{ctx-of}
\DeclareMathOperator{\typeof}{type-of}
\newcommand{\vdashS}{\ \vdash\ }
\DeclareMathOperator{\domain}{dom}
\DeclareMathOperator{\codomain}{cod}


\newcommand{\isoCtx}[1]{\stackrel{#1}{\cong}}
\newcommand{\isoTy}[2]{\stackrel{#1}{\cong}_{#2}}
\newcommand{\equSub}[1]{=_{#1}}
\newcommand{\equTm}[2]{=_{#1,#2}}
\newcommand{\TT}{\mathbf{T}}

\newtheorem{theorem}{Theorem}
\newcommand{\s}{\mathrm{s}}
\newcommand{\Rec}{\mathrm{R}}
\newcommand{\Ta}{\mathrm{T}}
\newcommand{\ta}{\mathrm{t}}
\newcommand{\Ru}{\mathcal{R}}
\newcommand{\Nhat}{\hat{\N}}
\newcommand{\Pihat}{\hat{\Pi}}
\newcommand{\Tan}{\Ta_n}
\newcommand{\Un}{\U_n}
\newcommand{\Nhatn}{\N^n}
\newcommand{\Pihatn}{\Pi^n}
\newcommand{\Nn}{\Nhatn}
\newcommand{\Pin}{\Pihatn}
\newcommand{\TRu}{\Ta_\Ru}
\newcommand{\URu}{\U_\Ru}
\newcommand{\NRu}{\N_\Ru}
\newcommand{\PiRu}{\Pi_\Ru}
\newcommand{\TRun}{{(\Ta_\Ru)}_n}
\newcommand{\URun}{{(\U_\Ru)}_n}
\newcommand{\NRun}{{(\N_\Ru)}^n}
\newcommand{\PiRun}{{(\Pi_\Ru)}^n}
\newcommand{\TRum}{{(\Ta_\Ru)}_m}
\newcommand{\URum}{{(\\U_\Ru)}_m}
\newcommand{\TC}{\Ta_\C}
\newcommand{\UC}{\U_\C}
\newcommand{\NC}{\N_\C}
\newcommand{\PiC}{\Pi_\C}
\newcommand{\Level}{\mathrm{Level}}
\def\Sort{\mathcal{S}}
\def\Op{\mathcal{O}}
\def\Eq{\mathcal{E}}
\def\D{\mathcal{D}}
\def\Cwf{\mathbf{CwF}}
\def\Obj{\mathrm{Obj}}
\def\Hom{\mathrm{Hom}}
\def\id{\mathrm{id}}



\title[Generalized Algebraic Theories and Categories with Families]{A Note on Generalized Algebraic Theories\\and Categories with Families}\author{Marc Bezem, Thierry Coquand, Peter Dybjer, Mart\'in Escard\'o}

\begin{document}

\maketitle

\begin{abstract}
We define a new finitary version of generalized algebraic theories. It differs from Cartmell's original syntactic definition in two respects. Whereas Cartmell allows a possibly infinite set of sort and operator symbols, we only allow a finite number. Moreover, our definition is categorical in the sense that it is based on the notion of initial cwf, and abstract in that it is independent of the particular construction of initial cwfs in terms of syntax and inference rules. This is in the spirit of Voevodsky's "initiality conjecture", whereby type theories are characterized as initial objects in appropriate categories of models of type theory.  Our main result is that the category of cwfs supporting a certain signature for gats has an initial object. This result is obtained by extending the construction of initial cwfs from Castellan, Clairambault, and Dybjer. We also point out that cwfs supporting the gats of categories cwfs, are cwfs with internal categories, cwfs, etc.
\end{abstract}

\section{Introduction}

Cartmell's definition of a generalized algebraic theory as a dependently typed generalization of many sorted algebraic theory \cite{cartmell:phd,cartmell:}. Mention his notion of contextual category. Mention Martin Hofmann's notion of category with attributes (cwa) and its relation to contextual categories. Also remark that Cartmell used the name cwa for a slightly different notion. Mention cwfs and Martin Hofmann's paper on "Syntax and Semantics of Dependent Types" which is based on cwfs. (Maybe one could actually elaborate this to a survey of Martin's contributions to the semantics of dependent type theory.) We could motivate this note by saying that it could be added to "Syntax and Semantics of Dependent Types" as yet another basic result for dependent type theory.

Then discuss Cartmell's definition a bit more and his notion of "derived rule". Cartmell's definition of a gat is purely syntactic, with no reference to categories. In particular it is not connected to initiality. On the other hand on of the theorems is an equivalence between the category of gats and the category of contextual categories. I would say that our notion is simpler and immediately gives rise to an initial structure.

Discuss the notion of cwf and why initial cwfs (with extra structure) is appropriate as an "abstract syntax" for dependent type theory, abstracting away from representation detail. Mention Voevodsky's initiality conjecture. Refer to Brunerie.

Also point out that the point of cwfs is to have a categorical notion of model which immediately gives rise to a generalized algebraic theory, a kind of idealized dependently typed syntax. Point out that cwfs and gats are mutually dependent of each other: To know what a cwf is we need to know what a gat is and to know what a gat is we need to know what a cwf is. Point out that cwfs appear on two levels in our account (i) the fundamental (set-theoretic) notion of cwf underlying the definition of gat (ii) the notion of "internal cwf" which is obtained by considering the models of the the gat of cwfs.

200724: 
\begin{itemize}
\item mention "line of work in the spirit of Voevodsky's initiality conjecture", general definition of "a type theory".
\item emphasize that Cartmell's original definition is syntactic. Is it infinitary?
\item typical phenomenon in category that you define something first and only later show that it exists.
\end{itemize}

\section{Categories with families}\label{sec:def_cwf}

We first define the category $\Fam$, whose objects are
set-indexed families of sets, denoted as $(U_x)_{x\in X}$.
A morphism of $\Fam$ with source $(U_x)_{x\in X}$ and target $(V_y)_{y\in Y}$
consists of a re-indexing function $f: X\to Y$ together with a family
$(g_x)_{x\in X}$ of functions $g_x : U_x \to V_{f(x)}$. %, for all $x\in X$.

The next step is to define the category $\Cwf$. First we define the
objects of $\Cwf$, which are called \emph{categories with families}.
Then we briefly describe the morphisms of $\Cwf$. Since $\Cwf$ has 
been developed as a categorical framework for the semantics of
type theory, much of the terminology (contexts, substitutions,
types, terms) refers the syntax of type theory, 
suggesting the intended interpretation of this syntax in the 
so-called $\Cwf$-semantics.

The main novelty of this note is to use $\Cwf$ as a framework
to define a new notion of a generalised algebraic theory. 
Contexts, substitutions and terms also make
sense in relation to gat's. Perhaps it would have been natural
to replace the phrase `type' by `sort' (ref to Makkai, FOLDS?),
but we have chosen to stick to existing terminology.
It is important to stress, however, that the definition
of the notion of a generalised algebraic theory in this note
is not a syntactical one. 


\begin{definition}\label{def:cwf} 
A category with families (cwf) consists of the following data:
\begin{itemize} 

\item A category $\C$ with a terminal object $1$ and maps
$\tuple{}_\Gamma \in \C(\Gamma, 1)$ for all objects $\Gamma$ of $\C$;

\item A $\Fam$-valued presheaf on $\C$, that is, a functor
$T : \Cop \to \Fam$;

\item Operations ${\cdot},~\p,~\q$ and $\tuple{\_,\_}$ 
explained in the following paragraphs.
\end{itemize}

We let $\Gamma, \Delta,\ldots$ range over objects of $\C$, 
and refer to them as \emph{contexts}. 
We let $\delta, \gamma,\ldots$ range over morphisms, 
and refer to them as \emph{substitutions}. 
We refer to $1$ as the \emph{empty context}; the terminal maps
$\tuple{}_\Gamma$ represent the \emph{empty substitution}.

If $T(\Gamma) = (U_x)_{x\in X}$, we write $\Ty(\Gamma)$ for $X$.
We call the elements of the set $\Ty(\Gamma)$
\emph{types in context $\Gamma$}, and let $A, B, C$ range over such types. 
Further, for $A \in \Ty(\Gamma)$, we write $\Tm(\Gamma, A)$ for $U_A$
and call the elements of the set $\Tm(\Gamma, A)$
\emph{terms of type $A$ in context $\Gamma$}. 

For $\gamma : \Delta \to \Gamma$,
the (contravariant) functorial action of $T$ yields a morphism
\[
T(\gamma) \in  \Fam((\Tm(\Gamma, A))_{A\in \Ty(\Gamma)}, % \to 
                (\Tm(\Delta, B))_{B\in \Ty(\Delta)})
\]
consisting of a reindexing function $\_\,[\gamma] : \Ty(\Gamma) \to
\Ty(\Delta)$ referred to as \emph{substitution in types}, and for each $A\in
\Ty(\Gamma)$ a function $\_\,[\gamma] : \Tm(\Gamma, A) \to \Tm(\Delta,
A[\gamma])$ referred to as \emph{substitution in terms}.
%\item a presheaf $\Ty : \Cop \to \Set$ of types ;
%\item a presheaf $\Tm : (\int^\C \Ty)^\op \to \Set$ of terms;
A \emph{context comprehension operation} which to a given context $\Gamma
\in \C$ and type $A \in \Ty(\Gamma)$ assigns a context $\Gamma \cext
A$ and two projections
\[
\p_{\Gamma, A} : \Gamma \cext A \to \Gamma
\qquad\qquad
\q_{\Gamma, A} \in \Tm(\Gamma\cext A, A[\p_{\Gamma,A}])
\] 
satisfying the following universal property: for all $\gamma : \Delta \to
\Gamma$, for all $a\in \Tm(\Delta, A[\gamma])$, there is a unique
$\tuple{\gamma, a} : \Delta \to \Gamma \cext A$ such that
\[
\p_{\Gamma, A} \circ \tuple{\gamma, a} = \gamma
\qquad \qquad
\q_{\Gamma, A} [\tuple{\gamma, a}] = a\,.
\]
%
%which is a terminal object in a category of triples
%$$
%(\Delta \in \C\Obj, \gamma \in \C(\Delta,\Gamma), a \in \Tm(\Delta,A[\gamma]))
%$$
%where we use the notation $A[\gamma] = \Ty(\gamma)(A)$ for $\gamma \in \C(\Delta,\Gamma)$.
We say that $(\Gamma\cext A, \p_{\Gamma, A}, \q_{\Gamma, A})$ is a
\emph{context comprehension} of $\Gamma$ and $A$.

\end{definition}

\section{An abstract definition of generalized algebraic theories}

The {\em signature} $\Sigma$ for a generalized algebraic theory (gat) is a triple $(\Sort,\Op,\Eq)$ consisting of a list of sort symbols $\Sort$, operator symbols $\Op$, and equations (between terms) $\Eq$, with type information, as follows. (Cartmell also allows equations between sort terms, but most examples do not make use of such, and we skip it for simplicity.) We shall now define how to build a valid gat signature $\Sigma$ and what it means for a cwf to support it. This definition relies on the construction of initial cwfs $\T_\Sigma$ supporting $\Sigma$. We proceed by induction on the size of $\Sigma$, but will postpone the construction of the initial cwfs until the next section. In this way we separate the abstract definition from the concrete syntactic details employed for building $\T_\Sigma$.
\begin{definition}
We define inductively how to build a valid signature $\Sigma$ and the category $\Cwf_\Sigma$ of cwfs that support $\Sigma$ and cwf-morphisms that preserve it.
\begin{description}
\item[The empty signature] The empty signature $\emptyset = ([],[],[])$ is valid and $\Cwf_\emptyset = \Cwf$. 
\end{description}
Assume now that $\Sigma$ is a valid signature and that the category $\Cwf_\Sigma$ has an initial object $\T_\Sigma$ with $\inte{-} : \T_\Sigma \to \C$ as the unique morphism into an object $\C$. Then we can add a new type symbol, or a new operator symbol, or a new equation, as follows:
\begin{description}
\item[Adding a new sort symbol] 
If $\Gamma$ is a context in $\T_\Sigma$, then we can extend $\Sigma$ with a new sort symbol $F$ (with context $\Gamma$) to obtain an extended gat $\Sigma'$. A cwf $\C$ supports $\Sigma'$ if it supports $\Sigma$ and there is $F_\C \in \Ty_\C(\inte{\Gamma})$. A cwf-morphism in $\C \to \D$ preserves $\Sigma'$ if it preserves $\Sigma$ and maps $F_\C$ to $F_\D$.
\item[Adding a new operator symbol] 
If $\Gamma$ is a context in $\T_\Sigma$ and $A \in \Ty_{\T_\Sigma}(\Gamma)$, then we can extend $\Sigma$ with a new operator symbol $f$ (with context $\Gamma$ and type $A$). A cwf $\C$ supports $\Sigma'$ if it supports $\Sigma$ and there is $f_\C \in\Tm_\C(\inte{\Gamma},\inte{A}_{\Gamma})$.
A cwf-morphism in $\C \to \D$ preserves $\Sigma'$ if it preserves $\Sigma$ and maps $f_\C$ to $f_\D$.
\item[Adding a new equation] 
If $\Gamma$ is a context in $\T_\Sigma$, $A \in \Ty_{\T_\Sigma}(\Gamma)$, and $a, a' \in \Tm_{\T_\Sigma}(\Gamma,A)$, then we can extend $\Sigma$ with a new equation $a = a'$ (with context $\Gamma$ and type $A$). Then a cwf $\C$ that supports $\Sigma'$ is obtained by adding the requirement that $\inte{a}_{\Gamma,A}= \inte{a'}_{\Gamma,A} \in \Tm_\C(\inte{\Gamma},\inte{A}_\Gamma)$.
\end{description}
\end{definition}

Note that the empty signature $\emptyset$ is the only valid signature that can be directly constructed from the definition. In order to form other signatures we need to construct initial cwfs supporting already constructed signatures. This is the topic of the next section.

\section{The construction of an initial cwf supporting a gat}

\begin{theorem}
The category $\Cwf_\Sigma$ has an initial object $\T_\Sigma$ for every valid signature $\Sigma$.
\end{theorem}

The proof is by induction on the construction of $\Sigma$.
\begin{description}
\item[The empty signature] 
The category $\Cwf = \Cwf_\emptyset$ of cwfs and strict cwf-morphisms has an initial object $\T_\emptyset$ \cite{castellan}.
\end{description}

We recall that this initial cwf is defined by first defining grammars for raw contexts, raw substitutions, raw types, and raw terms. Then four families of partial equivalence relations corresponding to the four judgment forms
\begin{eqnarray*}
\\&&\Gamma = \Gamma' \vdash
\\&&\Delta \vdash \gamma = \gamma' : \Gamma
\\&&\Gamma \vdash A = A'
\\&&\Gamma \vdash a = a' : A
\end{eqnarray*}
are defined inductively by inference rules. The four judgment forms
\begin{eqnarray*}
\\&&\Gamma \vdash
\\&&\Delta \vdash \gamma : \Gamma
\\&&\Gamma \vdash A
\\&&\Gamma \vdash a : A
\end{eqnarray*}
are then defined as the reflexive instances of the partial equivalence relations. The initial cwf $\T$ is then constructed from the equivalence classes of derivable judgments. For example, the contexts in $\T$ are equivalence classes $[\Gamma]$ with respect to the pers, such that $\Gamma \vdash$ See Castellan, Clairambault, and Dybjer \cite{castellan:warsaw,castellan:lmcs} for details.

Assume we have specified grammars and inference rules for $\T_\Sigma$ and that we write $\vdash_\Sigma$ for derivability by these inference rules. Let $\Sigma'$ be $\Sigma$ extended by a new sort symbol, a new operator symbol, or a new equation. We shall now explain how to build $\T_{\Sigma'}$, to extend the grammars and inference rules for $\T_\Sigma$.
\begin{description}
\item[Adding a new sort symbol] 
If $\Gamma$ is a context in $\T_\Sigma$ and $\Sigma'$ is $\Sigma$ extended with a new sort symbol $F$ (with context $\Gamma$). Then $\T_{\Sigma'}$ is defined by adding the production
$$
A ::= F
$$
to the grammar for raw types, and the inference rule
\begin{mathpar}
    \inferrule
    {}
    {\Gamma \vdash_{\Sigma'} F}
  \end{mathpar}
to the inference rules for $\Sigma$.

We then define $F_{\T_{\Sigma'}} = F$. It follows that $ \T_{\Sigma'}$ supports $\Sigma'$. 

Moreover, we extend the definition of the interpretation morphism $\inte{-}$  to an interpretation morphism $\inte{-}' : \T_{\Sigma'} \to \C$ by 
$$
\inte{[F]}' = F_\C
$$
It follows immediately that this is a morphism in $\Cwf_{\Sigma'}$ and that it is unique.

\item[Adding a new operator symbol] 
If $\Gamma$ is a context in $\T_\Sigma$, $A \in \Ty_{\T_\Sigma}(\Gamma)$ and $\Sigma'$ is $\Sigma$ extended with a new operator symbol $f$ (with context $\Gamma$ and type $A$). Then $\T_{\Sigma'}$ is defined by adding the production
$$
a ::= f
$$
to the grammar for raw terms, and the inference rule
\begin{mathpar}
    \inferrule
    {}
    {\Gamma \vdash_{\Sigma'} f : A}
\end{mathpar}
to the inference rules for $\Sigma$.

We then define $f_{\T_{\Sigma'}} = f$ and extend the definition of the interpretation morphism $\inte{-}$  to an interpretation morphism $\inte{-}' : \T_{\Sigma'} \to \C$ by 
$$
\inte{[f]}' = f_\C
$$
It follows immediately that this is a morphism in $\Cwf_{\Sigma'}$ and that it is unique.

\item[Adding a new equation] 
If $\Gamma$ is a context in $\T_\Sigma$, $A \in \Ty_{\T_\Sigma}(\Gamma)$, $a, a' \in \Tm_{\T_\Sigma}(\Gamma,A)$, and $\Sigma'$ is $\Sigma$ extended with a new equation $a = a'$ (with context $\Gamma$ and type $A$). Then $\T_{\Sigma'}$ is defined by adding the inference rule
 \begin{mathpar}
    \inferrule
    {}
    {\Gamma \vdash_{\Sigma'} a = a' : A}
\end{mathpar}
to the inference rules for $\Sigma$.

It is immediate that $\T_{\Sigma'}$ supports $\Sigma'$.
The interpretation morphism $\inte{-}'$ is defined as $\inte{-}$ but on the new equivalence classes. It immediately follows that it is a morphism in $\Cwf_{\Sigma'}$ and that it is unique.\end{description}

\section{Examples of generalized algebraic theories}

\subsection{Categories} 
The generalized algebraic theory of categories has sort symbols $\Obj$ 
and $\Hom$, operator symbols $\id$ for identity and $\circ$ for composition, and associativity and identity laws as equations. 
\begin{itemize}
\item The initial cwf $\T_\emptyset$ has only one object (context) 1, one equivalence class of morphisms $\id_1$, and no types and terms. Hence we can add a new constant sort $\Obj$ of (internal) objects with context $1$ to the signature. 
\item $\T_{([\Obj],[],[])}$ contains the context $(1.\Obj).\Obj[\p_{1,\Obj}]$ (corresponding to the context $x : \Obj, y : \Obj$ in usual notation with variables). Hence we can introduce a new sort $\Hom$ with this context.
\item $\T_{([\Obj, \Hom],[],[])}$ contains the context $1.\Obj$ (corresponding to the context $x : \Obj$ in usual notation) and the type $\Hom[\tuple{\id_{1.\Obj},\q_{1,\Obj}}]$ (corresponding to the type $\Hom(x,x)$). Hence we can introduce an operator symbol $\id$ with this context and type.
\item In a similar way we can add the operator symbol for composition and the equations, but we omit the details.
\end{itemize}

We say that a cwf that supports the generalized algebraic theory of categories is a cwf with an {\em internal category}. This is a cwf-based analogue of the usual notion of internal category in a category with finite limits.

\subsection{Example: (internal) cwfs} Similarly, we can define a cwf that supports the generalized algebraic theory of cwfs. We add sort symbols for types and terms, operator symbols for all the cwf-combinators, and all the cwf-equations. We say that a cwf that supports this generalized algebraic theory is a cwf with an internal cwf.

\subsection{Example: (internal) cwfs with $\Pi$-types} 
We add operator symbols $\Pi, \lambda, \app$ and equations $\beta, \eta$ to the generalized algebraic theory of cwfs. 

\subsection{Example: (internal) cwfs with $\N$-types} 
We add operator symbols $\N, 0, \s, \Rec$ and equations for $\Rec$.

\subsection{Example: (internal) cwfs with $\U_0$ closed under $\Pi$ and $\N$} 
We add operator symbols $\U_0, \Ta_0$, the code operations $\N^0, \Pi^0$, and the decoding equations.

\subsection{Example: (internal) cwfs with universe tower structures} We introduce a new sort symbol for levels and new operator symbols for 0, $\s$, and $\vee$ for levels with the equations for $\vee$. Then we have the other operator symbols and equations for universe tower structures.

Note that we have here extended the notion of cwf with a sort of levels, that is, we are no longer strictly within the framework of (internal) cwfs.

\subsection{Example: (internal) cwfs with universe polymorphic tower structures} Here we need to extend the cwf-framework further to take into account contexts with level variables, etc.

\section{Generalized algebraic theories and essentially algebraic theories}

It time permits. A remark about the biequivalence between categories with finite limits and democratic cwfs with $\Sigma$ and extensional identity types, as the basis for the correspondence between essentially and generalized algebraic theories.
\end{document}
