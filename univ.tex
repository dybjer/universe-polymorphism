%\documentclass[12pt,a4paper]{amsart}
\documentclass[11pt,a4paper]{article}
%\ifx\pdfpageheight\undefined\PassOptionsToPackage{dvips}{graphicx}\else%
%\PassOptionsToPackage{pdftex}{graphicx}
\PassOptionsToPackage{pdftex}{color}
%\fi

%\usepackage{diagrams}

%\usepackage[all]{xy}
\usepackage{url}
\usepackage[utf8]{inputenc}
\usepackage{verbatim}
\usepackage{latexsym}
\usepackage{amssymb,amstext,amsmath,amsthm}
\usepackage{epsf}
\usepackage{epsfig}
% \usepackage{isolatin1}
\usepackage{a4wide}
\usepackage{verbatim}
\usepackage{proof}
\usepackage{latexsym}
%\usepackage{mytheorems}
\newtheorem{theorem}{Theorem}[section]
\newtheorem{corollary}{Corollary}[theorem]
\newtheorem{lemma}{Lemma}[theorem]
\newtheorem{proposition}{Proposition}[theorem]
\theoremstyle{definition}
\newtheorem{definition}[theorem]{Definition}
\newtheorem{remark}{Remark}[theorem]
\newtheorem{TODO}{TODO}[theorem]

\usepackage{float}
\floatstyle{boxed}
\restylefloat{figure}


%%%%%%%%%copied from SymmetryBook by Marc

% hyperref should be the package loaded last
\usepackage[backref=page,
            colorlinks,
            citecolor=linkcolor,
            linkcolor=linkcolor,
            urlcolor=linkcolor,
            unicode,
            pdfauthor={CAS},
            pdftitle={Symmetry},
            pdfsubject={Mathematics},
            pdfkeywords={type theory, group theory, univalence axiom}]{hyperref}
% - except for cleveref!
\usepackage[capitalize]{cleveref}
%\usepackage{xifthen}
\usepackage{xcolor}
\definecolor{linkcolor}{rgb}{0,0,0.5}

%%%%%%%%%
\def\oge{\leavevmode\raise
.3ex\hbox{$\scriptscriptstyle\langle\!\langle\,$}}
\def\feg{\leavevmode\raise
.3ex\hbox{$\scriptscriptstyle\,\rangle\!\rangle$}}

%%%%%%%%%
\newcommand\myfrac[2]{
 \begin{array}{c}
 #1 \\
 \hline \hline
 #2
\end{array}}


\newcommand*{\Scale}[2][4]{\scalebox{#1}{$#2$}}%
\newcommand*{\Resize}[2]{\resizebox{#1}{!}{$#2$}}

\newcommand{\II}{\mathbb{I}}
\newcommand{\refl}{\mathsf{refl}}
\newcommand{\mkbox}[1]{\ensuremath{#1}}


\newcommand{\Id}{\mathsf{Id}}
\newcommand{\conv}{=}
%\newcommand{\conv}{\mathsf{conv}}
\newcommand{\lam}[2]{{\langle}#1{\rangle}#2}
\def\NN{\mathsf{N}}
\def\UU{\mathsf{U}}
\def\JJ{\mathsf{J}}
\def\Level{\mathsf{Level}}
%\def\Type{\hbox{\sf Type}}
\def\ZERO{\mathsf{0}}
\def\SUCC{\mathsf{S}}

\newcommand{\type}{\mathsf{type}}
\newcommand{\mypi}[3]{\Pi_{#1:#2}#3}
\newcommand{\mylam}[3]{\lambda_{#1:#2}#3}
\newcommand{\mysig}[3]{\Sigma_{#1:#2}#3}
\newcommand{\N}{\mathsf{N}}
\newcommand{\Set}{\mathsf{Set}}
\newcommand{\El}{\mathsf{El}}
%\newcommand{\U}{\mathsf{U}} clashes with def's in new packages
\newcommand{\T}{\mathsf{T}}
\newcommand{\Usuper}{\UU_{\mathrm{super}}}
\newcommand{\Tsuper}{\T_{\mathrm{super}}}
%\newcommand{\conv}{\mathrm{conv}}
\newcommand{\idtoeq}{\mathsf{idtoeq}}
\newcommand{\isEquiv}{\mathsf{isEquiv}}
\newcommand{\ua}{\mathsf{ua}}
\newcommand{\UA}{\mathsf{UA}}
%\newcommand{\Level}{\mathrm{Level}}
\def\Constraint{\mathsf{Constraint}}
\def\Ordo{\mathcal{O}}

\def\Ctx{\mathrm{Ctx}}
\def\Ty{\mathrm{Ty}}
\def\Tm{\mathrm{Tm}}

\def\CComega{\mathrm{CC}^\omega}
\setlength{\oddsidemargin}{0in} % so, left margin is 1in
\setlength{\textwidth}{6.27in} % so, right margin is 1in
\setlength{\topmargin}{0in} % so, top margin is 1in
\setlength{\headheight}{0in}
\setlength{\headsep}{0in}
\setlength{\textheight}{9.19in} % so, foot margin is 1.5in
\setlength{\footskip}{.8in}

% Definition of \placetitle
% Want to do an alternative which takes arguments
% for the names, authors etc.

\newcommand{\natrec}{\mathsf{natrec}}
%\rightfooter{}
\newcommand{\set}[1]{\{#1\}}
\newcommand{\sct}[1]{[\![#1]\!]}



\begin{document}

\title{A Note on Universe Levels}

\author{Marc Bezem, Thierry Coquand, Peter Dybjer, Mart\'in Escard\'o}
\date{}
\maketitle

\begin{abstract}
  The goal of this paper is to refine a proposal of Voevodsky \cite{VV} for handling universes in type theory.
  We recall first the version of type theory with an externally indexed sequence of universes \`a la Tarski and
  with and without cumulativity.
  We introduce then an internal notion of universe level.
  We add a new judgment $l\ \Level$ meaning that $l$ is a universe level, and $l = m$ meaning that $l$ and $m$ are equal universe levels, where level expressions are built up from level variables $\alpha$ by a successor operation $l^+$ and suprema $l \vee m$.
  %In this way judgments can be indexed by level variables, and we obtain universe polymorphism similar to polymorphism in the functional language ML.
  We then extend this theory with rules for level-indexed product types $[\alpha]A$ meaning ``$A$ is a type for all universe levels $\alpha$".
  The final theory extends a proposal by Voevodsky
  for a system with level constraints. We discuss the (solved) decision problems for sup-semilattices with
  successor that come with this approach.
\end{abstract}


\section{Introduction}\label{sec:intros}
\newcommand{\ttt}[1]{\text{\tt #1}}

The system of simple type theory, as introduced by Church \cite{church:formulation},
is elegant and the basis of several
proof assistants. It however has some unnatural limitation: it is not possible in this system to talk
about an arbitrary type, or about an arbitrary structure. It is also not possible to formulate the collection
of e.g. all groups, as needed in category theory. In order to address these limitations, Martin-L\"of
\cite{ML71,ML71a} introduced a system with a type $V$ of all types. A function $A\rightarrow V$ in this system can
then be seen as a family of types over a given type $A$, and it is natural in such a system to refine
the operations of simple type theory, exponential and cartesian product, by the operations of dependent products
and sums. After the discovery of Girard's paradox \cite{Girard71}, Martin-L\"of \cite{ML72}
introduced a distinction between
{\em small} and {\em large} type, similar to the distinction introduced in category theory between large and small sets,
and the type $V$ became the (large) type of small types.
The name ``universe'' for such a type was chosen
in analogy with the notion of universes introduced by Grothendieck to represent category theory
in set theory.


%% The earliest record of the use of universes in type theory that we could
%% find is in the paper \cite{deBruijn68} by de Bruijn on Automath.
%% In this paper, types are called categories, but not all categories are types.
%% Automath has a special symbol\footnote{PD: category?} \ttt{type} to make it possible to introduce
%% new categories, and only categories \ttt{c} introduced as $\ttt{c}:\ttt{type}$
%% are types, not \ttt{type} itself (so, not $\ttt{type}:\ttt{type}$).
%% One new type is $\ttt{bool}:\ttt{type}$,
%% which is the category of all propositions.
%% Propositions may or may not be asserted
%% and to this purpose there is a primitive notion\footnote{PD: category?} \ttt{TRUE}
%% of category \ttt{type}
%% in the context of a free variable of type \ttt{bool}.
%% By substitution one can form the type \ttt{TRUE(p)} for any $\ttt{p}:\ttt{bool}$.
%% In modern terminology we would call \ttt{type} a universe of
%% types and \ttt{bool} a subuniverse \emph{à la Tarski} of
%% (codes of) propositions with a decoding function \ttt{TRUE}.
%% \footnote{PD: this seems similar to Per's view of type theory in 1985, just before he introduced the logical framework. Then he talked about "the category of types" and "the category of objects of a type". Among the types he had a universe U which could be formulated a la Tarski with T(a) the type of elements of a : U. How was  \ttt{TRUE(p)} defined? Is it similar to T(a)?}
%At the very end, the paper discusses the possibility of adding a symbol
%\ttt{type*} and changing the category of \ttt{bool} from \ttt{type}
%to \ttt{type*}, disallowing functional abstraction with respect to \ttt{bool}.
%The possibility of functional abstraction with respect to \ttt{type} is mentioned and rejected.


%When representing mathematical reasoning using dependent types it is natural to introduce
%a sort for types
%\footnote{PD: This sentence is not clear: what is a "sort"?}
%, itself closed by dependent product, as was done in the system Automath \cite{deBruijn68}.
%% A little later, and independently, Martin-Löf \cite{ML71} introduced a dependent
%% type system with a \emph{type} $V$ of all types (so $V:V$).
%% After the discovery of Girard's Paradox \cite{Girard71}, this system was replaced by a predicative
%% version \cite{ML72}\footnote{PD: in that paper there were just two levels: "types" and "small types" (in the universe).}, more similar to the one used in Automath \cite{deBruijn68}.
Later, Martin-L\"of \cite{martinlof:predicative} introduced a countable sequence of universes
$$
\UU_0 : \UU_1 : \UU_2 : \cdots
$$
We refer to the indices $0, 1, 2, \ldots$ as {\em universe levels}. 

Before the advent of univalent foundations, one expected only the first few universe
levels to be relevant in practical formalisations. This suggests that it might be feasible
for a user of type theory to explicitly assign universe levels to their types, and whenever necessary
repeat definitions when they are needed on different levels. However, the number of copies of definitions
does not only grow with the level, but also with the number of type arguments in the definition of a type former.
The need for universe polymorphism has increased with the advent of Voevodsky's univalent foundations,
see for example \cite{VV}. One often wants to prove theorems uniformly for an arbitrary universe.

The {\em univalence axiom} states that for any two types $X,Y$ the canonical map
$$
\idtoeq_{X,Y} : (X=Y)\to (X\simeq Y)
$$
is an equivalence.
Formally, the univalence axiom is an axiom scheme which is added to Martin-Löf type theory. If we work in Martin-Löf type theory with a countable tower of universes, each type is a member of some universe $\UU_n$. Such a universe $\UU_n$ is {\em univalent} provided for all $X,Y : \UU_n$ the canonical map $\idtoeq_{X,Y}$ is an equivalence. Let $\UA_n$ be the type expressing the univalence of $\UU_n$, and
$$
\ua_n : \UA_n
$$
for $n = 0,1,\ldots$, be a sequence of constants expressing the instances of the univalence axiom. We note that $X = Y : \UU_{n+1}$ and $X\simeq Y : \UU_n$ and hence $\UA_n$ is in $\UU_{n+1}$.
If we have a type of levels, as in the Agda or Lean system,
we can express universe polymorphism as quantification over universe levels.
%% Then we can express univalence of all universes as one typing:
%% $$
%% \ua : (l : \Level) \to \UA_l
%% $$
%% in Agda's notation.

We remark that universes are more important in a predicative framework than in an impredicative one.
Consider for example the formalisation of real numbers as Dedekind cuts, or domain elements as filters of formal neighbourhoods, which belong to $\UU_1$ since they are properties of elements in $\UU_0$.
However, even in a system using an impredicative universe of propositions,
such as the ones in \cite{Huet87,moura:lean}, there is the need for the use of definitions parametric
in universe levels.

%{\color{red} Cite Lean \cite{moura:lean}?}

 To deal with this duplication problem Huet \cite{Huet87} and, later,
 Harper and Pollack \cite{HarperP91} introduced {\em universe polymorphism}.
 This "implicit"
 approach to universe polymorphisms however is problematic w.r.t. modularity, as pointed
 out in \cite{Courant02,Simpson04}: one can prove $A\rightarrow B$ in one file, and $B\rightarrow C$ in one other file, while
$A\rightarrow C$ is not valid. In order to cope with this issue, J. Courant \cite{Courant02}
suggested to have explicit level universes, with a sup operation (see also \cite{herbelin05}).
This approach is now followed in Agda and in Voevodsky's proposal \cite{VV}.

Our system complements
\cite{Courant02,VV} by handling constraints on universe levels and having instantiation operations.

 %This is a practical problem since universe level inference can be a costly operation, sometimes more costly than the usual type-checking.

%There is a connection between the dimension of a type and the level of a universe: for example, it is natural to consider the groupoid structure of the first universe, the 2-groupoid structure of the second universe, etc.\footnote{PD: I changed this paragraph. What do you think?}

%The starting point for the present discussions was Escard\'o's \cite{escardo} development of univalent mathematics in the Agda system. His development is {\em universe polymorphic} and makes heavy use of quantification over Agda's type $\Level$ of universe levels. However, the Agda system is not a precisely defined logical system, and we would like to introduce several candidate systems in which Escard\'o's development can be carried out. We will also present a system similar to Voevodsky's universe polymorphic system \cite{VV}. (** MODIFY THIS STAMEMENT AND ALL OTHER REFERENCES TO VV **) In this system there are variables ranging over universe levels and each judgment is made relative to a list of equational constraints between universe level expressions.


\paragraph{Plan.} We start by giving the rules for a basic version of dependent type theory with
$\Pi, \Sigma, \NN$, and an identity type former $\Id$.
We then explain how to add an externally indexed countable sequence of universes $\UU_n, \T_n$ \`a la Tarski with or without
cumulativity rules.
In Section \ref{sec:internal} we introduce the notion of universe level, and let judgments depend not only on a context of ordinary variables, but also on a list of level variables $\alpha_1, \ldots, \alpha_k$, giving rise to a theory with level polymorphism. This is a kind of ML-polymorphism since we only quantify over global level variables. We then further extend the theory with level-indexed products of types $[\alpha]A$.

In Section \ref{sec:constraints} we present (a variation of) Voevodsky's proposal with constraints between level variables  \cite{VV}, complementing his proposal with a way to eliminate constants introduced with level variables and constraints.

\section{Rules for a basic type theory}

We begin by listing some rules for a basic type theory with $\Pi, \Sigma, \N,$ and $\Id$, such as the rules of context formation and assumption:
$$
\frac{\Gamma\vdash A}{\Gamma,x:A\vdash}~~~~~~\frac{}{()\vdash}~~~~~~~
\frac{\Gamma\vdash}{\Gamma\vdash x:A}~(x\!:\! A~in~\Gamma)
$$
The judgment $\Gamma\vdash A$ expresses that $A$ is a type in context $\Gamma$.
We may write it $A~\type~(\Gamma)$ and may omit the global context $\Gamma$.
For instance, the rules
$$
%\frac{\Gamma,x:A\vdash B}{\Gamma\vdash \Pi (x:A) B}~~~~~~~~~
\frac{\Gamma\vdash A~~~~~~\Gamma,x:A\vdash B}{\Gamma\vdash \mypi{x}{A}{B}}~~~~~~~~~
\frac{\Gamma,x:A\vdash b:B}{\Gamma\vdash \mylam{x}{A}{b}:\mypi{x}{A}{B}}~~~~~~~~
\frac{\Gamma\vdash c:\mypi{x}{A}{B}~~~~~~\Gamma\vdash a:A}
     {\Gamma\vdash c~a:B(a/x)}
$$
can be written
$$
%\frac{\Gamma,x:A\vdash B}{\Gamma\vdash \Pi (x:A) B}~~~~~~~~~
\frac{A~\type~~~~~~B~\type~(x:A)}{\mypi{x}{A}{B}~\type}~~~~~~~~~
\frac{b:B~(x:A)}{\mylam{x}{A}{b}:\mypi{x}{A}{B}}~~~~~~~~
\frac{c:\mypi{x}{A}{B}~~~~~~a:A}
     {c~a:B(a/x)}
$$

We also have
$$
%\frac{\Gamma,x:A\vdash B}{\Gamma\vdash \mypi{x}{A}{B}}~~~~~~~~~
\frac{A~\type~~~~~~B~\type~(x:A)}{\mysig{x}{A}{B}~\type}~~~~~~~~~
\frac{a:A~~~~~~b:B(a/x)}{(a,b):\mysig{x}{A}{B}}~~~~~~~~
\frac{c:\mysig{x}{A}{B}}{c.1:A}~~~~~~~
\frac{c:\mysig{x}{A}{B}}{c.2:B(c.1/x)}
$$

We write $\conv$ for definitional equality (or conversion).
The main conversion rules are\footnote{We omit systematically {\em congruence} rules
  such as the rule that $c~a~\conv~c'~a'$ follows from $c~\conv~c'$ and $a~\conv~a'$.}
  \footnote{PD: We omit various other rules too.}
$$
\frac{ a:A~~~~~~ A~ \conv~ B}{ a:B}~~~~~~~~~
\frac{ a ~\conv~a':A~~~~~~ A  ~\conv~ B}{ a ~\conv~a':B}
$$
$$
\frac{b:B~(x:A)~~~~~~~~ a:A}{ (\mylam {x}{A}{b})~a  ~\conv~ b(a/x):B(a/x)}
~~~~~~~
\frac{f~x = g~x:B~(x:A)}{ f = g : \mypi{x}{A}{B}}
$$
$$
\frac{ a:A~~~~~~ b:B(a/x)}{ (a,b).1  ~\conv~ a:A}
~~~~~~~
\frac{ a:A~~~~~~ b:B(a/x)}{ (a,b).2  ~\conv~ b:B(a/x)}~~~~~~
\frac{ a.1 = a'.1:A~~~~~~~ a.2 = a'.2:B(t.1/x)}{ a = a' : \mysig{a}{A}{B}}
$$

We can introduce data types: the type of natural numbers $\NN$ has
constructors $\ZERO:\NN$ and $\SUCC:\NN\rightarrow\NN$ and elimination rules
$$
\frac{P~\type~(x:\NN)~~~~a:P(\ZERO/x)~~~~~g:\mypi{x}{\NN}{P\rightarrow P(\SUCC~x/x)}}{\natrec(a,g):\mypi{x}{\NN}{P}}
$$
with conversion rules $\natrec(a,g)~\ZERO = a$ and $\natrec(a,g)~(\SUCC~x) = g~x~(\natrec(a,g)~x)$.
%\footnote{change to version with $\natrec(a,g,n)$?}

We also have a type $\Id~A~a_0~a_1$ for $a_0:A$ and $a_1:A$ with $\refl~a:\Id~A~a~a$
and the elimination rule
$$\frac{a:A~~~~~C~\type~(x:A,p:\Id~A~a~x)~~~~~d:C(a/x,\refl~a/p)}{\JJ (a,d):\mypi{x}{A}{\mypi{p}{\Id~A~a~x}{C}}}$$
with the conversion rule $\JJ(a,d)~a~(\refl~a) = d$.

\section{Rules for an external sequence of universes}


  %PD: Is this set of rules is complete? What about $A=A':\UU_n$ implies $\T_n(A) = T_n(A')$?} set of rules for "universes as full reflections".
%in the style of the rules in Section \ref{sec:palmgren}.
We first present a version without
cumulativity\footnote{We recall that congurence rules are implicit.}.
$$
\UU_n~\type~~~~~~
\frac{A:\UU_{n}}{\T_{n}(A)~\type}
$$
$$
\frac{A:\UU_{n}~~~~~~B:\T_{n}(A)\rightarrow \UU_{m}}
     {\Pi^{n,m} A B:\UU_{n\vee m}}~~~~~~~~~
\frac{A:\UU_{n}~~~~~~B:\T_{n}(A)\rightarrow \UU_{m}}
     {\Sigma^{n,m} A B:\UU_{n\vee m}}~~~~~~~~~
$$
where $n$ and $m$ are external natural numbers, and $n \vee m$ is the maximum of $n$ and $m$. This means for example, that $\UU_n~\type$ is a {\em schema}, yielding one rule for each $n$.
The conversion rules are
$$
\T_{n\vee m}~(\Pi^{n,m} A B) = \mypi{x}{\T_{n}(A)}{ \T_{m}(B~x)}~~~~~~~
\T_{n\vee m}~(\Sigma^{n,m} A B) = \mysig {x}{\T_{n}(A)}{ \T_{m}(B~x)}~~~~~~~
$$
Furthermore we have $$\NN^{0}:\UU_{0}$$
$$\T_{0}(\NN^{0}) = \NN$$
and
$$
\frac{A:\UU_n~~~~~~a_0:\T_n(A)~~~~~~a_1:\T_n(A)}
{\Id^n~A~a_0~a_1:\UU_n}
$$
$$\T_n(\Id^n~A~a_0~a_1) = \Id~\T_n(A)~a_0~a_1$$
and a code for the previous universes, for $n>l$
$${\UU^{n}_{l}}:\UU_{n}$$
$$\T_{n}({\UU^{n}_{l}}) = \UU_{l}
\footnote{PD: wouldn't it be more natural to let $\UU^n$ be the code for $\UU_{n}$?}
$$

%% Note that $\UU_p, \T_p$ are given by separate inductive-recursive definition for each $p$. There is one introduction rule for each $\Pi^{n,m}$ and $\Sigma^{n,m}$ such that $p = m \vee n$ and introduction rules for $\NN^p, \Id^p$, and also for $\UU^{p-1}$ provided $p \geq 1$.

 \section{A type theory with universe levels and polymorphism }\label{sec:internal}

The problem with this approach is that we have to {\em duplicate} definitions that follow
the same pattern. For instance, we have the identity function $\mylam{X}{\UU_n}{\mylam{x}{\T_n(X)}{x}}$
of type $\mypi{X}{\UU_n}{\T_n(X)\rightarrow \T_n(X)}$ that can be defined for $n = 0,1,\dots$
We address now this issue by introducing {\em universe level}
expressions: we write $\alpha,\beta,\dots$
for {\em level variables}, and $l,m,\dots$ for {\em level expressions} which are built from level variables
by suprema $l \vee m$ and the next level operation $l^+$.
Level expressions form a sup semilattice $l\vee m$
with a next level operation $l^+$ such that $l \vee l^+ = l^+$
and $(l\vee m)^+ = l^+\vee m^+$. (We don't seem to need a $0$ element.)
We write $l\leqslant m$ for $l\vee m = m$ and $l<m$ for $l^+\leqslant m$.

%As expressed in \cite{VV}, this is essentially a tropical (max-plus)
%semiring, except that we don't actually seem to need a $0$ element.

We have a new context extension operation that adds a fresh level variable $\alpha$ to a context $\Gamma$ resulting in the extended context $\Gamma,\alpha~\Level$.
$$
\frac{}{\alpha~\Level~(\Gamma)}(\alpha~in~\Gamma)~~~~~~
\frac{l~\Level~~~~~m~\Level}{l\vee m~\Level}~~~~~~
\frac{l~\Level}{l^+~\Level}~~~~~~
$$
We also have level equality judgments $l = m$ and rules expressing that judgments are invariant under level equality.
For example,
$$
\frac{a(l/\alpha) : A(l/\alpha)\ \ \ l = m}
{a(m/\alpha) : A(m/\alpha)}
$$
We can  now add rules for level-indexed universes:
$$
\frac{l~\Level}{\UU_{l}~\type}~~~~~~
\frac{A:\UU_{l}}{\T_{l}(A)~\type}~~~~~~
$$
The remaining rules are completely analogous to the rules for externally indexed universes with external numbers replaced by internal levels. In this way we avoid having to duplicate definitions for different universe levels.
$$
\frac{A:\UU_{l}~~~~~~B:\T_{l}(A)\rightarrow \UU_{m}}
     {\Pi^{l,m} A B:\UU_{l\vee m}}~~~~~~~~~
\frac{A:\UU_{l}~~~~~~B:\T_{l}(A)\rightarrow \UU_{m}}
     {\Sigma^{l,m} A B:\UU_{l\vee m}}~~~~~~~~~
$$
with conversion rules
$$
\T_{l\vee m}~(\Pi^{l,m} A B) = \mypi{x}{\T_{l}(A)}{ \T_{m}(B~x)}~~~~~~~
\T_{l\vee m}~(\Sigma^{l,m} A B) = \mysig{x}{\T_{l}(A)}{ \T_{m}(B~x)}~~~~~~~
$$
Furthermore we have $$\NN^{l}:\UU_{l}$$
$$\T_{l}(\NN^{l}) = \NN$$
and
$$
\frac{A:\UU_l~~~~~~a_0:\T_l(A)~~~~~~a_1:\T_l(A)}
{\Id^l~A~a_0~a_1:\UU_l}
$$
$$\T_l(\Id^l~A~a_0~a_1) = \Id~\T_l(A)~a_0~a_1$$
and, for $n<l$
$${\UU_n^{l}}:\UU_{l}$$
$$\T_{l}({\UU_n^{l}}) = \UU_{n}$$

%
%\medskip
%
%We also have ${\UU^{l}}:\UU_{l^+}$ with
%$\T_{l^+}({\UU^{l}}) = \UU_{l}$.
%
%\medskip
%
%For data types, we introduce $\NN^{l}:\UU_{l}$ with the conversion rule
%$\T_{l}(\NN^{l}) = \NN$.
%
%We have $\Id^l~A~a_0~a_1:\UU_l$ for $A:\UU_l$ and $a_0:\T_l(A)$ and $a_1:\T_l(A)$
%with $\T_l(\Id^l~A~a_0~a_1) = \Id~\T_l(A)~a_0~a_1$.
%
%\medskip
%

The following rule should be admissible:
we can derive $A = B : \UU_l$ from $\T_l(A) = \T_l(B)$.
 It expresses that $\T_l$ is an embedding from the collection of elements of $\UU_l$
 to the collection of types.
 Voevodsky \cite[Rule 20 on p. 17]{VV} adds this as an internal rule, but then the system is not
 generalized algebraic anymore, since the embedding rule gives rise to a conditional equation.

 We expect that normalisation to hold for this system.
 This implies decidable type-checking. This also implies that if $a : \NN$ in a context with only level variables
then $a$ is convertible to a numeral.

%% As we mentioned in the introduction, we have now got a system in which we can express univalence of all universes as one typing
%% $$
%% \ua : (l : \Level) \to \UA_l
%% $$
%% since the type $(l : \Level) \to \UA_l$ is well-formed in the present system.
% and the
% fact that if all universes are univalent then they all satisfy function extensionality.



\paragraph{Interpreting the level-indexed system in the system with externally indexed universes.}

A judgment in the level-indexed system can be interpreted in the externally indexed system relative to an assignment $\rho$ of external natural numbers to level variables. We simply replace each level expression in the judgment by the corresponding natural number obtained by letting $l^+\,\rho = l\,\rho+1$ and $(l \vee m)\,\rho = \max(l\,\rho,m\,\rho)$.

\subsection*{Rules for level-indexed products}

In Agda $\Level$ is a type,
and it is thus possible to form level-indexed products of types.
In our system $\Level$ is not a type, but it is essential to be able to state operations polymorphic
in the levels. In order to do this, we extend the system with the following rules:
$$
\frac{A~\type~(\alpha~\Level)}{[\alpha]A~\type}~~~~~~~
\frac{t:[\alpha]A~~~~~l~\Level}
     {t~l:A(l/\alpha)}~~~~~~~~~
\frac{u:A~(\alpha~\Level)}{\lam{\alpha}{u}: [\alpha]A}~~~~~
\frac{t~\alpha = u~\alpha:A~(\alpha~\Level)}{t = u:[\alpha]A}
$$
with conversion rule $(\lam{\alpha}{u})~l = u(l/\alpha)$.

An example that uses level-indexed products is the following type which  expresses the theorem that univalence for universes of arbitrary level implies function extensionality for functions between universes of arbitrary levels.
$$
([\alpha]\mathsf{IsUnivalent}\, \UU_\alpha)
\to [\beta][\gamma] \mathsf{FunExt}\, \UU_\beta\, \UU_\gamma
$$

We conjecture that all functions in $[\alpha]\NN$ in our system
are constant.

%\paragraph{Interpreting the system with levels into extensional type theory with a super universe.} We can interpret the theory with rules for level-indexed products in extensional type theory with a (minimal) super universe, provided we add the rules for cumulativity. Levels will be interpreted by internal natural numbers $n : \NN$, and level-indexed products $[\alpha]A$ will be interpreted by ordinary indexed products $\Pi_{n : \NN}A$. Level application and level abstraction will be interpreted as ordinary application and abstraction.

%In intensional type theory we will not be able to justify all the laws for level expressions as definitional equalities. For example, we do not have $\max(m,n)=\max(n,m)$ definitionally.

\section{A system with level constraints}\label{sec:constraints}

To motivate why it may be useful to introduce the notion of judgment relative to a list of constraints on universe levels, consider the following type in a system without cumulativity\footnote{We use Russell's style notation}:
%[[The type constructor ``$\simeq$'' should be replaced by the identity type for this to work. I am not sure which notation for the identity type we want to use to formulate this example.]]
%$$
%    (A : \UU_l) \to (B : \UU_m) \to A \times B \simeq B \times A
%$$
%Here we get the constraint is that $l \vee m = m \vee l$, which holds for arbitrary levels $l, m$.
%
%Now  consider the theorem
%% $$
%%     \Pi_{A:\UU_l~{B}:{\UU_m}~{C}:{\UU_n}}
%%     {~~\Id\,\UU_{l \vee m}\, (A\times^{l \vee m} B)\,(C \times^{n \vee l} A)
%%     \to \Id\,\UU_{m \vee l} \, (B\times^{m \vee l} A)\,(C \times^{n \vee l} A)}
%% %    \to A \times B =_\delta C \times A \to B \times A =_\epsilon C \times A
%% $$
$$
    \Pi_{A:\UU_l~{B}:{\UU_m}~{C}:{\UU_n}}
    {~~\Id\,\UU_{l \vee m}\, (A\times B)\,(C \times A)
    \to \Id\,\UU_{m \vee l} \, (B\times A)\,(C \times A)}
%    \to A \times B =_\delta C \times A \to B \times A =_\epsilon C \times A
$$
It is well-formed provided $l \vee m = n \vee l$. There are three maximally general solutions:
\begin{eqnarray*}
&&l = \alpha, m = \beta, n = \alpha \vee \beta\\
&&l = \alpha, m = \gamma \vee \alpha, n = \gamma\\
&&l = \beta \vee \gamma, m = \beta, n = \gamma
\end{eqnarray*}
where $\alpha, \beta,$ and $\gamma$ are level
variables\footnote{Note the similarity with the Gustave function in stable domain theory.}.
%
%$$
%    (A : \UU_\alpha) \to (B : \UU_\beta) \to (C : \UU_{\alpha \vee \beta})
%    \to \Id\,\UU_{\alpha \vee \beta}\, (A\times^{\alpha \vee \beta} B)\,(C \times^{\alpha \vee \beta} A)
%    \to \Id\,\UU_{\alpha \vee \beta}\, (B\times^{\alpha \vee \beta} A)\,(C \times^{\alpha \vee \beta} A)
%$$
%$$
%    (A : \UU_\alpha) \to (B : \UU_{\gamma \vee \alpha}) \to (C : \UU_{\gamma})
%    \to \Id\,\UU_{\gamma \vee \alpha}\, (A\times^{\gamma \vee \alpha} B)\,(C \times^{\gamma \vee \alpha} A)
%    \to \Id\,\UU_{\gamma \vee \alpha}\, (B\times^{\gamma \vee \alpha} A)\,(C \times^{\gamma \vee \alpha} A)
%$$
%$$
%    (A : \UU_{\beta \vee \gamma}) \to (B : \UU_{\beta}) \to (C : \UU_{\gamma})
%    \to \Id\,\UU_{\beta \vee \gamma}\, (A\times^{\beta \vee \gamma} B)\,(C \times^{\beta \vee \gamma} A)
%    \to \Id\,\UU_{\beta \vee \gamma}\, (B\times^{\beta \vee \gamma} A)\,(C \times^{\beta \vee \gamma} A)
%$$

In a system with level constraints, we could instead derive the type
$$
    \Pi_{A:\UU_\alpha~{B}:{\UU_\beta}~{C}:{\UU_\gamma}}
    {~~\Id\,\UU_{\alpha \vee \beta}\, (A\times B)\,(C \times A)
    \to \Id\,\UU_{\alpha \vee \gamma}\, (B\times A)\,(C \times A)}
$$
which is valid under the constraint
$
\alpha \vee \beta = \alpha \vee \gamma.
$

\subsection{Rules for level constraints}%\label{ssec:VVsystem}

{\color{red}
  Remark about $Set_\UU$.}

A constraint is an equation $l = m$, where $l$ and $m$ are level expression.
A list of constraints is {\em consistent} if it does not create a {\em loop}, i.e.
a level expression $n$ such  that $n<n$ module this list of constraints.
The main result of \cite{bezem-coquand:lattices} is that it is decidable if a list
of constraint is consistent; furthermore, we can also decide if a given constraint
is a consequence.

Voevodsky \cite{VV} suggested to {\em introduce} universe levels with constraint. This corresponds
to mathematical practice: for instance, at the beginning of the book \cite{giraud:cohom-non-abel},
the author introduces two universes $U$ and $V$ with the constraint that $U$ is a member of $V$.
In our setting this will correspond to introducing two levels $\alpha$ and $\beta$ with the
constraint $\alpha<\beta$. To be explicit about universes can be important, as shown by the
example in \cite{chambert-loir:universes-matter,waterhouse:sheaves}.

Note that $\alpha<\beta$ holds if, and only if, $\beta$ is of the form $\gamma\vee\alpha^+$. We can thus
for this example avoid to have to declare a constraint by writing systematically $\gamma\vee\alpha^+$ for $\beta$.
This is what is done currently in the system Agda. However, this is a rather indirect way to express what is
going on. Furthermore, the example at the beginning of this section shows that this can lead to an artificial duplication
of definitions.

Voevodsky however does not describe a mechanism to {\em eliminate}
such levels, and this is what we present next.

\medskip

%Thus we have
%$$
%\frac{l_1~m_1~\dots~l_k~m_k~\Level}{l_1 = m_1,\dots,l_k = m_k~\Constraint}
%$$
%We write $l\leqslant m$ for the constraint $l\vee m = m$.

We  have a new context extension $\Gamma,\psi$ if $\psi$ is a {\em consistent}
finite set of constraints in $\Gamma$.

New judgment form $\psi~(\Gamma)$ which expresses that the constraint set $\psi$
holds in $\Gamma$. 
As shown in \cite{bezem-coquand:lattices}, this is a decidable judgement.
For instance we have $\alpha^+\leqslant\beta^+~(\alpha\leqslant\beta)$.

We recall that we have the following rule
$$
\frac{l~m~\Level~~~~~~l<m}{\UU_l:\UU_m}
$$
but now $l<m$ may hold since it is implied by the constraints in the context $\Gamma$.

%% In the system with constraints, we have judgments such as
%% $$
%% \UU_{\alpha}:\UU_{\beta}~(\alpha~\beta~\Level,\alpha<\beta)
%% $$
We introduce a ``restriction'' operation
$$
\frac{A~\type~(\psi)}{[\psi]A~\type}~~~~~~~~~~~~~~~~~\frac{t:A~(\psi)}{t:[\psi]A}
$$
with the rule $[\psi]A = A$ if $\psi = 1$.

\medskip

Here is a simple example of the use of this system. In order to represent set theory in type
theory, we can use a type $V$ satisfying the following equality $\Id~{\UU_{\beta}}~V~(\Sigma_{X:\UU_{\alpha}}X\rightarrow V)$.
This equation makes only sense modulo the constraint $\alpha<\beta$.

We can define in our system a constant
$$
c~=~\lam{\alpha~\beta}\lambda_{Y:\UU_{\beta}}\Id~{\UU_{\beta}}~Y~ (\Sigma_{X:\UU_{\alpha}}X\rightarrow Y)~~:~~
   [\alpha~\beta][\alpha<\beta]\UU_{\beta} \rightarrow \UU_{\beta^+}
$$

   This is because $\Sigma_{X:\UU_{\alpha}}X\rightarrow Y$ has type $\UU_{\beta}$ in the context

   $$\alpha:\Level,~\beta:\Level,~\alpha<\beta,~Y:\UU_{\beta}$$

   We can further instantiate this constant $c$ on two levels $l$ and $m$, and this will be of type
   $$[l<m]\UU_{m} \rightarrow \UU_{m^+}$$
   and this only can be used further if $l<m$ holds in the current context.

   \medskip


In the current system of Agda, the constraint $\alpha<\beta$ is represented indirectly by
writing $\beta$ on the form $\gamma\vee \alpha^+$ and $c$ is defined as
$$
c~=~\lam{\alpha~\gamma}\lambda_{Y:\UU_{\alpha^+\vee\gamma}}\Id~{\UU_{\alpha^+\vee\gamma}}~Y~ (\Sigma_{X:\UU_{\alpha}}X\rightarrow Y)~~:~~
   [\alpha~\gamma]\UU_{\alpha^+\vee\gamma} \rightarrow \UU_{\alpha^{++}\vee\gamma^+}
$$
   which arguably is less readable.
   
\medskip

In general, if we  build a term $t$ of type $A$ in a context using labels $\alpha_1,\dots,\alpha_m$
and constraint $\psi$ and variables $x_1:A_1,\dots,x_n:A_n$ we can introduce a constant
$$
c~=~ \lam{\alpha_1~\dots~\alpha_m}\lambda_{x_1~\dots~x_n}t ~:~
[\alpha_1~\dots~\alpha_m][\psi]\Pi_{x_1:A_1~\dots~x_n:A_n}A
$$
We can then instantiate this constant $c~l_1~\dots~l_m~u_1~\dots~u_n$, but only if the levels
$l_1~\dots~l_m$ satisfy the constraint $\psi$.

    We have thus complemented Voevodsky's proposal \cite{VV}: we can not only declare and define
    constants assuming universe levels satisfying some constraints, but we also provide a mechanism
    to {\em instantiate} these constants.

    

%% Here is an example showing how such a system can be used. With univalence, we can build a term $t$
%% $$
%% t~:~    \Pi_{A:\UU_\alpha~{B}:{\UU_\beta}~{C}:{\UU_\gamma}}[\alpha \vee \beta = \alpha \vee \gamma]
%%     {~~\Id\,\UU_{\alpha \vee \beta}\, (A\times B)\,(C \times A)
%%     \to \Id\,\UU_{\alpha \vee \gamma}\, (B\times A)\,(C \times A)}
%% $$

%%     We can then define $u = \lam{\alpha~\beta~\gamma}t$ which is of type
%%     $$[\alpha~\beta~\gamma][\psi]\Pi_{A:\UU_\alpha~{B}:{\UU_\beta}~{C}:{\UU_\gamma}}
%%     {~~\Id\,\UU_{\alpha \vee \beta}\, (A\times B)\,(C \times A)
%%     \to \Id\,\UU_{\alpha \vee \gamma}\, (B\times A)\,(C \times A)}
%% $$
%%     where $\psi$ is $\alpha \vee \beta = \alpha \vee \gamma$.

%%     It is now possible to instantiate this general operation to levels $u~l~m~n$. This has type
%%     $$[\psi(l,m,n)]\Pi_{A:\UU_l~{B}:{\UU_m}~{C}:{\UU_n}}
%%     {~~\Id\,\UU_{l \vee m}\, (A\times B)\,(C \times A)
%%     \to \Id\,\UU_{l \vee n}\, (B\times A)\,(C \times A)}
%% $$
%%     If $\psi(l,m,n)$ holds, i.e. $l,m,n$ satisfy the constraint $\psi$, we can instantiate
%%     further $u~l~m~n$ to $A:\UU_l,B:\UU_m,C:\UU_n$
%%     $$u~l~m~n~A~B~C~:~    {~~\Id\,\UU_{l \vee m}\, (A\times B)\,(C \times A)
%%     \to \Id\,\UU_{l \vee n}\, (B\times A)\,(C \times A)}$$

%% %Intuitively, we can use a term of type $[\psi]A$ only if all constraints in $\psi$ hold.

%% %% {\color{red}
%% %% TODO: state that decidability of type-checking requires (1) decidability of loop checking
%% %% (since $\UU:\UU$ makes type checking undecidable) and (2) decidablity of whether $\psi$
%% %% follows from the constraints in $\Gamma$.}

\begin{TODO}
  \begin{enumerate}
  \item Complete the rules.
  \item Add a remark saying that Voevodsky had all the rules for declaring level universes and
    constraints but not the above rules for instantiating operations defined using these levels
    supposed to satisfy these constraints.
  \item PD: We can postpone this question: TODO   can this be presented as GAT???
   \end{enumerate}
 \end{TODO}

\subsection{A decision problem for sup-semilattices with successor}

The type system in the previous subsection gives rise to the following
decision problem: given a context $\Gamma$ containing constraints $\phi_1,\ldots,\phi_n$,
can one infer the constraint $\psi$?

Voevodsky considered a similar decision problem in \cite[Section 2]{VV}.
His universe levels are natural numbers, with 0, successor, and maximum.
His constraints are equations between open terms in this vocabulary,
and for decidability he refers to Presburger Arithmetic,
in which these constraints can easily be expressed.

Although this is correct, we want to make a couple of remarks on Voevodsky's approach.
The use of Presburger Arithmetic seems a bit of an overkill here.
The only operations on universe levels that are directly motivated
from type theory are a join operation and a compatible successor function.
There is no need to assume a level 0, a well-founded total ordering,
nor does one need that the successor function is injective.
Since type theory is meant as a foundation of mathematics, it is,
at least from a philosophical point of view,
not a good idea to base the concept of a universe level
on more mathematical notions than stricly necessary.

We propose to apply Occam's Razor here. This means that universe levels
are elements of an abstract sup-semilattice with a successor function
that is compatible with the join operation. The logic is ordinary
equational logic.
This approach allows for possible other interpretations of universe levels
than natural numbers. At the same time the particular decision problem
called the \emph{uniform word problem} has to be solved for this equational theory.
This has been done by Bezem and Coquand in \cite{bezem-coquand:lattices},
proving the following theorem.

\begin{theorem}\label{thm:P-solvability}
For sup-semilattices with a successor, both the uniform word problem
and the problem of loop detection are decidable.
\end{theorem}
This theorem ensures that type checking remains decidable when
typings depend on universe level constraints, provided we have the expected property of normalisation.


\section{Related work}

(** Comments about groups **)
(** comments about Id depending on the universe **)

Huet \cite{Huet87} develops another algorithm for universe polymorphism
in the Calculus of Constructions extended with a predicative system of universes.
He drops the assumption that the universes form a linearly-ordered cumulative hierarchy indexed by the natural numbers.
%, and on this basis develops a more efficient consistency checking algorithm than Chan's.

Inspired by Huet's work, Harper and Pollack
\cite{HarperP91} consider consider both an externally indexed version and a version
with level expressions which are either numbers or variables, and with constraints of the form $l < m$ and $l \leq m$. They use an algorithm due to Chan
for solving those constraints in $\Ordo(c,v^3)$, where $c$ is the number of constraints and $v$ is the number of level variables. They also consider the level synthesis problem in a version with an anonymous universe Type.

This "implicit"
approach to universe polymorphisms however is problematic w.r.t. modularity, as pointed
out in \cite{Courant02,Simpson04}:
one can prove $A\rightarrow B$ in one file, and $B\rightarrow C$ in one other file, while
$A\rightarrow C$ is not valid. In order to cope with this issue, J. Courant \cite{Courant02}
suggested to have explicit level universes, with a sup operation (see also \cite{herbelin05}).
This approach is now followed in Agda and in Voevodsky's proposal \cite{VV}. Our system complements
\cite{Courant02,VV} by handling constraints on universe levels and having instantiation operations.
%The sup operation on universe levels, also used in Agda and in Voevodsky's
%proposal \cite{VV}, seems to appear first in the unpublished note \cite{herbelin05}.

%\begin{quotation}
%(cut-and-paste from Harper and Pollack)
%Huet, in an unpublished manuscript \cite{Huet87}, has independently developed an
%algorithm for handling universes in the Calculus of Constructions. His approach is to drop the assumption that the universes form a linearly-ordered
%cumulative hierarchy indexed by the natural numbers, and to consider instead a family of calculi in which there is some well-founded partial ordering
%of universes. The input language is correspondingly restricted so that specific universes are disallowed; only the anonymous universe Type may be
%used. The principal advantage of this approach over the one considered here
%is that the consistency checking algorithm is significantly more efficient than
%Chan's algorithm, reducing to an acyclicity check in a dependency graph of
%universe levels.
%\end{quotation}

Assaf \cite{Assaf14} considers an alternative version of the calculus of
constructions where subtyping is explicit. This new system avoids problems related to coercions and dependent types by using the Tarski style
of universes and by introducing additional equations to reflect equality. In particular he adds an explicit cumulativity map $\T^0_1 : \UU_0 \to \UU_1$. He argues that "full reflection" is necessary to achieve the expressivity of Russell style. He introduces the explicit cumulative calculus of constructions (CC$\uparrow$) which is closely related to our system of externally indexed Tarski style universes.
This is analysed further in the PhD thesis of F. Thir\'e \cite{Thire20}.

\paragraph{Acknowledgement.}
The authors acknowledge the support of the Centre for Advanced Study (CAS)
at the Norwegian Academy of Science and Letters
in Oslo, Norway, which funded and hosted the research project Homotopy
Type Theory and Univalent Foundations during the academic year 2018/19.
\bibliographystyle{plain}
\bibliography{refs}

\newpage

\section*{Appendix 1: formulation with cumulativity}

%% \subsection*{Rules for cumulativity}

%% With cumulativity, we introduce an operation
%% $$
%% \frac{A:\UU_{n}}
%% {\T_{n}^{m}(A):\UU_{m}}
%% n\leqslant m
%% $$
%% We can also formulate the side condition as an equation $m = n \vee m$.
%% We require for all $n,m$
%% \[
%% \T_m(\T_{n}^{m}(A)) = \T_{n}(A) \quad\text{and}\quad \T_{n}^{m}(\NN^{n}) = \NN^{m}.
%% \footnote{PD: should we not justify the fact that we can have those two rules together? Especially since Erik Palmgren in his work on universe operators has two versions, where the first rule is only admitted in the "uniform constructions" version and the second one in the "full reflection" version. The split arises because in the "uniform constructions" version the cumulativity maps are constructors, whereas in the "full reflection" case they are implicit operations computed by case analysis on the codes. My preferred way to justify both is to take the "full reflections" approach, but note that the rule $\T_m(\T_{n}^{m}(A)) = \T_{n}(A)$ is still valid in the "standard model" (the meaning explanations). Here this rule comes out as an extensional equality, but the rule $\T_{n}^{m}(\NN^{n}) = \NN^{m}$ comes out as a definitional equality. Interestingly, it seems that the resulting system is still normalizing and has decidable judgments. }
%% \]
%% We add for all $n,m,p$ with $m\leqslant n\leqslant p$
%% $$
%% \T_{n}^n(A) = A \quad\text{and}\quad \T_{n}^p\T_{m}^n = \T_m^p.
%% $$
%% We can then simplify the product and sum rules to
%% $$
%% \frac{A:\UU_{n}~~~~~~B:\T_{n}(A)\rightarrow \UU_{n}}
%%      {\Pi^{n} A B:\UU_{n}}~~~~~~~~~
%% \frac{A:\UU_{n}~~~~~~B:\T_{n}(A)\rightarrow \UU_{n}}
%%      {\Sigma^{n} A B:\UU_{n}}~~~~~~~~~
%% $$
%% with conversion rules
%% $$
%% \T_{n}~(\Pi^{n} A B) = \mypi{x}{\T_{n}(A)}{ \T_{n}(B~x)}~~~~~~~
%% \T_{n}~(\Sigma^{n} A B) = \mysig{x}{\T_{n}(A)}{ \T_{n}(B~x)}~~~~~~~
%% $$
%% and
%% $$
%% \T_{n}^{m}~(\Pi^{n} A B) = \Pi^{m} \T_{n}^{m}(A) (\mylam{x}{\T_{n}(A)}{\T_{n}^{m}(B~x)})~~~~~~
%% \T_{n}^{m}~(\Sigma^{n} A B) = \Sigma^{m} \T_{n}^{m}(A) (\mylam{x}{\T_{n}(A)}{\T_{n}^{m}(B~x)})~~~~~~
%% $$




%% These rules are also analogous to the rules for externally indexed universes.
%% For the sake of completeness we give them explicitly.

%% (** Here we do not need the $\vee$-operation **)
%% \footnote{PD: But we use $\vee$ in the rules below.}

We introduce an operation $\T_{l}^{m}(A):\UU_{m}$ if $A:\UU_{l}$
and $l\leqslant m$ (i.e., $m = l\vee m$).

We require $\T_{m}(\T_{l}^{m}(A)) = \T_{l}(A)$
and $\T_{l}^{m}(\NN^{l}) = \NN^{m}$.

We add $\T_{l}^l(A) = A$\footnote{Recall that the equality of universe levels is the one of
  sup semilattice with the next operation.}
and $\T_{m}^n\T_{l}^m = \T_l^n$ if $l\leqslant m\leqslant n$.
%Note that this cannot be simplified to $\T_{l}^l(A) = A$
%like $\T_{n}^n(A) = A$ in \cref{sec:internal}.
%The simplified equation would not give, e.g.,
%$\T_{\alpha\vee\beta}^{\beta\vee\alpha}(A) = A$.

We can then simplify the product and sum rules to
$$
\frac{A:\UU_{l}~~~~~~B:\T_{l}(A)\rightarrow \UU_{l}}
     {\Pi^{l} A B:\UU_{l}}~~~~~~~~~
\frac{A:\UU_{l}~~~~~~B:\T_{l}(A)\rightarrow \UU_{l}}
     {\Sigma^{l} A B:\UU_{l}}~~~~~~~~~
$$
with conversion rules
$$
\T_{l}~(\Pi^{l} A B) = \mypi{x}{\T_{l}(A)}{ \T_{l}(B~x)}~~~~~~~
\T_{l}~(\Sigma^{l} A B) =  \mysig{x}{\T_{l}(A)}{ \T_{l}(B~x)}~~~~~~~
$$
and
$$
\T_{l}^{m}~(\Pi^{l} A B) = \Pi^{m} \T_{l}^{m}(A) (\mylam {x}{\T_{l}(A)}{\T_{l}^{m}(B~x)})~~~~~~
\T_{l}^{m}~(\Sigma^{l} A B) = \Sigma^{m} \T_{l}^{m}(A) (\mylam {x}{\T_{l}(A)}{\T_{l}^{m}(B~x)})~~~~~~
$$

\section*{Appendix 2: formulation \`a la Russell}

\paragraph{Generalised algebraic presentation and Russell formulation}

It is possible to formulate these rules in the form of a generalised algebraic theory, which provides
a natural notion of {\em models} of type theory.

\begin{remark} \label{app:annotation}
As explained
in \cite{streicher:semtt}, in order to see the present rules as presenting the {\em initial} model,
it is enough to use a variation where application $c~a:B(a/x)$ for $c:\mypi{x}{A}B$ and $a:A$
is annoted by the type family $A,B$ (and similarly for the pairing operation).
\end{remark}

\medskip

There is another way to present this initial model, which is by using
a {\em Russell} formulation (using the terminology introduced in \cite{martinlof:padova}) of
universes). One expects all these formulations to be equivalent; for preliminary results in this
direction see \cite{Assaf14,Thire20}.

With this formulation, the version without cumulativity becomes

$$
\frac{A:\UU_{n}}{A~\type}
$$
$$
\frac{A:\UU_{n}~~~~~~B:\UU_m(x:A)}
     {\mypi{x}{A}{B}:\UU_{n\vee m}}~~~~~~~~~
\frac{A:\UU_{n}~~~~~~B:\UU_m(x:A)}
     {\mysig{x}{A}{B}:\UU_{n\vee m}}~~~~~~~~~
$$
$$\frac{l~\Level}{\NN:\UU_{l}}$$
$$
\frac{A:\UU_n~~~~~~a_0:A~~~~~~a_1:A}
{\Id~A~a_0~a_1:\UU_n}
$$
$$
\frac{l<n}{{\UU_l}:\UU_{n}}
$$

\medskip

For the version with cumulativity, we add the rules
$$
\frac{l\leqslant n~~~~~~~A:\UU_{l}}{A:\UU_{n}}
$$
and the rules for products and sums can be simplified to
$$
\frac{A:\UU_{n}~~~~~~B:\UU_n(x:A)}
     {\mypi{x}{A}{B}:\UU_{n}}~~~~~~~~~
\frac{A:\UU_{n}~~~~~~B:\UU_n(x:A)}
     {\mysig{x}{A}{B}:\UU_{n}}~~~~~~~~~
$$

\begin{remark}
  In this version \`a la Tarski, with or without cumulativity, terms have unique types, in the sense that if $t : A$ and $t : B$ then $A = B$, by induction on $t$. But for this to be valid, we need to annotate application as discussed in Remark~\ref{app:annotation}.
  Even with annotated application, the following property is not elementary: if $\UU_n$ and $\UU_m$ are convertible then $n$ is equal to $m$. This kind of property is needed for showing the equivalence between the Tarski and the Russell formulation.
\end{remark}

\begin{remark} \label{uniqueness:without:cumulativity}
  In a system without cumulativity, if we restrict $\NN$ to be of type $\UU_0$, and $\UU_n$ to be of type $\UU_{n+1}$ then well formed terms have unique types.
\end{remark}

\begin{remark}
  It should be the case that that the above formulation \`a la Russell presents the initial CwF with extra extructure for the standard type formers and a hierarchy of universes, but the proof doesn't seem to be trivial, due to Remark~\ref{uniqueness:without:cumulativity}.
\end{remark}




\end{document}
