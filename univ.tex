%\documentclass[12pt,a4paper]{amsart}
\documentclass[11pt,a4paper]{article}
%\ifx\pdfpageheight\undefined\PassOptionsToPackage{dvips}{graphicx}\else%
%\PassOptionsToPackage{pdftex}{graphicx}
\PassOptionsToPackage{pdftex}{color}
%\fi

%\usepackage{diagrams}

%\usepackage[all]{xy}
\usepackage{url}
\usepackage[utf8]{inputenc}
\usepackage{verbatim}
\usepackage{latexsym}
\usepackage{amssymb,amstext,amsmath,amsthm}
\usepackage{epsf}
\usepackage{epsfig}
% \usepackage{isolatin1}
\usepackage{a4wide}
\usepackage{verbatim}
\usepackage{proof}
\usepackage{latexsym}
%\usepackage{mytheorems}
\newtheorem{theorem}{Theorem}[section]
\newtheorem{corollary}{Corollary}[theorem]
\newtheorem{lemma}{Lemma}[theorem]
\newtheorem{proposition}{Proposition}[theorem]

\usepackage{float}
\floatstyle{boxed}
\restylefloat{figure}


%%%%%%%%%copied from SymmetryBook by Marc

% hyperref should be the package loaded last
\usepackage[backref=page,
            colorlinks,
            citecolor=linkcolor,
            linkcolor=linkcolor,
            urlcolor=linkcolor,
            unicode,
            pdfauthor={CAS},
            pdftitle={Symmetry},
            pdfsubject={Mathematics},
            pdfkeywords={type theory, group theory, univalence axiom}]{hyperref}
% - except for cleveref!
\usepackage[capitalize]{cleveref}
%\usepackage{xifthen}
\usepackage{xcolor}
\definecolor{linkcolor}{rgb}{0,0,0.5}

%%%%%%%%%
\def\oge{\leavevmode\raise
.3ex\hbox{$\scriptscriptstyle\langle\!\langle\,$}}
\def\feg{\leavevmode\raise
.3ex\hbox{$\scriptscriptstyle\,\rangle\!\rangle$}}

%%%%%%%%%
\newcommand\myfrac[2]{
 \begin{array}{c}
 #1 \\
 \hline \hline
 #2
\end{array}}


\newcommand*{\Scale}[2][4]{\scalebox{#1}{$#2$}}%
\newcommand*{\Resize}[2]{\resizebox{#1}{!}{$#2$}}

\newcommand{\II}{\mathbb{I}}
\newcommand{\refl}{\mathsf{refl}}
\newcommand{\mkbox}[1]{\ensuremath{#1}}


\newcommand{\Id}{\mathsf{Id}}
\newcommand{\conv}{=}
%\newcommand{\conv}{\mathsf{conv}}
\newcommand{\lam}[2]{{\langle}#1{\rangle}#2}
\def\NN{\mathsf{N}}
\def\UU{\mathsf{U}}
\def\JJ{\mathsf{J}}
\def\Level{\mathsf{Level}}
%\def\Type{\hbox{\sf Type}}
\def\ZERO{\mathsf{0}}
\def\SUCC{\mathsf{S}}

\newcommand{\type}{\mathsf{type}}
\newcommand{\mypi}[3]{\Pi_{#1:#2}#3}
\newcommand{\mylam}[3]{\lambda_{#1:#2}#3}
\newcommand{\mysig}[3]{\Sigma_{#1:#2}#3}
\newcommand{\N}{\mathsf{N}}
\newcommand{\Set}{\mathsf{Set}}
\newcommand{\El}{\mathsf{El}}
%\newcommand{\U}{\mathsf{U}} clashes with def's in new packages
\newcommand{\T}{\mathsf{T}}
\newcommand{\Usuper}{\UU_{\mathrm{super}}}
\newcommand{\Tsuper}{\T_{\mathrm{super}}}
%\newcommand{\conv}{\mathrm{conv}}
\newcommand{\idtoeq}{\mathsf{idtoeq}}
\newcommand{\isEquiv}{\mathsf{isEquiv}}
\newcommand{\ua}{\mathsf{ua}}
\newcommand{\UA}{\mathsf{UA}}
%\newcommand{\Level}{\mathrm{Level}}
\def\Constraint{\mathsf{Constraint}}
\def\Ordo{\mathcal{O}}

\def\Ctx{\mathrm{Ctx}}
\def\Ty{\mathrm{Ty}}
\def\Tm{\mathrm{Tm}}

\def\CComega{\mathrm{CC}^\omega}
\setlength{\oddsidemargin}{0in} % so, left margin is 1in
\setlength{\textwidth}{6.27in} % so, right margin is 1in
\setlength{\topmargin}{0in} % so, top margin is 1in
\setlength{\headheight}{0in}
\setlength{\headsep}{0in}
\setlength{\textheight}{9.19in} % so, foot margin is 1.5in
\setlength{\footskip}{.8in}

% Definition of \placetitle
% Want to do an alternative which takes arguments
% for the names, authors etc.

\newcommand{\natrec}{\mathsf{natrec}}
%\rightfooter{}
\newcommand{\set}[1]{\{#1\}}
\newcommand{\sct}[1]{[\![#1]\!]}



\begin{document}

\title{A Note on Universe Levels}

\author{Marc Bezem, Thierry Coquand, Peter Dybjer, Mart\'in Escard\'o}
\date{}
\maketitle

\begin{abstract}
  We present three versions of dependent type theory with a countable sequence of universes.
  The first version has an externally indexed sequence of universes \`a la Tarski and we present the rules with and without cumulativity.
  The second is a version where there is an internal notion of universe level. We add a new judgment $l\ \Level$ meaning that $l$ is a universe level, and $l = m$ meaning that $l$ and $m$ are equal universe levels, where level expressions are built up from level variables $\alpha$ by a successor operation $l^+$ and suprema $l \vee m$. In this way judgments can be indexed by level variables, and we obtain universe polymorphism similar to polymorphism in the functional language ML. 
  We then extend this theory with rules for level-indexed product types $[\alpha]A$ meaning ``$A$ is a type for all universe levels $\alpha$".
The final theory extends a proposal by Voevodsky
  for a system with level constraints. We discuss the (solved) decision problems for sup-semilattices with 
successor that come with this approach. 
\end{abstract}


\section{Introduction}\label{sec:intros}

When representing mathematical reasoning using dependent types it is natural to introduce
a sort for types
\footnote{PD: This sentence is not clear: what is a "sort"?}
, itself closed by dependent product, as was done in the system Automath \cite{deBruijn68}.
A little later, and independently, Martin-Löf introduced a dependent type system with
a type of all types \cite{ML71}.
After the discovery of Girard's Paradox \cite{Girard71}, this was replaced by a predicative
version \cite{ML72}, similar to the one used in Automath. Later, 
Martin-L\"of introduced a countable sequence of universes  \cite{martinlof:predicative}
$$
\UU_0 : \UU_1 : \UU_2 : \cdots
$$
We refer to the indices $0, 1, 2, \ldots$ as {\em universe levels}.

Before the advent of univalent foundations, one expected only the first few universe
 levels to be relevant in practical formalisations. This suggests that it might be feasible for a user of type theory to explicitly assign universe levels to their types, and whenever necessary repeat definitions when they are needed on different levels. However, the number of copies of definitions does not only grow with the level, but also with the number of type arguments in the definition of a type former. (We remark that universes are more important in a predicative framework than in an impredicative one. Consider for example the formalisation of real numbers as Dedekind cuts, or domain elements as filters of formal neighbourhoods, which belong to $\UU_1$ since they are properties of elements in $\UU_0$.)

 To deal with this duplication problem Huet \cite{Huet87} and Harper and Pollack \cite{HarperP91} introduced {\em universe polymorphism}. However, the precise formulation and implementation of universe polymorphism has been a subject of much debate, and different proof assistants use different approaches.
 %This is a practical problem since universe level inference can be a costly operation, sometimes more costly than the usual type-checking.

 The need for universe polymorphism has increased with the advent of Voevodsky's univalent foundations, see for example \cite{VV}. One often wants to prove theorems uniformly for an arbitrary universe.
%There is a connection between the dimension of a type and the level of a universe: for example, it is natural to consider the groupoid structure of the first universe, the 2-groupoid structure of the second universe, etc.\footnote{PD: I changed this paragraph. What do you think?}

%The starting point for the present discussions was Escard\'o's \cite{escardo} development of univalent mathematics in the Agda system. His development is {\em universe polymorphic} and makes heavy use of quantification over Agda's type $\Level$ of universe levels. However, the Agda system is not a precisely defined logical system, and we would like to introduce several candidate systems in which Escard\'o's development can be carried out. We will also present a system similar to Voevodsky's universe polymorphic system \cite{VV}. (** MODIFY THIS STAMEMENT AND ALL OTHER REFERENCES TO VV **) In this system there are variables ranging over universe levels and each judgment is made relative to a list of equational constraints between universe level expressions.

The {\em univalence axiom} states that for any two types $X,Y$ the canonical map
$$
\idtoeq_{X,Y} : (X=Y)\to (X\simeq Y)
$$
is an equivalence. Formally, the univalence axiom is an axiom scheme which is added to Martin-Löf type theory. If we work in Martin-Löf type theory with a countable tower of universes, each type is a member of some universe $\UU_n$. Such a universe $\UU_n$ is {\em univalent} provided for all $X,Y : \UU_n$ the canonical map $\idtoeq_{X,Y}$ is an equivalence. Let $\UA_n$ be the type expressing the univalence of $\UU_n$, and
$$
\ua_n : \UA_n
$$
for $n = 0,1,\ldots$, be a sequence of constants expressing the instances of the univalence axiom. We note that $X = Y : \UU_{n+1}$ and $X\simeq Y : \UU_n$ and hence $\UA_n$ is in $\UU_{n+1}$. If we have a type of levels, as in the Agda system, we can express universe polymorphism as quantification over universe levels. Then we can express univalence of all universes as one typing:
$$
\ua : (l : \Level) \to \UA_l
$$
in Agda's notation.

\paragraph{Plan.} We start by giving the rules for a basic version of dependent type theory with
$\Pi, \Sigma, \NN$, and an identity type former $\Id$.
We then explain how to add an externally indexed sequence of universes $\UU_n, \T_n$ \`a la Tarski with or without
cumulativity rules.
In Section \ref{sec:internal} we introduce the notion of universe level, and let judgments depend not only on a context of ordinary variables, but also on a list of level variables $\alpha_1, \ldots, \alpha_k$, giving rise to a theory with level polymorphism. This is a kind of ML-polymorphism since we only quantify over global level variables. We then further extend the theory with level-indexed products of types $[\alpha]A$.

In Section \ref{sec:constraints} we present (a variation of) Voevodsky's proposal with constraints between level variables  \cite{VV}. 

\section{Rules for a basic type theory}

We begin by listing some rules for a basic type theory with $\Pi, \Sigma, \N,$ and $\Id$, such as the rules of context formation and assumption:
$$
\frac{\Gamma\vdash A}{\Gamma,x:A\vdash}~~~~~~\frac{}{()\vdash}~~~~~~~
\frac{\Gamma\vdash}{\Gamma\vdash x:A}~(x\!:\! A~in~\Gamma)
$$
The judgment $\Gamma\vdash A$ expresses that $A$ is a type in context $\Gamma$.
We may write it $A~\type~(\Gamma)$ and may omit the global context $\Gamma$.
For instance, the rules
$$
%\frac{\Gamma,x:A\vdash B}{\Gamma\vdash \Pi (x:A) B}~~~~~~~~~
\frac{\Gamma\vdash A~~~~~~\Gamma,x:A\vdash B}{\Gamma\vdash \mypi{x}{A}{B}}~~~~~~~~~
\frac{\Gamma,x:A\vdash b:B}{\Gamma\vdash \mylam{x}{A}{b}:\mypi{x}{A}{B}}~~~~~~~~
\frac{\Gamma\vdash c:\mypi{x}{A}{B}~~~~~~\Gamma\vdash a:A}
     {\Gamma\vdash c~a:B(a/x)}
$$
can be written
$$
%\frac{\Gamma,x:A\vdash B}{\Gamma\vdash \Pi (x:A) B}~~~~~~~~~
\frac{A~\type~~~~~~B~\type~(x:A)}{\mypi{x}{A}{B}~\type}~~~~~~~~~
\frac{b:B~(x:A)}{\mylam{x}{A}{b}:\mypi{x}{A}{B}}~~~~~~~~
\frac{c:\mypi{x}{A}{B}~~~~~~a:A}
     {c~a:B(a/x)}
$$

We also have
$$
%\frac{\Gamma,x:A\vdash B}{\Gamma\vdash \mypi{x}{A}{B}}~~~~~~~~~
\frac{A~\type~~~~~~B~\type~(x:A)}{\mysig{x}{A}{B}~\type}~~~~~~~~~
\frac{a:A~~~~~~b:B(a/x)}{(a,b):\mysig{x}{A}{B}}~~~~~~~~
\frac{c:\mysig{x}{A}{B}}{c.1:A}~~~~~~~
\frac{c:\mysig{x}{A}{B}}{c.2:B(c.1/x)}
$$

We write $\conv$ for definitional equality (or conversion). The main conversion rules are
$$
\frac{ a:A~~~~~~ A~ \conv~ B}{ a:B}~~~~~~~~~
\frac{ a ~\conv~a':A~~~~~~ A  ~\conv~ B}{ a ~\conv~a':B}
$$
$$
\frac{b:B~(x:A)~~~~~~~~ a:A}{ (\mylam {x}{A}{b})~a  ~\conv~ b(a/x):B(a/x)}
~~~~~~~
\frac{f~x = g~x:B~(x:A)}{ f = g : \mypi{x}{A}{B}}
$$
$$
\frac{ a:A~~~~~~ b:B(a/x)}{ (a,b).1  ~\conv~ a:A}
~~~~~~~
\frac{ a:A~~~~~~ b:B(a/x)}{ (a,b).2  ~\conv~ b:B(a/x)}~~~~~~
\frac{ a.1 = a'.1:A~~~~~~~ a.2 = a'.2:B(t.1/x)}{ a = a' : \Sigma (x:A)B}
$$

We can introduce data types: the type of natural numbers $\NN$ has
constructors $\ZERO:\NN$ and $\SUCC:\NN\rightarrow\NN$ and elimination rules
$$
\frac{P~\type~(x:\NN)~~~~a:P(\ZERO/x)~~~~~g:\mypi{x}{\NN}{P\rightarrow P(\SUCC~x/x)}}{\natrec(a,g):\mypi{x}{\NN}{P}}
$$
with conversion rules $\natrec(a,g)~\ZERO = a$ and $\natrec(a,g)~(\SUCC~x) = g~x~(\natrec(a,g)~x)$.
%\footnote{change to version with $\natrec(a,g,n)$?}

We also have a type $\Id~A~a_0~a_1$ for $a_0:A$ and $a_1:A$ with $\refl~a:\Id~A~a~a$
and the elimination rule
$$\frac{a:A~~~~~C~\type~(x:A,p:\Id~A~a~x)~~~~~d:C(a/x,\refl~a/p)}{\JJ (a,d):\mypi{x}{A}{\mypi{p}{\Id~A~a~x}{C}}}$$
with the conversion rule $\JJ(a,d)~a~(\refl~a) = d$.

\section{Rules for an external sequence of universes}

We will now present a complete\footnote{PD: Is this set of rules is complete? What about $A=A':\UU_n$ implies $\T_n(A) = T_n(A')$?} set of rules for "universes as full reflections". 
%in the style of the rules in Section \ref{sec:palmgren}.
We first present a version without cumulativity.
$$
\UU_n~\type~~~~~~
\frac{A:\UU_{n}}{\T_{n}(A)~\type}
$$
$$
\frac{A:\UU_{n}~~~~~~B:\T_{n}(A)\rightarrow \UU_{m}}
     {\Pi^{n,m} A B:\UU_{n\vee m}}~~~~~~~~~
\frac{A:\UU_{n}~~~~~~B:\T_{n}(A)\rightarrow \UU_{m}}
     {\Sigma^{n,m} A B:\UU_{n\vee m}}~~~~~~~~~
$$
where $n$ and $m$ are external natural numbers, and $n \vee m$ is the maximum of $n$ and $m$. This means for example, that $\UU_n~\type$ is a {\em schema}, yielding one rule for each $n$.
The conversion rules are
$$
\T_{n\vee m}~(\Pi^{n,m} A B) = \Pi {x}{\T_{n}(A)}{ \T_{m}(B~x)}~~~~~~~
\T_{n\vee m}~(\Sigma^{n,m} A B) = \mysig {x}{\T_{n}(A)}{ \T_{m}(B~x)}~~~~~~~
$$
Furthermore we have $$\NN^{n}:\UU_{n}$$
$$\T_{n}(\NN^{n}) = \NN$$
and
$$
\frac{A:\UU_n~~~~~~a_0:\T_n(A)~~~~~~a_1:\T_n(A)}
{\Id^n~A~a_0~a_1:\UU_n}
$$
$$\T_n(\Id^n~A~a_0~a_1) = \Id~\T_n(A)~a_0~a_1$$
and a code for the previous universe
$${\UU^{n}}:\UU_{n + 1}$$
$$\T_{n + 1}({\UU^{n}}) = \UU_{n}$$

%% Note that $\UU_p, \T_p$ are given by separate inductive-recursive definition for each $p$. There is one introduction rule for each $\Pi^{n,m}$ and $\Sigma^{n,m}$ such that $p = m \vee n$ and introduction rules for $\NN^p, \Id^p$, and also for $\UU^{p-1}$ provided $p \geq 1$.

\subsection*{Rules for cumulativity}

With cumulativity, we introduce an operation
$$
\frac{A:\UU_{n}}
{\T_{n}^{m}(A):\UU_{m}}
n\leqslant m
$$
We can also formulate the side condition as an equation $m = n \vee m$.
We require for all $n,m$
\[
\T_m(\T_{n}^{m}(A)) = \T_{n}(A) \quad\text{and}\quad T_{n}^{m}(\NN^{n}) = \NN^{m}.
\]
We add for all $n,m,p$ with $m\leqslant n\leqslant p$
$$
T_{n}^n(A) = A \quad\text{and}\quad T_{n}^p\T_{m}^n = T_m^p.
$$
We can then simplify the product and sum rules to
$$
\frac{A:\UU_{n}~~~~~~B:\T_{n}(A)\rightarrow \UU_{n}}
     {\Pi^{n} A B:\UU_{n}}~~~~~~~~~
\frac{A:\UU_{n}~~~~~~B:\T_{n}(A)\rightarrow \UU_{n}}
     {\Sigma^{n} A B:\UU_{n}}~~~~~~~~~
$$
with conversion rules
$$
\T_{n}~(\Pi^{n} A B) = \mypi{x}{\T_{n}(A)}{ \T_{n}(B~x)}~~~~~~~
\T_{n}~(\Sigma^{n} A B) = \mysig{x}{\T_{n}(A)}{ \T_{n}(B~x)}~~~~~~~
$$
and
$$
\T_{n}^{m}~(\Pi^{n} A B) = \Pi^{m} \T_{n}^{m}(A) (\mylam{x}{\T_{n}(A)}{\T_{n}^{m}(B~x)})~~~~~~
\T_{n}^{m}~(\Sigma^{n} A B) = \Sigma^{m} \T_{n}^{m}(A) (\mylam{x}{\T_{n}(A)}{\T_{n}^{m}(B~x)})~~~~~~
$$


The problem with this approach is that we have to {\em duplicate} definitions that follow
the same pattern. For instance, we have the identity function $\mylam{X}{\UU_n}{\mylam{x}{\T_n(X)}{x}}$
of type $\mypi{X}{\UU_n}{\T_n(X)\rightarrow \T_n(X)}$ that can be defined for $n = 0,1,\dots$
The systems in the next sections address this issue.

\section{A type theory with universe levels and polymorphism }\label{sec:internal}

We introduce {\em universe level} expressions: we write $\alpha,\beta,\dots$
for {\em level variables}, and $l,m,\dots$ for {\em level expressions} which are built from level variables
by suprema $l \vee m$ and the next level operation $l^+$.
Level expressions form a sup semilattice $l\vee m$
with a next level operation $l^+$ such that $l \vee l^+ = l^+$
and $(l\vee m)^+ = l^+\vee m^+$. (We don't seem to need a $0$ element.)
%As expressed in \cite{VV}, this is essentially a tropical (max-plus)
%semiring, except that we don't actually seem to need a $0$ element.

We have a new context extension operation that adds a fresh level variable $\alpha$ to a context $\Gamma$ resulting in the extended context $\Gamma,\alpha~\Level$.
$$
\frac{}{\alpha~\Level~(\Gamma)}(\alpha~in~\Gamma)~~~~~~
\frac{l~\Level~~~~~m~\Level}{l\vee m~\Level}~~~~~~
\frac{l~\Level}{l^+~\Level}~~~~~~
$$
We also have level equality judgments $l = m$ and rules expressing that judgments are invariant under level equality. 
For example,
$$
\frac{a(l/\alpha) : A(l/\alpha)\ \ \ l = m}
{a(m/\alpha) : A(m/\alpha)}
$$
We can  now add rules for level-indexed universes:
$$
\frac{l~\Level}{\UU_{l}~\type}~~~~~~
\frac{A:\UU_{l}}{\T_{l}(A)~\type}~~~~~~
$$
The remaining rules are completely analogous to the rules for externally indexed universes with external numbers replaced by internal levels. In this way we avoid having to duplicate definitions for different universe levels.
$$
\frac{A:\UU_{l}~~~~~~B:\T_{l}(A)\rightarrow \UU_{m}}
     {\Pi^{l,m} A B:\UU_{l\vee m}}~~~~~~~~~
\frac{A:\UU_{l}~~~~~~B:\T_{l}(A)\rightarrow \UU_{m}}
     {\Sigma^{l,m} A B:\UU_{l\vee m}}~~~~~~~~~
$$
with conversion rules
$$
\T_{l\vee m}~(\Pi^{l,m} A B) = \mypi{x}{\T_{l}(A)}{ \T_{m}(B~x)}~~~~~~~
\T_{l\vee m}~(\Sigma^{l,m} A B) = \mysig{x}{\T_{l}(A)}{ \T_{m}(B~x)}~~~~~~~
$$
Furthermore we have $$\NN^{l}:\UU_{l}$$
$$\T_{l}(\NN^{l}) = \NN$$
and
$$
\frac{A:\UU_l~~~~~~a_0:\T_l(A)~~~~~~a_1:\T_l(A)}
{\Id^l~A~a_0~a_1:\UU_l}
$$
$$\T_l(\Id^l~A~a_0~a_1) = \Id~\T_l(A)~a_0~a_1$$
and a code for the previous universe
$${\UU^{l}}:\UU_{l + 1}$$
$$\T_{l + 1}({\UU^{l}}) = \UU_{l}$$

%
%\medskip
%
%We also have ${\UU^{l}}:\UU_{l^+}$ with
%$\T_{l^+}({\UU^{l}}) = \UU_{l}$.
%
%\medskip
%
%For data types, we introduce $\NN^{l}:\UU_{l}$ with the conversion rule
%$\T_{l}(\NN^{l}) = \NN$.
%
%We have $\Id^l~A~a_0~a_1:\UU_l$ for $A:\UU_l$ and $a_0:\T_l(A)$ and $a_1:\T_l(A)$
%with $\T_l(\Id^l~A~a_0~a_1) = \Id~\T_l(A)~a_0~a_1$.
%
%\medskip
%

The following rule should be admissible:
we can derive $A = B : \UU_l$ from $\T_l(A) = \T_l(B)$.
 It expresses that $\T_l$ is an embedding from the collection of elements of $\UU_l$
 to the collection of types.
 Voevodsky \cite[Rule 20 on p. 17]{VV} adds this as an internal rule, but then the system is not
 generalized algebraic anymore, since the embedding rule gives rise to a conditional equation.

We conjecture that if $a : \NN$ in a context with only level variables
then $a$ is convertible to a numeral.

%% As we mentioned in the introduction, we have now got a system in which we can express univalence of all universes as one typing
%% $$
%% \ua : (l : \Level) \to \UA_l
%% $$
%% since the type $(l : \Level) \to \UA_l$ is well-formed in the present system.
% and the
% fact that if all universes are univalent then they all satisfy function extensionality.

\section*{Rules for cumulativity}


(** Comments about groups **)
(** comments about Id depending on the universe **)




These rules are also analogous to the rules for externally indexed universes.
For the sake of completeness we give them explicitly.

(** Here we do not need the $\vee$-operation **)
\footnote{PD: But we use $\vee$ in the rules below.}

We introduce an operation $T_{l}^{m}(A):\UU_{m}$ if $A:\UU_{l}$
and $l\leqslant m$ (i.e., $m = l\vee m$).

We require $T_{m}(T_{l}^{m}(A)) = T_{l}(A)$
and $T_{l}^{m}(\NN^{l}) = \NN^{m}$.

We add $T_{l}^m(A) = A$ if $l = m$
and $T_{m}^nT_{l}^m = T_l^n$ if $l\leqslant m\leqslant n$.
Note that this cannot be simplified to $T_{l}^l(A) = A$
like $T_{n}^n(A) = A$ in \cref{sec:internal}. 
The simplified equation would not give, e.g., 
$T_{\alpha\vee\beta}^{\beta\vee\alpha}(A) = A$.

We can then simplify the product and sum rules to
$$
\frac{A:\UU_{l}~~~~~~B:T_{l}(A)\rightarrow \UU_{l}}
     {\Pi^{l} A B:\UU_{l}}~~~~~~~~~
\frac{A:\UU_{l}~~~~~~B:T_{l}(A)\rightarrow \UU_{l}}
     {\Sigma^{l} A B:\UU_{l}}~~~~~~~~~
$$
with conversion rules
$$
T_{l}~(\Pi^{l} A B) = \mypi{x}{T_{l}(A)}{ T_{l}(B~x)}~~~~~~~
T_{l}~(\Sigma^{l} A B) =  \mysig{x}{T_{l}(A)}{ T_{l}(B~x)}~~~~~~~
$$
and
$$
T_{l}^{m}~(\Pi^{l} A B) = \Pi^{m} T_{l}^{m}(A) (\mylam {x}{T_{l}(A)}{T_{l}^{m}(B~x)})~~~~~~
T_{l}^{m}~(\Sigma^{l} A B) = \Sigma^{m} T_{l}^{m}(A) (\mylam {x}{T_{l}(A)}{T_{l}^{m}(B~x)})~~~~~~
$$


\paragraph{Interpreting the level-indexed system in the system with externally indexed universes.}

A judgment in the level-indexed system can be interpreted in the externally indexed system relative to an assignment $\rho$ of external natural numbers to level variables. We simply replace each level expression in the judgment by the corresponding natural number obtained by letting $l^+\,\rho = l\,\rho+1$ and $(l \vee m)\,\rho = \max(l\,\rho,m\,\rho)$.

\subsection*{Rules for level-indexed products}

In Agda $\Level$ is a type, and it is thus possible to form level-indexed products of types. In our system $\Level$ is not a type, but if we want to be able to form level-indexed products we can extend the system with the following rules:
$$
\frac{A~\type~(\alpha~\Level)}{[\alpha]A~\type}~~~~~~~
\frac{t:[\alpha]A~~~~~l~\Level}
     {t~l:A(l/\alpha)}~~~~~~~~~
\frac{u:A~(\alpha~\Level)}{\lam{\alpha}{u}: [\alpha]A}~~~~~
\frac{t~\alpha = u~\alpha:A~(\alpha~\Level)}{t = u:[\alpha]A}
$$
with conversion rule $(\lam{\alpha}{u})~l = u(l/\alpha)$.

An example that uses level-indexed products is the following type which  expresses the theorem that univalence for universes of arbitrary level implies function extensionality for functions between universes of arbitrary levels.
$$
([\alpha]\mathsf{IsUnivalent}\, \UU_\alpha)
\to [\beta][\gamma] \mathsf{FunExt}\, \UU_\beta\, \UU_\gamma
$$   
               
We conjecture that all functions in $[\alpha]\NN$ in our system
are constant. 

%\paragraph{Interpreting the system with levels into extensional type theory with a super universe.} We can interpret the theory with rules for level-indexed products in extensional type theory with a (minimal) super universe, provided we add the rules for cumulativity. Levels will be interpreted by internal natural numbers $n : \NN$, and level-indexed products $[\alpha]A$ will be interpreted by ordinary indexed products $\Pi_{n : \NN}A$. Level application and level abstraction will be interpreted as ordinary application and abstraction.

%In intensional type theory we will not be able to justify all the laws for level expressions as definitional equalities. For example, we do not have $\max(m,n)=\max(n,m)$ definitionally.

\section{A system with level constraints}\label{sec:constraints}

\footnote{PD: We can postpone this question: TODO   can this be presented as GAT???}

To motivate why it may be useful to introduce the notion of judgment relative to a list of constraints on universe levels, consider the following type in a system without cumulativity:
%[[The type constructor ``$\simeq$'' should be replaced by the identity type for this to work. I am not sure which notation for the identity type we want to use to formulate this example.]]
%$$
%    (A : \UU_l) \to (B : \UU_m) \to A \times B \simeq B \times A
%$$
%Here we get the constraint is that $l \vee m = m \vee l$, which holds for arbitrary levels $l, m$.
%
%Now  consider the theorem
$$
    \Pi_{A:\UU_l~{B}:{\UU_m}~{C}:{\UU_n}}
    {~~\Id\,\UU_{l \vee m}\, (A\times^{l \vee m} B)\,(C \times^{n \vee l} A)
    \to \Id\,\UU_{m \vee l} \, (B\times^{m \vee l} A)\,(C \times^{n \vee l} A)}
%    \to A \times B =_\delta C \times A \to B \times A =_\epsilon C \times A
$$
It is well-formed provided $l \vee m = n \vee l$. There are three maximally general solutions: 
\begin{eqnarray*}
&&l = \alpha, m = \beta, n = \alpha \vee \beta\\
&&l = \alpha, m = \gamma \vee \alpha, n = \gamma\\
&&l = \beta \vee \gamma, m = \beta, n = \gamma
\end{eqnarray*}
where $\alpha, \beta,$ and $\gamma$ are level variables.
%
%$$
%    (A : \UU_\alpha) \to (B : \UU_\beta) \to (C : \UU_{\alpha \vee \beta}) 
%    \to \Id\,\UU_{\alpha \vee \beta}\, (A\times^{\alpha \vee \beta} B)\,(C \times^{\alpha \vee \beta} A)
%    \to \Id\,\UU_{\alpha \vee \beta}\, (B\times^{\alpha \vee \beta} A)\,(C \times^{\alpha \vee \beta} A)
%$$
%$$
%    (A : \UU_\alpha) \to (B : \UU_{\gamma \vee \alpha}) \to (C : \UU_{\gamma}) 
%    \to \Id\,\UU_{\gamma \vee \alpha}\, (A\times^{\gamma \vee \alpha} B)\,(C \times^{\gamma \vee \alpha} A)
%    \to \Id\,\UU_{\gamma \vee \alpha}\, (B\times^{\gamma \vee \alpha} A)\,(C \times^{\gamma \vee \alpha} A)
%$$
%$$
%    (A : \UU_{\beta \vee \gamma}) \to (B : \UU_{\beta}) \to (C : \UU_{\gamma}) 
%    \to \Id\,\UU_{\beta \vee \gamma}\, (A\times^{\beta \vee \gamma} B)\,(C \times^{\beta \vee \gamma} A)
%    \to \Id\,\UU_{\beta \vee \gamma}\, (B\times^{\beta \vee \gamma} A)\,(C \times^{\beta \vee \gamma} A)
%$$
(Note the similarity with the Gustave function in stable domain theory.)
In a system with level constraints, we could instead derive the type
$$
    \Pi_{A:\UU_\alpha~{B}:{\UU_\beta}~{C}:{\UU_\gamma}}
    {~~\Id\,\UU_{\alpha \vee \beta}\, (A\times B)\,(C \times A)
    \to \Id\,\UU_{\alpha \vee \gamma}\, (B\times A)\,(C \times A)}
$$
which is valid under the constraint
$
\alpha \vee \beta = \alpha \vee \gamma.
$

{\color{red}
TODO Cite \cite{giraud:cohom-non-abel}, \cite{chambert-loir:universes-matter} 
and \cite{waterhouse:sheaves}. Remark about $Set_\UU$.}

\subsection{Rules for level constraints}%\label{ssec:VVsystem}

A constraint is an equation $l = m$, where $l$ and $m$ are level expression. 
%Thus we have
%$$
%\frac{l_1~m_1~\dots~l_k~m_k~\Level}{l_1 = m_1,\dots,l_k = m_k~\Constraint}
%$$
%We write $l\leqslant m$ for the constraint $l\vee m = m$.

We then have a new context extension $\Gamma,\psi$ if $\psi$ is a finite set of constraints in $\Gamma$.

New judgment form $\psi~(\Gamma)$ which expresses that the constraint set $\psi$
holds in $\Gamma$. For instance we have $\alpha^+\leqslant\beta^+~(\alpha\leqslant\beta)$.

In order to have decidable type checking, we require that the constraints in a given context
should not be looping (which is decidable).

In the system with constraints, we have judgments such as
$$
T_{\alpha^+}^{\beta}(\UU^{\alpha}):\UU_{\beta}~(\alpha~\beta~\Level,\alpha^+\leqslant\beta)
$$
We introduce a new product operation
$$
\frac{t:A~(\psi)}{t:[\psi]A}
$$
with the rule $[\psi]A = A$ if $\psi = 1$.

%Intuitively, we can use a term of type $[\psi]A$ only if all constraints in $\psi$ hold.

{\color{red}
TODO: state that decidability of type-checking requires (1) decidability of loop checking
(since $\UU:\UU$ makes type checking undecidable) and (2) decidablity of whether $\psi$
follows from the constraints in $\Gamma$.}

\subsection{A decision problem for sup-semilattices with successor}

The type system in the previous (???) subsection gives rise to the following
decision problem: given a context $\Gamma$ containing constraints $\phi_1,\ldots,\phi_n$,
can one infer the constraint $\psi$?

Voevodsky considered a similar decision problem in \cite[Section 2]{VV}. 
His universe levels are natural numbers, with 0, successor, and maximum.
His constraints are equations between open terms in this vocabulary, 
and for decidability he refers to Presburger Arithmetic,
in which these constraints can easily be expressed.

Although this is correct, we want to make a couple of remarks on Voevodsky's approach.
The use of Presburger Arithmetic seems a bit of an overkill here.
The only operations on universe levels that are directly motivated
from type theory are a join operation and a compatible successor function.
There is no need to assume a level 0, a well-founded total ordering,
nor does one need that the successor function is injective.
Since type theory is meant as a foundation of mathematics, it is,
at least from a philosophical point of view, 
not a good idea to base the concept of a universe level 
on more mathematical notions than stricly necessary.

We propose to apply Occam's Razor here. This means that universe levels
are elements of an abstract sup-semilattice with a successor function
that is compatible with the join operation. The logic is ordinary 
equational logic.
This approach allows for possible other interpretations of universe levels
than natural numbers. At the same time the particular decision problem
called the \emph{uniform word problem} has to be solved for this equational theory. 
This has been done by Bezem and Coquand in \cite{bezem-coquand:lattices},
proving the following theorem.

\begin{theorem}\label{thm:P-solvability}
For sup-semilattices with a successor, both the uniform word problem  
and the problem of loop detection are decidable.
\end{theorem}
This theorem ensures that type checking remains decidable when
typings depend on universe level constraints


\section{Related work}

Harper and Pollack \cite{HarperP91} consider the system $\CComega$, the Calculus of Constructions with an sequence of universes \`a la Russell. They consider both an externally indexed version and a version with level expressions which are either numbers or variables, and with constraints of the form $l < m$ and $l \leq m$. Chan has an algorithm for solving those constraints in $\Ordo(c,v^3)$, where $c$ is the number of constraints and $v$ is the number of level variables. They also consider the level synthesis problem in a version with an anonymous universe Type.

Huet \cite{Huet87} develops another algorithm for universe polymorphism in the Calculus of Constructions. He drops the assumption that the universes form a linearly-ordered cumulative hierarchy indexed by the natural numbers, and on this basis develops a more efficient consistency checking algorithm than Chan's.
%\begin{quotation}
%(cut-and-paste from Harper and Pollack)
%Huet, in an unpublished manuscript \cite{Huet87}, has independently developed an
%algorithm for handling universes in the Calculus of Constructions. His approach is to drop the assumption that the universes form a linearly-ordered
%cumulative hierarchy indexed by the natural numbers, and to consider instead a family of calculi in which there is some well-founded partial ordering
%of universes. The input language is correspondingly restricted so that specific universes are disallowed; only the anonymous universe Type may be
%used. The principal advantage of this approach over the one considered here
%is that the consistency checking algorithm is significantly more efficient than
%Chan's algorithm, reducing to an acyclicity check in a dependency graph of
%universe levels. 
%\end{quotation}

Assaf \cite{Assaf14} considers an alternative version of the calculus of
constructions where subtyping is explicit. This new system avoids problems related to coercions and dependent types by using the Tarski style
of universes and by introducing additional equations to reflect equality. In particular he adds an explicit cumulativity map $\T^0_1 : \UU_0 \to \UU_1$. He argues that "full reflection" is necessary to achieve the expressivity of Russell style. He introduces the explicit cumulative calculus of constructions (CC$\uparrow$) which is closely related to our system of externally indexed Tarski style universes.

\paragraph{Acknowledgement.}
The authors acknowledge the support of the Centre for Advanced Study (CAS)
at the Norwegian Academy of Science and Letters
in Oslo, Norway, which funded and hosted the research project Homotopy  
Type Theory and Univalent Foundations during the academic year 2018/19.
\bibliographystyle{plain}
\bibliography{refs}

\end{document}
