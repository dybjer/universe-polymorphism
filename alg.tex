%\documentclass[12pt,a4paper]{amsart}
\documentclass[11pt,a4paper]{article}
%\ifx\pdfpageheight\undefined\PassOptionsToPackage{dvips}{graphicx}\else%
%\PassOptionsToPackage{pdftex}{graphicx}
\PassOptionsToPackage{pdftex}{color}
%\fi

%\usepackage{diagrams}

%\usepackage[all]{xy}
\usepackage{url}
\usepackage[utf8]{inputenc}
\usepackage{verbatim}
\usepackage{latexsym}
\usepackage{amssymb,amstext,amsmath,amsthm}
\usepackage{epsf}
\usepackage{epsfig}
% \usepackage{isolatin1}
\usepackage{a4wide}
\usepackage{verbatim}
\usepackage{proof}
\usepackage{latexsym}
%\usepackage{mytheorems}
\newtheorem{theorem}{Theorem}[section]
\newtheorem{corollary}{Corollary}[theorem]
\newtheorem{lemma}{Lemma}[theorem]
\newtheorem{proposition}{Proposition}[theorem]

\usepackage{float}
\floatstyle{boxed}
\restylefloat{figure}


%%%%%%%%%copied from SymmetryBook by Marc

% hyperref should be the package loaded last
\usepackage[backref=page,
            colorlinks,
            citecolor=linkcolor,
            linkcolor=linkcolor,
            urlcolor=linkcolor,
            unicode,
            pdfauthor={CAS},
            pdftitle={Symmetry},
            pdfsubject={Mathematics},
            pdfkeywords={type theory, group theory, univalence axiom}]{hyperref}
% - except for cleveref!
\usepackage[capitalize]{cleveref}
%\usepackage{xifthen}
\usepackage{xcolor}
\definecolor{linkcolor}{rgb}{0,0,0.5}

%%%%%%%%%
\def\oge{\leavevmode\raise
.3ex\hbox{$\scriptscriptstyle\langle\!\langle\,$}}
\def\feg{\leavevmode\raise
.3ex\hbox{$\scriptscriptstyle\,\rangle\!\rangle$}}

%%%%%%%%%
\newcommand\myfrac[2]{
 \begin{array}{c}
 #1 \\
 \hline \hline
 #2
\end{array}}


\newcommand*{\Scale}[2][4]{\scalebox{#1}{$#2$}}%
\newcommand*{\Resize}[2]{\resizebox{#1}{!}{$#2$}}

\newcommand{\II}{\mathbb{I}}
\newcommand{\refl}{\mathsf{refl}}
\newcommand{\mkbox}[1]{\ensuremath{#1}}


\newcommand{\Id}{\mathsf{Id}}
\newcommand{\conv}{=}
%\newcommand{\conv}{\mathsf{conv}}
\newcommand{\lam}[2]{{\langle}#1{\rangle}#2}
\def\NN{\mathsf{N}}
\def\UU{\mathsf{U}}
\def\JJ{\mathsf{J}}
\def\Level{\mathsf{Level}}
%\def\Type{\hbox{\sf Type}}
\def\ZERO{\mathsf{0}}
\def\SUCC{\mathsf{S}}

\newcommand{\type}{\mathsf{type}}
\newcommand{\N}{\mathsf{N}}
\newcommand{\Set}{\mathsf{Set}}
\newcommand{\El}{\mathsf{El}}
%\newcommand{\U}{\mathsf{U}} clashes with def's in new packages
\newcommand{\T}{\mathsf{T}}
\newcommand{\Usuper}{\UU_{\mathrm{super}}}
\newcommand{\Tsuper}{\T_{\mathrm{super}}}
%\newcommand{\conv}{\mathrm{conv}}
\newcommand{\idtoeq}{\mathsf{idtoeq}}
\newcommand{\isEquiv}{\mathsf{isEquiv}}
\newcommand{\ua}{\mathsf{ua}}
\newcommand{\UA}{\mathsf{UA}}
%\newcommand{\Level}{\mathrm{Level}}
\def\Constraint{\mathsf{Constraint}}
\def\Ordo{\mathcal{O}}

\def\Ctx{\mathrm{Ctx}}
\def\Ty{\mathrm{Ty}}
\def\Tm{\mathrm{Tm}}

\def\CComega{\mathrm{CC}^\omega}
\setlength{\oddsidemargin}{0in} % so, left margin is 1in
\setlength{\textwidth}{6.27in} % so, right margin is 1in
\setlength{\topmargin}{0in} % so, top margin is 1in
\setlength{\headheight}{0in}
\setlength{\headsep}{0in}
\setlength{\textheight}{9.19in} % so, foot margin is 1.5in
\setlength{\footskip}{.8in}

% Definition of \placetitle
% Want to do an alternative which takes arguments
% for the names, authors etc.

\newcommand{\natrec}{\mathsf{natrec}}
\newcommand{\nats}{\mathbb{N}}
%\rightfooter{}
\newcommand{\set}[1]{\{#1\}}
\newcommand{\sct}[1]{[\![#1]\!]}



\begin{document}

\title{An algorithm}

\author{Marc Bezem, Thierry Coquand, Peter Dybjer, Mart\'in Escard\'o}
\date{}
\maketitle



We assume given a finite set of variables $V$. If $x$ is in $V$
an {\em atom} is a formal sum $x+n$ with $n$ natural number. We identify
$x$ in $V$ with the atom $x+0$. We write $a,b,\dots$ for atoms.
If $a = x+n$ we write $a+k$ for $x+n+k$. If $A$ is a finite set of atoms $a_1,\dots,a_n$
we write $A+k$ for $a_1+k,\dots, a_n+k$.

 We add to $\nats$ a formal element $\infty$ with $n\leqslant \infty$ and $1+\infty = \infty$.

 Given a function $f(z)$ in $\nats + \infty$ we let $M(V,f)$ be the set of atoms
 $x+k$ with $k\leqslant f(z)$.

 Given a set $C$ of Horn clauses $A\rightarrow a$, a {\em model} of $C$ is
a set $M$ of atoms such that $A\subseteq M$ implies $a\in M$ for all $A\rightarrow a$
in $C$. We write $A\vdash a$ if we can deduce $a$ from $A$ using clauses in $C$.
The set of consequences of $A$ is the least model of $C$ containing $A$.

 We consider a set $C$ of Horn clauses such that
\begin{enumerate}
%\item if $A\rightarrow a$ in $C$ then $A$ is non empty
\item if $A\rightarrow y+n$ in $C$ and $x+m$ in $A$ then $n-m\leqslant 1$
\item if $B+1\rightarrow y+n$ in $C$ and $n>1$ then $B\rightarrow y+n-1$ in $C$
\item if $A\rightarrow a$ in $C$ then $A+1\rightarrow a+1$ in $C$
\item given $y$ in $V$ we can list all clauses $A\rightarrow y+k$ in $C$  
\end{enumerate}  

If $W\subseteq V$ we write $C_W$ the set of Horn clauses with only variables in $W$.
It is clear that $C_W$ satisfy the same conditions we just listed.

Given such a set $C$, we claim that we can compute a function $f(z)$ for $z$ in $V$
such that $M(V,f)$ is the set of consequences of $V$. This follows from the following lemma.


\begin{lemma}
  If $f(z)$ for $z$ in $W$ is such that $M(W,f)$ is the set of consequences of $W$ for $C_W$
  and we define $g(x) = f(x)+1$ for $x\in W$ and $g(x) = 0$ for $x\in V\setminus W$
  and there is no clause $A\rightarrow y+1$ with $A\subseteq M(V,g)$
  and we can derive $z+1$ from $V$ for all $z$ in $W$
  then $M(V,g)$ is the set of consequences of $V$. 
\end{lemma}  
 
\begin{proof}
  First, we can derive $x+k$ from $V$ if $k\leqslant g(x)$: this is clear if $x$ is in $V\setminus W$
  since then $g(x) = 0$, and if $x$ is in $W$ and $k>0$ we have $W\vdash x+k-1$ and so $W+1\vdash x+k$
  using condition (3) and we have $V\vdash z+1$ for all $z$ in $W$, hence $V\vdash x+k$.

  Second,  we show that $M(V,g)$ is a model of $C$ containing $V$. The only thing we have to show that is that
  if $A\rightarrow z+n$ in $C$ and $z$ in $W$ then $n\leqslant g(z)$.

  This is clear if $n\leqslant 1$ since $1\leqslant g(z)$.

  If $n>1$ then condition (1) shows that $A = B+1$ and we have
  $A\subseteq M(W,g)$ since $g(y) = 0$ if $y\in V\setminus W$. We then $B\vdash y+n-1$
  by condition (2) and $n-1\leqslant f(y)$ since $M(W,f)$ is a model of $C_W$.
  So we also have $n\leqslant f(y)+1 = g(y)$.
\end{proof}

\begin{theorem}
We can compute $f(z)$ for $z$ in $V$ such that $V\vdash z+n$ if, and only if, $n\leqslant f(z)$.
\end{theorem}  


Assume that we have $f(z)<\infty$ for all $z$ in $V$. This provides a {\em finite} model $M(V,f)$
of $C$. Using conditions (1) and (2), we see that $M(V, \lambda_z f(z) +k)$ is a (finite) model of $C$
for any $k$. As an application we get a decision algorithm for deciding if $a$ is a consequence
of a set $x+p(x),~x\in V$ for a given function $p(x)$ in $\nats$. Indeed we know that the set of
consequences is a subset of $M(V, \lambda_z f(z)+k)$ for $k$ maximum of all $p(x)$ and this set
is finite.




\bibliographystyle{plain}
\bibliography{refs}

\end{document}
