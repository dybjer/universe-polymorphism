\documentclass[11pt,a4paper]{article}
%\ifx\pdfpageheight\undefined\PassOptionsToPackage{dvips}{graphicx}\else%
%\PassOptionsToPackage{pdftex}{graphicx}
\PassOptionsToPackage{pdftex}{color}
%\fi

%\usepackage{diagrams}

%\usepackage[all]{xy}
\usepackage{url}
\usepackage{verbatim}
\usepackage{latexsym}
\usepackage{amssymb,amstext,amsmath,amsthm}
\usepackage{epsf}
\usepackage{epsfig}
% \usepackage{isolatin1}
\usepackage{a4wide}
\usepackage{verbatim}
\usepackage{proof}
\usepackage{latexsym}
%\usepackage{mytheorems}
\newtheorem{theorem}{Theorem}[section]
\newtheorem{corollary}{Corollary}[theorem]
\newtheorem{lemma}{Lemma}[theorem]
\newtheorem{proposition}{Proposition}[theorem]

\usepackage{float}
\floatstyle{boxed}
\restylefloat{figure}


%%%%%%%%%copied from SymmetryBook by Marc

% hyperref should be the package loaded last
\usepackage[backref=page,
            colorlinks,
            citecolor=linkcolor,
            linkcolor=linkcolor,
            urlcolor=linkcolor,
            unicode,
            pdfauthor={CAS},
            pdftitle={Symmetry},
            pdfsubject={Mathematics},
            pdfkeywords={type theory, group theory, univalence axiom}]{hyperref}
% - except for cleveref!
\usepackage[capitalize]{cleveref}
%\usepackage{xifthen}
\usepackage{xcolor}
\definecolor{linkcolor}{rgb}{0,0,0.5}

%%%%%%%%%
\def\oge{\leavevmode\raise
.3ex\hbox{$\scriptscriptstyle\langle\!\langle\,$}}
\def\feg{\leavevmode\raise
.3ex\hbox{$\scriptscriptstyle\,\rangle\!\rangle$}}

%%%%%%%%%
\newcommand\myfrac[2]{
 \begin{array}{c}
 #1 \\
 \hline \hline
 #2
\end{array}}


\newcommand*{\Scale}[2][4]{\scalebox{#1}{$#2$}}%
\newcommand*{\Resize}[2]{\resizebox{#1}{!}{$#2$}}

\newcommand{\II}{\mathbb{I}}
\newcommand{\refl}{\mathsf{refl}}
\newcommand{\mkbox}[1]{\ensuremath{#1}}


\newcommand{\Id}{\mathsf{Id}}
\newcommand{\conv}{=}
%\newcommand{\conv}{\mathsf{conv}}
\newcommand{\lam}[2]{{\langle}#1{\rangle}#2}
\def\NN{\mathsf{N}}
\def\UU{\mathsf{U}}
\def\JJ{\mathsf{J}}
\def\Level{\mathsf{Level}}
%\def\Type{\hbox{\sf Type}}
\def\ZERO{\mathsf{0}}
\def\SUCC{\mathsf{S}}

\newcommand{\type}{\mathsf{type}}
\newcommand{\N}{\mathsf{N}}
\newcommand{\Set}{\mathsf{Set}}
\newcommand{\El}{\mathsf{El}}
%\newcommand{\U}{\mathsf{U}} clashes with def's in new packages
\newcommand{\T}{\mathsf{T}}
\newcommand{\Usuper}{\UU_{\mathrm{super}}}
\newcommand{\Tsuper}{\T_{\mathrm{super}}}
%\newcommand{\conv}{\mathrm{conv}}
\newcommand{\idtoeq}{\mathsf{idtoeq}}
\newcommand{\isEquiv}{\mathsf{isEquiv}}
\newcommand{\ua}{\mathsf{ua}}
\newcommand{\UA}{\mathsf{UA}}
%\newcommand{\Level}{\mathrm{Level}}
\def\Constraint{\mathsf{Constraint}}
\def\Ordo{\mathcal{O}}

\def\Ctx{\mathrm{Ctx}}
\def\Ty{\mathrm{Ty}}
\def\Tm{\mathrm{Tm}}

\def\CComega{\mathrm{CC}^\omega}
\setlength{\oddsidemargin}{0in} % so, left margin is 1in
\setlength{\textwidth}{6.27in} % so, right margin is 1in
\setlength{\topmargin}{0in} % so, top margin is 1in
\setlength{\headheight}{0in}
\setlength{\headsep}{0in}
\setlength{\textheight}{9.19in} % so, foot margin is 1.5in
\setlength{\footskip}{.8in}

% Definition of \placetitle
% Want to do an alternative which takes arguments
% for the names, authors etc.

\newcommand{\natrec}{\mathsf{natrec}}
%\rightfooter{}
\newcommand{\set}[1]{\{#1\}}
\newcommand{\sct}[1]{[\![#1]\!]}



\begin{document}

\title{A Note on Universe Polymorphism\\
(some unfinished ideas not included in the paper)}

\author{Marc Bezem, Thierry Coquand, Peter Dybjer, Mart\'in Escard\'o}
\date{}
\maketitle

\section{Translation to Horn clauses}

There exist a well-known translation of lattice theory to Horn clauses
that we will explore now. The translation consists of several steps.

\begin{enumerate}
\item Every level expression is equal to its so-called
\emph{$+$-normal form} which is obtained by exhaustively using 
$(x\vee y)^+ = x^+ \vee y^+$ until the operation $\_^+$ is only applied
to terms without $\vee$. Let $\alpha+n$ denote $\alpha^{+\cdots+}$, the
operation $\_^+$ applied $n$ times, for any $n\in\mathbb{N}$.
Level expressions of the form $\alpha+n$ will be called \emph{atoms}.
Then a $+$-normal form is just a join of finitely many such atoms.
Level expressions in $+$-normal form will be denoted by capital letters $L,M$.

\item If $L$ is a level expression in $+$-normal form and $a$ an atom,
then we call $L\to a$ a \emph{Horn clause}. The interpretation $\sct{L\to a}_\rho$
is that the maximum of all $a_i\rho$ for $a_i$ in $L$ is greater than or equal $a\rho$.

\item Under the interpretation above, any constraint $L=M$ is
equivalent to the set of Horn clauses consisting of $L\to b$ for any $b$ in $M$
and $M\to a$ for any $a$ in $L$.

\item TBD

\end{enumerate}

\section{Related work}

Check related work of Ali Assaf, Luo, Mendler, Voevodsky, Huet, Harper and Pollack, Sozeau. Blogpost of McBride.

\section{Models in $\mathbb{Z}$ rather than $\mathbb{N}$?}

Discussion. Should we have operation $l^-$.

\section{Completeness}

Completeness of system wrt algorithm. Can we derive $l = l^+$ for some $l$ whenever the algorithm fails to find a model? Should we have the rule $l^+ = m^+$ implies $l = m$?

\section{Martin's examples}
Example of use of level-indexed products?

OK. Here is a story (although probably not in the format I would include 
it in a paper).

(1) With level judgments, one can work with global properties, either 
concrete, such as
$$
     is-prop : (l : Level) → U l → U l
$$
or hypothetical, such as $((l : Level) → is-univalent U)$ in
$$
     ua-gives-funext : ((l : Level) → is-univalent U)
                     → (l m : Level) → FunExt (U l) (U m))
$$
(2) One would hope that global univalence would give
$$
       (P : (l : Level) → U l → U l))
     → (l m : Level)
     → (X : U l)
     → (Y : U m)
     → X ≃ Y
     → P X
     → P Y
$$
That is, that under univalence, global properties would be invariant
under equivalence. But they are not (although it may be possible that
the *definable* global properties (by which I mean global properties
in an empty context) remain invariant under equivalence). This
possibly motivates the consideration of parametricity, or at least of
some other condition, as "invariance under equivalence" is at the
heart of univalent foundations. Such a condition can be syntactical,
modifying the type theory with level judgements, or in the form of an
axiom (like univalence is).

(3) Univalence implies univalence induction, which is just like J, but
     with $\simeq$ in place of the identity type.

     To show that all equivalences satisfy a property, it is enough to
     show that the identity equivalences satisfy the property.

     Simple example: it is difficult to prove directly that
     equivalences are half-adjoint equivalences, but it is immediate
     that identity functions are haes. Hence all equivalences are haes.

     (Replace "hae" by your favourite property of equivalences.)

     However, the above is a half-truth.

(3b) In my MGS lecture notes I tried to apply this to show that the
      equivalence (under univalence)
$$
      Id-to-Eq X Y : Id U X Y →  Eq U X Y
$$
      is a hae, which in turn can be used to give a simple proof of the
      structure identity principle (I chose to specialize this to
      magmas for didactical purposes).

      But this doesn't work, because

          $Id U X Y$ lives in the sucessor universe $U'$ with $U:U'$
          $Eq U X U$ lives in $U$.

      (I got a way around this by generalizing, in two ways,
      equivalence induction.)

(3c) You may hope that with cumulativity the problem in (3c) disappears.

      I conjectured that it doesn't.

      Mike Shulman confirmed that the proof that $Id-to-Eq X Y$ is a hae
      by equivalence induction doesn't type check with cumulativity and
      typical ambiguity in Coq.

      So it really doesn't.

      Cumulativity doesn't solve the problem. With typical ambiguity,
      you moreover get a "universe inconsistency" message with no
      further explanation.

I guess what I am doing above is to try to understand what happens
when we extend the type theory with level judgments.

I will report about one further example later: my student Tom de Jong
will present some work at Types on developing the Scott model of PCF
in a type theory with function extensionality, propositional
extensionality and propositional truncation and universes. This was
done in UniMath with type-in-type and on paper with
back-of-the-envelope universe level calculations, which are not really
discussed in the paper. We are re-formalizing this in Agda to keep
track more precisely the universe levels. Some interesting things are
happening. Firstly, although we haven't finished, can can already see
that everything will just work, although the reasons turn out to be
rather subtle. Secondly, the most natural way to do this *doesn't*
work unless Level is a type (the reason is that terms of ground type
have denotations in $U₀$, but terms of higher type have denotations in
$U₁$, and so we would need the level of a universe to depend on the type
of a PCF term. This *is* possible in Agda, because Level is a type,
but not in our type theory, where Level is a judgement. However,
luckly, we can in this simple example just "move up" the only thing in
$U₀$ to $U₁$ and avoid using Level as a type. But one can imagine
situations of languages more complex then PCF that would require a
type Level in the meta-language to interpret them. Anyway, I will
report back on this example, which we intend to complete to have ready
for the Types meeting.


\section*{The problem of universes}

Some questions to be explored

\medskip

injective : $[\alpha](\UU_{\alpha^+}\rightarrow \UU_{\alpha^+})$ if we have only one level

Super universe? Solution with a 2 level type system

Constraint problems: is there a Finite number of most general solutions?

E.g. system $(l\vee m) = n^+$, system $l\vee m  = l\vee n$

If we have a function $f:[\alpha](\UU_{\alpha}\rightarrow \UU_{\alpha^+})$ we cannot iterate
this function internally, for instance  $f = \lam{\alpha}{\lambda (X:\UU_{\alpha})~X\rightarrow\UU_{\alpha}}$

%\thebibliography

%\bibitem{VV}
%  V. Voevodsky
%  \newblock{Universe polymorphic type system.}
%  \newblock{Manuscript, 2014}
%
%\bibitem{BNR08}
%M.A.~Bezem, R.~Nieuwenhuis and E.~Rodr\'iguez.
%\newblock{The Max-Atom Problem and its Relevance}.
%In I.~Cervesato, H.~Veith and A.~Voronkov, editors,
%\emph{Proceedings LPAR-15}, LNAI 5330,
%pages 47--62, Springer-Verlag, Berlin, 2008.

%% \bibitem{Abel}
%% A. Abel and G. Scherer.
%% \newblock{On Irrelevance and Algorithmic Equality in Predicative Type Theory.}
%% \newblock{In Logical Methods in Computer Science, 8(1):1-36, 2012.}

%% \bibitem{Aczel}
%% P. Aczel.
%% \newblock{On Relating Type Theories and Set Theories.}
%% \newblock{{\em Types for proofs and programs}, 1–18, Lecture Notes in Comput. Sci., 1657, 1999.}

%% \bibitem{AK}
%% Th. Altenkirch and A. Kaposi.
%% \newblock{Normalisation by Evaluation for Dependent Type Theory.}
%% \newblock{FSCD, 2016.}

%% \bibitem{AHS}
%% Th. Altenkirch, M. Hofmann and Th. Streicher.
%% \newblock{Reduction-free normalisation for system F.}
%% \newblock{Unpublished note, 1997.}

%% \bibitem{BJ}
%% J.-Ph. Bernardy, P. Jansson, R. Paterson.
%% \newblock{Parametricity and dependent types.}
%% \newblock{ ICFP 2010: 345-356.}

%% \bibitem{BS}
%% U. Berger and H. Schwichtenberg.
%% \newblock{An inverse of the evaluation functional for typed lambda-calculus.}
%% \newblock{Proceedings of LICS 1991.}

%% \bibitem{Bickford}
%% M. Bickford.
%% \newblock{Formalizing Category Theory and Presheaf Models of Type Theory in Nuprl.}
%% \newblock{Preprint, https://arxiv.org/abs/1806.06114, 2018.}

%% %% \bibitem{CD}
%% %% Th. Coquand and P. Dybjer.
%% %% \newblock{Intuitionistic Model Constructions and Normalization Proofs.}
%% %% \newblock{Mathematical Structures in Computer Science 7(1): 75-94 (1997).}

%% \bibitem{Coq}
%% Th. Coquand.
%% \newblock{An algorithm for testing conversion in type theory.}
%% \newblock{In {\em Logical frameworks}, p. 255-279, Cambridge University Press, 1991.}

%% \bibitem{CCHM}
%% C. Cohen, Th. Coquand, S. Huber, A. M\"ortberg.
%% \newblock{Cubical type theory: a constructive interpretation of the univalence axiom.}
%% \newblock{Proceeding of the Type Conference, 2015.}

%% \bibitem{Rathjen}
%% L. Crosilla and ML. Rathjen
%% \newblock{Inaccessible set axioms may have little consistency strength.}
%% \newblock{Ann. Pure Appl. Log. 115, 33–70 (2002).}

%% \bibitem{Dybjer}
%% P. Dybjer.
%% \newblock{Internal Type Theory.}
%% \newblock{in {\em Types for Programs and Proofs}, Springer, 1996.}

%% %% \bibitem{Fiore}
%% %% M. Fiore, G. Pltokin and D. Turi.
%% %% \newblock{Abstract Syntax and Variable Binding.}
%% %% \newblock{Proc.\ 14$^{\rm th}$ LICS Conf., 1999.}

%% \bibitem{FLD}
%% S. Fortune, D. Leivant, M. O'Donnell.
%% \newblock{The Expressiveness of Simple and Second-Order Type Structures.}
%% \newblock{Journal of the ACM, Volume 30 Issue 1, p. 151-185, 1983.}

%% \bibitem{Godel}
%% K. G\"odel.
%% \newblock{\"Uber eine bisher noch nicht ben\"utzte Erweiterung des finiten Standpunktes.}
%% \newblock{Dialectica, 12, pp. 280-287, 1958.}

%% \bibitem{Hofmann1}
%% M. Hofmann.
%% \newblock{Syntax and semantics of dependent type theory.}
%% \newblock{In {\em Semantics of Logic of Computation}, Cambridge University Press, 1997.}

%% %% \bibitem{Hofmann}
%% %% M. Hofmann.
%% %% \newblock{Semantics analysis of higher-order syntax.}
%% %% \newblock{Proc.\ 14$^{\rm th}$ LICS Conf., 1999.}

%% \bibitem{LS}
%% J. Lambek and P.J. Scott.
%% \newblock{{\it Introduction to higher order categorical logic.}}
%% \newblock{Cambridge studies in advanced mathematics 7, 1986.}


%% \bibitem{ML72}
%% P. Martin-L\"of.
%% \newblock{An intuitionistic theory of types.}
%% \newblock{Preliminary version 1972; published in {\em 25 Years of Type Theory}, 1995.}

%% \bibitem{ML73}
%% P. Martin-L\"of.
%% \newblock{An intuitionistic theory of types: predicative part.}
%% \newblock{Logic Colloquium '73 (Bristol, 1973), pp. 73–118.}

%% \bibitem{ML74}
%% P. Martin-L\"of.
%% \newblock{About Models for Intuitionistic Type Theories and the Notion of Definitional Equality.}
%% \newblock{Proceedings of the Third Scandinavian Logic Symposium, 1975, Pages 81-109.}

%% \bibitem{ML79}
%% P. Martin-L\"of.
%% \newblock{Constructive mathematics and computer programming.}
%% \newblock{Logic, methodology and philosophy of science, VI (Hannover, 1979),  pp. 153--175,
%%  Stud. Logic Found. Math., 104, North-Holland, Amsterdam, 1982.}

%% \bibitem{Shoenfield}
%% J.R. Shoenfield.
%% \newblock{{\em Mathematical Logic.}}
%% \newblock{Addison-Wesley, 1967.}

%% \bibitem{Shulman}
%% M. Shulman.
%% \newblock{Univalence for inverse diagrams and homotopy canonicity.}
%% \newblock{{\em Mathematical Structures in Computer Science}, 25:05, p. 1203–1277, 2014.}

%% \bibitem{Tait}
%% W.W. Tait.
%% \newblock{Intensional interpretations of functionals of finite type, part I.}
%% \newblock{Journal of Symbolic Logic, 32, pp. 198-212, 1967.}



%\end{thebibliography}

\end{document}

