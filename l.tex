%\documentclass[12pt,a4paper]{amsart}
\documentclass[11pt,a4paper]{article}
%\ifx\pdfpageheight\undefined\PassOptionsToPackage{dvips}{graphicx}\else%
%\PassOptionsToPackage{pdftex}{graphicx}
\PassOptionsToPackage{pdftex}{color}
%\fi

%\usepackage{diagrams}

%\usepackage[all]{xy}
\usepackage{url}
\usepackage[utf8]{inputenc}
\usepackage{verbatim}
\usepackage{latexsym}
\usepackage{amssymb,amstext,amsmath,amsthm}
\usepackage{epsf}
\usepackage{epsfig}
% \usepackage{isolatin1}
\usepackage{a4wide}
\usepackage{verbatim}
\usepackage{proof}
\usepackage{latexsym}
%\usepackage{mytheorems}
\newtheorem{theorem}{Theorem}[section]
\newtheorem{corollary}[theorem]{Corollary}
\newtheorem{lemma}[theorem]{Lemma}
%\newtheorem{proposition}{Proposition}[theorem]
%\newtheorem{example}{Example}[theorem]

\usepackage{float}
\floatstyle{boxed}
\restylefloat{figure}


%%%%%%%%%copied from SymmetryBook by Marc

% hyperref should be the package loaded last
\usepackage[backref=page,
            colorlinks,
            citecolor=linkcolor,
            linkcolor=linkcolor,
            urlcolor=linkcolor,
            unicode,
            pdfauthor={CAS},
            pdftitle={Symmetry},
            pdfsubject={Mathematics},
            pdfkeywords={type theory, group theory, univalence axiom}]{hyperref}
% - except for cleveref!
\usepackage[capitalize]{cleveref}
%\usepackage{xifthen}
\usepackage{xcolor}
\definecolor{linkcolor}{rgb}{0,0,0.5}

%%%%%%%%%
\def\oge{\leavevmode\raise
.3ex\hbox{$\scriptscriptstyle\langle\!\langle\,$}}
\def\feg{\leavevmode\raise
.3ex\hbox{$\scriptscriptstyle\,\rangle\!\rangle$}}

%%%%%%%%%
\newcommand\myfrac[2]{
 \begin{array}{c}
 #1 \\
 \hline \hline
 #2
\end{array}}


\newcommand*{\Scale}[2][4]{\scalebox{#1}{$#2$}}%
\newcommand*{\Resize}[2]{\resizebox{#1}{!}{$#2$}}

\newcommand{\II}{\mathbb{I}}
\newcommand{\refl}{\mathsf{refl}}
\newcommand{\mkbox}[1]{\ensuremath{#1}}


\newcommand{\Id}{\mathsf{Id}}
\newcommand{\conv}{=}
%\newcommand{\conv}{\mathsf{conv}}
\newcommand{\lam}[2]{{\langle}#1{\rangle}#2}
\def\NN{\mathsf{N}}
\def\UU{\mathsf{U}}
\def\JJ{\mathsf{J}}
\def\Level{\mathsf{Level}}
%\def\Type{\hbox{\sf Type}}
\def\ZERO{\mathsf{0}}
\def\SUCC{\mathsf{S}}

\newcommand{\type}{\mathsf{type}}
\newcommand{\N}{\mathsf{N}}
\newcommand{\Set}{\mathsf{Set}}
\newcommand{\El}{\mathsf{El}}
%\newcommand{\U}{\mathsf{U}} clashes with def's in new packages
\newcommand{\T}{\mathsf{T}}
\newcommand{\Usuper}{\UU_{\mathrm{super}}}
\newcommand{\Tsuper}{\T_{\mathrm{super}}}
%\newcommand{\conv}{\mathrm{conv}}
\newcommand{\idtoeq}{\mathsf{idtoeq}}
\newcommand{\isEquiv}{\mathsf{isEquiv}}
\newcommand{\ua}{\mathsf{ua}}
\newcommand{\UA}{\mathsf{UA}}
%\newcommand{\Level}{\mathrm{Level}}
\def\Constraint{\mathsf{Constraint}}
\def\Ordo{\mathcal{O}}

\def\Ctx{\mathrm{Ctx}}
\def\Ty{\mathrm{Ty}}
\def\Tm{\mathrm{Tm}}

\def\CComega{\mathrm{CC}^\omega}
\setlength{\oddsidemargin}{0in} % so, left margin is 1in
\setlength{\textwidth}{6.27in} % so, right margin is 1in
\setlength{\topmargin}{0in} % so, top margin is 1in
\setlength{\headheight}{0in}
\setlength{\headsep}{0in}
\setlength{\textheight}{9.19in} % so, foot margin is 1.5in
\setlength{\footskip}{.8in}

% Definition of \placetitle
% Want to do an alternative which takes arguments
% for the names, authors etc.

\newcommand{\natrec}{\mathsf{natrec}}
\newcommand{\nats}{\mathbb{N}}
%\rightfooter{}

%added by Marc

\newcommand\set[1]{\{#1\}}
\newcommand\sct[1]{[\![#1]\!]}
\newcommand\jterm[3]{{{#1_1}+{#2_1}}\vee\cdots\vee{{#1_#3}+{#2_#3}}}
\newcommand\jbody[3]{{{#1_1}+{#2_1}},\ldots,{{#1_#3}+{#2_#3}}}
\newcommand\lathy{{\cal L}}
\newcommand\prvL{\vdash_{\lathy}}
\newcommand\prvH{\vdash_{\cal H}}
\newcommand\Ninf{\N^\infty}
\newcommand\M{\mathsf{Maxgain}}
\newcommand\upS[1]{\overline{S_{#1}}}
\newcommand{\FYI}[1]{{\color{red}#1}}




\begin{document}

\title{Loop-checking and the uniform word problem
for join-semilattices with an inflationary endomorphism}

\author{Marc Bezem%
\thanks{Department of Informatics, University of Bergen, {\tt Marc.Bezem@uib.no}}
~and Thierry Coquand
\thanks{Department of Computer Science and Engineering, 
Chalmers/University of Gothenburg, {\tt coquand@chalmers.se}}
}
\date{}


\maketitle


\section{Introduction}\label{sec:intro}
The \emph{uniform word problem} for an equational theory $T$ is
to determine, given a finite set $E$ of relations between generators,
whether a given relation is provable from $E$ in $T$.
The theory of join-semilattices with an inflationary endomorphism
is a special case of a certain description logic, that was shown
to have an EXPTIME-complete uniform word problem by Hofmann \cite{Hofmann}
and Baader \cite{Baader}.
Here we show that the special case of one inflationary endomorphism
has a uniform word problem that can be solved in PTIME. We also
show that loop-checking can be done in PTIME.
Our special case is relevant for dependent type theory with universes.

We start from the equational definition of semilattices in which the
join, denoted by $\vee$, is an associative, commutative, and idempotent
binary operation. The endomorphism, denoted by ${\_}^+$,
satisfies the following two equational axioms:

\[
x \vee x^+ = x^+ \quad \quad (x\vee y)^+ = x^+ \vee y^+
\] 
The logic is ordinary equational logic.
We denote the resulting theory by $\lathy$.
For example, we can prove $(t^+)^+ \vee t = (t^+)^+$ in $\lathy$,
for any term $t$. Also, we can infer $s^+ = (t^+)^+$
from $s = t^+$, but not conversely. 
As is customary, we let $s\geq t$ abbreviate $s\vee t = s$.
\FYI{Throughout this note we call a join-semilattice with an inflationary 
endomorphism simply a \emph{semilattice}. 
We call ${t}^+$ the \emph{successor} of $t$.}


\section{Semilattice presentations and Horn clauses}\label{sec:latt-Horn}

A \emph{semilattice presentation} consists of a
set $V$ of \emph{generators} and a set $C$ of \emph{relations}.
We will colloquially call the generators also variables, and the
relations constraints.
For any semilattice term $t$ and natural number $k$, 
let $t+k$ denote the $k$-fold successor of $t$.
Thus $t+0 = t$, and we may use $t+1$ and $t^+$ interchangeably. 
A \emph{term over} $V$ is
a term of the form $\jterm{x}{k}{m}$, 
with all $x_i\in V$ and $k_i\in \N$.
 
Since the endomorphism commutes with the join operation,
every semilattice term $t$ is equal to a term of the
form $\jterm{x}{k}{m}$, with all variables $x_i$ occurring in $t$
and all $k_i\in \N$.

A \emph{relation} is an equation $s=t$ with $s,t$ terms over $V$.
A constraint like $x=y^+$ with $x,y\in V$
expresses a relation between the generators $x$ and $y$ and 
should not be read as an implicitly universally quantified axiom
in which the $x$ and $y$ can be instantiated.

The semilattice \emph{presented by} $(V,C)$ consists of terms over $V$
modulo provable equality from $C$. The latter will be denoted
by $C\prvL s=t$. In the next section we will prove that
$C\prvL s=t$ is decidable in polynomial time for finite semilattice presentations $(V,C)$.
The results in this section hold for arbitrary semilattice presentations.

We follow Lorenzen \cite[Section 2]{Lorenzen51} for an equivalent 
characterisation of $C\prvL s=t$ using Horn clauses.
Let $(V,C)$ be a semilattice presentation.
Let $s:= \jterm{x}{k}{m}$ and $t:= \jterm{y}{l}{n}$ be
terms over $V$. From the constraint $s=t$
we can prove $m+n$ inequalities which we write as
Horn clauses by replacing join by conjunction (written as ``,'') and $\geq$
by implication.\footnote{In this note all clauses are positive,
propositional Horn clauses with a non-empty body,
with abstract atoms of the form $x+k$ with $x\in V$ and $k\in \N$.} 
Thus we get the set $S_{s=t}$ consisting of the following Horn clauses:
\begin{align*}
\jbody{x}{k}{m} &\to y_1+l_1 \\
&\ldots  \\
\jbody{x}{k}{m} &\to y_n+l_n 
\end{align*}
and
\begin{align*}
\jbody{y}{l}{n} &\to x_1+k_1 \\
&\ldots \\
\jbody{y}{l}{n} &\to x_m+k_m 
\end{align*}
Let $S_C$ be the union of all $S_{s=t}$ with $s=t$ 
a constraint in $C$.

We reflect for a moment on which other clauses we need.
Consider the axiom $x \vee x^+ = x^+$. This would lead to three
clauses: $x,x^+ \to x^+$, $x^+ \to x^+$, and $x^+ \to x$.
Only the last is non-trivial, we call it a \emph{predecessor clause}.
The next question is: for which $x$ do we need a predecessor clause?
Since the axiom $x \vee x^+ = x^+$ is implicitly universally quantified,
we would need all instances with $x$ a term over $V$.
For this it suffices to have all predecessor clauses
$x+k+1 \to x+k$ with $x\in V$ and all $k\in \N$.

The axiom stating that endomorphism and join commute is built-in
in the notion of term over $V$ and does not require extra clauses.
However, we should not forget the congruence axioms from
equational logic. Congruence of the endomorphism means that $s=t$
implies $s+1=t+1$. This requires that we close the set of clauses
under \emph{shifting upwards}: if $A\to b$ is in the clause set,
then so is $A+1\to b+1$, where $A+1$ is the set of atoms of
the form $a+1$ with $a\in A$. 
Congruence of join means that $s=t$ implies $r\vee s=r\vee t$.
It is easy to see that this does not require extra clauses. 

In summary: given a semilattice presentation $(V,C)$,
let $\upS{C}$ be the smallest set of clauses that 
is closed under shifting upwards and contains
the set $S_C$ coming from the constraints in $C$,
as well as all predecessor clauses for each $v\in V$. 

Given a set $S$ of Horn clauses,
let $S\prvH A\to b$ denote provability from $S$.
One convenient way to define this is by induction:
$S\prvH A\to b$ if 
(i) $b\in A$ or 
(ii) $S\prvH A,c\to b$ for some $c$ such that there exists
a clause $A'\to c$ in $S$ with $A'\subseteq A$.
This is the inductive way of defining forward reasoning,
that is, using the Horn clauses in $S$ to generate atoms from $A$.
We can also use this definition if $A$ is infinite.
Then it is more customary to write $S,A \prvH b$.

For $X$ a set of atoms, define $X^+ := \set{x^+ \mid x\in X}$
and $X+k := \set{x+k \mid x\in X}$ for all $k\in\N$.
For $S$ a set of clauses, define 
$S^+ := \set{X^+\to y^+ \mid \text{$X\to y$ in $S$}}$.
The following lemma will be used later on.

\begin{lemma}\label{lem:shift-up}
Let $V$ be a finite set of variables, and $A\to b$ a Horn clause.
Let $S$ and $T$ be sets of Horn clauses, where all clauses in $T$ 
have conclusion \emph{in} $V$. Then the following three are equivalent:
(1) $S\prvH A\to b$; (2) $S^+\prvH A^+\to b^+$; (3) $S^+ \cup T \prvH V,A^+\to b^+$.
\end{lemma} 
\begin{proof} 
Immediate by induction on the definition of $\prvH$.
\end{proof}


We have the following theorem generalizing \cite[Theorem 3]{Lorenzen51}.

\begin{theorem}\label{thm:LvsH}
For all terms $\jbody{x}{k}{m},y+l$ over $V$ we have:
\[
C\prvL \jterm{x}{k}{m}\geq y+l 
\quad\text{iff}\quad
\upS{C}\prvH \jbody{x}{k}{m}\to y+l.
\]
\end{theorem} 
\begin{proof}
The if-part is a simple induction on derivation of $\prvH$:
all steps can easily be mimicked in semilattice theory.
The converse implication is more interesting.
For any set of Horn clauses $X,Y$, let $X \prvH Y$ mean
that $X \prvH A\to b$ for all clauses $A\to b$ in $Y$.
For any two terms $s,t$ over $V$, define $s\equiv t$ by
$\upS{C} \prvH S_{s=t}$. We define $s\vee t$ in the obvious way,
and $(\jterm{x}{k}{m})^+ := x_1+k_1+1 \vee\cdots\vee x_m+k_m+1$.
The latter is well-defined, since we have $s^+\equiv t^+$ 
if $s\equiv t$, as $\upS{C}$ is closed under shifting upwards.

Now one can show that all axioms and rules are satisfied modulo $\equiv$. 
For example, $\equiv$ is an equivalence relation and
a congruence with respect to join and endomorphism (successor).
All semilattice axioms are satisfied, for example,
the predecessor clauses prove $s\vee s^+ \equiv s^+$. 
Moreover, for each constraint $s=t$ in $C$ 
we have $s\equiv t$, as $S_{s=t}$ is included in $\upS{C}$.

By soundness, if $C\prvL s=t$, then $s\equiv t$.
In particular we have the only-if part of the theorem.
\end{proof}

\section{Decidability}\label{sec:decidability}

In this section we first prove the decidability of $\upS{C}\prvH A\to b$.
In the next section we show that our decision procedure is in PTIME. 
By Theorem~\ref{thm:LvsH} this is sufficient for
the decidability of $C\prvL s=t$. We recall a basic fact
about Horn clauses: the models (as satisfying sets of atoms)
are closed under intersection. Moreover, every set $X$ of atoms
can be extended to a unique minimal model; this minimal model
consists of all atoms that can be inferred from $X$ using the
Horn clauses as generating rules, as defined just before Lemma~\ref{lem:shift-up}. 

We proceed by defining an auxiliary notion that we call `gain'.
The \emph{gain} of a clause $\jbody{x}{k}{m}\to y+l$ is $l$ 
minus the minimum of $\set{k_1, ... ,k_m}$.
For example, predecessor clauses have gain $-1$.
The gain of a clause is invariant under shifting.

Let $\Ninf$ be $\N$ extended with $\infty$, totally ordered
by $n < \infty$ for all $n\in\N$. 
Given a finite semilattice presentation $(V,C)$, we view
a function $f: V\to\Ninf$ as specifying the downward closed set
of atoms $\set{v+k \mid v\in V,~k\in\N,~k \leq f(v)}$.
We are interested in such sets as models of $\upS{C}$.
\FYI{A clause $A\to b$ with 
$A = \jbody{x}{k}{m}$ and $b= y+l$, all $x_i,y \in V$,
is satisfied by $f$ if $l\leq f(v)$ when all $k_i \leq f(x_i)$.
Predecessor clauses are of course satisfied by downward closure.

\begin{lemma}\label{lem:f-sat}
Given $f: V\to\Ninf$ and a clause $A\to b$, let $P$ be the problem whether or not
$A+k\to b+k$ is satified by $f$ for all $k\in\N$. Then $P$ is decidable.
\end{lemma}
\begin{proof}
Assume $A = \jbody{x}{k}{m}$ and $b= y+l$ with $x_i,y \in V$.
Let $W$ consist of all variables $x_i$ in $A$ that satisfy $f(x_i)<\infty$.
If $W$ is empty, then $P$ is equivalent to $f(y)=\infty$.
Otherwise, let $k_0 = \min_{\set{i \mid x_i \in W}} (f(x_i)-k_i)$.
If $k_0 < 0$, then $P$ holds. If $k_0 \geq 0$, 
then $P$ is equivalent to $l+k_0 \leq f(y)$.
\end{proof}}

Given a finite semilattice presentation $(V,C)$
and a subset $W$ of $V$, we denote by $\upS{C}|W$ 
the set of clauses in $\upS{C}$ mentioning only variables in $W$,
and by $\upS{C}{\downarrow}W$
the set of clauses in $\upS{C}$ with conclusion over $W$.

\begin{theorem}\label{thm:main}
Let  $(V,C)$ be a finite semilattice presentation.
For any $f: V\to\Ninf$ we can compute
the least $g \geq f$ that is a model of $\upS{C}$.
\end{theorem}

We prepare the proof of this theorem with a lemma.

\begin{lemma}\label{lem:secondary}
Let  $(V,C)$ be a finite semilattice presentation.
Let $W$ be a strict subset of $V$ such that 
for any $f: W\to\Ninf$ we can compute
the least $g \geq f$ that is a model of $\upS{C}|W$.
Then for any $f: V\to\Ninf$ with $f(V-W)\subseteq \N$ we can compute
the least $h \geq f$ that is a model of $\upS{C}{\downarrow}W$.
\end{lemma}

\begin{proof}
Let conditions be as stated in the lemma.
\FYI{Since $(V,C)$ is finite we can compute the number
$\M$ such that each clause in $S_C$ has gain at most $\M$.}
Let $f: V\to\Ninf$ with $f(V-W)\subseteq \N$ be given
and denote its restriction to $W$ by $f_W$.
By the definition of $\upS{C}{\downarrow}W$, any minimal $h\geq f$
that is a model of $\upS{C}{\downarrow}W$
coincides with $f$ on $V-W$, so we focus on its values on $W$.
By assumption we can compute the least $g_f \geq f_W$ that 
is a model of $\upS{C}|W$.
Now we look at clauses in $\upS{C}{\downarrow}W - \upS{C}|W$. 
Such clauses are of the form $X,Y \to w+k$ with $X$ a non-empty
set of atoms over $V-W$, and possibly empty $Y$ over $W$. 
If $X = \ldots,x_i+k_i,\ldots$ is satisfied by $f$, 
then by the definition of $\M$, using $f(V-W)\subseteq \N$,
we have:
\begin{equation}\label{eq:f-ub}
k \leq\min_i(f(x_i)) + \M \leq \max(f(V-W)) + \M \in \N.
\end{equation}
\FYI{Inequality (\ref{eq:f-ub}) gives an upper bound on values that clauses 
in $\upS{C}{\downarrow}W - \upS{C}|W$ can generate.} Define:
\[
M(g_f) := \sum_{w \in W}  \max(0, \max(f(V-W)) + \M - g_f(w)).
\] 

\FYI{After these preparations we are ready to prove the lemma
by induction on $M(g_f)$. More precisely, we prove for all $n\in\N$
and $f: V\to\Ninf$ with $f(V-W)\subseteq \N$, if $M(g_f)=n$,
then we can compute
the least $h \geq f$ that is a model of $\upS{C}{\downarrow}W$.}

In the base case $M(g_f)=0$ we have $k\leq g_f(w)$ for all $w\in W$
and all clauses in $\upS{C}{\downarrow}W$ are satisfied by $g_f$.
In this case we take $h=g_f$ on $W$ and $h=f$ on $V-W$ and we are done.

For the induction step, let $M(g_f)>0$ and assume
the lemma has been proved for smaller values of $M(\_)$.
We now make a case distinction that is decidable
by Lemma~\ref{lem:f-sat} since
$S_C$ is finite (even though $\upS{C}$ is not).
Thus we only have to check finitely many clauses.
If all clauses in $\upS{C}{\downarrow}W - \upS{C}|W$ are satisfied 
by $g_f$ we are again done, like in the base case.
Otherwise, one such clause gives value $g_f(w)+k+1$ for some $w\in W$ 
and $k\in\N$, using values of $f$ on $V-W$ and values of $g_f$ on $W$. 
Then we know by (\ref{eq:f-ub}) that the term with $g_f(w)$
in the sum defining $M(g_f)$ is positive.
Define $f' : V \to \Ninf$ by 
\begin{align*}
f'(x)&= f(x)   \quad\quad\text{for $x$ in $V-W$,}\\ 
f'(y)&= g_f(y) \quad\quad\text{for $y$ in $W-\set{w}$, and}\\ 
f'(w)&= g_f(w)+k+1. 
\end{align*}
We have $g_{f'}(w) \geq f'(w) > g_f(w)$, so $M(g_{f'}) < M(g_f)$
and we can apply the induction hypothesis to $f'$.
The resulting $h$ for $f'$ also works for $f$,
since every step in the sequence 
$h \geq f' > g_f \geq f$
is by adding atoms that can be inferred from $f$.
\end{proof}

We now return to the proof of Theorem~\ref{thm:main}.

\begin{proof}
Let  $(V,C)$ be a finite semilattice presentation
such that each clause in $S_C$ has gain at most $\M$.
By induction on $|V|$ we compute, for any $f: V\to\Ninf$,
the least $g \geq f$ that is a model of $\upS{C}$.

In the base case $|V|=0$ there is nothing to prove.

For the induction step, let $|V|>0$ and assume the
theorem has been proved for smaller values of $|V|$.
Let $f: V\to\Ninf$. If $f(v)=\infty$ for some $v\in V$,
then we can eliminate $v$ form $\upS{C}$. Recall that
$f(v)=\infty$ means that all atoms of the form $v+k$
are true. This means that all clauses of the form
$\ldots\to v+k$ can be left out from $\upS{C}$.
Also, in each clause $A\to b$ in $\upS{C}$
we can leave out all atoms of the form $v+k$ from $A$.
If we get a (forbidden) clause $\emptyset\to v'+l$,
we can infer $f(v')=\infty$ and continue.
We end up with a strict subset $V'$ of $V$ and a (restricted)
$f : V' \to \N$.

Alternatively, we can do the simplification of the semilattice
presentation and end up with $(V',C')$. Here $C'$
is obtained from $C$ by removing all $v+k$ from the joins,
taking care to continue if a join becomes empty, and so on.%
\footnote{One can take $f(v)=\infty$ to mean $\bot=v=v+1$,
which yields $\bot=v+k$ for all $k\in\N$.}
Both methods lead to the same set of clauses $\upS{C'}$,
and this set satisfies all requirements, in particular
each clause has gain at most $\M$.

In such a case we can directly apply the induction hypothesis
to the simplified semilattice presentation, and extend the
function with values $\infty$ for all $v \in V-V'$.
Otherwise we have $f: V\to\N$.
We compute first the subset of variables $x\in V$ 
whose value $f(x)$ can be improved in one step,
by a clause $A\to x+k$ in $\upS{C}$ such that $A$ is
satisfied by $f$, yielding the result $k$ for $x$. 
Let $g(x)$ be the maximum of $f(x)$ and the results
for $x$ as obtained in this way. Since $S_C$ is finite,
$g$ is a function from $V$ to $\N$ whose values we can
compute. Let $W$ be the subset of $V$ consisting of
all $x$ such that $g(x) > f(x)$ and distinguish the
following cases.

If $W$ is empty we are done since then $f$ itself 
is a model of $\upS{C}$.

If $W = V$ we are done since then $h(x) = \infty$ for all 
$x \in V$ is the least $h \geq f$ that is a model of $\upS{C}$.
Proof: by using the predecessor clauses we can infer
$x+f(x)+1$ from every $x+g(x)$ with $g(x)>f(x)$ as above,
for every $x\in V$. By shifting upwards we can infer
$x+f(x)+k$, and hence $x+k$, for every $k\in\N$ and $x\in V$.

The third case is that $W$ is is a non-empty strict 
subset of $V$, and we can apply the induction
hypothesis to $W$ to satisfy the condition of 
Lemma~\ref{lem:secondary}. We apply the conclusion of 
the lemma to $g$, noting that $g(V-W)=f(V-W)$ is
a subset of $\N$. Hence we can compute the least 
$h\geq g$ that is a model of $\upS{C}{\downarrow}W$.
This function $h$ coincides with $g$ and $f$ on $V-W$. 

If $h(w) = \infty$ for some $w\in W$, 
then we simplify and apply the induction and are done 
as described in the first paragraphs of the step case.
Otherwise we have $h: V\to\N$.
We now make a case distinction that is decidable by 
Lemma~\ref{lem:f-sat}, since $S_C$ is finite.
If all clauses in $\upS{C}$ are satisfied by $h$ then we are done.
Otherwise, we can infer in one step a value $h(y)+k+1$ 
for some variable(s) $y$. Such $y$ must be in $V-W$ 
since $h$ is a model of $\upS{C}{\downarrow}W$.
For every $y\in V$, let $j(y)$ be the maximum of $h(y)$
and the values $h(y)+k+1$ that can possibly
be inferred in one step.
We extend $W$ to $W'$ with all $y$ such that $j(y)>h(y)$ 
and proceed with $W'$ and $j$ (to keep all work done)
in the same way as with $W$ and $g$ above.
This terminates since we exhaust $V$.
\end{proof}


From Theorem~\ref{thm:main} we get the decidability of $\upS{C}\prvH A\to b$
when $A$ can be represented by a function $f: V\to\N$. If every
variable $v$ in $V$ occurs in $A$, then $f(v)$ is simply
the maximal $k$ such that $v+k \in A$. We can then simply check
$\upS{C}\prvH A\to b$ by computing the least $g\geq f$ that is
a model of $\upS{C}$ and check whether atom $b$ is satisfied by $g$.

The decision method above for $\upS{C}\prvH A\to b$ only works
if every variable in $V$ occurs in $A$.
However, it is not difficult to extend Theorem~\ref{thm:main}
so that we get decidability of $\upS{C}\prvH A\to b$ in general. 
Let $T=\set{v^+\to v \mid v\in V}$. By Lemma~\ref{lem:shift-up}
we see that $\upS{C}\prvH A\to b$ is equivalent to
$\upS{C}^+ \cup T \prvH V,A^+ \to b^+$, and
$\upS{C}^+ \cup T$ is in fact $\upS{C^+}$, where
$C^+$ is the set of constraints $s^+ = t^+$ with $s=t$ in $C$.
 Thus we get:

\begin{corollary}\label{cor:decidability}
For all $A,b,s,t$, $\upS{C}\prvH A\to b$ and $C\prvL s=t$ are decidable.
\end{corollary}

Another application of Theorem~\ref{thm:main} is loop checking.
Given a finite semilattice presentation $(V,C)$, a \emph{loop} is a 
term $t$ over $V$ such that $C\prvL t^+=t$. Let $L$ be the
semilattice presented by $(V,C)$. Let $N$ be the semilattice $\N$ with
the usual $\max$ and endomorphism.

\begin{corollary}\label{cor:loopchecking}
Exactly one of the following two decidable cases holds:
(1) There is a loop;
(2) There is a homomorphism $h: L\to N$.
\end{corollary}
\begin{proof}
Let $m$ be maximal such that $x+m$ occurs in the body of a clause in $S_C$.
Take $f: V\to\N$ to be the constant $m$ function and compute $g$ according
to Theorem~\ref{thm:main}. Let $W$ be the subset of $V$ such that 
$g(w)=\infty$ for all $w\in W$. Claim: if $W$ is not empty,
then we have a loop, case (1), because there exists an $n\in\N$ such that 
$\upS{C}\prvH W+n \to w+n+1$ for all $w\in W$.

Proof of claim: if $W=V$, then $n=m$ and we are done,
otherwise take $n= \max(g(V-W))+\M$. The idea of this choice of $n$ is 
that variables in $V-W$ cannot play a role above $n$.
In order to see this, define $f': V\to\N$ by $f'(x)=g(x)$ if $x\in V-W$
and $f'(w)=n$ if $w\in W$. Then $g\geq f'\geq f$, so $g$ is also the minimal 
model when starting from $f'$. Since $f'$ and $g$ coincide on $V-W$ we have
that all clauses in $\upS{C}{\downarrow}(V-W)$ are satisfied by $f'$.
By the particular choice of $n$, using same reasoning as in the proof of 
Lemma~\ref{lem:secondary}, albeit with $f'$ instead of $f$, also
$\upS{C}{\downarrow}W - \upS{C}|W$ is satisfied by $f'$.
Hence the only clauses that play a role in computing $g$ are
clauses from $\upS{C}|W$, so we must have 
$\upS{C}\prvH W+n \to w+n+1$ for all $w\in W$.
It follows that $\vee_{w\in W} w+n$ is a loop.

If $W$ is empty we can construct a homomorphism $h: L\to N$, case (2).
Define $h(x) = \max(g(V))- g(x)$ for all $x\in V$. 
Extend $h$ to terms over $V$ by $h(t^+) = h(t)+1$ 
and $h(s\vee t) = \max(h(s),h(t))$.
We have to make sure that definition of $h$ respects
equality in $L$, that is, if $C\prvL s=t$, then $h(s)=h(t)$.
For this it suffices to show $h(s)=h(t)$ for all $s=t$ in $C$.
This can in turn be simplified to: $h(\jterm{x}{k}{m}) \geq h(y)+l$ 
for every $\jbody{x}{k}{m} \to y+l$ in $S_C$.
Easy calculations show that we must prove
$\min(g(x_1)-k_1,\ldots,g(x_m)-k_m) \leq g(y)-l$.
Wlog we assume that $g(x_1)-k_1$ is the minimum on the left.
Since $\jbody{x}{k}{m} \to y+l$ in $S_C$ we know 
that $g(x_1)\geq f(x_1)\geq k_1$. Shifting the clause
upwards by $g(x_1)-k_1$ we get the clause
$x_1+g(x_1),\ldots,x_m+k_m+g(x_1)-k_1 \to y+l+g(x_1)-k_1$
in $S_C^+$. Due to the assumption that $g(x_1)-k_1$ is minimal,
the body of this clause is satisfied by $g$.
Since $g$ is a model of $S_C^+$ by Theorem~\ref{thm:main},
the conclusion is also satisfied by $g$, that is,
$g(y)\geq l+g(x_1)-k_1$, so $g(y)-l\geq g(x_1)-k_1$.
This completes the proof that $h$ respects equality in $L$.

It should be clear that (1) and (2) exclude each other.
\end{proof}

\section{Complexity analysis}\label{sec:complexity}

All proofs in this note are constructive, so that they in fact
contain algorithms. In this section we shall show
that these algorithms are polynomial. 
The small-model property in Corollary~\ref{cor:small-model} below, 
a refinement of the small-model property in \cite{BNR},
will be instrumental.

Let's define the input size. The size $|E|$ of logical expression $E$ is
the number of symbols in $E$. The size $|f|$ of a function $f: V\to\Ninf$
is taken to be the \emph{maximum of all its values} $<\infty$. %$|V|$ plus ?
This choice for the size of $f$ implies that the complexity of some
algorithms depending on $f$ is weakly polynomial. 
However, for the important
Corollary~\ref{cor:decidability} and \ref{cor:loopchecking}
the algorithms are strongly polynomial, that is, in PTIME.
%The correspondence between semilattices and Horn clauses in 
%Section~\ref{sec:latt-Horn} is well-known to be polynomial.

Our algorithms are essentially performing just forward reasoning. 
However, since we have an infinite language, one has to take care
to terminate, which is explained in the inductive proofs. 
Moreover, termination should happen in a polynomial number of reasoning
steps, each taking at most polynomial time. We prepare 
by the following lemma.


\begin{lemma}\label{thm:compress-model}
Let  $(V,C)$ be a finite semilattice presentation such that all
clauses have maximal gain $\M$.
Let $f: V\to\Ninf$ be a model of $\upS{C}$ such that $V$ can be
partitioned as $V = L\cup H\cup I$ with $I= \set{v\in V \mid f(v)=\infty}$
and $f(x)-f(y) > \M$ for all $x\in H$ and $y\in L$.
Then $g: V\to\Ninf$ defined by $g(x)=f(x)-1$ for all $x \in H$ and $g(y)=f(y)$
for all $y\in L\cup I$ is also a model of $\upS{C}$.
\end{lemma}
\begin{proof}
We only have to check clauses with conclusion over $H$.
Let $\jbody{y}{k}{m} \to x+l$ be a clause in $\upS{C}$ with $y_i \in V$
and $x\in H$. If the premiss is satisfied in $g$, and some $y_i\in L$, then
\[
g(x)+1 = f(x) > f(y_i)+\M = g(y_i)+\M \geq k_i+\M \geq l.
\]
Hence $g(x)\geq l$, so also the conclusion holds in $g$.
If no $y_i\in L$, then we use that any clause $A\to b$ 
that only mentions variables in $H\cup I$
is satisfied in $g$ when $A^+ \to b^+$ is satisfied in $f$. 
\end{proof}

We immediately get the following small-model property.
\begin{corollary}\label{cor:small-model}
For any $f$, the least $g\geq f$ that is a model of $\upS{C}$
is bounded by $|f|+|V|*\M$.
\end{corollary}

We now analyse the complexity of various results point for point.

\begin{itemize}

\item\label{it:f-sat} The complexity of the test in Lemma~\ref{lem:f-sat}
is clearly polynomial in $|A\to b|$ and $|f|$.

\item In Theorem~\ref{thm:main} our choice for the size of $f$ becomes clear:
with only one clause $x,y \to y+1$ in $S_C$ and $f(y)=0$, forward reasoning
takes $f(x)+1$ steps to arrive at the model $f(y)=f(x)+1$.
This is polynomial in the value of $f(x)$, but not in its
binary representation.

The proof of Theorem~\ref{thm:main} is intertwined with the
proof of Lemma~\ref{thm:main}. In view of Corollary~\ref{cor:small-model}
it suffices that there is a polynomial bound on the work done for each 
forward inference and that there is steady progress in the global state encoded by
functions $f: V\to\Ninf$, until the algorithm terminates.
Both are easily verified by inspection of the proofs,
using that the test in Lemma~\ref{lem:f-sat} is polynomial . 
 
\item\label{it:Coros} In the statement of Corollary~\ref{cor:decidability}
and \ref{cor:decidability} there is no function $f: V\to\Ninf$. 
However, the proofs apply Theorem~\ref{thm:main} with such a function,
satisfying $|f| \leq |A|$ and $|f| \leq |S_C|$, respectively.
Hence both corollaries yield algorithms that are polynomial in the input size.

\end{itemize}

One may wonder what is the role of the assumption that the endomorphism
is inflationary, leading to the predecessor axioms in $\upS{C}$. A first
answer is that models of the predecessor axioms are downward closed,
leading to an efficient representation of models by functions $f: V\to\Ninf$.
Moreover we have the following example.

Let $p_i$ be the $i$-th prime number and consider clauses $x_i \to x_i + p_i$,
and $x_1+1,\ldots,x_n+1 \to y+1$, and $y+1 \to y$ as the only predecessor axiom.
Then we can infer $x_1,\ldots,x_n \to y$, but forward reasoning takes
exponentially many steps. 


\section{Discussion}\label{sec:discussion}

\subsection{Motivation}

The motivation for this problem comes from dependent type theory, 
where the relevant operations on universe levels are to take the
supremum of two levels, and to increment a level.

In order to avoid universe inconsistencies in type theory, 
it has been suggested in \cite{Huet87}, \cite{HarperP91}, \cite{VV}
to use constraints on universe levels.
In \cite{Huet87}, \cite{HarperP91} these constraints
are linear inequalities between universe levels.
In \cite{VV} also the maximum of two universe levels is used.
A typing would then
only be valid if its constraints can be inferred from the set
of constraints in the context. 
Moreover, the latter set should
be consistent in the sense that there are no loops. 
As defined above, a loop is a semilattice term that is equal to its successor; 
a good intuition is that loops lead to universe inconsistencies 
comparable to the paradoxes in set theory.

In type theory it is important that typing checking is decidable.
%and preferably even efficient. 
The results of this note show that having typings depend
on a set of universe level constraints preserves the decidability
of type checking.

Since dependent type theory is meant to be a foundation of mathematics,
we want to make minimal mathematical assumptions about the universe levels.
For example, we don't say that they are natural numbers
with a zero, successor and a maximum function (like Voevodsky in \cite{VV},
referring to Presburger Arithmetic for decidability).
Such assumptions would weaken, at least philosophically,
the foundational claim of dependent type theory:
natural numbers are introduced as an inductive type
at some later point in the development. For similar reasons
we don't assume that universe levels are totally ordered,
nor that the endomorphism is injective.

\subsection{Example}
Let $S_C$ consist of the clauses 
$a,b\to b+1$; $b\to c+3$; $c+1\to d$; $b,d+2\to e$.
We shall show how the proof of Theorem~\ref{thm:main} works to find the 
minimal model above the function that is constant $0$.
Sets of variables will be denoted by a string, e.g., $V=abcde$. 
We denote functions with domain $V$ by a string of values, e.g., $00000$.
(Digits will suffice in this example.) We have $\M=3$.

First we compute the function $g_0 = 01300$ with the maximal values
that can be obtained \emph{in one step} from $00000$. We have $W_0 = bc$.
(We give indices to $g,W$ from the proof since the proof is inductive 
and we'll need several local versions of these parameters.) 
The proof of Theorem~\ref{thm:main} now invokes
Lemma~\ref{lem:secondary} to compute the minimal model 
above $g_0$ of all clauses in $S$ with conclusion over $W_0$,
$a,b\to b+1$ and $b\to c+3$. Also the proof of the lemma is inductive,
but we immediately get that this minimal model is $h_0 = 01400$.

We now check whether $h_0$ is a model of $S_C$. It is not:
(only) the clause $c+1\to d$ is not satisfied, and the
maximal value for $d$ is $3$. So we continue with
$g_1 = 01430$ and $W_1 = bcd$ and compute the minimal
model $h_1$ of the clauses with conclusion over $W_1$,
which happens to be $g_1$ itself.

One more, very similar round yields
$g_2 = 01431$ and $W_2 = bcde$ and $h_2=g_2$,
which satisfies all clauses of $S_C$, and is the minimal 
model starting from the function that is constant $0$.

An interesting variation would be to add the clause
$e\to a$ and to see that the algorithm then detects
the loop. This is indeed the
case because $e\to a$ is not satisfied by $h_2$ and so
$g_3 = 11431$ and $W_3=abcde$ are computed.
Now $V=W_3$ and the loop has been detected,
the minimal model is the function that is constant $\infty$.


\subsection{Related work}


In the literature one can find a considerable body of research 
on efficient constraint solving in the same language as in this note,
but interpreted in the integers or in the rational numbers.
The latter are totally ordered, which makes the problem different.
Consider, for example, the constraint $x\vee y = x^+$.
When the ordering is total, this constraint implies $y = x^+$.
However, if $x$ and $y$ are incomparable, there are models
of $x\vee y = x^+$ in which $y = x^+$ is false. The simplest
such countermodel has three elements. The model of $x\vee y = x^+$
as described in the proof of Theorem~\ref{thm:LvsH} is based
on terms over $V=\set{x,y}$, modulo the congruence $\equiv$.
This congruence gives ${x+k \vee y+l} \equiv {x+l+1}$ if $k\leq l$
(use $x,y \to x+1$),
and ${x+k \vee y+l} \equiv {x+k}$ if $k>l$
(use $x+1\to y$).
In this model $y = x^+$ is false ($y\to x+1$ cannot be derived).

Another connection is with work on uniform word problems with
endomorphisms. Both \cite{Baader,Hofmann} show that this problem is decidable
but EXPTIME hard. We describe a PTIME algorithm in a special case: for
only {\em one} endomorphism which is furthermore inflationary.
It seems to be an open problem if there is a PTIME algorithm for
one endomorphism without any extra assumption. A similar
question can be asked in the case of a family endomorphisms
that are all {\em inflationary}.

It is simple to describe M. Hofmann \cite{Hofmann} algorithm with
our notation. This is a EXPTIME algorithm in the case where
we take away the rule $a+1\to a$. The idea is to provide a complete cut-free
derivation system. The search of a cut-free
proof of $A\to a$ given a set $R$ of primitive rules can then be done by a search of cut-free
proofs in this system. The derivations rules are the following.

\begin{enumerate}
\item $A\to a$ if $a$ is in $A$
\item $A,B+1\to b+1$ if $B\to b$
\item $A\to a$ if there is a rule $c_1,\dots,c_n\to d$ in $R$ such that 
  $A\to c_1,\dots,A\to c_n,A,d\to a$
\end{enumerate}




\bibliographystyle{plain}
\bibliography{refs}
\begin{thebibliography}{99}

\bibitem{Baader}
  Franz Baader, Sebastian Brandt and Carsten Lutz.
  \newblock{Pushing the EL Envelope.}
  \newblock{IJCAI 2005: 364-369.}

\bibitem{BNR}
M.A.~Bezem, R.~Nieuwenhuis and E.~Rodr\'iguez.
\newblock{The Max-Atom Problem and its Relevance}.
In I.~Cervesato, H.~Veith and A.~Voronkov, editors,
\emph{Proceedings LPAR-15}, LNAI 5330,
pages 47--62, Springer-Verlag, Berlin, 2008.

\bibitem{HarperP91}
Robert Harper and Robert Pollack.
\newblock\emph{Type checking with universes},
Theoretical Computer Science, 89, 107--136, 1991.

\bibitem{Hofmann}
  Martin Hofmann.
  \newblock{Proof-theoretic approach to description logic.}
  \newblock{LICS 2005: 229-237.}

\bibitem{Huet87}
G\'erard Huet.
\newblock\emph{Extending the calculus of constructions with {Type:Type}},
\newblock unpublished manuscript, April 1987.

\bibitem{Lorenzen51}
Paul Lorenzen.
  \newblock\emph{Algebraische und logistische Untersuchungen über freie Verbände},
Journal of Symbolic Logic, 16(2), 81--106, 1951.
English translation by Stefan Neuwirth: \url{https://arxiv.org/abs/1710.08138}

\bibitem{VV}
  Vladimir Voevodsky.
 \newblock\emph{A universe polymorphic type system},
manuscript, 
 \newblock \url{http://www.math.ias.edu/Voevodsky/voevodsky-publications_abstracts.html\#UPTS}, {2014}.



\end{thebibliography}

\end{document}

\begin{example}~\label{exa:abcd}

Clauses: $a \to b+1$, $b \to c+1$, $c \to d+1$, with
$V = a,b,c,d$

\begin{enumerate}
\item $W = b,c,d$, 
recursive call with $b \to c+1$, $c \to d+1$

\begin{enumerate}
\item  $W = c,d$, 
  recursive call with $c \to d+1$

\begin{enumerate}
\item  $W = d$,
    recursive call with empty set of clauses
\item     return $n(d) = 0$
\end{enumerate}

\item  return $n(c) = 0$,  $n(d) = 1$
\end{enumerate}

\item return $n(b) = 0$, $n(c) = 1$, $n(d) = 2$

\item no new activated clauses,
end result $n(a) = 0$, $n(b) = 1$, $n(c) = 2$, $n(d) = 3$
\end{enumerate}

\end{example}

\begin{example}

Clauses: $a \to b+1$, $b \to c+1$, $c \to d+1$, $d+1 \to e$, $e+1 \to f$, $f+1 \to g$,
with $V = a,b,c,d,e,f,g$

\begin{enumerate}
\item $W = b,c,d$, 
follows verbatim Example~\ref{exa:abcd} with
result $n(b) = 1$, $n(c) = 2$, $n(d) = 3$

\item only $d+1 \to e$ gets activated, we infer $e+1$ (this is inefficient, we could
infer $e+2$ and thereafter $f+1$)

\item $W = b,c,d,e$,
recursive call with $b \to c+1$, $c \to d+1$, $d+1 \to e$,
complete recomputation of
result $n(b) = 1$, $n(c) = 2$, $n(d) = 3$, extended with $n(e) = 2$ (highly inefficient)

\item only $e+1 \to f$ gets activated, we infer f+1

\item $W = b,c,d,e,f$,
recursive call with $b \to c+1$, $c \to d+1$, $d+1 \to e$, $e+1 \to f$,
complete recomputation of
result $n(b) = 1$, $n(c) = 2$, $n(d) = 3$, $n(e) = 2$, extended with $n(f) = 1$
(highly inefficient)

\item no new activated clauses,
end result $n(a) = 0$, $n(g) = 0$, $n(b) = 1$, $n(c) = 2$, $n(d) = 3$, $n(e) = 2$, $n(f) = 1$

\end{enumerate}

\end{example}
