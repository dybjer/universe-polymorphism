%\documentclass[12pt,a4paper]{amsart}
\documentclass[11pt,a4paper]{article}
%\ifx\pdfpageheight\undefined\PassOptionsToPackage{dvips}{graphicx}\else%
%\PassOptionsToPackage{pdftex}{graphicx}
\PassOptionsToPackage{pdftex}{color}
%\fi

%\usepackage{diagrams}

%\usepackage[all]{xy}
\usepackage{url}
\usepackage[utf8]{inputenc}
\usepackage{verbatim}
\usepackage{latexsym}
\usepackage{amssymb,amstext,amsmath,amsthm}
\usepackage{epsf}
\usepackage{epsfig}
% \usepackage{isolatin1}
\usepackage{a4wide}
\usepackage{verbatim}
\usepackage{proof}
\usepackage{latexsym}
%\usepackage{mytheorems}
\newtheorem{theorem}{Theorem}[section]
\newtheorem{corollary}{Corollary}[theorem]
\newtheorem{lemma}{Lemma}[theorem]
\newtheorem{proposition}{Proposition}[theorem]
\newtheorem{example}{Example}[theorem]

\usepackage{float}
\floatstyle{boxed}
\restylefloat{figure}


%%%%%%%%%copied from SymmetryBook by Marc

% hyperref should be the package loaded last
\usepackage[backref=page,
            colorlinks,
            citecolor=linkcolor,
            linkcolor=linkcolor,
            urlcolor=linkcolor,
            unicode,
            pdfauthor={CAS},
            pdftitle={Symmetry},
            pdfsubject={Mathematics},
            pdfkeywords={type theory, group theory, univalence axiom}]{hyperref}
% - except for cleveref!
\usepackage[capitalize]{cleveref}
%\usepackage{xifthen}
\usepackage{xcolor}
\definecolor{linkcolor}{rgb}{0,0,0.5}

%%%%%%%%%
\def\oge{\leavevmode\raise
.3ex\hbox{$\scriptscriptstyle\langle\!\langle\,$}}
\def\feg{\leavevmode\raise
.3ex\hbox{$\scriptscriptstyle\,\rangle\!\rangle$}}

%%%%%%%%%
\newcommand\myfrac[2]{
 \begin{array}{c}
 #1 \\
 \hline \hline
 #2
\end{array}}


\newcommand*{\Scale}[2][4]{\scalebox{#1}{$#2$}}%
\newcommand*{\Resize}[2]{\resizebox{#1}{!}{$#2$}}

\newcommand{\II}{\mathbb{I}}
\newcommand{\refl}{\mathsf{refl}}
\newcommand{\mkbox}[1]{\ensuremath{#1}}


\newcommand{\Id}{\mathsf{Id}}
\newcommand{\conv}{=}
%\newcommand{\conv}{\mathsf{conv}}
\newcommand{\lam}[2]{{\langle}#1{\rangle}#2}
\def\NN{\mathsf{N}}
\def\UU{\mathsf{U}}
\def\JJ{\mathsf{J}}
\def\Level{\mathsf{Level}}
%\def\Type{\hbox{\sf Type}}
\def\ZERO{\mathsf{0}}
\def\SUCC{\mathsf{S}}

\newcommand{\type}{\mathsf{type}}
\newcommand{\N}{\mathsf{N}}
\newcommand{\Set}{\mathsf{Set}}
\newcommand{\El}{\mathsf{El}}
%\newcommand{\U}{\mathsf{U}} clashes with def's in new packages
\newcommand{\T}{\mathsf{T}}
\newcommand{\Usuper}{\UU_{\mathrm{super}}}
\newcommand{\Tsuper}{\T_{\mathrm{super}}}
%\newcommand{\conv}{\mathrm{conv}}
\newcommand{\idtoeq}{\mathsf{idtoeq}}
\newcommand{\isEquiv}{\mathsf{isEquiv}}
\newcommand{\ua}{\mathsf{ua}}
\newcommand{\UA}{\mathsf{UA}}
%\newcommand{\Level}{\mathrm{Level}}
\def\Constraint{\mathsf{Constraint}}
\def\Ordo{\mathcal{O}}

\def\Ctx{\mathrm{Ctx}}
\def\Ty{\mathrm{Ty}}
\def\Tm{\mathrm{Tm}}

\def\CComega{\mathrm{CC}^\omega}
\setlength{\oddsidemargin}{0in} % so, left margin is 1in
\setlength{\textwidth}{6.27in} % so, right margin is 1in
\setlength{\topmargin}{0in} % so, top margin is 1in
\setlength{\headheight}{0in}
\setlength{\headsep}{0in}
\setlength{\textheight}{9.19in} % so, foot margin is 1.5in
\setlength{\footskip}{.8in}

% Definition of \placetitle
% Want to do an alternative which takes arguments
% for the names, authors etc.

\newcommand{\natrec}{\mathsf{natrec}}
\newcommand{\nats}{\mathbb{N}}
%\rightfooter{}

%added by Marc

\newcommand\set[1]{\{#1\}}
\newcommand\sct[1]{[\![#1]\!]}
\newcommand\jterm[3]{{{#1_1}+{#2_1}}\vee\cdots\vee{{#1_#3}+{#2_#3}}}
\newcommand\jbody[3]{{{#1_1}+{#2_1}},\ldots,{{#1_#3}+{#2_#3}}}
\newcommand\lathy{{\cal L}}
\newcommand\prvL{\vdash_{\lathy}}
\newcommand\prvH{\vdash_{\cal H}}
\newcommand\Ninf{\N^+}
\newcommand\M{\mathsf{Maxgain}}




\begin{document}

\title{Decidability of the theory of join-semilattices with a successor}
%\title{Decidability of join-semilattices with a successor}

%\author{Marc Bezem, Thierry Coquand, Peter Dybjer, Mart\'in Escard\'o}
\author{} \date{}


\maketitle


\section{Introduction}
We prove the decidability of join-semilattice theory with
a successor function.
We start from an algebraic definition of semilattices in which the
join, denoted by $\vee$, is an associative, commutative, and idempotent
binary operation. The successor function, denoted by ${\_}^+$,
satisfies the following two equational axioms:

\[
x \vee x^+ = x^+ \quad \quad (x\vee y)^+ = x^+ \vee y^+
\] 
The logic is ordinary equational logic.
We denote the resulting theory by $\lathy$.
For example, we can prove $(t^+)^+ \vee t = (t^+)^+$ in $\lathy$,
for any lattice term $t$. Also, we can infer $s^+ = (t^+)^+$
from $s = t^+$, but not conversely (?? see below for a discussion
of injectivity of the successor function). 
As is customary, we let $s\geq t$ abbreviate $s\vee t = s$.


\section{Lattice presentations and Horn clauses}\label{sec:latt-Horn}

For any lattice term $t$ and natural number $k$, 
let $t+k$ denote the $k$-fold successor of $t$.
Since the successor function commutes we the join operation,
every lattice term $t$ is equal to a term of the
form $\jterm{x}{k}{m}$, with all variables $x_i$ occurring in $t$
and all $k_i\in \N$.

A \emph{lattice presentation} consists of a
set $V$ of \emph{generators} and
a set $C$ of \emph{relations}. A \emph{term over} $V$ is
a term of the form $\jterm{x}{k}{m}$, 
with all $x_i\in V$ and $k_i\in \N$.
A relation is an equation $s=t$ with $s,t$ terms over $V$.
We will colloquially call the generators also variables, and the
relations constraints. A constraint like $x=y^+$ with $x,y\in V$
expresses a relation between the generators $x$ and $y$ and 
should not be read as an implicitly universally quantified axiom
in which the $x$ and $y$ can be instantiated.

The lattice presented by $(V,C)$ consists of terms over $V$
modulo provable equality from $C$. The latter will be denoted
by $C\prvL s=t$. In the next section we will prove that
$C\prvL s=t$ is decidable for finite lattice presentations $(V,C)$.
The results below hold for arbitrary lattice presentations.

We follow Skolem (??), Lorenzen (??) for an equivalent 
characterisation of $C\prvL s=t$ using Horn clauses.
Let $(V,C)$ be a lattice presentation.
Let $s:= \jterm{x}{k}{m}$ and $t:= \jterm{y}{l}{n}$ be
terms over $V$. From the constraint $s=t$
we can prove $m+n$ inequalities which we write as 
the set $S_{s=t}$ consisting of the following 
Horn clauses\footnote{In this note all clauses are positive,
propositional Horn clauses with a non-empty body.}
:
\begin{align*}
\jbody{x}{k}{m} &\to y_1+l_1 \\
&\ldots  \\
\jbody{x}{k}{m} &\to y_n+l_n 
\end{align*}
and
\begin{align*}
\jbody{y}{l}{n} &\to x_1+k_1 \\
&\ldots \\
\jbody{y}{l}{n} &\to x_m+k_m 
\end{align*}
Let $S_C$ be the union of all $S_{s=t}$ with $s=t$ 
a constraint in $C$.
Such Horn clauses are to be read in propositional logic
with abstract atoms of the form $x+k$ with $x\in V$ and $k\in \N$.

We reflect for a moment on which other clauses we need.
Consider the axiom $x \vee x^+ = x^+$. This would lead to three
clauses: $x,x^+ \to x^+$, $x^+ \to x^+$, and $x^+ \to x$.
Only the last is non-trivial, we call it a \emph{predecessor clause}.
The next question is: for which $x$ do we need a predecessor clause?
Since $x \vee x^+ = x^+$ is implicitly universally quantified,
we would need all instances with $x$ a term over $V$.
For this it suffices to have all predecessor clauses
$x+k+1 \to x+k$ with $x\in V$ and all $k\in \N$.

The axiom stating that successor and join commute is built-in
in the notion of term over $V$ and does not require extra clauses.
However, we should not forget the congruence axioms from
equational logic. Congruence of the successor means that $s=t$
implies $s+1=t+1$. This requires that we close the set of clauses
under \emph{shifting upwards}: if $A\to b$ is in the clause set,
then so is $A+1\to b+1$, where $A+1$ is the set of atoms of
the form $a+1$ with $a\in A$. 
Congruence of join means that $s=t$ implies $r\vee s=r\vee t$.
It is easy to see that this does not require extra clauses. 

In summary: given a lattice presentation $(V,C)$,
let $S^+_C$ be the smallest set of clauses that 
is closed under shifting upwards and contains
the set $S_C$ coming from the constraint in $C$,
as well as all predecessor clauses for each $v\in V$. 

Given a set $S$ of Horn clauses,
let $S\prvH A\to b$ denote provability from $S$.
Recall that provability from $S$ is defined inductively by: 
$S\prvH a\to a$;
$S\prvH A\to b$ if $A\to b$ in $S$;
$S\prvH A,c\to b$ if $S\prvH A\to b$;
$S\prvH A\to b$ if $S\prvH A\to c$ and $S\prvH A,c\to b$. 

We now have (cf.\ Lorenzen, Thm 3) the following theorem.

\begin{theorem}\label{thm:LvsH}
For all terms $\jbody{x}{k}{m},y+l$ over $V$ we have:
\[
C\prvL \jterm{x}{k}{m}\geq y+l 
\quad\text{iff}\quad
S^+_C\prvH \jbody{x}{k}{m}\to y+l.
\]
\end{theorem} 
\begin{proof}
The if-part is a simple induction on derivation of $\prvH$:
all steps can easily be mimicked in lattice theory.
The converse implication is more interesting.
For any set of Horn clauses $X,Y$, let $X \prvH Y$ mean
that $X \prvH A\to b$ for all clauses $A\to b$ in $Y$.
For any two terms $s,t$ over $V$, define $s\equiv t$ by
$S^+_C \prvH S_{s=t}$. We define $s\vee t$ in the obvious way,
and $(\jterm{x}{k}{m})^+ := x_1+k_1+1 \vee\cdots\vee x_m+k_m+1$.
The latter is well-defined, since we have $s^+\equiv t^+$ 
if $s\equiv t$, as $S^+_C$ is closed under shifting upwards.

Now one can show that all axioms are satisfied modulo $\equiv$. 
For example, $\equiv$ is an equivalence relation and
a congruence with respect to join and successor.
All lattice axioms are satisfied, for example,
the predecessor clauses prove $s\vee s^+ \equiv s^+$. 
Moreover, for each constraint $s=t$ in $C$ 
we have $s\equiv t$, as $S_{s=t}$ is included in $S^+_C$.

By soundness, if $C\prvL s=t$, then $s\equiv t$.
In particular we have the only-if part of the theorem.
\end{proof}

\section{Decidability}

In this section we prove the decidability of $S^+_C\prvH A\to b$.
By Theorem~\ref{thm:LvsH} this is sufficient for
the decidability of $C\prvL s=t$. We recall a basic fact
about Horn clauses: the models (as satisfying sets of atoms)
are closed under intersection. Moreover, every set $X$ of atoms
can be extended to a unique minimal model; this minimal model
consists of all atoms that can be inferred from $X$ using the
Horn clauses as generating rules. 

We proceed by defining an auxiliary notion that we call `gain'.
The \emph{gain} of a clause $\jbody{x}{k}{m}\to y+l$ is $l$ 
minus the minimum of $\set{k_1, ... ,k_m}$.
For example, predecessor clauses have gain $-1$s.
The gain of a clause is invariant under shifting.
In this section we fix a constant $\M$ and consider
finite lattice presentations $(V,C)$ such that
each clause in $S_C$ has a gain at most $\M$.

Let $\Ninf$ be $\N$ extended with $\infty$, totally ordered
by $n < \infty$ for all $n\in\N$. 
Given a finite lattice presentation $(V,C)$, we view
a function $f: V\to\Ninf$ as specifying the set
of atoms $\set{v+k \mid v\in V,~k\in\N,~k \leq f(v)}$.
We are interested in such sets as models of $S^+_C$.
We shall prove the following theorem.

Given a finite lattice presentation $(V,C)$
and a subset $W$ of $V$, we denote by $S^+_C|W$ 
the set of clauses in $S^+_C$ mentioning only variables in $W$,
and by $S^+_C{\downarrow}W$
the set of clauses in $S^+_C$ with conclusion over $W$.

\begin{theorem}\label{thm:main}
Let  $(V,C)$ be a finite lattice presentation
such that each clause in $S_C$ has gain at most $\M$.
For any $f: V\to\Ninf$ we can compute
the least $g \geq f$ that is a model of $S^+_C$.
\end{theorem}

We prepare the proof of this theorem with a lemma.

\begin{lemma}\label{lem:secondary}
Let  $(V,C)$ be a finite lattice presentation
such that each clause in $S_C$ has gain at most $\M$.
Let $W$ be a strict subset of $V$ such that 
for any $f: W\to\Ninf$ we can compute
the least $g \geq f$ that is a model of $S^+_C|W$.
Then for any $f: V\to\Ninf$ with $f(V-W)\subseteq \N$ we can compute
the least $h \geq f$ that is a model of $S^+_C{\downarrow}W$.
\end{lemma}

\begin{proof}
Let conditions be as stated in the lemma.
By the definition of $S^+_C{\downarrow}W$, any minimal $h\geq f$
that is a model of $S^+_C{\downarrow}W$
coincides with $f$ on $V-W$, so we focus on its values on $W$.
For any $f: V\to \Ninf$, let $f_W$ be its restriction to $W$.
By assumption we can compute the least $g_f \geq f_W$ that 
is a model of $S^+_C|W$.
Now we look at clauses in $S^+_C{\downarrow}W - S^+_C|W$. 
Such clauses are of the form $X,Y \to w+k$ with $X$ a non-empty
set of atoms over $V-W$, and possibly empty $Y$ over $W$. 
If $X = \ldots,x_i+k_i,\ldots$ is satisfied by $f$, 
then 
\[
k \leq\min_i(f(x_i)) + \M \leq \max(f(V-W)) + \M.
\]								
Using that $f(V-W)$ is a subset of $\N$, we do induction on 
\[
M(g_f) := \sum_{w \in W}  \max(0, \max(f(V-W)) + \M - g_f(w)).
\] 

In the base case we have $M(g_f)=0$.
Then all clauses in $S^+_C{\downarrow}W$ are satisfied,
and hence we can take $h=g_f$ on $W$ and $h=f$ on $V-W$.

For the induction step, let $M(g_f)>0$ and assume
the lemma has been proved for smaller values of $M(g_f)$.
We now make a case distinction that is decidable since
the lattice presentation $(V,C)$ is finite,
and hence $S_C$ is finite (even though $S^+_C$ is not).
Thus we only have to check finitely many clauses.
If all clauses in $S^+_C{\downarrow}W - S^+_C|W$ are satisfied 
by $g_f$ we are again done, like in the base case.
Otherwise, assume we can infer the value $g_f(w)+k+1$
for $w\in W$, using values of $f$ on $V-W$ and values 
of $g_f$ on $W$. Then we know that the term with $g_f(w)$
in the sum defining $M(g_f)$ is positive.
Define $f' : V -> \Ninf$ by 
\begin{align*}
f'(x)&= f(x)   \quad\quad\text{for $x$ in $V-W$,}\\ 
f'(y)&= g_f(y) \quad\quad\text{for $y$ in $W-\set{w}$, and}\\ 
f'(w)&= g_f(w)+k+1. 
\end{align*}
We have $g_{f'}(w) \geq f'(w) > g_f(w)$, so $M(g_{f'}) < M(g_f)$
and we can apply the induction hypothesis to $f'$.
The resulting $h$ for $f'$ also works for $f$,
since every step in the sequence 
$h \geq f' > g_f \geq f$
is by adding atoms that can be inferred from $f$.
\end{proof}

We now return to the proof of Theorem~\ref{thm:main}.

\begin{proof}
Let  $(V,C)$ be a finite lattice presentation
such that each clause in $S_C$ has gain at most $\M$.
By induction on $|V|$ we compute, for any $f: V\to\Ninf$,
the least $g \geq f$ that is a model of $S^+_C$.

In the base case $|V|=0$ there is nothing to prove.

For the induction step, let $|V|>0$ and assume the
theorem has been proved for smaller values of $|V|$.
Let $f: V\to\Ninf$. If $f(v)=\infty$ for some $v\in V$,
then we can eliminate $v$ form $S^+_C$. Recall that
$f(v)=\infty$ means that all atoms of the form $v+k$
are true. This means that all clauses of the form
$\ldots\to v+k$ can be left out from $S^+_C$.
Also, in each clause $A\to b$ in $S^+_C$
we can leave out all atoms of the form $v+k$ from $A$.
If we get a (forbidden) clause $\emptyset\to v'+l$,
we can infer $f(v')=\infty$ and continue.
We end up with a strict subset $V'$ of $V$ and a (restricted)
$f : V' \to \N$.

Alternatively, we can do the simplification of the lattice
presentation and end up with $(V',C')$. Here $C'$
is obtained from $C$ by removing all $v+k$ from the joins,
taking care to continue if a join becomes empty, and so on.%
\footnote{One can take $f(v)=\infty$ to mean $\bot=v=v+1$,
which yields $\bot=v+k$ for all $k\in\N$.}
Both methods lead to the same set of clauses $S^+_{C'}$,
and this set satisfies all requirements, in particular
each clause has gain at most $\M$.

In such a case we can directly apply the induction hypothesis
to the simplified lattice presentation, and extend the
function with values $\infty$ for all $v \in V-V'$.
Otherwise we have $f: V\to\N$.
We compute first the subset of variables $x\in V$ 
whose value $f(x)$ can be improved in one step,
by a clause $A\to x+k$ in $S^+_C$ such that $A$ is
satisfied by $f$, yielding the result $k$ for $x$. 
Let $g(x)$ be the maximum of $f(x)$ and the results
for $x$ as obtained in this way. Since $S_C$ is finite,
$g$ is a function from $V$ to $\N$ whose values we can
compute. Let $W$ be the subset of $V$ consisting of
all $x$ such that $g(x) > f(x)$ and distinguish the
following cases.

If $W$ is empty we are done since then $f$ itself 
is a model of $S^+_C$.

If $W = V$ we are done since then $h(x) = \infty$ for all 
$x \in V$ is the least $h \geq f$ that is a model of $S^+_C$.
Proof: by using the predecessor clauses we can infer
$x+f(x)+1$ from every $x+g(x)$ with $g(x)>f(x)$ as above,
for every $x\in V$. By shifting upwards we can infer
$x+f(x)+k$, and hence $x+k$, for every $k\in\N$ and $x\in V$.

The third case is that $W$ is is a non-empty strict 
subset of $V$, and we can apply the induction
hypothesis to $W$ to satisfy the condition of 
Lemma~\ref{lem:secondary}. We apply the conclusion of 
the lemma to $g$, noting that $g(V-W)=f(V-W)$ is
a subset of $\N$. Hence we can compute the least 
$h\geq g$ that is a model of $S^+_C{\downarrow}W$.
This function $h$ coincides with $g$ and $f$ on $V-W$. 

If $h(w) = \infty$ for some $w\in W$, 
then we simplify and apply the induction and are done 
as described in the first paragraphs of the step case.
Otherwise we have $h: V\to\N$.
Like in the proof of Lemma~\ref{lem:secondary},
we make a case distinction that is decidable since
the lattice presentation $(V,C)$ is finite,
and hence $S_C$ is finite (even though $S^+_C$ is not).
Thus we only have to check finitely many clauses.
If all clauses in $S^+_C$ are satisfied by $h$ then we are done.
Otherwise, we can infer in one step a value $h(y)+k+1$ 
for some variable(s) $y$. Such $y$ must be in $V-W$ 
since $h$ is a model of $S^+_C{\downarrow}W$.
For every $y\in V$, let $j(y)$ be the maximum of $h(y)$
and the values $h(y)+k+1$ that can possibly
be inferred in one step.
We extend $W$ to $W'$ with all $y$ such that $j(y)>h(y)$ 
and proceed with $W'$ and $j$ (to keep all work done)
in the same way as with $W$ and $g$ above.
This terminates since we exhaust $V$.
\end{proof}

{\color{blue}
From Theorem~\ref{thm:main} we get the decidability of $S^+_C\prvH A\to b$
when $A$ can be represented by a function $f: V\to\N$. If every
variable $v$ in $V$ occurs in $A$, then $f(v)$ is simply
the maximal $k$ such that $v+k \in A$. We can then simply check
$S^+_C\prvH A\to b$ by computing the least $g\geq f$ that is
a model of $S^+_C$ and check whether atom $b$ is satisfied by $g$.

Another application of Theorem~\ref{thm:main} is loop checking.
Let $k$ be maximal such that $x+k$ occurs in the body of
a clause in $S_C$ ($x\in V$).
Take $f: V\to\N$ to be the constant $k$ function and compute $g$.
Let $W$ be the subset of $V$ such that $g(w)=\infty$ for
all $w\in W$. Then there exist $n\in\N$ such that 
$S^+_C\prvH W+n \to w+n+1$ for all $w\in W$.
Proof: if $W=V$, then $n=k$, otherwise $n= \max(g(V-W))+\M$,
since variables in $V-W$ cannot play a role above $n$.

The decision method above for $S^+_C\prvH A\to b$ only works
if every variable in $V$ occurs in $A$.
However, it is not difficult to extend Theorem~\ref{thm:main}
such that we get decidability of $S^+_C\prvH A\to b$ in
general. For example, we can allow $f: V\to\Ninf$ to be partial.
If $f(x)$ is undefined, we must put $k\leq f(x)$ false,
since $\set{x+k \mid k\leq f(x)}$ must be empty.
On the other hand, it is possible that we can infer a value for $x$
even if $f(x)$ is undefined, so we must put $f(x)\leq k$ true
for all $k\in\Ninf$. An elegant solution to this
problem is to use $f(x)=-1$ to mean that $f(x)$ is undefined.
One can then verify that Theorem~\ref{thm:main} and
Lemma~\ref{lem:secondary} go through with minimal adaptations
in their proofs.
} %end color

\section{Discussion}

\subsection{Motivation}

The motivation for this problem comes from dependent type theory, 
where the relevant operations on universe levels are taking the
supremum of two levels, and incrementing a level.

One novel, rigorous way to avoid so-called universe inconsistencies
in type theory would be to let the typings depend not only
on a context assigning types to variables, but also on a
finite set of universe level constraints. A typing would then
only be valid if its constraints can be inferred from the set
of constraints in the context. Moreover, the latter set should
be consistent in the sense that there are no loops. 
A loop is a lattice term that is equal to its successor; 
a good intuition is that loops lead to universe-inconsistencies 
comparable to the paradoxes in set theory.

In type theory it is important that typing checking is decidable.
%and preferably even efficient. 
The results of this note show that having typings depend
on a set of universe level constraints does not make 
type checking undecidable.

{\color{blue}
Since dependent type theory is meant to be a foundation of mathematics,
we want to make minimal mathematical assumptions about the universe levels.
For example, we don't say that they are natural numbers
with a zero, successor and a maximum function.
Such assumptions would weaken, at least philosophically,
the foundational claim of dependent type theory:
natural numbers are introduced as an inductive type
at some later point in the development. For similar reasons
we don't assume that universe levels are totally ordered,
nor that the successor function is injective.
} %end color

\subsection{Related work}

{\color{blue}
There exist in the literature a considerable body of research 
on efficient constraint solving in the same language as in this note,
but interpreted in the integers or in the rational numbers.
The latter are totally ordered, which makes the problem different.
Consider, for example, the constraint $x\vee y = x^+$.
When the ordering is total, this constraint implies $y = x^+$.
However, if $x$ and $y$ are incomparable, there are countermodels
of $x\vee y = x^+$ in which $y = x^+$ is false. The simplest
countermodel has three points. The model of $x\vee y = x^+$
as described in the proof of Theorem~\ref{thm:LvsH} is based
on $V$-terms with $V=\set{x,y}$, modulo the congruence $\equiv$.
This congruence gives ${x+k \vee y+l} \equiv {x+l+1}$ if $k\leq l$
(use $x,y \to x+1$),
and ${x+k \vee y+l} \equiv {x+k}$ if $k>l$
(use $x+1\to y$).
In this model $y = x^+$ is false ($y\to x+1$ cannot be derived).
} %end color

\bibliographystyle{plain}
\bibliography{refs}
\begin{thebibliography}{99}
\bibitem{VV}
  Vladimir Voevodsky,
 \newblock\emph{Universe polymorphic type system}.
Manuscript, \url{http://www.math.ias.edu/Voevodsky/voevodsky-publications_abstracts.html\#UPTS},
  {2014}
\end{thebibliography}

\end{document}

\begin{example}~\label{exa:abcd}

Clauses: $a \to b+1$, $b \to c+1$, $c \to d+1$, with
$V = a,b,c,d$

\begin{enumerate}
\item $W = b,c,d$, 
recursive call with $b \to c+1$, $c \to d+1$

\begin{enumerate}
\item  $W = c,d$, 
  recursive call with $c \to d+1$

\begin{enumerate}
\item  $W = d$,
    recursive call with empty set of clauses
\item     return $n(d) = 0$
\end{enumerate}

\item  return $n(c) = 0$,  $n(d) = 1$
\end{enumerate}

\item return $n(b) = 0$, $n(c) = 1$, $n(d) = 2$

\item no new activated clauses,
end result $n(a) = 0$, $n(b) = 1$, $n(c) = 2$, $n(d) = 3$
\end{enumerate}

\end{example}

\begin{example}

Clauses: $a \to b+1$, $b \to c+1$, $c \to d+1$, $d+1 \to e$, $e+1 \to f$, $f+1 \to g$,
with $V = a,b,c,d,e,f,g$

\begin{enumerate}
\item $W = b,c,d$, 
follows verbatim Example~\ref{exa:abcd} with
result $n(b) = 1$, $n(c) = 2$, $n(d) = 3$

\item only $d+1 \to e$ gets activated, we infer $e+1$ (this is inefficient, we could
infer $e+2$ and thereafter $f+1$)

\item $W = b,c,d,e$,
recursive call with $b \to c+1$, $c \to d+1$, $d+1 \to e$,
complete recomputation of
result $n(b) = 1$, $n(c) = 2$, $n(d) = 3$, extended with $n(e) = 2$ (highly inefficient)

\item only $e+1 \to f$ gets activated, we infer f+1

\item $W = b,c,d,e,f$,
recursive call with $b \to c+1$, $c \to d+1$, $d+1 \to e$, $e+1 \to f$,
complete recomputation of
result $n(b) = 1$, $n(c) = 2$, $n(d) = 3$, $n(e) = 2$, extended with $n(f) = 1$
(highly inefficient)

\item no new activated clauses,
end result $n(a) = 0$, $n(g) = 0$, $n(b) = 1$, $n(c) = 2$, $n(d) = 3$, $n(e) = 2$, $n(f) = 1$

\end{enumerate}

\end{example}
